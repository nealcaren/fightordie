%% Generated by Sphinx.
\def\sphinxdocclass{jupyterBook}
\documentclass[letterpaper,10pt,english]{jupyterBook}
\ifdefined\pdfpxdimen
   \let\sphinxpxdimen\pdfpxdimen\else\newdimen\sphinxpxdimen
\fi \sphinxpxdimen=.75bp\relax
\ifdefined\pdfimageresolution
    \pdfimageresolution= \numexpr \dimexpr1in\relax/\sphinxpxdimen\relax
\fi
%% let collapsible pdf bookmarks panel have high depth per default
\PassOptionsToPackage{bookmarksdepth=5}{hyperref}
%% turn off hyperref patch of \index as sphinx.xdy xindy module takes care of
%% suitable \hyperpage mark-up, working around hyperref-xindy incompatibility
\PassOptionsToPackage{hyperindex=false}{hyperref}
%% memoir class requires extra handling
\makeatletter\@ifclassloaded{memoir}
{\ifdefined\memhyperindexfalse\memhyperindexfalse\fi}{}\makeatother

\PassOptionsToPackage{warn}{textcomp}

\catcode`^^^^00a0\active\protected\def^^^^00a0{\leavevmode\nobreak\ }
\usepackage{cmap}
\usepackage{fontspec}
\defaultfontfeatures[\rmfamily,\sffamily,\ttfamily]{}
\usepackage{amsmath,amssymb,amstext}
\usepackage{polyglossia}
\setmainlanguage{english}



\setmainfont{FreeSerif}[
  Extension      = .otf,
  UprightFont    = *,
  ItalicFont     = *Italic,
  BoldFont       = *Bold,
  BoldItalicFont = *BoldItalic
]
\setsansfont{FreeSans}[
  Extension      = .otf,
  UprightFont    = *,
  ItalicFont     = *Oblique,
  BoldFont       = *Bold,
  BoldItalicFont = *BoldOblique,
]
\setmonofont{FreeMono}[
  Extension      = .otf,
  UprightFont    = *,
  ItalicFont     = *Oblique,
  BoldFont       = *Bold,
  BoldItalicFont = *BoldOblique,
]



\usepackage[Bjarne]{fncychap}
\usepackage[,numfigreset=1,mathnumfig]{sphinx}

\fvset{fontsize=\small}
\usepackage{geometry}


% Include hyperref last.
\usepackage{hyperref}
% Fix anchor placement for figures with captions.
\usepackage{hypcap}% it must be loaded after hyperref.
% Set up styles of URL: it should be placed after hyperref.
\urlstyle{same}


\usepackage{sphinxmessages}



        % Start of preamble defined in sphinx-jupyterbook-latex %
         \usepackage[Latin,Greek]{ucharclasses}
        \usepackage{unicode-math}
        % fixing title of the toc
        \addto\captionsenglish{\renewcommand{\contentsname}{Contents}}
        \hypersetup{
            pdfencoding=auto,
            psdextra
        }
        % End of preamble defined in sphinx-jupyterbook-latex %
        

\title{Dare you Fight&#58;<br/>W.E.B. Du Bois in The Crisis}
\date{Apr 12, 2022}
\release{}
\author{William E.\@{} Burghardt Du Bois}
\newcommand{\sphinxlogo}{\vbox{}}
\renewcommand{\releasename}{}
\makeindex
\begin{document}

\pagestyle{empty}
\sphinxmaketitle
\pagestyle{plain}
\sphinxtableofcontents
\pagestyle{normal}
\phantomsection\label{\detokenize{index::doc}}


\sphinxAtStartPar
This is an ongoing project to make available some of the editorials published by The Crisis, the official journal of the NAACP, between 1910 and 1934, when it was edited by W. E. B. Du Bois.
\begin{itemize}
\item {} 
\sphinxAtStartPar
{\hyperref[\detokenize{introduction::doc}]{\sphinxcrossref{Introduction}}}

\item {} 
\sphinxAtStartPar
{\hyperref[\detokenize{Sections/socialchange::doc}]{\sphinxcrossref{Politics}}}
\begin{itemize}
\item {} 
\sphinxAtStartPar
{\hyperref[\detokenize{Sections/protest::doc}]{\sphinxcrossref{Protest}}}
\begin{itemize}
\item {} 
\sphinxAtStartPar
{\hyperref[\detokenize{Volumes/01/01/Agitation::doc}]{\sphinxcrossref{Agitation (1910)}}}

\item {} 
\sphinxAtStartPar
{\hyperref[\detokenize{Volumes/05/05/proper_way::doc}]{\sphinxcrossref{The Proper Way (1913)}}}

\item {} 
\sphinxAtStartPar
{\hyperref[\detokenize{Volumes/16/03/close_ranks::doc}]{\sphinxcrossref{Close Ranks (1918)}}}

\item {} 
\sphinxAtStartPar
{\hyperref[\detokenize{Volumes/19/01/statement::doc}]{\sphinxcrossref{A Statement (1919)}}}

\item {} 
\sphinxAtStartPar
{\hyperref[\detokenize{Volumes/12/06/negro_party::doc}]{\sphinxcrossref{The Negro Party (1916)}}}

\item {} 
\sphinxAtStartPar
{\hyperref[\detokenize{Volumes/22/01/inter-racial_comity::doc}]{\sphinxcrossref{Inter\sphinxhyphen{}Racial Comity (1921)}}}

\item {} 
\sphinxAtStartPar
{\hyperref[\detokenize{Volumes/24/01/publicity::doc}]{\sphinxcrossref{Publicity (1922)}}}

\item {} 
\sphinxAtStartPar
{\hyperref[\detokenize{Volumes/41/05/violence::doc}]{\sphinxcrossref{Violence (1934)}}}

\item {} 
\sphinxAtStartPar
{\hyperref[\detokenize{Volumes/41/06/counsels_of_despair::doc}]{\sphinxcrossref{Counsels of Despair (1934)}}}

\end{itemize}

\item {} 
\sphinxAtStartPar
{\hyperref[\detokenize{Sections/voting::doc}]{\sphinxcrossref{Disenfranchisement}}}
\begin{itemize}
\item {} 
\sphinxAtStartPar
{\hyperref[\detokenize{Volumes/30/02/disenfranchisement::doc}]{\sphinxcrossref{Disenfranchisement (1925)}}}

\item {} 
\sphinxAtStartPar
{\hyperref[\detokenize{Volumes/36/05/negro_citizen::doc}]{\sphinxcrossref{The Negro Citizen (1929)}}}

\item {} 
\sphinxAtStartPar
{\hyperref[\detokenize{Volumes/36/11/negro_in_politics::doc}]{\sphinxcrossref{The Negro in Politics}}}

\end{itemize}

\item {} 
\sphinxAtStartPar
{\hyperref[\detokenize{Sections/electoral::doc}]{\sphinxcrossref{Voting and Parties}}}
\begin{itemize}
\item {} 
\sphinxAtStartPar
{\hyperref[\detokenize{Volumes/04/04/politics::doc}]{\sphinxcrossref{Politics (1912)}}}

\item {} 
\sphinxAtStartPar
{\hyperref[\detokenize{Volumes/15/01/oath_of_the_negro_voter::doc}]{\sphinxcrossref{The Oath of the Negro Voter (1917)}}}

\item {} 
\sphinxAtStartPar
{\hyperref[\detokenize{Volumes/21/02/unreal_campaign::doc}]{\sphinxcrossref{The Unreal Campaign (1920)}}}

\item {} 
\sphinxAtStartPar
{\hyperref[\detokenize{Volumes/22/01/drive::doc}]{\sphinxcrossref{The Drive (1921)}}}

\item {} 
\sphinxAtStartPar
{\hyperref[\detokenize{Volumes/34/04/chicago::doc}]{\sphinxcrossref{Chicago (1927)}}}

\item {} 
\sphinxAtStartPar
{\hyperref[\detokenize{Volumes/35/11/on_the_fence::doc}]{\sphinxcrossref{On the Fence (1928)}}}

\item {} 
\sphinxAtStartPar
{\hyperref[\detokenize{Volumes/35/11/third_party::doc}]{\sphinxcrossref{A Third Party (1928)}}}

\end{itemize}

\end{itemize}

\item {} 
\sphinxAtStartPar
{\hyperref[\detokenize{Sections/NAACP::doc}]{\sphinxcrossref{NAACP}}}
\begin{itemize}
\item {} 
\sphinxAtStartPar
{\hyperref[\detokenize{Volumes/01/02/NAACP::doc}]{\sphinxcrossref{N.A.A.C.P. (1910)}}}

\item {} 
\sphinxAtStartPar
{\hyperref[\detokenize{Volumes/01/01/TheCrisis::doc}]{\sphinxcrossref{The Crisis (1910)}}}

\item {} 
\sphinxAtStartPar
{\hyperref[\detokenize{Volumes/06/01/vigilance_committee::doc}]{\sphinxcrossref{The Vigilance Committee: A Call To Arms (1913)}}}

\item {} 
\sphinxAtStartPar
{\hyperref[\detokenize{Volumes/07/03/fightordie::doc}]{\sphinxcrossref{Join or Die (1914)}}}

\item {} 
\sphinxAtStartPar
{\hyperref[\detokenize{Volumes/30/01/new_crisis::doc}]{\sphinxcrossref{The New Crisis (1925)}}}

\item {} 
\sphinxAtStartPar
{\hyperref[\detokenize{Volumes/33/03/our_methods::doc}]{\sphinxcrossref{Our Methods (1927)}}}

\item {} 
\sphinxAtStartPar
{\hyperref[\detokenize{Volumes/40/01/toward_a_new_racial_philosophy::doc}]{\sphinxcrossref{Toward a New Racial Philosophy (1933)}}}

\end{itemize}

\item {} 
\sphinxAtStartPar
{\hyperref[\detokenize{Sections/lynching::doc}]{\sphinxcrossref{Lynching}}}
\begin{itemize}
\item {} 
\sphinxAtStartPar
{\hyperref[\detokenize{Volumes/02/04/lynching::doc}]{\sphinxcrossref{Lynching (1911)}}}

\item {} 
\sphinxAtStartPar
{\hyperref[\detokenize{Volumes/08/03/cause_of_lynching::doc}]{\sphinxcrossref{The Cause of Lynching (1914)}}}

\item {} 
\sphinxAtStartPar
{\hyperref[\detokenize{Volumes/12/06/cowardice::doc}]{\sphinxcrossref{Cowardice (1916)}}}

\item {} 
\sphinxAtStartPar
{\hyperref[\detokenize{Volumes/22/01/anti-lynching_legislation::doc}]{\sphinxcrossref{Anti\sphinxhyphen{}Lynching Legislation (1921)}}}

\item {} 
\sphinxAtStartPar
{\hyperref[\detokenize{Volumes/26/02/university_course_in_lynching::doc}]{\sphinxcrossref{A University Course in Lynching (1923)}}}

\item {} 
\sphinxAtStartPar
{\hyperref[\detokenize{Volumes/32/01/lynching::doc}]{\sphinxcrossref{Lynching (1926)}}}

\item {} 
\sphinxAtStartPar
{\hyperref[\detokenize{Volumes/34/01/aiken::doc}]{\sphinxcrossref{Aiken (1927)}}}

\item {} 
\sphinxAtStartPar
{\hyperref[\detokenize{Volumes/33/04/lynching::doc}]{\sphinxcrossref{Lynching (1927)}}}

\item {} 
\sphinxAtStartPar
{\hyperref[\detokenize{Volumes/38/04/causes_of_lynching::doc}]{\sphinxcrossref{Causes of Lynching (1931)}}}

\end{itemize}

\item {} 
\sphinxAtStartPar
{\hyperref[\detokenize{Sections/segregation::doc}]{\sphinxcrossref{Segregation}}}
\begin{itemize}
\item {} 
\sphinxAtStartPar
{\hyperref[\detokenize{Volumes/01/01/Segregation::doc}]{\sphinxcrossref{Segregation (1910)}}}

\item {} 
\sphinxAtStartPar
{\hyperref[\detokenize{Volumes/03/05/homes::doc}]{\sphinxcrossref{Homes (1912)}}}

\item {} 
\sphinxAtStartPar
{\hyperref[\detokenize{Volumes/05/04/blesseddiscrimination::doc}]{\sphinxcrossref{Blessed Discrimination (1913)}}}

\item {} 
\sphinxAtStartPar
{\hyperref[\detokenize{Volumes/13/06/perpetual_dilemma::doc}]{\sphinxcrossref{The Perpetual Dilemma (1917)}}}

\item {} 
\sphinxAtStartPar
{\hyperref[\detokenize{Volumes/17/03/jim_crow::doc}]{\sphinxcrossref{Jim Crow (1919)}}}

\item {} 
\sphinxAtStartPar
{\hyperref[\detokenize{Volumes/18/01/returning_soldiers::doc}]{\sphinxcrossref{Returning Soldiers (1919)}}}

\item {} 
\sphinxAtStartPar
{\hyperref[\detokenize{Volumes/26/02/on_being_crazy::doc}]{\sphinxcrossref{On Being Crazy (1923)}}}

\item {} 
\sphinxAtStartPar
{\hyperref[\detokenize{Volumes/31/01/challenge_of_detroit::doc}]{\sphinxcrossref{The Challenge of Detroit (1925)}}}

\item {} 
\sphinxAtStartPar
{\hyperref[\detokenize{Volumes/41/02/naacp_and_race_segregation::doc}]{\sphinxcrossref{The N.A.A.C.P. and Race Segregation (1934)}}}

\item {} 
\sphinxAtStartPar
{\hyperref[\detokenize{Volumes/41/04/segregation_in_the_north::doc}]{\sphinxcrossref{Segregation in the North (1934)}}}

\item {} 
\sphinxAtStartPar
{\hyperref[\detokenize{Volumes/41/05/segregation::doc}]{\sphinxcrossref{Segregation (1934)}}}

\item {} 
\sphinxAtStartPar
{\hyperref[\detokenize{Volumes/40/03/Color_caste_in_the_united_states::doc}]{\sphinxcrossref{Color Caste in the United States (1933)}}}

\end{itemize}

\item {} 
\sphinxAtStartPar
{\hyperref[\detokenize{Sections/racism::doc}]{\sphinxcrossref{Racism and Discrimination}}}
\begin{itemize}
\item {} 
\sphinxAtStartPar
{\hyperref[\detokenize{Volumes/01/03/social_equality::doc}]{\sphinxcrossref{“Social Equality” (1911)}}}

\item {} 
\sphinxAtStartPar
{\hyperref[\detokenize{Volumes/01/03/ashamed::doc}]{\sphinxcrossref{“Ashamed” (1911)}}}

\item {} 
\sphinxAtStartPar
{\hyperref[\detokenize{Volumes/03/04/light::doc}]{\sphinxcrossref{Light (1912)}}}

\item {} 
\sphinxAtStartPar
{\hyperref[\detokenize{Volumes/06/02/logic::doc}]{\sphinxcrossref{Logic (1913)}}}

\item {} 
\sphinxAtStartPar
{\hyperref[\detokenize{Volumes/10/01/clansman::doc}]{\sphinxcrossref{The Clansman (1915)}}}

\item {} 
\sphinxAtStartPar
{\hyperref[\detokenize{Volumes/12/05/conduct_not_color::doc}]{\sphinxcrossref{Conduct, Not Color (1916)}}}

\item {} 
\sphinxAtStartPar
{\hyperref[\detokenize{Volumes/12/06/migration::doc}]{\sphinxcrossref{Migration (1916)}}}

\item {} 
\sphinxAtStartPar
{\hyperref[\detokenize{Volumes/24/01/slavery::doc}]{\sphinxcrossref{Slavery (1921)}}}

\item {} 
\sphinxAtStartPar
{\hyperref[\detokenize{Volumes/31/04/newer_south::doc}]{\sphinxcrossref{The Newer South (1926)}}}

\item {} 
\sphinxAtStartPar
{\hyperref[\detokenize{Volumes/34/02/higher_friction::doc}]{\sphinxcrossref{The Higher Friction (1927)}}}

\item {} 
\sphinxAtStartPar
{\hyperref[\detokenize{Volumes/34/09/prejudice::doc}]{\sphinxcrossref{Prejudice (1927)}}}

\item {} 
\sphinxAtStartPar
{\hyperref[\detokenize{Volumes/40/07/protest::doc}]{\sphinxcrossref{A Protest (1933)}}}

\end{itemize}

\item {} 
\sphinxAtStartPar
{\hyperref[\detokenize{Sections/economics::doc}]{\sphinxcrossref{Economics}}}
\begin{itemize}
\item {} 
\sphinxAtStartPar
{\hyperref[\detokenize{Sections/labor::doc}]{\sphinxcrossref{Labor}}}
\begin{itemize}
\item {} 
\sphinxAtStartPar
{\hyperref[\detokenize{Volumes/03/06/servant_in_the_south::doc}]{\sphinxcrossref{The Servant in the South (1912)}}}

\item {} 
\sphinxAtStartPar
{\hyperref[\detokenize{Volumes/15/05/the_black_man_and_the_unions::doc}]{\sphinxcrossref{The Black Man and the Unions (1918)}}}

\item {} 
\sphinxAtStartPar
{\hyperref[\detokenize{Volumes/18/05/labor_omnia_vincit::doc}]{\sphinxcrossref{Labor Omnia Vincit (1919)}}}

\item {} 
\sphinxAtStartPar
{\hyperref[\detokenize{Volumes/31/02/black_man_and_labor::doc}]{\sphinxcrossref{The Black Man and Labor (1925)}}}

\item {} 
\sphinxAtStartPar
{\hyperref[\detokenize{Volumes/31/06/again_pullman_porters::doc}]{\sphinxcrossref{Again, Pullman Porters (1926)}}}

\end{itemize}

\item {} 
\sphinxAtStartPar
{\hyperref[\detokenize{Sections/socialism::doc}]{\sphinxcrossref{Socialism and Communism}}}
\begin{itemize}
\item {} 
\sphinxAtStartPar
{\hyperref[\detokenize{Volumes/19/04/cooperation::doc}]{\sphinxcrossref{Coöperation (1920)}}}

\item {} 
\sphinxAtStartPar
{\hyperref[\detokenize{Volumes/22/03/negro_and_radical_thought::doc}]{\sphinxcrossref{The Negro and Radical Thought (1921)}}}

\item {} 
\sphinxAtStartPar
{\hyperref[\detokenize{Volumes/22/04/class_struggle::doc}]{\sphinxcrossref{The Class Struggle (1921)}}}

\item {} 
\sphinxAtStartPar
{\hyperref[\detokenize{Volumes/22/06/socialism_and_the_negro::doc}]{\sphinxcrossref{Socialism and the Negro (1921)}}}

\item {} 
\sphinxAtStartPar
{\hyperref[\detokenize{Volumes/22/06/single_tax::doc}]{\sphinxcrossref{The Single Tax (1921)}}}

\item {} 
\sphinxAtStartPar
{\hyperref[\detokenize{Volumes/27/03/black_man_and_the_wounded_world::doc}]{\sphinxcrossref{The Black Man and the Wounded World (1923)}}}

\item {} 
\sphinxAtStartPar
{\hyperref[\detokenize{Volumes/33/01/russia_1926::doc}]{\sphinxcrossref{Russia, 1926 (1926)}}}

\item {} 
\sphinxAtStartPar
{\hyperref[\detokenize{Volumes/33/04/judging_russia::doc}]{\sphinxcrossref{Judging Russia (1927)}}}

\item {} 
\sphinxAtStartPar
{\hyperref[\detokenize{Volumes/35/11/dunbar_national_bank::doc}]{\sphinxcrossref{The Dunbar National Bank (1928)}}}

\item {} 
\sphinxAtStartPar
{\hyperref[\detokenize{Volumes/38/09/negro_and_communism::doc}]{\sphinxcrossref{The Negro and Communism (1931)}}}

\item {} 
\sphinxAtStartPar
{\hyperref[\detokenize{Volumes/40/07/our_class_struggle::doc}]{\sphinxcrossref{Our Class Struggle}}}

\end{itemize}

\end{itemize}

\item {} 
\sphinxAtStartPar
{\hyperref[\detokenize{Sections/womansuffrage::doc}]{\sphinxcrossref{Woman Suffrage}}}
\begin{itemize}
\item {} 
\sphinxAtStartPar
{\hyperref[\detokenize{Volumes/04/04/ohio::doc}]{\sphinxcrossref{Ohio (1912)}}}

\item {} 
\sphinxAtStartPar
{\hyperref[\detokenize{Volumes/09/06/womansuffrage::doc}]{\sphinxcrossref{Woman Suffrage (1915)}}}

\item {} 
\sphinxAtStartPar
{\hyperref[\detokenize{Volumes/09/03/agility::doc}]{\sphinxcrossref{Agility (1915)}}}

\item {} 
\sphinxAtStartPar
{\hyperref[\detokenize{Volumes/15/01/votes_for_women::doc}]{\sphinxcrossref{Votes for Women (1917)}}}

\end{itemize}

\item {} 
\sphinxAtStartPar
{\hyperref[\detokenize{Sections/race_riots::doc}]{\sphinxcrossref{Race Riots}}}
\begin{itemize}
\item {} 
\sphinxAtStartPar
{\hyperref[\detokenize{Volumes/18/06/shillady_and_texas::doc}]{\sphinxcrossref{Shillady and Texas (1919)}}}

\item {} 
\sphinxAtStartPar
{\hyperref[\detokenize{Volumes/34/06/mob_tactics::doc}]{\sphinxcrossref{Mob Tactics (1927)}}}

\item {} 
\sphinxAtStartPar
{\hyperref[\detokenize{Volumes/19/03/brothers_come_north::doc}]{\sphinxcrossref{Brothers, Come North (1920)}}}

\end{itemize}

\item {} 
\sphinxAtStartPar
{\hyperref[\detokenize{Sections/other::doc}]{\sphinxcrossref{Miscellaneous}}}
\begin{itemize}
\item {} 
\sphinxAtStartPar
{\hyperref[\detokenize{Sections/racescience::doc}]{\sphinxcrossref{Science of race}}}
\begin{itemize}
\item {} 
\sphinxAtStartPar
{\hyperref[\detokenize{Volumes/02/04/races::doc}]{\sphinxcrossref{Races (1911)}}}

\end{itemize}

\item {} 
\sphinxAtStartPar
{\hyperref[\detokenize{Sections/criminal_justice::doc}]{\sphinxcrossref{Criminal Justice}}}
\begin{itemize}
\item {} 
\sphinxAtStartPar
{\hyperref[\detokenize{Volumes/15/03/thirteen::doc}]{\sphinxcrossref{Thirteen (1919)}}}

\item {} 
\sphinxAtStartPar
{\hyperref[\detokenize{Volumes/19/04/crime::doc}]{\sphinxcrossref{Crime (1920)}}}

\item {} 
\sphinxAtStartPar
{\hyperref[\detokenize{Volumes/32/01/crime::doc}]{\sphinxcrossref{Crime (1926)}}}

\item {} 
\sphinxAtStartPar
{\hyperref[\detokenize{Volumes/39/04/courts_and_jails::doc}]{\sphinxcrossref{Courts and Jails (1932)}}}

\end{itemize}

\item {} 
\sphinxAtStartPar
{\hyperref[\detokenize{Sections/intermarriage::doc}]{\sphinxcrossref{Intermarriage}}}
\begin{itemize}
\item {} 
\sphinxAtStartPar
{\hyperref[\detokenize{Volumes/05/04/intermarriage::doc}]{\sphinxcrossref{Intermarriage (1913)}}}

\item {} 
\sphinxAtStartPar
{\hyperref[\detokenize{Volumes/19/03/sex_equality::doc}]{\sphinxcrossref{Sex Equality (1920)}}}

\item {} 
\sphinxAtStartPar
{\hyperref[\detokenize{Volumes/31/05/correspondence::doc}]{\sphinxcrossref{Correspondence (1926)}}}

\end{itemize}

\item {} 
\sphinxAtStartPar
{\hyperref[\detokenize{Sections/reparations::doc}]{\sphinxcrossref{Reparations}}}
\begin{itemize}
\item {} 
\sphinxAtStartPar
{\hyperref[\detokenize{Volumes/24/02/whitecharity::doc}]{\sphinxcrossref{White Charity (1922)}}}

\end{itemize}

\item {} 
\sphinxAtStartPar
{\hyperref[\detokenize{Sections/colonialization::doc}]{\sphinxcrossref{Colonialization}}}
\begin{itemize}
\item {} 
\sphinxAtStartPar
{\hyperref[\detokenize{Volumes/09/01/worldwar::doc}]{\sphinxcrossref{World War and the Color Line (1914)}}}

\item {} 
\sphinxAtStartPar
{\hyperref[\detokenize{Volumes/17/04/reconstruction_and_africa::doc}]{\sphinxcrossref{Reconstruction and Africa (1919)}}}

\item {} 
\sphinxAtStartPar
{\hyperref[\detokenize{Volumes/19/03/race_pride::doc}]{\sphinxcrossref{Race Pride (1920)}}}

\item {} 
\sphinxAtStartPar
{\hyperref[\detokenize{Volumes/31/02/firing_line::doc}]{\sphinxcrossref{The Firing Line (1925)}}}

\item {} 
\sphinxAtStartPar
{\hyperref[\detokenize{Volumes/40/01/listen_japan_and_china::doc}]{\sphinxcrossref{Listen, Japan and China (1933)}}}

\end{itemize}

\item {} 
\sphinxAtStartPar
{\hyperref[\detokenize{Sections/misc::doc}]{\sphinxcrossref{Miscellaneous}}}
\begin{itemize}
\item {} 
\sphinxAtStartPar
{\hyperref[\detokenize{Volumes/03/05/lee::doc}]{\sphinxcrossref{Lee (1912)}}}

\item {} 
\sphinxAtStartPar
{\hyperref[\detokenize{Volumes/09/02/negro::doc}]{\sphinxcrossref{Negro (1914)}}}

\item {} 
\sphinxAtStartPar
{\hyperref[\detokenize{Volumes/06/07/national_emancipation_exposition::doc}]{\sphinxcrossref{The National Emancipation Exposition (1913)}}}

\item {} 
\sphinxAtStartPar
{\hyperref[\detokenize{Volumes/11/02/star_of_ethiopia::doc}]{\sphinxcrossref{The Star of Ethiopia (1915)}}}

\item {} 
\sphinxAtStartPar
{\hyperref[\detokenize{Volumes/18/03/reconstruction::doc}]{\sphinxcrossref{Reconstruction (1919)}}}

\item {} 
\sphinxAtStartPar
{\hyperref[\detokenize{Volumes/19/01/social_equity::doc}]{\sphinxcrossref{Social Equity (1919)}}}

\item {} 
\sphinxAtStartPar
{\hyperref[\detokenize{Volumes/34/02/farmers::doc}]{\sphinxcrossref{Farmers (1927)}}}

\item {} 
\sphinxAtStartPar
{\hyperref[\detokenize{Volumes/39/01/john_brown::doc}]{\sphinxcrossref{John Brown (1932)}}}

\end{itemize}

\end{itemize}

\item {} 
\sphinxAtStartPar
{\hyperref[\detokenize{random::doc}]{\sphinxcrossref{Random Encounters}}}

\end{itemize}

\sphinxAtStartPar
\sphinxincludegraphics{{nypl.digitalcollections.510d47dc-8fb3-a3d9-e040-e00a18064a99.001.w}.jpg}
\sphinxstyleemphasis{W. E. B. Dubois in the office of The Crisis. Source: \sphinxhref{https://digitalcollections.nypl.org/items/510d47dc-8fb3-a3d9-e040-e00a18064a99}{Schomburg Center for Research in Black Culture}}


\chapter{Introduction}
\label{\detokenize{introduction:introduction}}\label{\detokenize{introduction::doc}}
\sphinxAtStartPar
This volume was compiled by Neal Caren.

\sphinxAtStartPar
Resources used:


\chapter{Politics}
\label{\detokenize{Sections/socialchange:politics}}\label{\detokenize{Sections/socialchange::doc}}
\sphinxAtStartPar
Editorials on electoral politics, protest and movement strategy.
\begin{itemize}
\item {} 
\sphinxAtStartPar
{\hyperref[\detokenize{Sections/protest::doc}]{\sphinxcrossref{Protest}}}
\begin{itemize}
\item {} 
\sphinxAtStartPar
{\hyperref[\detokenize{Volumes/01/01/Agitation::doc}]{\sphinxcrossref{Agitation (1910)}}}

\item {} 
\sphinxAtStartPar
{\hyperref[\detokenize{Volumes/05/05/proper_way::doc}]{\sphinxcrossref{The Proper Way (1913)}}}

\item {} 
\sphinxAtStartPar
{\hyperref[\detokenize{Volumes/16/03/close_ranks::doc}]{\sphinxcrossref{Close Ranks (1918)}}}

\item {} 
\sphinxAtStartPar
{\hyperref[\detokenize{Volumes/19/01/statement::doc}]{\sphinxcrossref{A Statement (1919)}}}

\item {} 
\sphinxAtStartPar
{\hyperref[\detokenize{Volumes/12/06/negro_party::doc}]{\sphinxcrossref{The Negro Party (1916)}}}

\item {} 
\sphinxAtStartPar
{\hyperref[\detokenize{Volumes/22/01/inter-racial_comity::doc}]{\sphinxcrossref{Inter\sphinxhyphen{}Racial Comity (1921)}}}

\item {} 
\sphinxAtStartPar
{\hyperref[\detokenize{Volumes/24/01/publicity::doc}]{\sphinxcrossref{Publicity (1922)}}}

\item {} 
\sphinxAtStartPar
{\hyperref[\detokenize{Volumes/41/05/violence::doc}]{\sphinxcrossref{Violence (1934)}}}

\item {} 
\sphinxAtStartPar
{\hyperref[\detokenize{Volumes/41/06/counsels_of_despair::doc}]{\sphinxcrossref{Counsels of Despair (1934)}}}

\end{itemize}

\item {} 
\sphinxAtStartPar
{\hyperref[\detokenize{Sections/voting::doc}]{\sphinxcrossref{Disenfranchisement}}}
\begin{itemize}
\item {} 
\sphinxAtStartPar
{\hyperref[\detokenize{Volumes/30/02/disenfranchisement::doc}]{\sphinxcrossref{Disenfranchisement (1925)}}}

\item {} 
\sphinxAtStartPar
{\hyperref[\detokenize{Volumes/36/05/negro_citizen::doc}]{\sphinxcrossref{The Negro Citizen (1929)}}}

\item {} 
\sphinxAtStartPar
{\hyperref[\detokenize{Volumes/36/11/negro_in_politics::doc}]{\sphinxcrossref{The Negro in Politics}}}

\end{itemize}

\item {} 
\sphinxAtStartPar
{\hyperref[\detokenize{Sections/electoral::doc}]{\sphinxcrossref{Voting and Parties}}}
\begin{itemize}
\item {} 
\sphinxAtStartPar
{\hyperref[\detokenize{Volumes/04/04/politics::doc}]{\sphinxcrossref{Politics (1912)}}}

\item {} 
\sphinxAtStartPar
{\hyperref[\detokenize{Volumes/15/01/oath_of_the_negro_voter::doc}]{\sphinxcrossref{The Oath of the Negro Voter (1917)}}}

\item {} 
\sphinxAtStartPar
{\hyperref[\detokenize{Volumes/21/02/unreal_campaign::doc}]{\sphinxcrossref{The Unreal Campaign (1920)}}}

\item {} 
\sphinxAtStartPar
{\hyperref[\detokenize{Volumes/22/01/drive::doc}]{\sphinxcrossref{The Drive (1921)}}}

\item {} 
\sphinxAtStartPar
{\hyperref[\detokenize{Volumes/34/04/chicago::doc}]{\sphinxcrossref{Chicago (1927)}}}

\item {} 
\sphinxAtStartPar
{\hyperref[\detokenize{Volumes/35/11/on_the_fence::doc}]{\sphinxcrossref{On the Fence (1928)}}}

\item {} 
\sphinxAtStartPar
{\hyperref[\detokenize{Volumes/35/11/third_party::doc}]{\sphinxcrossref{A Third Party (1928)}}}

\end{itemize}

\end{itemize}


\section{Protest}
\label{\detokenize{Sections/protest:protest}}\label{\detokenize{Sections/protest::doc}}
\sphinxAtStartPar
Editorials on protest and social movement strategy.
\begin{itemize}
\item {} 
\sphinxAtStartPar
{\hyperref[\detokenize{Volumes/01/01/Agitation::doc}]{\sphinxcrossref{Agitation (1910)}}}

\item {} 
\sphinxAtStartPar
{\hyperref[\detokenize{Volumes/05/05/proper_way::doc}]{\sphinxcrossref{The Proper Way (1913)}}}

\item {} 
\sphinxAtStartPar
{\hyperref[\detokenize{Volumes/16/03/close_ranks::doc}]{\sphinxcrossref{Close Ranks (1918)}}}

\item {} 
\sphinxAtStartPar
{\hyperref[\detokenize{Volumes/19/01/statement::doc}]{\sphinxcrossref{A Statement (1919)}}}

\item {} 
\sphinxAtStartPar
{\hyperref[\detokenize{Volumes/12/06/negro_party::doc}]{\sphinxcrossref{The Negro Party (1916)}}}

\item {} 
\sphinxAtStartPar
{\hyperref[\detokenize{Volumes/22/01/inter-racial_comity::doc}]{\sphinxcrossref{Inter\sphinxhyphen{}Racial Comity (1921)}}}

\item {} 
\sphinxAtStartPar
{\hyperref[\detokenize{Volumes/24/01/publicity::doc}]{\sphinxcrossref{Publicity (1922)}}}

\item {} 
\sphinxAtStartPar
{\hyperref[\detokenize{Volumes/41/05/violence::doc}]{\sphinxcrossref{Violence (1934)}}}

\item {} 
\sphinxAtStartPar
{\hyperref[\detokenize{Volumes/41/06/counsels_of_despair::doc}]{\sphinxcrossref{Counsels of Despair (1934)}}}

\end{itemize}


\subsection{Agitation (1910)}
\label{\detokenize{Volumes/01/01/Agitation:agitation-1910}}\label{\detokenize{Volumes/01/01/Agitation::doc}}
\sphinxAtStartPar
Some good friends of the cause we represent fear agitation. They say: “Do not agitate—do not make a noise; work.” They add, “Agitation is destructive or at best negative—what is wanted is positive constructive work.”

\sphinxAtStartPar
Such honest critics mistake the function of agitation. A toothache is agitation. Is a toothache a good thing? No . Is it therefore useless? No. It is supremely useful, for it tells the body of decay, dyspepsia and death. Without it the body would suffer unknowingly. It would think: All is well, when lo! danger lurks.

\sphinxAtStartPar
The same is true of the Social Body. Agitation is a necessary evil to tell of the ills of the Suffering. Without it many a nation has been lulled to false security and preened itself with virtues it did not possess.

\sphinxAtStartPar
The function of this Association is to tell this nation the crying evil of race prejudice. It is a hard duty but a necessary one—a divine one. It is Pain; Pain is not good but Pain is necessary. Pain does not aggravate disease — Disease causes Pain. Agitation does not mean Aggravation—Aggravation calls for Agi tation in order that Remedy may be found.


\bigskip\hrule\bigskip


\sphinxAtStartPar
\sphinxstyleemphasis{Citation:} Dubois, W.E.B. 1910. “Agitation.”  1910. \sphinxstyleemphasis{The Crisis}. 1(1):11.


\subsection{The Proper Way (1913)}
\label{\detokenize{Volumes/05/05/proper_way:the-proper-way-1913}}\label{\detokenize{Volumes/05/05/proper_way::doc}}
\sphinxAtStartPar
The editor of the Cleveland \sphinxstyleemphasis{Gazette} names three main points of attack for any national association which aims to help colored people:
\begin{enumerate}
\sphinxsetlistlabels{\arabic}{enumi}{enumii}{}{.}%
\item {} 
\sphinxAtStartPar
Disfranchisement.

\item {} 
\sphinxAtStartPar
Interstate “Jim Crow”

\item {} 
\sphinxAtStartPar
Lynchings.

\end{enumerate}

\sphinxAtStartPar
This is perfectly true, and the National Association for the Advancement of Colored People recognizes this and is straining every nerve to attack these evils. As to disfranchisement we are making every effort to get the proper ease before the Supreme Court. We have already helped by briefs and contributions the Oklahoma ease, and when it comes before the court we have offered the services of two of the most eminent lawyers in the United States. We are represented on the counsel of the Mississippi. ‘‘Jim Crow’’ case; the briefs are being examined by our lawyers, and we are making every effort to get the question before the court in the right way.

\sphinxAtStartPar
But the \sphinxstyleemphasis{Gazette} should know that cases before the Supreme Court are delicate matters. It does not do to rush into court with any haphazard ease. If anyone has a case or knows of a case which will bring out the proper points we should be glad to have it. Theoretically, it would seem very easy to settle such matters. Practically, it is very hard, but we propose to keep at it.

\begin{sphinxShadowBox}
\sphinxstylesidebartitle{}

\sphinxAtStartPar
The “Coatesville matter” is the 1911 \sphinxhref{https://en.wikipedia.org/wiki/Lynching\_of\_Zachariah\_Walker}{lynching of Zachariah Walker} in Coatesville, Pennsylvania.
\end{sphinxShadowBox}

\sphinxAtStartPar
As to lynching, there are four things to do: Publish the facts, appeal to the authorities, agitate publicity and employ detectives. Every one of these things we have done. \sphinxstylestrong{The Crisis} publishes the facts monthly over the protest of sensitive readers. We have sent telegrams and appeals to governors, sheriffs and the President; we have held mass meetings: we have sent distinguished writers and: investigators; we have secured publicity in prominent magazines, and we spent thousands of dollars in putting Burns’ detectives on the Coatesville matter.
What else can we do? We want suggestions. Meantime we shall keep up our present agitation.

\sphinxAtStartPar
Some folk seem to imagine that the walls of caste and prejudice in America will fall at a blast of the trumpet, if the blast be loud enough. Consequently, when an association like the National Association for the Advancement of Colored People does something, they say querulously: ‘‘But nothing has happened.’’ They ought to say: Nothing has yet happened, for that is true and that is expected. If in fifty or a hundred years \sphinxstylestrong{The Crisis} can point to a distinct lessening of disfranchisement. and an undoubted reduction of lynching, and more decent traveling accommodations, this will be a great, an enormous accomplishment. Would God all this could be done to\sphinxhyphen{}morrow, but this is not humanly possible.

\sphinxAtStartPar
What is possible to\sphinxhyphen{}day and tomorrow and every day is to keep up necessary agitation, make unfaltering protest, fill the courts and legislatures and executive chambers, and keep ever lastingly at the work of protest in season and out of season. The weak and silly part of the program of those who deprecate complaint and agitation is that a moment’s let up, a moment’s acquiescence, means a chance for the wolves of prejudice to get at our necks. It is not that we have too many organizations; it is that we have too few effective workers in the great cause of Negro emancipation in America. Let us from this movement join in a frontal attack on disfranchiseemnt, “Jim Crow’’ ears and lynching. We shall not win today or to\sphinxhyphen{}morrow, but some day we shall win if we faint not.


\bigskip\hrule\bigskip


\sphinxAtStartPar
\sphinxstyleemphasis{Citation:} Du Bois, W.E.B. 1913. “The Proper Way” Editorial. \sphinxstyleemphasis{The Crisis}. 5(5): 238\sphinxhyphen{}239.


\subsection{Close Ranks (1918)}
\label{\detokenize{Volumes/16/03/close_ranks:close-ranks-1918}}\label{\detokenize{Volumes/16/03/close_ranks::doc}}
\begin{sphinxShadowBox}
\sphinxstylesidebartitle{}

\sphinxAtStartPar
This is one the most controversial and well\sphinxhyphen{}known editorials produced by Du Bois during his time at the Crisis. For example, the \sphinxstyleemphasis{Richmond Planet} ran a response headlined, “DuBois, One\sphinxhyphen{}Time Radical Leader, Deserts and Betrays Cause of His Race.”
\end{sphinxShadowBox}

\sphinxAtStartPar
This is the crisis of the world.

\sphinxAtStartPar
For all the long years to come men will point to the at year 1918 as the great Day of Decision, the day when the world decided whether it would submit to military despotism and an endless armed peace—if peace it could be called—or whether they would put down the menace of German militarism and inaugurate the United States of the World.

\sphinxAtStartPar
We of the colored race have no ordinary interest in the outcome. That which the German power represents today spells death to the aspirations of Negroes and all darker races for equality, freedom and democracy. Let us not hesitate. Let us, while this war lasts, forget our special grievances and close our ranks shoulder to shoulder with our own white fellow citizens and the allied nations that are fighting for democracy. We make no ordinary sacrifice, but we make it gladly and willingly with our eyes lifted to the hills.


\bigskip\hrule\bigskip


\sphinxAtStartPar
\sphinxstyleemphasis{Citation:} Du Bois, W.E.B. 1918. “Close Ranks.” \sphinxstyleemphasis{The Crisis}. 16(3): 111.


\subsection{A Statement (1919)}
\label{\detokenize{Volumes/19/01/statement:a-statement-1919}}\label{\detokenize{Volumes/19/01/statement::doc}}
\sphinxAtStartPar
At no previous period in the history of the Negro in America has he been confronted with a more critical situation than today. The forces of prejudice against which we are fighting seem determined to keep twelve millions of Americans in that bondage of prejudice because of race, while those twelve millions are determined as never before to achieve the status of citizens—full and unlimited by caste or color. When two forces of such magnitude meet, a critical situation is inevitable and such a one is upon us today.

\sphinxAtStartPar
There are those who are attempting to becloud the issue by declaring that the present unrest and discontent is due to influences other than natural resentment against wrong. Such persons are making such absurd statements either through ignorance of the facts or because they know the facts and are attempting to shift the responsibility for the half\sphinxhyphen{}century of lynching, disfranchisement. peonage, “Jim\sphinxhyphen{}Crowism” and injustice of every sort practiced on the Negro. We do not countenance violence. Our fight is against violence. We are fighting—as we always have fought —for the reign of law over the reign of the mob. No sane man or woman
can for a minute advise any group to use the torch or the gun to right the wrong of violence. Only in self defense can such a course ever be considered justifiable.

\sphinxAtStartPar
But we must fight and we are going to fight in every legitimate and lawful way until our problem is entirely settled. To do this we need greater organization, energy, funds and courage than ever before. The call is to you to rally to the support of the National Association for the Advancement of Colored People and help in the fight to make America safe for the colored man.


\bigskip\hrule\bigskip


\sphinxAtStartPar
\sphinxstyleemphasis{Citation:} Du Bois, W.E.B. 1919. “A Statement.” \sphinxstyleemphasis{The Crisis}. 19(1): 335.


\subsection{The Negro Party (1916)}
\label{\detokenize{Volumes/12/06/negro_party:the-negro-party-1916}}\label{\detokenize{Volumes/12/06/negro_party::doc}}
\sphinxAtStartPar
There is for the future one and only one effective political move for colored voters. We have long foreseen it, have sought to avoid it. It is a move of segregation, it “hyphenates” us, it separates us from our fellow Americans; but self\sphinxhyphen{}defense knows no nice hesitations. The American Negro must either vote as a unit or continue to be politically emasculated as at present.

\begin{sphinxShadowBox}
\sphinxstylesidebartitle{}

\sphinxAtStartPar
\sphinxhref{https://en.wikipedia.org/wiki/Robert\_Reed\_Church}{Robert Reed Church}
\end{sphinxShadowBox}

\sphinxAtStartPar
Miss Inez Milholland, in a recent address, outlined with singular clearness and force a Negro Party on the lines of the recently formed Woman’s Party. Mr. R. R. Church, Jr., of Tennessee, and certain leading colored men in New Jersey, Ohio and elsewhere have unconsciously and effectively followed her advice.

\sphinxAtStartPar
The situation is this: At present the Democratic party can maintain its ascendency only by the help of the Solid South. The Solid South is built on the hate and fear of Negroes; consequently it can never, as a party, effectively bid for the Negro vote. The Republican party is the party of wealth and big business and, as such, is the natural enemy of the humble working people who compose the mass of Negroes. Between these two great parties, as parties there is little to choose.

\sphinxAtStartPar
On the other hand, parties are represented by individual candidates. Negroes can have choice in the naming of these candidates and they can vote for or against them. Their only effective method in the future is to organize in every congressional district as a Negro Party to endorse those candidates, Republican, Democratic, Socialist, or What\sphinxhyphen{}not, whose promises and past performances give greatest hope for the remedying of the wrongs done the Negro race. If no candidate fills this bill they should nominate a candidate of their own and give that candidate their solid vote. This policy effectively and consistently carried out throughout the United States, North and South, by colored voters who refuse the bribe of petty office and money, would make the Negro vote one of the most powerful and effective of the group votes in the United States.

\sphinxAtStartPar
This is the program which we must follow. We may hesitate and argue about it, but if we are a sensible, reasonable people we will come to it and the quicker the better.


\bigskip\hrule\bigskip


\sphinxAtStartPar
\sphinxstyleemphasis{Citation:} “The Negro Party.” Editorial. 1916. \sphinxstyleemphasis{The Crisis}. 12(6): 267\sphinxhyphen{}268.


\subsection{Inter\sphinxhyphen{}Racial Comity (1921)}
\label{\detokenize{Volumes/22/01/inter-racial_comity:inter-racial-comity-1921}}\label{\detokenize{Volumes/22/01/inter-racial_comity::doc}}
\sphinxAtStartPar
There are persons who assume that the N.A.A.C.P. and particularly the CRISIS are opposed to the Inter\sphinxhyphen{}racial Committees in the South and to any efforts of white Southerners to settle the problems of race. This is a singu­lar misapprehension. On the contrary, we count it as one of the great results of the N.A.A.C.P., that its persistent fight in the last ten years has aroused and even compelled the South to attempt its own internal reform Can we not remember the day, but 20 years since, when conferences on the Negro were confined to white men because white Southerners would not sit with Negroes? That day is gone and gone forever, and the N.A.A.C.P. prepared its passports.

\sphinxAtStartPar
The Inter\sphinxhyphen{}racial movement sprang from the fight we have made. If it accomplishes anything, it will be because of our continued and persistent fighting. If it fails, it will show the need of redoubled effort on our part. If lulled by false hope and vague promises, we cease our vigilant effort, the Inter\sphinxhyphen{}racial movement would drop dead before the cry: “The Negroes are satisfied; why stir up trouble’.”

\sphinxAtStartPar
This has been the history of all such movements in the past. If the present movement succeeds (and God grant it may) it will be because the N.A.A.C.P. neither slumbers nor sleeps but keeps to its God\sphinxhyphen{}appointed task of making every black slave in the United States dissatisfied with his slavery, and every white slave\sphinxhyphen{}driver conscious of his guilt.

\sphinxAtStartPar
Meantime, may we not advise our Inter\sphinxhyphen{}racial friends,—do not fill your committees with “pussy footers” like Robert Moton or “white\sphinxhyphen{}folks’ N{[}*****{]}” like Isaac Fisher. Get more real black men who dare to look you in the eye and speak the truth and who refuse to fawn and lie. An ounce of truth outweighs a ton of impudence. Do not seek to mislead or lull the Negro with ancient platitudes and generalities. Let your “black mam­ my” sleep and show your “best friendship” by deeds and not words. Do something. Do not dodge and duck. Face the fundamental problems: the Vote, the “Jim\sphinxhyphen{}Crow” car, Peonage and Mob\sphinxhyphen{}law.


\bigskip\hrule\bigskip


\sphinxAtStartPar
\sphinxstyleemphasis{Citation:} “Inter\sphinxhyphen{}Racial Comity” Editorial. 1922. \sphinxstyleemphasis{The Crisis}. 22(1): 6\sphinxhyphen{}7.


\subsection{Publicity (1922)}
\label{\detokenize{Volumes/24/01/publicity:publicity-1922}}\label{\detokenize{Volumes/24/01/publicity::doc}}
\sphinxAtStartPar
We learned during the Great War what Publicity could do. We saw its good effects in bringing the truth before the people; we saw its bad effects in making millions believe lies. We are thinking of these bad effects so persistently since the war that Propaganda is in bad odor. But let us remember that in pitiless Publicity we have perhaps the greatest militant organ of social reform at our hands.

\sphinxAtStartPar
In our own problem, the N.A.A.C.P. at the very beginning looked upon \sphinxstylestrong{The Crisis} as a first and absolutely necessary step. Until the best black and white people realized the facts concerning the Negro problem, there was no use discussing remedies. It is as true today as it was then.

\sphinxAtStartPar
But further than that, if we want the economic conditions upon which modern life is based to be changed and changed for the better, we need first of all Publicity. The mass of men do not know the facts and there is not today any adequate effort to make all these facts known to the public. Not only that, but law and custom conspire to conceal the truth.

\sphinxAtStartPar
What is the \sphinxstyleemphasis{first} knowledge which any reformer should have who wishes to improve or rebuild modern industry? It is the facts concerning \sphinxstyleemphasis{Income}. The income of every human being, far from being a closely guarded secret, should be the most easily ascertainable economic fact. \sphinxstyleemphasis{Secondly}, the basis of that income should be known. It should be a matter of public knowledge by what work each individual gains his income and the character and extent of this work everybody should know or be able to find out.

\sphinxAtStartPar
If the institution of private property is to persist and if it ought to persist, the fundamental fact concerning it should be easily ascertainable; and that is, its exact and precise ownership and whence that ownership came; and if the property is alienated, to whom the ownership is transferred.

\sphinxAtStartPar
If individuals must be called upon to support the government, as they certainly must, it should be a matter of public information as to how much each individual contributes toward the public support.

\sphinxAtStartPar
These are all simple fundamental facts. Progress, to be sure, has been made in the last few years in making these facts known. It is not too much to say that economic reform has succeeded in so far and only in so far as it was based upon the revelation of such facts. There was a time when a man’s income was considered an absolutely private matter. Today it is at least partially public through the working of the income tax. Tomorrow it will be absolutely public. Today it is only with great difficulty that we can surmise the ownership of anonymous corporations. Tomorrow we will allow no corporation to exist whose ownership and control is not always a matter of accessible public record. Today a man’s occupation is considered his own business. Tomorrow it will be the business and the prime business of each one of his neighbors.


\bigskip\hrule\bigskip


\sphinxAtStartPar
\sphinxstyleemphasis{Citation:} “Publicity” Editorial. 1922. \sphinxstyleemphasis{The Crisis}. 24(1): 9\sphinxhyphen{}10.


\subsection{Violence (1934)}
\label{\detokenize{Volumes/41/05/violence:violence-1934}}\label{\detokenize{Volumes/41/05/violence::doc}}
\sphinxAtStartPar
A certain group of young, American Negroes, inspired by white Radicals, are distinctly looking toward violence as the only method of settling the Negro problems.

\sphinxAtStartPar
When I was speaking in Chicago, recently, a woman of mature age, phrased the argument somewhat heatedly. She said: “Why can’t we fight? What is the difference between our situation and the situation of the English colonies before the Revolution? What is the difference between our plight and that of the slaves in Haiti? Or of the peasants in France before the Revolution?”

\sphinxAtStartPar
Now the answer to this is almost too obvious to require stating. In 1776, England and her armies were something like six months distant from the Thirteen Colonies. Even if England had had the men to spare for fighting overseas, and she hadn’t, it would have required the transporting of large numbers of men over long periods of time, and at impossible cost, to have beaten the Colonies into submission. They had the forests of the wild West for refuge, and as long as they fought a defensive war, they were invincible.

\sphinxAtStartPar
At the time of the Haitian Revolution, there were forty thousand white people on the island, surrounded four hundred and fifty\sphinxhyphen{}two thousand blacks, not to mention the twenty\sphinxhyphen{}eight thousand mulattoes. In other words, the whites were outnumbered ten to one. To be sure, there were the armies and navies of. France, England and Spain, but there again loomed the difficulty of transportation, the difficulty of co\sphinxhyphen{}operation, and the World Wars which were keeping them all busy.

\sphinxAtStartPar
On the other hand, here in the United States are twelve million Negroes, totally surrounded by over one hundred and ten million whites. Under such circumstances, to talk about force, is little less than idiotic.

\sphinxAtStartPar
But, someone added, could we not induce a large enough number of white people to sympathize with us, so that combining with them, we could gain their rights? The answer to this is simple. We could not. If we could, there would be no need to appeal to force. If the justice of our cause appealed to a majority of white Americans, then our cause would be won by peaceful votes, and pressure of overwhelming public opinion. To such public opinion, we have been appealing for several centuries, and with redoubled and systematic effort, during the last twenty\sphinxhyphen{}five years. Our appeal has had some results, but not great results, nor results that give us the slightest reason for thinking that any considerable minority of American white people sympathize with our condition.

\sphinxAtStartPar
But the situation is worse than this. Just as soon as Negroes resort to violence or thoughts of violence, they will solidify white opposition. They will give every enemy of the Negro race a chance for using world\sphinxhyphen{}wide propaganda to prove that we are enemies of the United States, and of European civilization. With the same tactics that Hitler is using in Germany, they would seek to annihilate, and spiritually, even physically, reinslave the black folk of America. Of this, there is no reasonable doubt, and it is, therefore, our clear policy not to appeal to force until clearly and evidently, there is no other way.


\bigskip\hrule\bigskip


\sphinxAtStartPar
\sphinxstyleemphasis{Citation:} “Violence”.” 1934. Editorial.  41(5):147\sphinxhyphen{}148.


\subsection{Counsels of Despair (1934)}
\label{\detokenize{Volumes/41/06/counsels_of_despair:counsels-of-despair-1934}}\label{\detokenize{Volumes/41/06/counsels_of_despair::doc}}
\sphinxAtStartPar
Many persons have interpreted my reassertion of our current attitude toward segregation as a counsel of despair. We can’t win, therefore, give up and accept the inevitable. Never, and nonsense. Our business in this world is to fight and fight again, and never to yield. But after all, one must fight with his brains, if he has any. He gathers strength to fight. He gathers knowledge, and he raises children who are proud to fight and who know what they are fighting about. And above all, they learn that what they are fighting for is the opportunity and the chance to know and associate with black folk. They are not fighting to escape themselves. They are fighting to say to the world: the opportunity of knowing Negroes is worth so much to us and is so appreciated, that we want you to know them too.

\sphinxAtStartPar
Negroes are not extraordinary human beings. They are just like other human beings, with all their foibles and ignorance and mistakes. But they are human beings, and human nature is always worth knowing, and withal, splendid in its manifestations. Therefore, we are fighting to keep open the avenues of human contact; but in the meantime, we are taking every advantage of what opportunities of contact are already open. to us, and among those opportunities which are open, and which are splendid and inspiring, is the opportunity of Negroes to work together in the twentieth century for the uplift and development of the Negro race. It is no counsel of despair to emphasize and hail the opportunity for such work.


\subsubsection{The Anti\sphinxhyphen{}Segregation Campaign}
\label{\detokenize{Volumes/41/06/counsels_of_despair:the-anti-segregation-campaign}}
\sphinxAtStartPar
The assumptions of the anti\sphinxhyphen{}segregation campaign have been all wrong. This is not our fault, but it is our misfortune. When I went to Atlanta University to teach in 1897, and to study the Negro problem, I said, confidently, that the basic problem is our racial ignorance and lack of culture. That once Negroes know civilization, and whites know Negroes, then the problem is solved. This proposition is still true, but the solution is much further away that my youth dreamed. Negroes are still ignorant, but the disconcerting thing is that white people on the whole are just as much opposed to Negroes of education and culture, as to any other kind, and perhaps more so. Not all whites, to be sure, but the overwhelming majority.

\sphinxAtStartPar
Our main method, then, falls flat. We stop training ability. We lose our manners. We swallow our pride, and beg for things. We agitate and get angry. And with all that, we face the blank fact: Negroes are not wanted; neither as scholars nor as business men; neither as clerks nor as artisans; neither as artists nor as writers. What can we do about it? We cannot use force. We cannot enforce law, even if we get it on the statute books. So long as overwhelming public opinion sanctions and justifies and defends color segregation, we are helpless, and without remedy. We are segregated. We are cast back upon ourselves, to an Island Within; “To your tents, Oh Israel!”

\sphinxAtStartPar
Surely then, in this period of frustration and disappointment, we must turn from negation to affirmation, from the ever\sphinxhyphen{}lasting “No” to the ever\sphinxhyphen{}lasting “Yes”. Instead of sitting, sapped of all initiative and independence; instead of drowning our originality in imitation of mediocre white folks; instead of being afraid of ourselves and cultivating the art of skulking to escape the Color Line; we have got to renounce a program that always involves humiliating self\sphinxhyphen{}stultifying scrambling to crawl somewhere where we are not wanted; where we crouch panting like a whipped dog. We have got to stop this and learn that on such a program they cannot build manhood. No, by God, stand erect in a mud\sphinxhyphen{}puddle and tell the white world to go to hell, rather than lick boots in a parlor.

\sphinxAtStartPar
Affirm, as you have a right to affirm, that the Negro race is one of the great human races, inferior to none in its accomplishment and in its ability. Different, it is true, and for most of the difference, let us reverently thank God. And this race, with its vantage grounds in modern days, can go forward of its own will, of its own power, and its own initiative. It is led by twelve million American Negroes of average modern intelligence; three or four million educated African Negroes are their full equals, and several million Negroes in the West Indies and South America. This body of at least twenty\sphinxhyphen{}five million modern: men are not called upon to commit suicide because somebody doesn’t like their complexion or their hair. It is their opportunity and their day to stand up and make themselves heard and felt in the modern world.

\sphinxAtStartPar
Indeed, there is nothing else we can do. If you have passed your resolution, “No segregation, Never and Nowhere,” what are you going to do about it? Let me tell you what you are going to do. You are going back to continue to make your living in a Jim\sphinxhyphen{}Crow school; you are going to dwell in a segregated section of the city; you are going to pastor a Jim\sphinxhyphen{}Crow Church; you are going to occupy political office because of Jim\sphinxhyphen{}Crow political organizations that stand back of you and force you into office. All these things and a thousand others you are going to \sphinxhyphen{}do because you have got to.

\sphinxAtStartPar
If you are going to do this, why not say so? What are you afraid of? Do you believe in the Negro race or do you not? If you do not, naturally, you are justified in keeping still. But if you do believe in the extraordinary accomplishment of the Negro church and the Negro college, the Negro school and the Negro newspaper, then say so and say so plainly, not only for the sake of those who have given their lives to make these things worthwhile, but for those young people whom you are teaching, by that negative attitude, that there is nothing that they can do, nobody that they can emulate, and no field worthwhile working in. Think of what Negro art and literature has yet to accomplish if it can only be free and untrammeled by the necessity of pleasing white folks! Think of the splendid moral appeal that you can make to a million children tomorrow, if once you can get them to see the possibilities of the American Negro today and now, whether he is segregated or not, or in spite of all possible segregation.


\subsubsection{Protest}
\label{\detokenize{Volumes/41/06/counsels_of_despair:protest}}
\sphinxAtStartPar
Some people seem to think that the fight against segregation consists merely of one damned protest after another. That the technique is to protest and wail and protest again, and to keep this thing up until the gates of public opinion and the walls of segregation fall down.

\sphinxAtStartPar
The difficulty with this program is that it is physically and psychologically impossible. It would be stopped by cold and hunger and strained voices, and it is an undignified and impossible attitude and method to maintain indefinitely. Let us, therefore, remember that this program must be modified by adding to it a positive side. Make the protest, and keep on making it, systematically and thoughtfully. Perhaps now and then even hysterically and theatrically; but at the same time, go to work to prepare methods and institutions which will supply those things and those opportunities which we lack because of segregation. Stage boycotts which will put Negro clerks in the stores which exploit Negro neighborhoods. Build a 15th Street Presbyterian Church, when the First Presbyterian would rather love Jesus without your presence. Establish and elaborate a Washington system of public schools, comparable to any set of public schools in the nation; and then when you have done this, and as you are doing it, and while in the process you are saving your voice and your temper, say softly to the world: see what a precious fool you are. Here are stores as efficiently clerked as any where you trade. Here is a church better than most of yours. Here are a set of schools where you should be proud to send your children.


\subsubsection{The Conservation of Races}
\label{\detokenize{Volumes/41/06/counsels_of_despair:the-conservation-of-races}}
\sphinxAtStartPar
The Second Occasional Papers published by The American Negro Academy was “The Conservation of Races” by W. E. B. DuBois, and was published in 1897. On page 11, I read with interest this bit:
\begin{quote}

\sphinxAtStartPar
“Here, then, is the dilemma, and it is a puzzling one, I admit. No Negro who has given earnest thought to the situation of his people in America has failed, at some time in life, to find himself at these cross\sphinxhyphen{}roads; has failed to ask himself at some time: What, after all, am I? Am I an American or am I a Negro? Can I be both? Or is it my duty to cease to be a Negro as soon as possible and be an American? If I strive as a Negro, am I not perpetuating the very cleft that threatens and separates Black and White America? Is not my only possible practical aim the subduction of all that is Negro in me to the American? Does my black blood place upon me any more obligation to assert my nationality than German, or Irish or Italian blood would?
\end{quote}
\begin{quote}

\sphinxAtStartPar
“It is such incessant self\sphinxhyphen{}questioning and the hesitation that arises from it, that is making the present period a time of vacillation and contradiction for the American Negro; combined race action is stifled, race responsibility is shirked, race enterprises languish, and the best blood, the best talent, the best energy of the Negro people cannot be marshalled to do the bidding of the race. They stand back to make room for every rascal and demagogue who chooses to cloak his selfish deviltry under the veil of race pride.
\end{quote}
\begin{quote}

\sphinxAtStartPar
“Is this right? Is it rational? Is it good policy? Have we in America a distinct mission as a race—a distinct sphere of action and an opportunity for race development, or is self\sphinxhyphen{}obliteration the highest end to which Negro blood dare aspire?”
\end{quote}

\sphinxAtStartPar
On the whole, I am rather pleased to find myself still so much in sympathy with myself.


\subsubsection{Methods of Attack}
\label{\detokenize{Volumes/41/06/counsels_of_despair:methods-of-attack}}
\sphinxAtStartPar
When an army moves to attack, there are two methods which it may pursue. The older method, included brilliant forays with bugles and loud fanfare of trumpets, with waving swords, and shining uniforms. In Coryn’s “The Black Eagle”, which tells the story of Bertrand du Guesclin, one sees that kind of fighting power in the fourteenth century. It was thrilling, but messy, and on the whole rather ineffective.

\sphinxAtStartPar
The modern method of fighting, is not nearly as spectacular. It is preceded by careful, very careful planning. Soldiers are clad in drab and rather dirty khaki. Officers are not riding out in front and using their swords; they sit in the rear and use their brains. The whole army digs in and stays hidden. The advance is a slow, calculated forward mass movement. Now going forward, now advancing in the center, now running around by the flank. Often retreating to positions that can be better defended. And the whole thing depending upon G.H.Q.; that is, the thought and knowledge and calculations of the great general staff. This is not nearly as spectacular as the older method of fighting, but it is much more effective, and against the enemy of present days, it is the only effective way. It is common sense based on modern technique.

\sphinxAtStartPar
And this is the kind of method which we trust use to solve the Negro problem and to win our fight against segregation. There are times when a brilliant display of eloquence and picketing and other theatrical and spectacular things are not only excusable but actually gain ground. But in practically all cases, this is true simply because of the careful thought and planning that has gone before. And it is a waste of time and effort to think that the spectacular demonstration is the real battle.

\sphinxAtStartPar
The real battle is a matter of study and thought; of the building up of loyalties; of the long training of men; of the growth of institutions; of the inculcation of racial and national ideals. It is not a publicity stunt. It is a life.


\subsubsection{The New Negro Alliance}
\label{\detokenize{Volumes/41/06/counsels_of_despair:the-new-negro-alliance}}
\begin{sphinxShadowBox}
\sphinxstylesidebartitle{}

\sphinxAtStartPar
The arrests of members of the New Negro Alliance led to a \sphinxhref{https://en.wikipedia.org/wiki/New\_Negro\_Alliance\_v.\_Sanitary\_Grocery\_Co}{Supreme Court decision} in 1938 establishing the right to a labor picket.
\end{sphinxShadowBox}

\sphinxAtStartPar
We find ourselves in sudden and apparently complete agreement with our young friends of Washington. It seems that the alliance fell afoul of ordinances against picketing, but that this did not result altogether in failure. Two pickets were arrested, and finally, after a month or so, the complaints were dismissed. In another case, a complaint and temporary injunction is still being fought out before the courts. This is fine. We are glad that the picketing has met with so much of success and we hope that in Washington, as in Chicago, ultimate success will come.

\sphinxAtStartPar
Further than this, the alliance explains that what it is doing, is asking for clerks whose color in the main shall correspond to neighborhoods. If there is a store in a black neighborhood, there should be at least some black clerks in the store. With this, we quite agree, and say, as we said in the Chicago case, that this is fighting segregation with segregation. If there are, for instance (and there certainly are in Washington), segregated neighborhoods, don’t squat before segregation and bawl. Use segregation. Use every bit that comes your way and transmute it into power. Power that some day will smash all race separation. In the meantime, call it what you will. If the Negro Alliance wishes to say that it is not fighting segregation with segregation, it can call the thing that it is doing Transubstantiation or Willipuswallipus. Whatever they call it, that is what we both mean.


\subsubsection{Negro Fraternities}
\label{\detokenize{Volumes/41/06/counsels_of_despair:negro-fraternities}}
\sphinxAtStartPar
Nothing better illustrates our current philosophy and practice in segregation, than the rise and development of Negro fraternities in colleges. When I was a student, fraternities were not allowed in Negro colleges and in the white colleges almost no fraternity ever accepted a Negro member. For a long time, Negro students went their way accepting this situation. When given opportunity, they protested against the Color Line in fraternities, and in a few cases, where the admission to fraternities depended upon scholarship, they succeeded in breaking the Color Line.

\sphinxAtStartPar
Nevertheless, it soon became manifest that there were certain things that the college fraternity could do for a student, which colored students were not getting in the large Northern universities. They lacked very often dormitory facilities; they had no place where they could entertain visiting friends; they had no social center; they had no opportunity for companionship and conference and mutual inspiration.

\sphinxAtStartPar
At Cornell, therefore, in 1906, a group of students formed the Alpha Phi Alpha Fraternity. There were many Negro students there, and in other places, at that time and since, who have condemned this movement as segregation while others excused it as voluntary segregation. It was segregation; and nevertheless, it was necessary; and it was voluntary only in the sense that either Negroes must have their own fraternity or forego fraternal advantages. It was, therefore, as a matter of fact just as compulsory as the “Jim\sphinxhyphen{}Crow” car.

\sphinxAtStartPar
This fraternity movement: has spread all over the United States. It has resulted in colored fraternities and sororities, whose membership runs into the thousands. If anyone has any doubt as to the meaning and inspiration of these fraternities, they should attend one of their national meetings and see the type of men and women that they are bringing together: the splendid enthusiasm, the inspiration and nationwide friendship. This is the kind of segregation that is forced upon us, and it is the kind of segregation in which we glory and which we are going to make the very finest type of institution that the United States has ever seen. And moreover, this is the singular and contradictory result: more Negroes have been taken into white fraternities since Negro fraternities started than ever before. The number thus admitted is still small, but it is not, as the timid argued, smaller; it is much larger.

\sphinxAtStartPar
\sphinxstyleemphasis{Editors Note: This is the final editorial written by Du Bois for The Crisis.}


\bigskip\hrule\bigskip


\sphinxAtStartPar
\sphinxstyleemphasis{Citation:} “Counsels of Despair.” 1934. Editorial.  41(6):182\sphinxhyphen{}184.


\section{Disenfranchisement}
\label{\detokenize{Sections/voting:disenfranchisement}}\label{\detokenize{Sections/voting::doc}}
\sphinxAtStartPar
Editorials on Disenfranchisement.
\begin{itemize}
\item {} 
\sphinxAtStartPar
{\hyperref[\detokenize{Volumes/30/02/disenfranchisement::doc}]{\sphinxcrossref{Disenfranchisement (1925)}}}

\item {} 
\sphinxAtStartPar
{\hyperref[\detokenize{Volumes/36/05/negro_citizen::doc}]{\sphinxcrossref{The Negro Citizen (1929)}}}

\item {} 
\sphinxAtStartPar
{\hyperref[\detokenize{Volumes/36/11/negro_in_politics::doc}]{\sphinxcrossref{The Negro in Politics}}}

\end{itemize}


\subsection{Disenfranchisement (1925)}
\label{\detokenize{Volumes/30/02/disenfranchisement:disenfranchisement-1925}}\label{\detokenize{Volumes/30/02/disenfranchisement::doc}}
\sphinxAtStartPar
How is the Negro disenfranchised? The process is so complicated that few Negroes themselves know definitely. Beginning with 1890 laws have been passed in various Southern states which today disfranchise approximately four million Negroes 21 years of age and over, over half of whom can read and write and who own property which runs into the hundreds of millions. The restrictions by which these have been accomplished are eight in number: 1. Illiteracy: The voter must be able to read and write. 2. Property: The voter must own a certain amount of property. 3. Poll Tax: The voter must have paid his poll tax for the present year or for a series of years. 4. Employment: The voter must have regular employment. 5. Army service: Soldiers in the Civil War and certain other wars, or their descendants, may vote. 6. Reputation: Persons of good reputation who understand the duties of a citizen may vote. 7. “Grandfather” clause: Persons who could vote before the freedmen were enfranchised or descendants of such persons may vote. 8. Understanding clause: Persons may vote who understand some selected clause of the Constitution and can explain it to the satisfaction of the registration officials.

\sphinxAtStartPar
The laws are often obscurely drawn and many of them have not had full judicial determination but apparently these restrictions are distributed as follows:

\sphinxAtStartPar
1890—Mississippi (1 or 8) +3.
1895—South Carolina 1 or 2 or 8.
1898—Louisiana (1+2) or 7.
1901—North Carolina (1+3) or 7.
1901—Alabama (1+4) or 2 or 5 or 6.
1902—Virginia (1+3 or 5) or 8 or 2.
1909—Georgia 1 or 2 or 5 or 6.

\sphinxAtStartPar
That is, in Mississippi the voter must be able to read and write or he must understand and explain a section of the Constitution read to him and, in addition to that, he must have paid his poll tax, etc.

\sphinxAtStartPar
The Grandfather Clause, No. 7, has been declared unconstitutional but as it was in force for nearly a generation most illiterate white people were able to register under it.

\sphinxAtStartPar
The effect of these laws can be illustrated in the case of Louisiana where out of 147,348 colored men 21 years of age and over, of whom 57,000 were reported to be able to read and write, there were in 1908 only 1,743 voters.

\sphinxAtStartPar
But all this amounts to nothing as compared with the effect of the White Primary. The White Primary is based on law and custom and is legally the primary election of the Democratic Party. In fact, the Democratic Party admits to this primary any white person who wishes to vote on condition that he pledge himself to stand by the decision of the primary. On the other hand, no Negro is allowed to vote in the White Primary save in exceptional cases. The White Primary therefore becomes the real election and all over the country the newspapers report the results of primary elections in the South as the real decision. When the legal election takes place very few people vote. I lived 13 years in the city of Atlanta where in a population increasing from 100,000 to 200,000 people, usually 700 votes were cast in the legal election.

\sphinxAtStartPar
In addition to the power of disfranchisement thus held by the White Primary there is social and economic pressure. Colored men are continually told to keep out of politics or lose their jobs and it has become a point of honor with many Negroes of education and character not to vote nor even to attempt to vote. It is in this way that democratic government is made of no account in a large portion of the United States and it is against this that the N.A.A.C.P. is fighting in its latest judicial case against the White Primary of Texas.


\bigskip\hrule\bigskip


\sphinxAtStartPar
\sphinxstyleemphasis{Citation:} “Disenfranchisement.” 1925. Editorial.  30(2):62\sphinxhyphen{}63.


\subsection{The Negro Citizen (1929)}
\label{\detokenize{Volumes/36/05/negro_citizen:the-negro-citizen-1929}}\label{\detokenize{Volumes/36/05/negro_citizen::doc}}
\sphinxAtStartPar
\sphinxstyleemphasis{This article was read before the National Inter\sphinxhyphen{}racial Conference at Washington, D. C., December 19, 1928. It is published here entire, with only verbal changes.}

\sphinxAtStartPar
What we know about the civil and political rights of Negroes in the United States; what significance this knowledge has for social organizations whose purpose it is to improve conditions; and what further study by universities and research organizations is called for, is the subject of this paper.

\sphinxAtStartPar
Our general knowledge may thus be summarized: There is a system of color caste in the United States based on legal and customary race distinctions and discriminations, having to do with separation in travel, in schools, in public accommodations, in residence and in family relations. There is discrimination in the kind and amount of public school education and in civil rights of various sorts and in courts, jails and fines. There is disfranchisement of voters by means of various tests, including restrictions as to registration, and as to voting in primaries; and including the right of summary administrative decisions; and finally there is lynching and mob violence.

\sphinxAtStartPar
Over against this there are the war amendments of the Constitution and various civil right laws of the states and the decisions of the courts in these matters.

\sphinxAtStartPar
The results of these discriminations have been pretty carefully studied in the case of education and lynching, but have received little systematic study in the matter of voting and civil rights.

\sphinxAtStartPar
I doubt if it would be worth while to examine and expatiate on the general and pretty well\sphinxhyphen{}known facts of Negro citizenship and caste. I, therefore, pass to the matter of the significance of this general knowledge for social organizations whose purpose is to improve conditions.


\subsubsection{Afraid of Facts}
\label{\detokenize{Volumes/36/05/negro_citizen:afraid-of-facts}}
\sphinxAtStartPar
We are confronted not simply by lack of exact data but by a clear disposition not to investigate or even to discuss. I know of no organization that has ever proposed to study Negro suffrage.

\sphinxAtStartPar
I distinctly remember when this recoiling from the facts covered other fields. There was a time when social studies, having to do primarily with the health, physique and growth of the Negro population, were of pressing importance because of the widespread assumption that the Negro was not adapted to the American climate or to conditions of life under freedom and that he was bound sooner or later to die out.

\sphinxAtStartPar
It was necessary, therefore, to test by such scientific measurements as were available these assumptions. Yet for a long time universities and social organizations refused to touch the matter and philanthropists refused funds and encouragement when Atlanta University attempted its wretchedly restricted pioneer work. Times changed. Today, tests and measurements have gone so far that there is no further question of the survival of the Negro race in America and the physical studies connected with him are no different and demand no different technique or organization from the general physical studies carried on in the nation. The real question narrows down to matters of sanitation, hospitals and income. What has Negro suffrage to do with these?

\sphinxAtStartPar
 

\sphinxAtStartPar
Again, between the years 1890 and 1910, the right of the American Negro to modern education had to be established and proven. It was assumed that the ability of the Negro to assimilate a college education was at least questionable; and it was dogmatically stated ‘that the economic future of the Negro in America was such that all that he needed was industrial training to make him a contented laborer and servant; that this class of people did not need political power and could not use it; but that on the contrary their disfranchisement would free the South so that it could divide its vote on pressing political matters; and that the South could be depended upon to guard the rights of this working caste.

\sphinxAtStartPar
The fight was bitter and long drawn out. Those of us who insisted that in modern industrial life no laboring class could maintain itself without educational leadership and political power were assailed, put out of court, accused of jealousy, and of an overwhelming desire to promote miscegenation.

\sphinxAtStartPar
Today finds the educational part of our contention answered by facts. We have twelve thousand college students, where we had less than one thousand in 1900, and we are graduating today annually 1500 Bachelors of Art, when in 1900 we sent out less than 150. It is admitted now without serious question that the American Negro can use modern education for his group development, in economic and spiritual life.

\sphinxAtStartPar
There is, however, still the feeling that the present problems of Negro education are problems of charity, good will, self\sphinxhyphen{}sacrifice and double taxation and not problems which depend primarily for their final solution upon political power.

\sphinxAtStartPar
So, too, in the matter of housing, recreation and crime, we seem to assume that a knowledge of the facts of discrimination and of the needs of the colored public are sufficient, with faith, hope and charity, to bring ultimate betterment; and that in presenting demands to the government of city, state and nation, we have only to prove that Negro poverty, disease and crime hurt white citizens in order to induce the lawmakers elected by white citizens to do justice to black citizens.


\subsubsection{Political Power}
\label{\detokenize{Volumes/36/05/negro_citizen:political-power}}
\sphinxAtStartPar
In the matter of occupation and income the need of political power in any laboring class is conceded by every social student; for the American Negro or his friends to dream that he can sustain himself as a peasant proprietor, an artisan or day laborer, and secure recognition from his organized voting white fellow worker and a decent wage from his employer, without a vote, is extraordinary. It is a conceded impossibility in every modern land.

\sphinxAtStartPar
We can point with some pride to what has been accomplished in the courts in breaking down caste and establishing Negro citizenship, and in the abolition of mob law and lynching. But we are still uncertain in estimating the cause and effects of such actions.

\sphinxAtStartPar
I have heard a number of plausible and attractive explanations of the decline of lynching from 226 in 1896 to 11 in 1928. Some attribute it to prayer, and others to inter\sphinxhyphen{}racial resolutions; but I see it differently. I see lynching increase and decrease indifferently, until in 1919 a nationwide agitation was begun by the N.A.A.C.P., backed by statistics, advertisements and meetings. The curve of mob murder fell lazily. Then suddenly in a single year it dropped 75\%. I study the occurrences of that year, 1922. And that study leads me to believe that the effective check to lynching was the organized political power of Northern Negroes that put the Dyer Anti\sphinxhyphen{}Lynching Bill through the House of Representatives January 26, 1922, by a vote of 230 to 119.


\subsubsection{The Filibuster}
\label{\detokenize{Volumes/36/05/negro_citizen:the-filibuster}}
\sphinxAtStartPar
The bill was forced through a senate committee and reported to the Senate with a majority pledged to its passage. The only way that the South accomplished its defeat was by refusing to allow the government of the United States to function. Knowing that such high\sphinxhyphen{}handed measures were going a bit too far, the South promised to stop lynching and it has pretty nearly kept its word. And yet consider the cost: there has not been in Poland or in Haiti, in Russia or in the Balkans, a more open, impudent, and shameless holding up of Democracy than the senators of the Bourbon South, holding office on the disfranchised Negro vote, accomplished in November, 1922.

\sphinxAtStartPar
The success which we have had before the courts in abolishing the hereditary right to vote which the “Grandfather” clauses bestowed on white Southerners; the fight against segregation in residence and its spread in schools; the fight against the white primary and numerous Civil Rights cases have not simply been brought to successful issue because of our present small but increasing political power, but are without significance unless they point to fuller political power.

\sphinxAtStartPar
I do not for a moment argue that political power will immediately abolish color caste, make ignorant men intelligent or bad men good. We have caste and discrimination in the North with the vote and social progress in some parts of the South without it. But there is this vast difference: in states like New York where we are beginning to learn the meaning and use of the ballot we are building a firm and unshakable basis of permanent freedom. While every advance in the South unprotected by political power is based on chance and changing personalities and may at any time be vetoed by a hostile voting group. I maintain that political power is the beginning of all permanent reform and the only hope for maintaining gains.


\subsubsection{The Negro Vote}
\label{\detokenize{Volumes/36/05/negro_citizen:the-negro-vote}}
\sphinxAtStartPar
There is today a surprisingly large number of intelligent and sincere people, both white and black, who really believe that the Negro problem in the United States can ultimately be solved without our being compelled to face and settle the question of the Negro vote.

\sphinxAtStartPar
Nearly all of our social studies apparently come to this conclusion, either openly or by assumption, and do not say, as they ought to say, and as everyone knows in the long run they must say, that granted impulse by philanthropy, help by enlightened public opinion, and the aid of time, no permanent improvement in the economic and social condition of Negroes is going to be made so long as they are deprived of political power to support and defend it.

\sphinxAtStartPar
Nowhere else in the world is there any suggestion that a modern laboring class can permanently better itself without political power. It may be a question, it certainly is a question, as to just how labor is going to use this power ultimately so as to raise its economic and social status. But there is no question, but that such power must be had and today the world over it is being used.

\sphinxAtStartPar
With all the research that has gone on formerly, and especially in the last few years with regard to the American Negro, with singular equanimity, nothing has been said or done with regard to the Negro vote. I am, therefore, stressing in this paper the significance and the danger of this omission and I am seeking to say that of all the questions that are before us today that of political power on the part of the American Negro occupies, to my mind, the key position, and is the question which peculiarly tests the good faith of the American people, the honesty of philanthropy in America toward the Negro, and the sincerity of the National Inter\sphinxhyphen{}racial Conference.


\subsubsection{A Debate}
\label{\detokenize{Volumes/36/05/negro_citizen:a-debate}}
\sphinxAtStartPar
I listened yesterday with mounting astonishment to a discussion of school betterment in the South. I am convinced that in no other civilized country in the world could such a discussion have taken place. The crucial problem was that of raising local funds for schools and of having the National Government supplement those funds in the poorer states: and the essential point in the whole matter was surely the selection of local officials who would spend the money as the local voting population wished; would raise funds by local taxation fairly placed on local wealth and would expend National monies equitably. In any other land the first point of the debate would have been the question of the selection of such proper officials and of the democratic control of their actions.

\sphinxAtStartPar
That question in the debate to which I listened was never raised. It was assumed that, although there were to be separate schools for Negroes, Negroes were to have no voice in the selection of local officials, no control of their own taxation, no vote on expenditure; and that despite this through philanthropy and good will you were going to get and maintain a decent and adequate school system for them.

\sphinxAtStartPar
If the present rulers of Russia had heard this debate they would have gone into gales of laughter; and if any government had attempted to carry on a debate on these lines in the English Parliament, the German Reichstag or the French Chamber of Deputies, the government would have been thrown out forthwith. Every Englishman, Frenchman and German would have said, without qualification, that education today can not be carried on as a matter of philanthropy and good will: that it is the duty of the State and that back of the State must stand some effective democratic control.


\subsubsection{Democracy}
\label{\detokenize{Volumes/36/05/negro_citizen:democracy}}
\sphinxAtStartPar
Most nations would have made this control the ballot in the hands of all adult citizens and even Italy and Russia and Turkey would affirm that this is the ideal toward which they consistently and steadily march.

\sphinxAtStartPar
It is of extraordinary significance that in an intelligent and open\sphinxhyphen{}hearted assembly, such a clear and obvious point was either not thought of or worse yet, the members did not have the courage to make it.

\sphinxAtStartPar
In the question of the lack of public funds for growing expense in education one cannot assume that Americans do not know what the public thought of the world in the most progressive countries is doing, in insisting that wealth bear a greater burden of taxation and that poverty be exempt. The United States is the one great country of the world where wealth is escaping taxation and where the burden of public contributions that falls upon the farmer, the small householder, the laborer and particularly the black laborer is crushing in its incidence; and yet how little is said of drafting by universal suffrage sufficient wealth for the public good to pay every reasonable expense and of putting the people, black and white, back of such draft.

\sphinxAtStartPar
\sphinxstyleemphasis{I hold this truth to be self evident, that a disfranchised working\sphinxhyphen{}class in modern industrial civilization is worse than helpless. It is a menace, not simply to itself, but to every other group in the community. It will be diseased; it will be criminal; it will be ignorant; it will be the plaything of mobs, and it will be insulted by caste restrictions.}

\sphinxAtStartPar
So far we are upon old ground. This argument has been urged many times in the past. It has failed to impress the people of the United States simply because so many folk do not care about the future of American Negroes. They once almost hoped that the problem would be settled by the Negroes dying out or migrating, or bowing in dumb submission to any kind of treatment that the people of the United States decided to give them.

\sphinxAtStartPar
But, today, the matter is changed, and it is changed because those Americans who have any ability to see and think are beginning slowly to realize that when Democracy fails for one group in the United States, it fails for the nation; and when it fails for the United States it fails for the world. A disfranchised group compels the disfranchisement of other groups. ‘The white primary system in the South is simply a system which compels the white man to disfranchise himself in order to take the vote away from the Negro.

\sphinxAtStartPar
The present extraordinary political psychology of the Negro in the South; namely that the voluntary disfranchisement of intelligent and thrifty black men is helping to solve the Negro problem, is simply putting into the hands of scoundrels and grafters white and black the meagre remains of those political rights which 200,000 black civil war soldiers fought to gain.


\subsubsection{The Bourbon South}
\label{\detokenize{Volumes/36/05/negro_citizen:the-bourbon-south}}
\sphinxAtStartPar
All this has led to extraordinary results. In the past we have deplored disfranchisement in the South because of its effect on the Negro. But it is not simply that the Negro remains a slave as long as he is disfranchised, but that Southern white laborers are dragged inevitably down to the Negro’s position, and that the decent white South is not only deprived of decent government, but of all real voice in both local and national government. It is as true today as it ever was that the nation cannot exist half slave and half free.

\sphinxAtStartPar
Today, in the South, politicians have every incentive to cut down the number of voters, black and white. The Republican organization, in nine cases out of ten, becomes simply the tail to the Democratic kite. Party government disappears. Political power is vested in the hands of a clique of professional politicians, white and black, and there is nothing that has been done in dirty politics by Tammany in New York, by Thompson in Chicago, or Vare in Philadelphia, that you cannot find duplicated by the political oligarchies which rule the Southern South.

\sphinxAtStartPar
Political ignorance in the South has grown by leaps and bounds. The mass of people in the South today have no knowledge as to how they are governed or by whom. Elections have nothing to do with broad policies and social development but are matters of the selection of friends to lucrative offices and punishment of personal enemies. Local administration is a purposely disguised system of intrigue which not even an expert could unravel.

\sphinxAtStartPar
Today, a small group of Western Congressmen, to the dismay of East and South, are investigating the sale of offices by black Republicans in the South; but offices from the highest to the lowest have been regularly sold by white Republicans and white Democrats in the South and are being sold today.

\sphinxAtStartPar
And yet, of all this, there must be no criticism, no exposure, no real investigation, no political revolt, because decent white South lacks the moral courage to expose and punish rascals even though they are white and to stand up for democracy even if it includes black folk.

\sphinxAtStartPar
I yield to no man in my admiration for what the new young South is doing in liberalizing race relations and humanizing thought, but I maintain that until the liberal white south has the guts to stand up for democracy regardless of race there will be no solution of the Negro problem and no solution of the problem of popular government in America. You cannot build bricks of molasses.

\sphinxAtStartPar
Nor is this all. Because of the rotten boroughs of the South, real Democratic government is impossible in the North. The Democratic Party cannot become a liberal body because the bulk of its support depends upon disfranchisement, caste and race hate in the South. It depends on minimizing participation in politics by all people, black and white, and stifling of discussion. It is the only part of the nation where the woman suffrage amendment is largely ignored and yet the white women do not dare to open their mouths to protest.

\sphinxAtStartPar
So long as this party holds this grip on 114 electoral votes despite argument, with no reference to dominant political questions and with no reference to the way in which votes are actually cast, this party cannot be displaced by a Third Party. With no Third Party corrective for a discredited minority, democratic government becomes simply impossible without something resembling revolution.


\subsubsection{Recent Eletions}
\label{\detokenize{Volumes/36/05/negro_citizen:recent-eletions}}
\sphinxAtStartPar
When in 1912 Roosevelt tried to appeal to liberal thought in the United States against the reactionary Republicans and the bourbon Democrats, he only succeeded in putting the Democrats in power. When La Follette tried to do the same thing in 1924, he simply scared the country into larger reaction, since they realized that they had to choose between bourbon Democracy and organized privilege.

\sphinxAtStartPar
In 1928, we had an extraordinary spectacle. It is too well\sphinxhyphen{}known for me to comment. I only remind the reader that the right of Southern white men to vote as they wished on public questions was openly and vehemently denied and the right of dominant political cliques holding their power by disfranchising four million white and black voters, to make their own election returns as to the vote cast, without state or national investigation or inquiry, was successfully maintained. This is the only modern nation in the world which does not control its own elections.


\subsubsection{The Future}
\label{\detokenize{Volumes/36/05/negro_citizen:the-future}}
\sphinxAtStartPar
How is all this going to be remedied? How are we going to restore normal democracy in the United States? It is not a question of the millennium; of being able through democratic Government to do everything immediately. But it is a question, and a grave and insistent question, whether the United States of America is going to maintain or surrender democracy as the fundamental starting point of permanent human uplift. If democracy is still our corner stone, must it be smashed because of twelve million Negroes? Better cut their throats quickly and build on.

\sphinxAtStartPar
On the other hand, if Democracy fails in the United States, and fails because of our attitude toward a darker people, what about Democracy in the world, and particularly in India, in China, in Japan and in Egypt? We have got a chance today, and an unrivaled chance, again to rescue and guide the world, as we did at the end of the 18th Century. And we have the same kind of dilemma.

\sphinxAtStartPar
In those days when we started to build a nation of equal citizens, Negro slavery could have been abolished; its abolition was begun even in the South; but the respectable people, the smug people, sat down before it and organized the American Colonization Society, which was the interracial movement of that day; and instead of fighting evil they were content to congratulate themselves on the good already accomplished. In the long run, they did less than nothing.

\sphinxAtStartPar
So today it is fortunate that people can sit down at Interracial Conferences and find so much to congratulate themselves about in the improved relation between races, and the increased knowledge which they have of each other. But all of this is going to be of no avail in the crisis approaching unless we take advantage of the present desire for knowledge and willingness to study and willingness to listen, and attack the main problem which is and has been the question of political power for the Negro citizens of the United States.

\sphinxAtStartPar
I do not for a moment minimize the difficulty of inaugurating in a land but a generation removed from slavery, of universal suffrage which includes children of slaves. It is extraordinarily difficult and calls for patience and tolerance. But my point is that the sooner we face the goal the quicker we will reach it. We are not going to make democracy in the South possible by admitting its impossibility and refusing to study and discuss the facts. Let us first of all say, and broadcast the fact, that all Americans of adult age and sufficient character and intelligence must vote and that any interference with or postponement of this realization is a danger to every other American—a danger to be attacked now and continuously and with dogged determination with a clear avowal of intention by every open\sphinxhyphen{}minded man.

\sphinxAtStartPar
What then is called for? Facts. A foundation of actual fact concerning the political situation of Negroes; their voting, their representation in local, state and national government; their taxation, their party affiliation and subserviance to political machines; the economic nexus between political power and occupation and income.

\sphinxAtStartPar
This study beginning with Negroes should extend to whites. We must lift the curtain from democracy and view it into the open. We must insist that politics is no secret, shameful thing known only to ward heelers and political bosses, and to the corporations who buy and sell them. Here is the greatest and most insistent field of scientific investigation open to the social reformer.


\bigskip\hrule\bigskip


\sphinxAtStartPar
\sphinxstyleemphasis{Citation:} “The Negro Citizen.” 1929. Editorial.  36(5):156\sphinxhyphen{}157, 171\sphinxhyphen{}173.


\subsection{The Negro in Politics}
\label{\detokenize{Volumes/36/11/negro_in_politics:the-negro-in-politics}}\label{\detokenize{Volumes/36/11/negro_in_politics::doc}}
\sphinxAtStartPar
The political situation of the American Negro this fall has many anomalies, contradictions and encouragements. A white Southerner has been made Chairman of the Republican National Committee. He will undoubtedly do what he can to eliminate the Negroes from political activity in the South. If he is successful, there will grow up in the South two parties dominated by white men. This alarms some Negroes and certain of their friends. \sphinxstylestrong{The Crisis} is not alarmed. If these are two real parties and not merely one party with two faces, then they will need votes; and the more progressive their program is, the greater will be their need. This will be the Negroes’ opportunity.

\sphinxAtStartPar
But this means that the Negro must become opportunist in politics. No area illustrates this better than Harlem. Black Harlem cares nothing for political labels. Candidates may be marked Democrat, Republican, Socialist, Harlem votes for the candidate. Moreover, it votes very largely for local reasons. It will support Mayor Walker, not because he is the best candidate. He is not. Norman Thomas, the Socialist candidate, is by long odds the best man running. But he has no chance of election. On the other hand, Mayor Walker has done a great deal for Harlem and is a much better man than the Republican, La Guardia; and Harlem will stand by him. Harlem will vote for a colored Congressman, for two colored Aldermen, for a colored member of the Assembly, for a colored leaders in Republican organizations.

\sphinxAtStartPar
It is, of course, unfortunate that in all this, Harlem and the Negroes of the United States, must vote “‘colored”’; but the fault is not theirs. So long as the color of a man’s skin means more, to most candidates, than the tariff, democratic government, prohibition, war, or any other issue, just so long  the black man must vote with his eye on this fact. In these larger issues, he disfranchises himself, but he refuses to commit suicide in order to save a white world.


\bigskip\hrule\bigskip


\sphinxAtStartPar
\sphinxstyleemphasis{Citation:} “The Negro in Politics.” 1929. \sphinxstyleemphasis{The Crisis}. 36(11): 387.


\section{Voting and Parties}
\label{\detokenize{Sections/electoral:voting-and-parties}}\label{\detokenize{Sections/electoral::doc}}
\sphinxAtStartPar
Editorials on electoral politics and voting.
\begin{itemize}
\item {} 
\sphinxAtStartPar
{\hyperref[\detokenize{Volumes/04/04/politics::doc}]{\sphinxcrossref{Politics (1912)}}}

\item {} 
\sphinxAtStartPar
{\hyperref[\detokenize{Volumes/15/01/oath_of_the_negro_voter::doc}]{\sphinxcrossref{The Oath of the Negro Voter (1917)}}}

\item {} 
\sphinxAtStartPar
{\hyperref[\detokenize{Volumes/21/02/unreal_campaign::doc}]{\sphinxcrossref{The Unreal Campaign (1920)}}}

\item {} 
\sphinxAtStartPar
{\hyperref[\detokenize{Volumes/22/01/drive::doc}]{\sphinxcrossref{The Drive (1921)}}}

\item {} 
\sphinxAtStartPar
{\hyperref[\detokenize{Volumes/34/04/chicago::doc}]{\sphinxcrossref{Chicago (1927)}}}

\item {} 
\sphinxAtStartPar
{\hyperref[\detokenize{Volumes/35/11/on_the_fence::doc}]{\sphinxcrossref{On the Fence (1928)}}}

\item {} 
\sphinxAtStartPar
{\hyperref[\detokenize{Volumes/35/11/third_party::doc}]{\sphinxcrossref{A Third Party (1928)}}}

\end{itemize}


\subsection{Politics (1912)}
\label{\detokenize{Volumes/04/04/politics:politics-1912}}\label{\detokenize{Volumes/04/04/politics::doc}}
\sphinxAtStartPar
The colored voter now stands face to face with the great question of the proper use of his electoral franchise. Under normal conditions 2,000,000 of the 20,000,000 votes which might be cast at a presidential election would belong to the race, and some day, despite every effort of fraud and race prejudice, those votes are going to be cast.

\sphinxAtStartPar
To\sphinxhyphen{}day, however, of the 15,000,000 or more votes which will actually be cast for President, some 500,000 will be black men’s votes.

\sphinxAtStartPar
What shall we do with these 500,000 ballots?

\sphinxAtStartPar
First of all we must teach ourselves to regard them seriously. The Negro\sphinxhyphen{}American is not disfranchised; on the contrary, he is a half million votes this side of disfranchisement and that is a long, long way. There have been ‘but two or three presidential elections since the war which have not been settled by a margin of less than a half million votes, and in every single election since the proslavery compromise of 1850 such a number of votes distributed at strategic points would easily have decided the presidency.

\sphinxAtStartPar
The votes of black Americans are today at strategic points. We may, of course, leave the South out of account: on account of illegal enactments and brazen fraud, democratic government exists in the South only in inchoate and incomplete form. The presidential election is probably going to be decided by the Middle West and the States of New York and New Jersey. New York and

\sphinxAtStartPar
Ohio have each between 40,000 and 50,000 colored votes; New Jersey, Illinois and Indiana have each 30,000 or more. Is there any political prophet who would risk his reputation on the possibility of any one of these States being carried by more than 20,000 majority? There may be majorities of 50,000 or 100,000 in one or two of the States, but the chances are that the colored voters hold the balance of power in every one of their States and thus have the power to say whether William Howard Taft or Woodrow Wilson shall be the next President.

\sphinxAtStartPar
If colored America had long political experience and wide knowledge of men and measures, it would organize the black voters of each State into a solid phalanx. It would say to this phalanx: white and colored voters in this land are selling their votes too cheaply. By the use of a “slush fund” of \$3,000,000 Theodore Roosevelt was able almost to split the Republican party. You could easily sell your votes next November for one or two millions of dollars, but that is too cheap. You could easily sell your votes for an Assistant Attorney General, a Register of the Treasury, a Recorder of Deeds and a few other black wooden men whose duty it is to look pleasant, say nothing and have no opinions that a white man is bound to respect. This also is too cheap—it is dirt cheap.

\sphinxAtStartPar
What price should you ask for 500,000 votes, black America? You should ask this:
\begin{enumerate}
\sphinxsetlistlabels{\arabic}{enumi}{enumii}{}{.}%
\item {} 
\sphinxAtStartPar
The abolition of the interstate “Jim Crow” car.

\item {} 
\sphinxAtStartPar
The enforcement of the Thirteenth Amendment by the suppression of peonage.

\item {} 
\sphinxAtStartPar
The enforcement of the Fourteenth Amendment by cutting down the representation in Congress of the rotten boroughs of the South.

\item {} 
\sphinxAtStartPar
National aid to elementary public schools without class or racial discrimination.

\end{enumerate}

\sphinxAtStartPar
Is this price too much to pay for a presidency? It is not if you dare ask it.

\sphinxAtStartPar
Who then would pay it? Would William Howard Taft pay it? There has not been in the presidential chair for fifty years a man so utterly lacking in initiative and ideal as Mr. Taft. His abject surrender to Southern prejudice and reaction has been simply pitiable: He began his career by defending disfranchisement; he followed this by promising to appoint no black man to office if any white man protested; and in spite of the fact that over 200 Negroes have been publicly murdered without trial during his administration, the utmost that 10,000,000 black men have elicited from his lips is a hesitating statement that he is sorry—and helpless. Any colored man who votes for Mr. Taft will do so on the assumption that zero is better than minus one.

\sphinxAtStartPar
As to Mr. Wilson, there are, one must confess, disquieting facts: he was born in Virginia and he was long president of a college which did not admit Negro students and yet was not honest enough to say so, resorting rather to subterfuge and evasion. A man, however, is not wholly responsible for his birthplace or his college. On the whole, we do not believe that Woodrow Wilson admires Negroes. Left to himself, we suspect he would be like Mr. Johnson, the new dean of Yale. Mr. Johnson is a Southerner, and recently told a colored applicant that Yale did not want “Chinese, Jews or Negroes.” The ideal of such folk would be a world inhabited by flaxen\sphinxhyphen{}haired wax dolls with or without brains.

\begin{sphinxShadowBox}
\sphinxstylesidebartitle{}

\sphinxAtStartPar
“\sphinxhref{https://en.wikipedia.org/wiki/Benjamin\_Tillman}{Tillman}, \sphinxhref{https://en.wikipedia.org/wiki/James\_K.\_Vardaman}{Vardaman}, \sphinxhref{https://en.wikipedia.org/wiki/M.\_Hoke\_Smith}{Hoke Smith} and \sphinxhref{https://en.wikipedia.org/wiki/Coleman\_Livingston\_Blease}{Blease}” were white supremacist Southern elected officials.
\end{sphinxShadowBox}

\sphinxAtStartPar
Notwithstanding such possible preferences, Woodrow Wilson is a cultivated scholar and he has brains. We know that there are several hundred millions of “Chinese, Jews and Negroes” who have to be reckoned with, and that the date at which the “blond beast” will inherit this earth has been, to put it mildly, indefinitely postponed. We have, therefore, a conviction that Mr. Wilson will treat black men and their interests with farsighted fairness. He will not be our friend, but he will not belong to the gang of which Tillman, Vardaman, Hoke Smith and Blease are the brilliant expositors. He will not advance the cause of oligarchy in the South, he will not seek further means of “Jim Crow” insult, he will not dismiss black men wholesale from office, and he will remember that the Negro in the United States has a right to be heard and considered; and if he becomes President by the grace of the black man’s vote, his Democratic successors may be more willing to pay the black man’s price of decent travel, free labor, votes and education.

\begin{sphinxShadowBox}
\sphinxstylesidebartitle{}

\sphinxAtStartPar
After losing the Republican nomination, Roosevelt, running under the banner of the Progressive Party, finished in second place to Wilson in the 1912 presidential elections. Debs came in fourth with 6\%.
\end{sphinxShadowBox}

\sphinxAtStartPar
Outside of these two men, what else? We thank God that Theodore Roosevelt has been eliminated. How many black men, with the memory of Brownsville, could support such a man passes our comprehension. Of Eugene V. Debs, the Socialist candidate, we can only say this frankly: if it lay in our power to make him President of the United States we would do so, for of the four men mentioned he alone, by word and deed, stands squarely on a platform of human rights regardless of race or class.


\bigskip\hrule\bigskip


\sphinxAtStartPar
\sphinxstyleemphasis{Citation:} Du Bois, W.E.B. 1912. “Politics.”  \sphinxstyleemphasis{The Crisis}. 4(4):180\sphinxhyphen{}181.


\subsection{The Oath of the Negro Voter (1917)}
\label{\detokenize{Volumes/15/01/oath_of_the_negro_voter:the-oath-of-the-negro-voter-1917}}\label{\detokenize{Volumes/15/01/oath_of_the_negro_voter::doc}}
\sphinxAtStartPar
As one of the Earth’s Disowned I swear to hold my Ballot as the sacred pawn of Liberty for all mankind and for my prisoned race.

\sphinxAtStartPar
I will accept no price for my priceless Vote, save alone just laws, honestly dealt, without regard to color, wealth or strength. I will make the first and foremost aim of my voting the Enfranchisement of every citizen, male and female; and particularly the restoring of the stolen franchise to my people, by which continuing theft the 1 enemies of the Negro race sit in high places today and wretchedly misgovern.

\sphinxAtStartPar
I will make the second object of my voting the division of the Social Income on the principle that he who does not work, be he rich or poor, may not eat; and that Land and Capital ought to belong to the Many and not to the Few.

\sphinxAtStartPar
I will accept no Office which I cannot efficiently fill; I will judge all Officials by their service to the common weal and I will not regard the mere giving of Office to my friends as payment for my support of any party.

\sphinxAtStartPar
I will judge all Political Parties not by their past deeds or their future promises but simply by the present acts of the Officials who represent them, and I will cast my vote for or against those officials accordingly.

\sphinxAtStartPar
I will scan carefully the Record of every candidate for whom I must vote and especially of Congressmen, Legislators and local Officials, learning what manner of man each is, how he has carried out his trust and what pledges he makes in general; and in particular I will ask his attitude toward my race.

\sphinxAtStartPar
I hereby solemnly pledge myself to join with others like\sphinxhyphen{}minded to myself in thus before each election, agreeing upon a list of suitable Candidates who by their records or promises seem most likely to secure good government and justice to black folk, and I will vote for these candidates, regardless of their party, race or sex.

\sphinxAtStartPar
I will have firm faith in Democracy, despite its mistakes and inefficiency, knowing that in no other way can the common Experience, Want and Will be pooled for the common good, and that no Despot or Aristocrat can ever be wise or good enough to rule his fellowmen.

\sphinxAtStartPar
In order to accomplish the above ends I hereby entrust the National Association for the Advancement of Colored People and its local branches with the investigation and co\sphinxhyphen{}operation necessary to the listing of suitable candidates for my franchise at each election.

\sphinxAtStartPar
Persons minded to sign the above oath will send their names to the Editor of THE CRISIS.


\bigskip\hrule\bigskip


\sphinxAtStartPar
\sphinxstyleemphasis{Citation:} “The Oath of the Negro Vote” Editorial. 1917. \sphinxstyleemphasis{The Crisis}. 15(1): 7.


\subsection{The Unreal Campaign (1920)}
\label{\detokenize{Volumes/21/02/unreal_campaign:the-unreal-campaign-1920}}\label{\detokenize{Volumes/21/02/unreal_campaign::doc}}
\sphinxAtStartPar
Never have the American people endured such a Presidential campaign. It is true that in no campaign are the great issues always distinct and clear and the alignments definite and understandable. But surely in no campaign has there ever been such a lack of alignment and such deliberate smearing of issues. One may, as one is constituted, regard voting as a test of conscience and principle or as a practical make\sphinxhyphen{}shift. In the first, the land may easily go to the dogs while the Dreamer walks his high and isolated path. In the other case, we may too easily forget our dreams.

\sphinxAtStartPar
But both sorts of Voters on November 2, 1920, were in maddening puzzlement. The League of Nations? There was no real difference between the parties; between Johnson and Wilson there was a world of argument; but Taft and Holt, Hoover and Cox? It was all a matter of punctuation and style. No man then could vote either the Republican or the Democratic ticket because he favored or opposed the League. And outside the League, What? Imperialism, labor and wage, the power of capital, the marketing of farm products, the building of homes, the training of children, the ownership of land, the freedom of suffrage—in these and all else the major parties were mere fog or reaction. Indeed the only real, stinging, fighting questions in the whole campaign were President Wilson and the Negro.


\subsubsection{The Family Tree}
\label{\detokenize{Volumes/21/02/unreal_campaign:the-family-tree}}
\sphinxAtStartPar
The allegation was that the President\sphinxhyphen{}elect had a black man somewhere among his remote ancestors. Can you conceive how real a fury this query roused? How else could it be with our deliberate cultivation of race hate and with Pat Harrison of Mississippi heading one of the Speakers’ Bureaus?

\sphinxAtStartPar
Suppose President Harding is colored—What of it? He would be but one of hundreds of distinguished Americans who served their country well from the day of Alexander Hamilton to that of Lew Wallace. Nefertari and Amenhotep, Candace and Terence, Askia and Tamuramaro, Browning and Lafcadio Hearn—how many of the world’s heroes have shared the black blood of Africa! It is an ancient and noble lineage; as high and deserving as that of any race on earth, until Ethiopia was raped and murdered and despoiled by Europe. God knows that all of us— black, white, red and yellow, are low enough in ancestry and service, but “Pure White America” excels most lands in being able to boast an ancestry which includes far more jails, asylums and gutters and far fewer palaces than most nations. Is this her shame or glory? And which was worse: the shrieking whispers of the Democrats, or the vociferous denials of the Republicans of the taint! Taint, forsooth! what could taint America?

\sphinxAtStartPar
Ohio, once the home of freedom, led in Anti\sphinxhyphen{}Negro propaganda, egged on by Cox and the new white southern immigration. “Timely warnings” to white folk were distributed by the hundred thousand declaring, “Ohioans should remember that the time has come when we must handle this problem in somewhat the same way as the South is handling it!” As a result many of the colored candidates for the legislature were defeated, but more were elected than ever before and many friends like Governor Davis, formerly Mayor of Cleveland, triumphed.


\subsubsection{Third Parties}
\label{\detokenize{Volumes/21/02/unreal_campaign:third-parties}}
\sphinxAtStartPar
The third parties have made a singularly spiritless campaign. As residual legatees of all protest and unrest they did almost nothing to answer the searching query of desperate strivers for light. Not the ultimate dogma of Marxian socialism or the undigested hodge\sphinxhyphen{}podge of Farmer\sphinxhyphen{}Laborers sufficed. Men asked and were not answered: What is the next step and how does it lead to a definite, clearmarked way? The great mass of laboring people, the thrifty, hardworking farmers and small capitalists, lacked dismally here that leadership which through the great Labor party of England and the right wing Socialists of Italy are hewing, waveringly but stubbornly, a real path, leaving on the one side intransigeant communism and on the other, organized and reactionary theft. Not so America. We, the burden\sphinxhyphen{}bearers, could not even agree to disagree and delivered the masses bound into the hands of the Phillistines.

\sphinxAtStartPar
And the Black Man. He had no chance. He was less than free and more than a slave. He was a machine —an automatic registration mark for the Republican party. He could not be otherwise. From the day Woodrow Wilson shamelessly betrayed his black supporters of 1912 to the day when the flippant Cox of Ohio built his Ohio campaign on the cheapest brand of “n{[}*****{]}”\sphinxhyphen{}hatred, the black American had but one political choice or mission: to defeat the South\sphinxhyphen{}ridden Democratic party. He could not even think of taking an off\sphinxhyphen{}shot at the Millenium by voting Socialist or Farmer Labor—he must defeat the Democrats.

\sphinxAtStartPar
And he did his bit.

\sphinxAtStartPar
And so the great farce ends. The People have spoken—and said nothing.


\bigskip\hrule\bigskip


\sphinxAtStartPar
\sphinxstyleemphasis{Citation:} “The Unreal Campaign” Editorial. 1920. \sphinxstyleemphasis{The Crisis}. 21(2): 54\sphinxhyphen{}56.


\subsection{The Drive (1921)}
\label{\detokenize{Volumes/22/01/drive:the-drive-1921}}\label{\detokenize{Volumes/22/01/drive::doc}}
\sphinxAtStartPar
Again the pendulum has swung: it is no longer a question of educating the Negro to his just demands as an American citizen. He has ceased to be beguiled by the silly philosophy that a voteless, spineless suppliant who owns a three\sphinxhyphen{}story brick house is going to command anybody’s respect.

\sphinxAtStartPar
But today comes the question of practical, efficient means of getting the rights which he has at last been persuaded he wants.

\sphinxAtStartPar
Into the field have jumped a hoard of scoundrels and bubble\sphinxhyphen{}blowers, ready to conquer Africa, join the Russian revolution, and vote in the Kingdom of God tomorrow. It is without doubt certain that Africa will some day belong to the Africans; that steamship lines and grocery stores, properly organized and run, are excellent civilizers; and that we are in desperate need today of organized industry and organized righteousness. But what are the practical steps to these things? By yelling? By pouring out invective and vituperation against all white folk? By collecting the pennies of the ignorant poor in shovelsful and refusing to account for them, save with bombast and lies?

\sphinxAtStartPar
Or is it reason and decency to unite on a program which says: the battle of Negro rights is to be fought right here in America; that here we must unite to fight lynching and “Jim\sphinxhyphen{}Crow” cars, to settle our status in the courts, to put our children in school and maintain our free ballot.

\sphinxAtStartPar
Far from being discouraged in the light, we are daily more and more triumphant. Yesterday 1,650 Negro women voted in New Orleans. Never since 1876 have so many Negroes voted in the South as in the last election. Our fight for right has the enemy on the run. He has had to retreat to mob violence, secret and silly mummery, clumsy and hypocritical promises. Twenty\sphinxhyphen{}five years more of the intelligent fighting that the N.A.A.C.P. has led will make the black man in the United States free and equal.

\sphinxAtStartPar
Our enemies know this. They are scared. They are hastening to lay down a barrage of suspicion and personal bickering. They are encouraging and advertising any and all crazy schemes, to cut and run from the hard and bloody battle here, to Africa and the South Seas. Africa needs her children, but she needs them triumphant, victorious, and not as poverty\sphinxhyphen{}
stricken and cowering refugees.

\sphinxAtStartPar
Are we going to be misled fools, or are we going to put a quarter of a million level\sphinxhyphen{}headed, determined and unwavering black men and women back of the N.A.A.C.P. to continue the battle so nobly and successfully be­ gun? Answer, black folk of America, this month!


\bigskip\hrule\bigskip


\sphinxAtStartPar
\sphinxstyleemphasis{Citation:} “The Drive” Editorial. 1922. \sphinxstyleemphasis{The Crisis}. 22(1): 8.


\subsection{Chicago (1927)}
\label{\detokenize{Volumes/34/04/chicago:chicago-1927}}\label{\detokenize{Volumes/34/04/chicago::doc}}
\begin{sphinxShadowBox}
\sphinxstylesidebartitle{}

\sphinxAtStartPar
The \sphinxhref{https://en.wikipedia.org/wiki/1927\_Chicago\_mayoral\_election}{1927 Chicago mayoral election} featured Republican William Thomson running an America First campaign aligned with local organized crime, including Al Capone, against incumbent Democrat William Dever, who emphasized “decency” and anti\sphinxhyphen{}Black racism.
\end{sphinxShadowBox}

\sphinxAtStartPar
The late Chicago election was a serious misfortune. The Democratic mayor had previously made overtures to colored people. He had even spoken at the annual N.A.A.C.P. Conference. Thompson, the Republican, is a well\sphinxhyphen{}known demagogue, who represents open house to gamblers, bootleggers and prostitutes. Wise white Americans would have advised colored people in a campaign of this sort to vote for the Democrats. If they had been permitted, many far\sphinxhyphen{}sighted Negroes would have taken this advice. Indeed the most astute Negro politician in Chicago fought Thompson to the last in the Republican Primary and lost his city patronage when Thompson won.

\sphinxAtStartPar
The silly Democrats of Chicago did not think that the intelligent Negro vote would be as valuable to them as the votes of the Negro haters and the Ku Klux Klan. They therefore flooded the city with anti\sphinxhyphen{}Negro propaganda; they sang “Bye\sphinxhyphen{}Bye, Blackbirds” everywhere. They sent out placards saying: “Don’t vote yourselves N{[}*****{]} wages”; they distributed cartoons of a train loaded with Negroes and the legend: “Big Bill’s express will start for Chicago April 6th unless you stop it April 5th”; and then they asked white voters if they wanted Negro teachers to teach their children; if they wanted Negroes to work in the stock yards and the factories; if they could stand a colored judge, two colored aldermen, a colored Senator and 6 colored representatives. They pointed to 300 colored policemen and 200 colored firemen and other colored civilian employees. They issued dodgers showing Mayor Thompson kissing a black baby and bearing the statement: “Thompson, Africa first”.

\sphinxAtStartPar
What was the result of this campaign? It resulted in bringing in race and national propaganda of other sorts: attacks upon England; appeals to the bitter memories of Germans and the like. But above all, it forced every Negro voter, no matter what his attitude toward Thompson was, to vote for Thompson and against Dever. He did not even dare throw his vote away on a third candidate. He was compelled to deliver himself bound hand and foot to one of the worse representatives of the Republican machine.

\begin{sphinxShadowBox}
\sphinxstylesidebartitle{}

\sphinxAtStartPar
Thompson won \sphinxhref{https://en.wikipedia.org/wiki/1927\_Chicago\_mayoral\_election}{more than 90\%} of the Black vote, which the Associated Press claimed was decisive in his victory.
\end{sphinxShadowBox}

\sphinxAtStartPar
All of which brings us to remark: that for bull\sphinxhyphen{}headed asininity, commend us to the Democratic party; only in New York City and in Tammany Hall does it appear to have glimmerings of common sense.


\bigskip\hrule\bigskip


\sphinxAtStartPar
\sphinxstyleemphasis{Citation:} “Chicago.” 1927. Editorial. \sphinxstyleemphasis{The Crisis} 34(4):131.


\subsection{On the Fence (1928)}
\label{\detokenize{Volumes/35/11/on_the_fence:on-the-fence-1928}}\label{\detokenize{Volumes/35/11/on_the_fence::doc}}
\sphinxAtStartPar
In this presidential campaign \sphinxstylestrong{The Crisis} is sitting squarely on the fence, naked and unashamed and without apology. It is certain that either Herbert Hoover or Al Smith is going to be elected President of the United States, and in our humble opinion, it does not matter a tinker’s damn which of these gentlemen succeed. With minor exceptions, they stand for exactly the same thing: oligarchy in the South; color caste in national office holding; and recognition of the rule of organized wealth. We do not believe that there is a jot or tittle of difference between these two men in their attitude toward these fundamental matters, and we sincerely advise our readers to vote for neither of them.

\sphinxAtStartPar
On the other hand, we repeat advice which we have given many times before: it is of the utmost importance that Negro voters register and vote for Congressmen who are the friends of Industrial Democracy and who do not believe in the color bar. A Congress filled with men of this sort will do much to curb the Bourbon South led by Al Smith, or entrenched plutocracy represented by Herbert Hoover.


\bigskip\hrule\bigskip


\sphinxAtStartPar
\sphinxstyleemphasis{Citation:} “On the Fence.” 1928. Editorial.  35(11):381.


\subsection{A Third Party (1928)}
\label{\detokenize{Volumes/35/11/third_party:a-third-party-1928}}\label{\detokenize{Volumes/35/11/third_party::doc}}
\sphinxAtStartPar
The political theory of the Third Party in a Republican form of Government is that when the two chief parties cease to stand for distinct policies and principles, one of them will disappear, or the two will coalesce, and that a Third Party will arise and become one of the main contenders for the popular vote. In this way, it is assumed that there will always be a real difference of aim and principle between the main political parties. It was thus that the Republican Party arose and triumphed in the election of Abraham Lincoln. In this way the English Liberal Party displaced the Whigs, and the Labor Party now bids fair to displace the Liberals.

\sphinxAtStartPar
Many Americans place their hopes of political reform in the United States on the rise of a Third Party which will register the fact that the present Republican and Democratic parties no longer differ in any essential respect. That both represent the rule of organized wealth, and neither of them has been willing to take radical ground with regard to the tariff, the farmer, labor, or the Negro.

\sphinxAtStartPar
The efforts, however, to organize a Third Party movement have not been successful. ‘The Populists failed. The Socialists failed. The Progressives failed. The Farmer\sphinxhyphen{}Labor movement Many reasons have been advanced for these failures, but by common consent the real effective reason has seldom been discussed and that reason is in the Solid South: the fact is that no party in American politics can disappear if it is sure of 136 electoral votes.

\sphinxAtStartPar
This number of votes the Democratic Party is practically sure of in the Solid South: Virginia, 12; North Carolina, 12; Tennessee, 12; South Carolina, 9; Georgia, 14; Florida, 6; Alabama, 12; Mississippi, 10; Arkansas, 9; Louisiana, 10; Oklahoma, 10; Texas, 20.

\sphinxAtStartPar
There is a possibility that Tennessee’s 12 votes might now and then be cast for the Republican Party and a still slimmer possibility of Oklahoma’s 10, and North Carolina’s 12. For the most part, however, a presidential election in the United States has to do only with 395 of the 531 electoral votes. In order to win an election, a party must carry 266 votes. Any Third Party, therefore, in the United States to be successful would have to find its 266 votes among the 395 votes.

\sphinxAtStartPar
It must receive all but 29 of the electoral votes of the North and West. If it fails to do this, then it not only fails to carry the election, but it throws the election into the hands of the least liberal of the two old parties.

\sphinxAtStartPar
The least liberal party must be the Democratic Party:iy because that party must place its main dependence upon the Solid South. The Solid South has the greatest percentage of illiteracy, the greatest percentage of lynchings and lawlessness, the greatest amount of religious bigotry, the least liberal laws as to labor of men, women and children, and is, in fine, because of its economic history, the most backward part of the whole nation. It does not make any difference how far the Democratic Party of the North may be stirred by liberal leanings, its very dependence upon the Solid South compels it to be a reactionary party. We might, of course, imagine liberal and radical elements among the Democrats making a strong appeal to the party and to the nation, but could we imagine any such political leadership voluntarily relinquishing 136 electoral votes?

\sphinxAtStartPar
Suppose we represent the electoral vote of the United States by the figure 16, and assume that today this power is divided almost evenly between Democrats and Republicans, each with 8 votes. To 4 of these votes, cast by the Democrats and representing the Solid South, a Third Party could make no appeal at all. This would leave 12 votes open to liberal appeal. Assume that the principles of the Third Party are so strong and compelling that they convince half of those voters; that is, half of those open to conviction. What is the result? The result is the triumph of the Democratic Party by a vote of 7 Democrats, 6 Third Party men, and 3 Republicans. This illustrates what is bound to take place as long as there is a Solid South.


\bigskip\hrule\bigskip


\sphinxAtStartPar
\sphinxstyleemphasis{Citation:} “A Third Party.” 1928. Editorial.  35(11):381.


\chapter{NAACP}
\label{\detokenize{Sections/NAACP:naacp}}\label{\detokenize{Sections/NAACP::doc}}
\sphinxAtStartPar
Editorials on the NAACP and their journal, The Crisis.
\begin{itemize}
\item {} 
\sphinxAtStartPar
{\hyperref[\detokenize{Volumes/01/02/NAACP::doc}]{\sphinxcrossref{N.A.A.C.P. (1910)}}}

\item {} 
\sphinxAtStartPar
{\hyperref[\detokenize{Volumes/01/01/TheCrisis::doc}]{\sphinxcrossref{The Crisis (1910)}}}

\item {} 
\sphinxAtStartPar
{\hyperref[\detokenize{Volumes/06/01/vigilance_committee::doc}]{\sphinxcrossref{The Vigilance Committee: A Call To Arms (1913)}}}

\item {} 
\sphinxAtStartPar
{\hyperref[\detokenize{Volumes/07/03/fightordie::doc}]{\sphinxcrossref{Join or Die (1914)}}}

\item {} 
\sphinxAtStartPar
{\hyperref[\detokenize{Volumes/30/01/new_crisis::doc}]{\sphinxcrossref{The New Crisis (1925)}}}

\item {} 
\sphinxAtStartPar
{\hyperref[\detokenize{Volumes/33/03/our_methods::doc}]{\sphinxcrossref{Our Methods (1927)}}}

\item {} 
\sphinxAtStartPar
{\hyperref[\detokenize{Volumes/40/01/toward_a_new_racial_philosophy::doc}]{\sphinxcrossref{Toward a New Racial Philosophy (1933)}}}

\end{itemize}


\section{N.A.A.C.P. (1910)}
\label{\detokenize{Volumes/01/02/NAACP:n-a-a-c-p-1910}}\label{\detokenize{Volumes/01/02/NAACP::doc}}
\sphinxAtStartPar
What is the National Association for the Advancement of Colored People? It is a union of those who believe that earnest, active opposition is the only effective way of meeting the forces of evil. They believe that the growth of race prejudice in the United States is evil. It is not always consciously evil. Much of it is born of ignorance and misapprehension, honest mistake and misguided zeal. However caused, it is none the less evil, wrong, dangerous, fertile of harm. For this reason it must be combatted. It is neither safe nor sane to sit down dumbly before such human error or to seek to combat it with smiles and hushed whispers. Fight the wrong with every human weapon in every civilized way.

\sphinxAtStartPar
The National Association for the Advancement of Colored People is organized to fight the wrong of race prejudice:



\sphinxAtStartPar
This is a large program of reform? It is, and this is because the evil is large. There is not to\sphinxhyphen{}day in human affairs a more subtle and awful enemy of human progress, of peace and sympathy than the reaction war and hatred that lurks in the indefinite thing which we call race prejudice. Does it not call for opposition—determined, persistent opposition? Aree rational beings justified in sitting silently, willingly dumb and blind to the growth of this crime? We believe not. We are organized, then to say to the world and our country:

\sphinxAtStartPar
Negroes are men with the foibles and virtues of men.

\sphinxAtStartPar
To treat evil as though it were good and good as though it were evil is not only wrong but dangerous, since in the end it encourages evil and discourages good.

\sphinxAtStartPar
To treat all Negroes alike is treating evil as good and good as evil.

\sphinxAtStartPar
To draw a crass and dogged undeviated color line in human affairs is dangerous—as dangerous to those who draw it as to those against whom it is drawn.

\sphinxAtStartPar
We are organized to fight this great modern danger. How may we fight it?
\begin{enumerate}
\sphinxsetlistlabels{\arabic}{enumi}{enumii}{}{.}%
\item {} 
\sphinxAtStartPar
By the argument of the printed word in a periodical like this, and in pamphlets and tracts.

\item {} 
\sphinxAtStartPar
By the spoken word in talk and lecture.

\item {} 
\sphinxAtStartPar
By correspondence.

\item {} 
\sphinxAtStartPar
By encouraging all efforts at social uplift.

\item {} 
\sphinxAtStartPar
By careful investigation of the truth in matters of social condition and race contact—not the truth as we want or you want it, but as it really is.

\item {} 
\sphinxAtStartPar
By individual relief of the wretched.

\end{enumerate}

\sphinxAtStartPar
If you think such work is worth while aid us by joining our organization and contributing to it.


\bigskip\hrule\bigskip


\sphinxAtStartPar
\sphinxstyleemphasis{Citation:} Du Bois, W.E.B. 1910. “N.A.A.C.P.”  \sphinxstyleemphasis{The Crisis}. 1(2): 16\sphinxhyphen{}17.


\section{The Crisis (1910)}
\label{\detokenize{Volumes/01/01/TheCrisis:the-crisis-1910}}\label{\detokenize{Volumes/01/01/TheCrisis::doc}}
\sphinxAtStartPar
The object of this publication is to set forth those facts and arguments which show the danger of race prejudice, particularly as manifested to\sphinxhyphen{}day toward colored people. It takes its name from the fact that the editors believe that this is a critical time in the history of the advancement of men. Catholicity and tolerance, reason and forbearance can to\sphinxhyphen{}day make the world\sphinxhyphen{}old dream of human brotherhood approach realization; while bigotry and prejudice, emphasized race consciousness and force can repeat the awful his tory of the contact of nations and groups in the past. We strive for this higher and broader vision of Peace and Good Will.

\sphinxAtStartPar
The policy of THE CRISIS will be simple and well defined:

\sphinxAtStartPar
It will first and foremost be a newspaper: it will record important happenings and movements in the world which bear on the great problem of inter\sphinxhyphen{}racial relations, and especially those which affect the Negro\sphinxhyphen{}American.

\sphinxAtStartPar
Secondly, it will be a review of opinion and literature, recording briefly books, articles, and important expressions of opinion in the white and colored press on the race problem.

\sphinxAtStartPar
Thirdly, it will publish a few short articles.

\sphinxAtStartPar
Finally, its editorial page will stand for the rights of men, irrespective of color or race, for the highest ideals of American democracy, and for reasonable but earnest and persistent attempt to gain these rights and realize these ideals. The magazine will be the organ of no clique or party and will avoid personal rancor of all sorts. In the absence of proof to the contrary it will assume honesty of purpose on the part of all men, North and South, white and black.
\begin{itemize}
\item {} 
\sphinxAtStartPar
Dubois, W.E.B. 1910. “The Crisis.” \sphinxstyleemphasis{The Crisis}. 1(1): 10.

\end{itemize}


\section{The Vigilance Committee: A Call To Arms (1913)}
\label{\detokenize{Volumes/06/01/vigilance_committee:the-vigilance-committee-a-call-to-arms-1913}}\label{\detokenize{Volumes/06/01/vigilance_committee::doc}}
\sphinxAtStartPar
There is scarcely a community in the United States where a group of colored people live that has not its vigilance committee. Sometimes this committee is organized and has a name indicating its function. Sometimes it is organized for other purposes and becomes a vigilance committee on occasion. In other cases the committee has no regular organization or membership; it springs into being on occasion, but consists of approximately the same group of persons from year to year.

\sphinxAtStartPar
The work of these vigilance committees is to protect the colored people in their several communities from aggression.

\sphinxAtStartPar
The aggression takes the form of hostile laws and ordinances. curtailment of civil rights, new racial discriminations, overtax or oversevere enforcement of the law, curtailment of opportunities, etc. Sometimes this aggression is but the careless act of thoughtless folk and needs but a word in season to correct it. More often it is a part of that persistent underground campaign centering largely among white Americans of Southern birth which is determined so to intrench color caste in the United States as to make it impossible for any person of Negro blood to be more than a menial.

\sphinxAtStartPar
Against both sorts of racial aggression organized effort is necessary. Many thoughtful colored people have sought to avoid this; to act independently and to refuse to meet organization by organization. This in most cases has been found impossible. The blows of racial and color prejudice fall on all alike, rich and poor, educated and ignorant, and all must stand together and fight.

\sphinxAtStartPar
The methods of these vigilance committees are various. The simplest action is the appointment of a committee of one or more to call on some official or person of influence; from this, action extends to letters and the press, pamphlets, legislative hearings, mass meetings, petitions, etc. In a few cases threats and violence have been attempted, but these are at present exceptional.

\sphinxAtStartPar
From this procedure on the part of tens of thousands of largely isolated groups much actual good has been done and much experience accumulated.

\sphinxAtStartPar
The time is now evidently at hand to fund and pool this nation\sphinxhyphen{}wide experience, and to systematize this scattered local effort into steady, persistent and unwavering pressure. As it is, unorganized local effort loses much time and energy in reorganizing for every new object: organized local efforts lack experience and knowledge of similar action elsewhere. Henceforth we must act together and we must fight continuously.

\sphinxAtStartPar
The object of the National Association for the Advancement of Colored People is to federate local vigilance committees among colored people in every community in the United States; to coordinate their activities, to exchange experiences and to concentrate the application of funds where the need is greatest.

\sphinxAtStartPar
Hitherto we have spoken almost exclusively of the central office of the association in New York and its work. The central office is now permanently established, with executive officers and an organ of publicity read by at least 150,000 of the most intelligent colored people in the land.

\sphinxAtStartPar
We are now turning our attention to the branches in order, on the one hand, to build a sure foundation and support for the national body and, on the other hand, chiefly to federate and organize the local battle against race prejudice.

\sphinxAtStartPar
What then is a local branch of the National Association for the Advancement of Colored People?

\sphinxAtStartPar
It is an organization of men and women, white and colored, working in a given locality for the overthrow of race prejudice and color caste. It should have, among other things, twelve principal functions:
\begin{enumerate}
\sphinxsetlistlabels{\arabic}{enumi}{enumii}{}{.}%
\item {} 
\sphinxAtStartPar
\sphinxstyleemphasis{Legislation.} It should watch hostile and discriminating legislation, hostile and discriminating administration of the law and injustice in the courts.

\item {} 
\sphinxAtStartPar
\sphinxstyleemphasis{Discrimination.} It should note the barometer of racial discrimination and see that it does not fall a single degree in the matters of civil rights, in parks, museums, theatres and places of public accommodation and amusement. It should note new efforts at discrimination, have them systematically and promptly reported. It should note old habits of discrimination and have them wisely but persistently opposed. It should note the action of the police and discrimination in charitable and settlement work.

\item {} 
\sphinxAtStartPar
\sphinxstyleemphasis{Legal Redress.} It should see that good test cases of the rights of Negro citizens are brought before the courts and strong decisions obtained.

\item {} 
\sphinxAtStartPar
\sphinxstyleemphasis{Laws.} It should seek to secure new laws and ordinances to protect the lives and property of citizens and to prevent race discrimination. In cases where race discrimination is too strongly intrenched to be attacked at present it should secure at least equal rights and accommodations for colored citizens.

\item {} 
\sphinxAtStartPar
\sphinxstyleemphasis{Education.} It should see that every colored child between the ages of 5 and 14 years is in school; that the largest possible number of colored children finish the high school; that every colored boy and girl who shows good ability goes to a good college; that careful technical training in some branch of modern industry is furnished all colored children.
It should see that libraries, museums, ete., are open to colored folk and that they use them.

\item {} 
\sphinxAtStartPar
\sphinxstyleemphasis{Health.} It should conserve the health and healthful habits of the colored people, particularly in the matters of fresh air, sensible clothing, good food and healthy amusement.

\item {} 
\sphinxAtStartPar
\sphinxstyleemphasis{Occupations.} It should see that the colored youth have a larger opportunity for employment at better wages than now; that they have a chance of promotion according to merit; that they are urged into new and higher avenues of endeavor, especially in lines of literature, pictorial art, music, business, executive work and skilled labor of the higher sorts and scientific farming.
It should see that co\sphinxhyphen{}operative effort to furnish capital is encouraged and wise investments extended and guarded, and capital put at the disposal of the honest and efficient.

\item {} 
\sphinxAtStartPar
\sphinxstyleemphasis{Co\sphinxhyphen{}operation.} It should endeavor to co\sphinxhyphen{}operate and advise with all general philanthropic effort and have the colored people represented on boards of control.

\item {} 
\sphinxAtStartPar
\sphinxstyleemphasis{Publicity and Information.} It should stop the conscious and unconscious enmity of the daily and weekly press and seek to abate scurrilous headlines and contemptuous and belittling reports; it should send letters to newspapers. answer attacks, visit the editors; furnish the papers with news of events; give facilities to reporters to see the best and follow them up; it should see that the cause of the Negro is represented on all public occasions; it should send lecturers and lantern slides to clubs and meetings, etc.
It should publish pamphlets and distribute them and use every opportunity to make the Negro church a vehicle of uplifting information to Negroes.

\item {} 
\sphinxAtStartPar
\sphinxstyleemphasis{Racial Contact.} It should use every opportunity to bring the best representatives of both races into helpful and enlightening contact; it should bring white lecturers on all subjects to colored audiences and colored lecturers to white audiences; it should arrange conferences.

\item {} 
\sphinxAtStartPar
\sphinxstyleemphasis{Political Action.} It should see that colored people qualify as voters according to law and vote intelligently at every election. It should keep the records of legislators and Congressmen on racial discrimination and publish the record before each election with such promises for the future as can be obtained. It should discourage and expose bribery, and support only the best qualified candidates, black and white.

\item {} 
\sphinxAtStartPar
\sphinxstyleemphasis{Meetings.} The branch should have an executive committee, which should meet regularly at least once a month for reports and plan of campaign. It should have a secretary with an office open each day. It should arrange at least four times a year large meetings of members and friends for lectures, reports and protests.

\end{enumerate}

\sphinxAtStartPar
In fine, the local branch should try, in each community, North and South, East and West, to solve the Negro problem in that community by making the injustice of discrimination clear to all, and the need of equal opportunity plain to the most prejudiced.

\sphinxAtStartPar
Finally, let the locals support the National Association for the Advancement of Colored People. We have 1,500 members. We want 10,000 members who will contribute at least a dollar apiece to the national body in order to effect the second and final emancipation of the Negro in America; in order that the national body may become a great clearing house for information and experience, and be able to concentrate money and help on particular plague spots of prejudice.

\sphinxAtStartPar
We have ten branches which are beginning work as outlined above. Who will be the next?


\bigskip\hrule\bigskip


\sphinxAtStartPar
\sphinxstyleemphasis{Citation:} Du Bois, W.E.B. 1913. “The Vigilance Committee: A Call To Arms.” \sphinxstyleemphasis{The Crisis}. 6(1): 26\sphinxhyphen{}29.


\section{Join or Die (1914)}
\label{\detokenize{Volumes/07/03/fightordie:join-or-die-1914}}\label{\detokenize{Volumes/07/03/fightordie::doc}}
\sphinxAtStartPar
THE CRISIS exists for one great purpose.

\sphinxAtStartPar
THE CRISIS seeks to entertain its readers, and entertainment is a legitimate object for a popular magazine. Our pictures, our stories and our ornamentation are for entertainment. But the object of THE CRISIS is not simply to entertain.

\sphinxAtStartPar
THE CRISIS seeks to inform. It is in the large sense a newspaper. It deals not so much with immediate news as with forgotten, neglected and concealed news. This is its great and notable
function, and gives THE CRISIS a peculiar place among American periodicals.

\sphinxAtStartPar
But THE CRISIS is not merely an entertaining newspaper.

\sphinxAtStartPar
THE CRISIS entertains and informs its readers for the one object and the sole object of arousing their fighting blood. THE CRISIS means fight and fight for Right.

\sphinxAtStartPar
More than that, THE CRISIS stands for a definite method of fighting. THE CRISIS does not simply protest or simply tell the unpalatable truth or simply cry fight wildly and crazily.

\sphinxAtStartPar
THE CRISIS cries fight and adds: Here are weapons, and here is the battle line:

\sphinxAtStartPar
Join or die!

\sphinxAtStartPar
What are the weapons? Organized publicity, advertisement, public meetings, petitions, arrest, law suits, protest, investigation, research , resistance—every way in which the civilized world has fought and must fight and will fight wrong.

\sphinxAtStartPar
Where is the battle line?

\sphinxAtStartPar
The battle line is the NATIONAL ASSOCIATION FOR THE ADVANCEMENT OF COLORED PEOPLE.

\sphinxAtStartPar
There are individual sharpshooters fighting their own effective guerrilla warfare. We greet them and give them all credit.

\sphinxAtStartPar
There are a few organizations here and there with some activity. We would not detract a moment from the value of their work.

\sphinxAtStartPar
But the flat fact remains:

\sphinxAtStartPar
There is in the United States but one organization with permanent headquarters, paid officials, active nation­wide membership, live local branches, a national organ, law officers and traveling organizers, all organized and prepared to make a front forward fight on racial prejudice in this land.
Here is the organization.

\sphinxAtStartPar
Here is the work to do.

\sphinxAtStartPar
Here are twenty local branches and over 2,100 members already banded for the warfare.
What are you doing?

\sphinxAtStartPar
Join or die!

\sphinxAtStartPar
Join the NATIONAL ASSOCIATION FOR THE ADVANCEMENT OF COLORED PEOPLE or be strangled to a slow and awful death by growing prejudice.

\sphinxAtStartPar
There is no excuse.

\sphinxAtStartPar
The cost? It costs but a dollar.

\sphinxAtStartPar
The need? We need 200,000 members instead of a paltry 2,000.

\sphinxAtStartPar
The use? Merciful God! does a drowning man ask the USE of struggles? No, he struggles, he fights. He wins the shore or dies fighting.

\sphinxAtStartPar
Black men of America, are you men? Dare you fight?

\sphinxAtStartPar
Join us and fight, then.

\sphinxAtStartPar
Join or die!


\bigskip\hrule\bigskip


\sphinxAtStartPar
\sphinxstyleemphasis{Citation:} “Fight or Die” Editorial. 1914. \sphinxstyleemphasis{The Crisis}. 7(3): 133\sphinxhyphen{}134.

\sphinxAtStartPar
133\sphinxhyphen{}134


\section{The New Crisis (1925)}
\label{\detokenize{Volumes/30/01/new_crisis:the-new-crisis-1925}}\label{\detokenize{Volumes/30/01/new_crisis::doc}}
\sphinxAtStartPar
We have assumed, with the Spring, with the beginning  of our 30th semi\sphinxhyphen{}annual volume, with our 175th number and with the closing of a fateful quarter century, something of a new dress and a certain renewal of spirit.

\sphinxAtStartPar
How long may a \sphinxstylestrong{Crisis} last? one might ask, sensing between our name and age some contradiction. To which we answer: What is long? 15 or 5000 years? But even in 15 years we see curious and suggestive change. In November, 1910, we wrote:
\begin{quote}

\sphinxAtStartPar
“The object of this publication is to set forth those facts and arguments which show the danger of race prejudice, particularly as manifested today toward colored people. It takes its name from the fact that the editors believe that this is a critical time in the history of the advancement of men. Catholicity and tolerance. reason and forbearance can today make the world\sphinxhyphen{}old dream of human brotherhood approach realization; while bigotry and prejudice, emphasized race consciousness and force can repeat the awful history of the contact of nations and groups in the past. We strive for this higher and broader vision of Peace and Good Will.”
\end{quote}

\sphinxAtStartPar
Then we set forth the plan to make \sphinxstylestrong{The Crisis} (1) a newspaper, (2) a review of opinion, (3) a magazine with “a few short articles”. This initial program has unfolded itself, changed and developed. There is no longer need of a monthly newspaper for colored folk. Colored weeklies have arisen with an efficiency and scope in news\sphinxhyphen{}gathering that was not dreamed of in 1910. Our news therefore has transformed itself into a sort of permanent record of a few matters of widespread and historic importance. Our review of opinion continues in both “Opinion” and “Looking Glass”, but rather as interpretation than as mere quotation. Particularly has our policy changed as to articles. They have increased in number, length and authority. And above all, out of the broad vagueness of our general policy have emerged certain definite matters which we shall pursue with increased earnestness. We name them in something like the order in which they appeal to us now:
\begin{enumerate}
\sphinxsetlistlabels{\arabic}{enumi}{enumii}{}{.}%
\item {} 
\sphinxAtStartPar
Economic Development

\sphinxAtStartPar
At Philadelphia, the N.A.A.C.P. made a suggestion of alliance among the laboring people of the United States across the color line. The American Federation of Labor has as yet made no active response to our overtures. Meantime, however, we are not waiting and we \sphinxhyphen{}propose to make a crusade in \sphinxstylestrong{The Crisis} covering the next three years and taking up in succession the history and significance of the Labor Movement in the modern world, the present actual relation of Negroes to labor unions and a practical plan of future cooperation.

\item {} 
\sphinxAtStartPar
Political Independence

\sphinxAtStartPar
We shall stress as never before political independence. No longer must Negroes be born into the Republican Party. If they vote the Republican ticket or any other ticket it must be because the candidates of that party in any given election make the best promises for the future and show the best record in the past. Above all we shall urge all Negroes, male and female, to register and vote and to study political ethics and machinery.

\item {} 
\sphinxAtStartPar
Education and Talent

\sphinxAtStartPar
We shall stress the education of Negro youth and the discovery of Negro talent. Our schools must be emancipated from the secret domination of the Bourbon white South. Teachers, white or black, in Negro schools who cannot receive and treat their pupils as social equals must go. We must develop brains, ambition, efficiency and ideals without limit or circumscription. If our own Southern colleges will not do this, and whether they do it or not, we must  continue to force our way into Northern colleges in larger and larger numbers and to club their doors open with our votes. We must provide larger scholarship funds to support Negroes of talent here and abroad.

\item {} 
\sphinxAtStartPar
Art

\sphinxAtStartPar
We shall stress Beauty—all Beauty, but especially the beauty of Negro life and character; its music, its dancing, its drawing and painting and the new birth of its literature. This growth which THE CRISIS long since predicted is sprouting and coming to flower. We shall encourage it in every way—by reproduction, by publication, by personal mention— keeping the while a high standard of merit and stooping never to cheap flattery and misspent kindliness.

\item {} 
\sphinxAtStartPar
Peace and International Understanding

\sphinxAtStartPar
Through the Pan\sphinxhyphen{}African movement we shall press for better knowledge of each other by groups of the peoples of African descent; we shall seek wider understanding with the brown and yellow peoples of the world and thus, by the combined impact of an appeal to decency and humanity from the oppressed and insulted to those fairer races who today accidentally rule the world, we shall seek universal peace by abolishing the rivalries and hatreds and economic competition that lead to organized murder.

\item {} 
\sphinxAtStartPar
The Church

\sphinxAtStartPar
We shall recognize and stress the fact that the American Negro church is doing the greatest work in social uplift of any present agency. We criticise our churches bitterly and in these plaints \sphinxstylestrong{The Crisis} has often joined. At the same time we know that without the help of the Negro church neither the N.A.A.C.P. nor \sphinxstylestrong{The Crisis} could have come into being nor could they for a single day continue to exist. Despite an outworn creed and ancient methods of worship the black church is leading the religious world in real human brotherhood, in personal charity, in social uplift and in economic teaching. No such tremendous force can be neglected or ignored by a journal which seeks to portray and expound the truth. We shall essay, then, the contradictory task of showing month by month the accomplishment of black religious organization in America and at the same time seeking to free the minds of our people from the futile dogma that makes for unreason and intolerance.

\item {} 
\sphinxAtStartPar
Self\sphinxhyphen{}criticism

\sphinxAtStartPar
\sphinxstylestrong{The Crisis} is going to be more frankly critical of the Negro group. In our fight for the sheer crumbs of decent treatment we have become habituated to regarding ourselves as always right and resenting criticism from whites and furiously opposing self\sphinxhyphen{}criticism from within. We are seriously crippling Negro art and literature by refusing to contemplate any but handsome heroes, unblemished heroines and flawless defenders; we insist on being always and everywhere all right and often we ruin our cause by claiming too much and admitting no fault. Here \sphinxstylestrong{The Crisis} has sinned with its group and it purposes hereafter to examine from time to time judicially the extraordinary number of very human faults among us—both those common to mankind and those born of our extraordinary history and experiences.

\item {} 
\sphinxAtStartPar
Criticism

\sphinxAtStartPar
This does not mean that we propose for a single issue to cease playing the gadfly to the Bourbon South and the Copperhead North, to hypocritical Philanthropy and fraudulent Science, to race hate and human degradation.

\end{enumerate}

\sphinxAtStartPar
All this, we admit, is an enormous task for a magazine of 52 pages, selling for 15 cents and paying all of its own expenses out of that 15 cents and not out of the bribes of Big Business.

\sphinxAtStartPar
We shall probably fall far short of its well doing but we shall make the attempt in all seriousness and good will. And, Good Reader, what will you do? Write and tell us.


\bigskip\hrule\bigskip


\sphinxAtStartPar
\sphinxstyleemphasis{Citation:} “The New Crisis.” 1925. Editorial.  30(1):7\sphinxhyphen{}9.


\section{Our Methods (1927)}
\label{\detokenize{Volumes/33/03/our_methods:our-methods-1927}}\label{\detokenize{Volumes/33/03/our_methods::doc}}
\sphinxAtStartPar
Some assiduous friends have recalled the remark of a white secretary when he left the N.A.A.C.P. voicing serious doubts as to our methods. There has long been and long will be controversy over the stand which the N.A.A.C.P. has taken. There are those who still believe that our rights as American citizens can be won by cajoling and kow\sphinxhyphen{}towing and by the various methods of the so called “white\sphinxhyphen{}folks n*****”.

\sphinxAtStartPar
It is not, on the other hand, to be supposed that the N.A.A.C.P. is doing aimless and ill\sphinxhyphen{}considered fighting. First of all, it is untrue that our officers reside in a “safety zone” and “dictate” to the black South. Walter White did not dictate to Aiken from New York. He went to Aiken and risked life and limb in so doing. During the present year our officers have visited and spoken in Texas, Louisiana, North Carolina, South Carolina, Virginia, Georgia, Alabama, Arkansas, Mississippi, New Mexico, Arizona, Tennessee, West Virginia and Missouri. During the last five years there is not a single Southern state that our Secretaries have not visited and most of them repeatedly. Not only that, but the N.A.A.C.P. took its whole paraphernalia and went down to Georgia in 1920 and held its annual conference. And it said in Atlanta exactly the same sort of thing which it said last year in Chicago. The editor of \sphinxstylestrong{The Crisis} in the last ten years has lectured in Virginia, North and South Carolina, Georgia, Florida, Kentucky, Tennessee, Alabama, Texas, Louisiana, Arkansas and Oklahoma.

\sphinxAtStartPar
The N.A.A.C.P. has always cooperated with other agencies, be they radical or conservative, white or black, Northern or Southern. It has, for instance, repeatedly cooperated with Monroe Trotter and the Equal Rights League, financially and otherwise.

\sphinxAtStartPar
The N.A.A.C.P. has attacked the white people of the South and as long as they are responsible for lynching and disfranchisement and Jim Crowism, it is going to continue to attack them. But far from the fact that its outspoken methods have made the relations between the races worse in the South, there is testimony from the South itself that race relations have been steadily improving since 1910 when we began our work; and we maintain that they have been improving because the Negroes have been increasing and reiterating their demands for treatment as men.

\sphinxAtStartPar
Again we have been accused of selecting “notorious” cases for defense. This is putting the cart before the horse. Cases become notorious because we select them. Aiken got on the front pages of the newspapers because we put it there. The Sweet case never would have been heard of by most of the country if we had not hired Clarence Darrow and flooded the land with propaganda. The East St. Louis riots would have been hushed up if America had been able to hush up the N.A.A.C.P.

\sphinxAtStartPar
The great majority of the persons we have defended have been ordinary working people like the twelve peons in Arkansas; but we have never refused help just because the victim was educated, decent and practicing a profession.

\sphinxAtStartPar
It has been charged that the management of the N.A.A.C.P. is undemocratic. If by this is meant that it is impossible for any temporary mob to turn our work up\sphinxhyphen{}side\sphinxhyphen{}down in fifteen minutes of yelling, this is perfectly true. On the other hand, each member of this organization has a full and effective voice in the conduct of the organization. The members of the Board of Directors are elected at an annual meeting at which every member has a right to vote. Nominations to the Board can be made by anybody who wishes. The Board of Directors thus elected appoints the Executive Officers. This is the method pursued by practically every organization which is permanently effective.

\sphinxAtStartPar
Finally some people who find pleasure and profit in opposing the officers of the N.A.A.C.P. get very much up\sphinxhyphen{}set if these same officers answer attacks in words just as sharp as those of the attacking parties; but surely all this is a matter of taste. If a man calls me a thief there are various kinds of answers recommended by human experience. But certainly pained surprise is the last attitude for the accuser to take if I reply that the gentleman looks to me distinctly like a liar. The facts of the case however are what the public is really interested in and these facts are clear.


\bigskip\hrule\bigskip


\sphinxAtStartPar
\sphinxstyleemphasis{Citation:} Du Bois, W.E.B. 1927. “Our Methods.”  33(3):129\sphinxhyphen{}130.


\section{Toward a New Racial Philosophy (1933)}
\label{\detokenize{Volumes/40/01/toward_a_new_racial_philosophy:toward-a-new-racial-philosophy-1933}}\label{\detokenize{Volumes/40/01/toward_a_new_racial_philosophy::doc}}
\sphinxAtStartPar
A college graduate came to me yesterday and asked: “What has the N.A.A.C.P. published concerning the present problems of the Negro, and especially of young Negroes just out of college?”

\sphinxAtStartPar
I started to answer with stereotyped remarks; and then I said suddenly, “Nothing.”

\sphinxAtStartPar
The N.A.A.C.P., beginning nearly a quarter of a century ago, formulated on the basis of the problems which then faced Negroes, a clear\sphinxhyphen{}cut and definite program. This program we have followed ever since with unusual success for it has expressed during these years the aspirations and lines of effort among 12 millions and their friends.

\sphinxAtStartPar
Today the situation has changed enormously in its trend, objects and details, and there is both need and widespread demand for a re\sphinxhyphen{}examination of what is called the Negro problem from the point of view of the middle of the 20th Century. \sphinxstylestrong{The Crisis} realizes this and it proposes during 1933 to discuss the present Negro problems from 12 points of view. Tentatively, these points seem to us now to be something as follows, although we may change many of them:
\begin{enumerate}
\sphinxsetlistlabels{\arabic}{enumi}{enumii}{}{.}%
\item {} 
\sphinxAtStartPar
\sphinxstyleemphasis{Birth}. The physical survival of the Negro in America is discussed in this number.

\item {} 
\sphinxAtStartPar
\sphinxstyleemphasis{Health}. In February, we shall ask: How can the infant mortality, the great loss from sickness and the general death rate among Negroes, be lowered? What is the duty of colored physicians toward  Negro health? What is the extent of available hospitalization? How can we extend the life term and meet such enemies of our race as tuberculosis, pneumonia, syphilis and cancer.

Later, we shall treat our problems in something like the following order:


\item {} 
\sphinxAtStartPar
\sphinxstyleemphasis{The Home}. Should a Negro family live in the city or in the country, in the North or in the South, in a single house or in an apartment? And on what facts should an individual family base its decision? How can housework be reduced and systematized? Must paid household help be an ideal? Should Negroes seek to live in their own neighborhoods or in white neighborhoods?

\item {} 
\sphinxAtStartPar
\sphinxstyleemphasis{Occupations}. What kinds of work do Negroes want to do and what kind can they do and what kinds are they allowed to do? How far shall they be farmers, artisans, artists, professional men, merchants or financiers? What is their relation to the great economic and industrial changes going on now in the world of work? How far are they being displaced by machines and technique and by new organizations of capital? Can they achieve a place of power and efficiency in the present oligarchy of white capital or in the present labor union or in any future industrial democracy?

\item {} 
\sphinxAtStartPar
\sphinxstyleemphasis{Education}. Should a colored child be sent to a white school or to a colored school? Should it be educated in the North or South or even abroad? What should be the kind and aim and length of its education? Should the boy or girl go to college? Should they go to technical school? Should they go to professional school? Should they be apprenticed to manual work? Should they depend for their education upon experience? How far is their education dependent upon contact with the white group and its larger opportunities? How shall the cost of education be met?

\item {} 
\sphinxAtStartPar
\sphinxstyleemphasis{Income}. What should be the ideal and standards of living among colored people? How much income must the average colored family have and how should it be spent? How early can our young people get married? What things must they regard as luxuries and beyond them and what as necessities? Shall we aim to be rich or make poverty an ideal?

\item {} 
\sphinxAtStartPar
\sphinxstyleemphasis{Discrimination}. Accepting race and color discrimination as a fact which despite all effort ‘s bound to last in some form at least through this generation, if not longer, what shall be our general attitude toward it? How can we avoid in its face an inferiority complex? How far must we be belligerent or acquiescent? Can we meet discrimination by ignoring it or by fighting it? When shall we fight and how shall we fight, and what is the cost of effective fighting? What types of organized effort are needed in this fighting? How much of co\sphinxhyphen{}operation with the whites must be sought or accepted? How far must we be willing recipients of white co\sphinxhyphen{}operation, philanthropy and charity? How can we escape discrimination by emigration to other states, countries or continents ?

\item {} 
\sphinxAtStartPar
\sphinxstyleemphasis{Government and Law}. How far must we be obedient to government and law when they are unfair to us? Can we adopt an attitude of defiance? How can we change our legal status or reform evil administration? Is revolution by force advisable, possible or probable? Can we use our right to vote, curtailed as it is, for our emancipation and change of status? Should we vote for the Republican Party or the Democratic Party; or for Socialists or Communists? Or should we adopt a method of independent opportunism in our voting? In states and cases where we are disfranchised, what shall we do about it? Is it worth while to register, even though we cannot vote or cannot vote effectively? Should we join the Democratic White Primary in the South? Should we in our voting keep our racial needs and demands in mind, or should we have an eye upon the good of the majority of the nation? And what shall we do in case these two ideals clash? How far are we criminal and antisocial? What causes and what can prevent crime? How are our criminals treated?

\item {} 
\sphinxAtStartPar
\sphinxstyleemphasis{Race Pride}. How far shall American Negroes remember and preserve their history, keep track of their ancestry, build up a racial literature and a group patriotism? What does loyalty to the race mean? How far shall we have distinctively race organizations, and how far shall we seek to join organizations regardless of race? If we do have organizations, what sorts are needed for the various ends we have in view? How far do present organizations fill our need or how shall they be changed and what new ones must come? Shall we imitate and duplicate, on our side of the color line, the organizations that white folk have? Especially, of what use are secret fraternities and how can their functioning be made to help our general uplift? What is our relation to Africa, to the West Indies, to Asia, to the colored world in general? Is it profitable or advisable consciously to build up by race pride a nation within a nation, or races within the world?

\item {} 
\sphinxAtStartPar
\sphinxstyleemphasis{Religion}. We are by tradition a religious people and the “old\sphinxhyphen{}time religion” still has a strong hold on our masses. The present Negro church more nearly represents the mass of people than any other organization and its ministers are its spokesmen. Nevertheless, the number of colored people who do not go to church is large and growing. Is this right, and if not, what is the remedy? What should be the function of the Negro church? How far should there be churches divided along the color line? What is the present status of creeds? How far do we dare disturb simple religious faith, “evangelical” dogma and ordinary religious superstition? Should a man join a church and work with it if he does not believe completely in its dogma? Can the Negro church be made a center and unit of racial and cultural and social development? Will Creed and Culture, Reason and Faith, Science and Superstition clash as in other groups and ages? What should be the attitude of the Negro church toward white churches? How far should they co\sphinxhyphen{}operate in missions and philanthropy? How far is the white church, with its greater wealth and experience, pauperizing certain colored churches? What is the remedy? Will Christianity abolish or emphasize the Color Line? How can we, with or without religion, encourage courtesy, honor, unselfishness, sacrifice, self restraint, the ideals of the higher spiritual life, the recognition of beauty in Art and deed?

\item {} 
\sphinxAtStartPar
\sphinxstyleemphasis{Social Contacts}. Is the method of advance among colored people today a building up of social classes so that the educated, the rich, the well\sphinxhyphen{}to\sphinxhyphen{}do and the moral can separate themselves from the poor, the ignorant and the criminal? Can this class\sphinxhyphen{}building technique of civilization be ignored in our case and something better substituted? How far must colored people try to accumulate wealth and become the employers of other Negroes and even of whites? Must we have a bourgeoisie for defense in a bourgeois world? What should be our social standards in marriage? Should we encourage our children to inter\sphinxhyphen{}marry with white folk, or at any rate, to increase the social contacts between colored and white people with the ultimate ideal of marriage? And if not, where can the line be drawn? In the absence of social contact with more favored persons and races, is it possible for culture among us to grow or to grow as fast? Should we demand and practice social equality? Should we regard the development of lower masses among our people as inevitable, and if so, what should be our attitude toward these masses? What should be our attitude toward social questions beyond our own racial orders, toward world problems of peace and war, the labor movement, the status of women, education, health, social and economic reform?

\item {} 
\sphinxAtStartPar
\sphinxstyleemphasis{Recreation}. How can we get the relaxation of play and recreation without having it spoiled by discrimination along the color line? Should we travel, and if we travel, should we seek or avoid white people? Where can we spend our vacations? Where can we bathe in the ocean unmolested and not insulted? What should be our attitude toward discrimination in transport, railroad trains, buses and hotels? Where shall we be willing to sit in theatres and at concerts? If we seek recreation only in those places where there is no discrimination, will that help uplift by increasing our pleasure or will it encourage the growth of further discrimination?

\end{enumerate}

\sphinxAtStartPar
The above is a tentative outline of 12 sets of problems. Further reflection will doubtless change and add to them. There are doubtless important omissions. \sphinxstylestrong{The Crisis} would welcome from readers suggestions as to these heads.

\sphinxAtStartPar
In the meantime, it is our present plan to publish a discussion of one of these subjects in each of the next 12 numbers \sphinxstylestrong{The Crisis}, which will exhibit different points of view concerning the main problems suggested. We would be glad to have contributions or suggestions as to persons who might contribute to these symposiums. Of necessity, the contributions must be terse and to the point, and, of course, we cannot publish all. Beside the editorial statement of the problem, and papers discussing two or more sides of the problem, we are going to try to get hold of 12 pieces of fiction which will illustrate the problems humanly. We admit it is going to be a hard thing to do this.

\sphinxAtStartPar
What we want is suggested by the story this month—“The Three Mosquitoes.” Do you like it? Is it worth while? Such stories must be short,— not more than two pages of \sphinxstylestrong{The Crisis},— and they must illustrate the difficulties and contradictions of each of these 12 suggested matters of thought. They must not be “defeatist;” we want them to be artistic; we want them true: but we do not propose to have every story end in a lynching or a suicide, for the simple reason that we do not believe that death is the necessary answer to any of these situations.

\sphinxAtStartPar
Finally, we welcome from everybody, terse, definite and pointed opinions, which we shall try to reflect and quote.


\bigskip\hrule\bigskip


\sphinxAtStartPar
\sphinxstyleemphasis{Citation:} Du Bois, W.E.B. 1933. “Toward a New Racial Philosophy”  40(1):20\sphinxhyphen{}22.


\chapter{Lynching}
\label{\detokenize{Sections/lynching:lynching}}\label{\detokenize{Sections/lynching::doc}}
\sphinxAtStartPar
Editorials on lynching and anti\sphinxhyphen{}lynching legislation.
\begin{itemize}
\item {} 
\sphinxAtStartPar
{\hyperref[\detokenize{Volumes/02/04/lynching::doc}]{\sphinxcrossref{Lynching (1911)}}}

\item {} 
\sphinxAtStartPar
{\hyperref[\detokenize{Volumes/08/03/cause_of_lynching::doc}]{\sphinxcrossref{The Cause of Lynching (1914)}}}

\item {} 
\sphinxAtStartPar
{\hyperref[\detokenize{Volumes/12/06/cowardice::doc}]{\sphinxcrossref{Cowardice (1916)}}}

\item {} 
\sphinxAtStartPar
{\hyperref[\detokenize{Volumes/22/01/anti-lynching_legislation::doc}]{\sphinxcrossref{Anti\sphinxhyphen{}Lynching Legislation (1921)}}}

\item {} 
\sphinxAtStartPar
{\hyperref[\detokenize{Volumes/26/02/university_course_in_lynching::doc}]{\sphinxcrossref{A University Course in Lynching (1923)}}}

\item {} 
\sphinxAtStartPar
{\hyperref[\detokenize{Volumes/32/01/lynching::doc}]{\sphinxcrossref{Lynching (1926)}}}

\item {} 
\sphinxAtStartPar
{\hyperref[\detokenize{Volumes/34/01/aiken::doc}]{\sphinxcrossref{Aiken (1927)}}}

\item {} 
\sphinxAtStartPar
{\hyperref[\detokenize{Volumes/33/04/lynching::doc}]{\sphinxcrossref{Lynching (1927)}}}

\item {} 
\sphinxAtStartPar
{\hyperref[\detokenize{Volumes/38/04/causes_of_lynching::doc}]{\sphinxcrossref{Causes of Lynching (1931)}}}

\end{itemize}


\section{Lynching (1911)}
\label{\detokenize{Volumes/02/04/lynching:lynching-1911}}\label{\detokenize{Volumes/02/04/lynching::doc}}
\sphinxAtStartPar
The mob spirit in America is far from dead. Time and time again the disappearance of lynching has been confidently announced. Still this species of murder and lawlessness flourishes blithely. Its sickening details in the last few weeks have been as bad as could be imagined. The cause of this is obvious: a disrespect for law and a growing cheapness of human life. Why should America lose respect for law? Because for years some of its best brains have been striving both in the profession of law and on the bench to show how worthless legislation is and help­ less to accomplish its ends. To cite an instance: The Constitution of the United States, the highest law of the land, says that citizens of the United States cannot be disfranchised on ac­ count of race or color. Yet every schoolboy knows that Negro Americans are disfranchised in large areas of the South for no other reason than race and color. This is but one in­ stance of our laughing at law.

\sphinxAtStartPar
Why should America count human life cheap? Because it is cheap. Because it is difficult to punish a rich murderer and extremely difficult for a black suspect to escape lynching. Back of the despising of life lies the con­ tempt for men who live. They are not ends, but means— “hands” for doing my work, “masses” for me to contemplate, “n{[}******{]}” for me to keep down. Their lives, their hurts, their thoughts and aspirations—what is that to me as long as I live and enjoy and rise? Shall my race be disturbed, my fortune taxed, my world turned upside down because six black men in Florida are murdered or a woman and a child hanged by ruffians at Oklahoma? Nonsense. They are not worth it. They can be bought for fifty cents a day. Thus we despise life.

\sphinxAtStartPar
The result is mob and murder. The result is barbarism and cruelty. The result is human hatred. Come, Americans who love America, is it not time to rub our eyes and awake and act?


\bigskip\hrule\bigskip


\sphinxAtStartPar
\sphinxstyleemphasis{Citation:} Du Bois, W.E.B. 1911. “Lynching.”  \sphinxstyleemphasis{The Crisis}. 2(4): 158\sphinxhyphen{}159.


\section{The Cause of Lynching (1914)}
\label{\detokenize{Volumes/08/03/cause_of_lynching:the-cause-of-lynching-1914}}\label{\detokenize{Volumes/08/03/cause_of_lynching::doc}}
\sphinxAtStartPar
It is exceedingly difficult to get at the real cause of lynching but \sphinxstylestrong{The Crisis} is more and more convinced that the real cause is seldom the one In the barbaric Oklahoma case of the lynching of a woman, the press despatches made it a quarrel in a “redlight” district, but two private letters in our hands from apparently trustworthy persons declare that it was the case of a seventeen\sphinxhyphen{}year\sphinxhyphen{}old girl defending her own honor.

\sphinxAtStartPar
From Shreveport there are newspaper accounts of a horrible lynching of a Negro boy for attacking a ten\sphinxhyphen{}year\sphinxhyphen{}old child. But again a private letter tells us that the girl was old enough to be ticket seller in a theatre; that she was not injured in the slightest degree, but was found “hale and hearty and singing” the day after; and that, as a matter of fact, the boy was lynched because of his relations with another and older white woman.

\sphinxAtStartPar
We have no way of proving these assertions; but they have many ear\sphinxhyphen{}marks of truth and their very assertion is an astounding indictment of modern American barbarism. \sphinxstylestrong{The Crisis} knows that Negroes are human and it does not for a moment presume that every Negro accused of a horrible crime is innocent. It wants, and wants for the sake of colored people even more than of others, that colored criminals be treated so as to decrease crime, whatever that treatment may be. It is painfully significant that of all methods of suppressing crime lynching has certainly failed in Shreveport; in that city and parish seven Negroes have been lynched in two years, not counting ordinary murders.


\bigskip\hrule\bigskip


\sphinxAtStartPar
\sphinxstyleemphasis{Citation:} “The Cause of Lynching” Editorial. 1914. \sphinxstyleemphasis{The Crisis}. 8(3): 126\sphinxhyphen{}127.


\section{Cowardice (1916)}
\label{\detokenize{Volumes/12/06/cowardice:cowardice-1916}}\label{\detokenize{Volumes/12/06/cowardice::doc}}
\begin{sphinxShadowBox}
\sphinxstylesidebartitle{}

\sphinxAtStartPar
The \sphinxhref{https://en.wikipedia.org/wiki/Newberry\_Six\_lynchings}{Newberry Six} were lynched by a posse organized by the local sheriff. Boisey Long, the “wretched man” whose arrest Du Bose describes, was executed after being convicted by an all\sphinxhyphen{}white jury.
\end{sphinxShadowBox}

\sphinxAtStartPar
No colored man can read an account of the recent lynching at Gainesville, Fla., without being ashamed of his people.

\sphinxAtStartPar
The action was characteristic. White officers, knowing themselves in the wrong and afraid of the resistance of colored men, sneaked in at midnight to serve a warrant on a person who they hoped would be helpless and ignorant of their intentions. Two of them seized the man in his house and after the melee one of the white men was dead and the other seriously wounded. Of the right and wrong of this no one will ever be really sure. There is no proof that the black man was guilty; there is no proof that he knowingly resisted arrest. There is proof, on the other hand, that after this extraordinary attack his colored fellows acted like a set of cowardly sheep. Without resistance they let a white mob whom they outnumbered two to one, torture, harry and murder their women, shoot down innocent men entirely unconnected with the alleged crime, and finally to cap the climax, they caught and surrendered the wretched man whose attempted arrest caused the difficulty.

\sphinxAtStartPar
No people who behave with the absolute cowardice shown by these colored people can hope to have the sympathy or help of the civilized folk. The men and women who had nothing to do with the alleged crime should have fought in self\sphinxhyphen{}defense to the last ditch if they had killed every white man in the county and themselves been killed. The man who surrendered to a lynching mob the victim of the sheriff ought himself to have been locked up.

\sphinxAtStartPar
In the last analysis lynching of Negroes is going to stop in the South when the cowardly mob is faced by effective guns in the hands of the people determined to sell their souls dearly.
\begin{quote}

\sphinxAtStartPar
It may be a good thing to forget and forgive; but it is altogether too easy a trick to forget and be forgiven.

\begin{flushright}
---G.K. Chesterton
\end{flushright}
\end{quote}


\bigskip\hrule\bigskip


\sphinxAtStartPar
\sphinxstyleemphasis{Citation:} “Cowardice.” Editorial. 1916. \sphinxstyleemphasis{The Crisis}. 12(6): 270\sphinxhyphen{}271.


\section{Anti\sphinxhyphen{}Lynching Legislation (1921)}
\label{\detokenize{Volumes/22/01/anti-lynching_legislation:anti-lynching-legislation-1921}}\label{\detokenize{Volumes/22/01/anti-lynching_legislation::doc}}
\index{lynching@\spxentry{lynching}}\ignorespaces 
\index{NAACP@\spxentry{NAACP}}\ignorespaces 
\sphinxAtStartPar
During the last five years the N.A.A.C.P. has carried on a campaign against lynching, conducted in accordance with the most approved methods of public­ity. This campaign has been financed by philanthropists North and South, and black and white, and by wide\sphinxhyphen{}spread popular subscription. We have never heard any serious criticism of our campaign save, naturally, from those who oppose any effort to defend and emancipate black folk.

\sphinxAtStartPar
Our accomplishment from all our effort may thus be summarized:
\begin{enumerate}
\sphinxsetlistlabels{\arabic}{enumi}{enumii}{}{.}%
\item {} 
\sphinxAtStartPar
Investigated more than a score of lynchings and race riots

\item {} 
\sphinxAtStartPar
Kept an accurate record of lynchings

\item {} 
\sphinxAtStartPar
Published the only statistical study of lynchings ever compiled

\item {} 
\sphinxAtStartPar
Sent 4,462,899 copies of the CRISIS to every state in the Union and to every country on the globe

\item {} 
\sphinxAtStartPar
Sent out hundreds of news releases. During the year of 1920 alone 131 press stories were sent out from the National Office

\item {} 
\sphinxAtStartPar
Held more than 2,000 public meetings, attended by more than 3,000,000 people, at which speak­ ers acquainted the audiences with the facts about lynchings

\item {} 
\sphinxAtStartPar
Sent literature on lynching, with actual photographs of lynchings and burnings at the stake, all over the civilized world

\item {} 
\sphinxAtStartPar
Fought successfully three notable extradition cases when attempts were made to carry colored men accused of crimes from Northern States back to sections of the South where the men would probably have been lynched

\item {} 
\sphinxAtStartPar
Held in New York City in May, 1919, a great and successful conference against lynching. The call for this conference was signed by 120 of the leading citizens of the country, 20 of them from Southern States. The number included 5 governors, 4 ex\sphinxhyphen{}governors, members of Congress, judges of the higher courts, members of the President’s cabinet, and other men and women prominent in the nation’s affairs

\item {} 
\sphinxAtStartPar
Issued the “Address to the Nation”

\item {} 
\sphinxAtStartPar
Raised and expended since 1911 in the fight against lynching \$33,928.56 this amount being spent for investigations, printing and publicity of all kinds, meetings and conferences

\item {} 
\sphinxAtStartPar
Secured the passage of a law in Kentucky providing for the punishment of any peace officer who allows a prisoner to be taken from him and lynched

\item {} 
\sphinxAtStartPar
Secured the introduction of anti\sphinxhyphen{}lynching bills in both houses of Congress. For the first time the House Committee on the Judiciary reported favorably on such a bill during the second ses­sion of the 66th Congress

\item {} 
\sphinxAtStartPar
Presented to committees of both the Senate and the House of Representatives of Congress, through its representatives, evidence and testimony showing the necessity of a Federal law against lynching

\item {} 
\sphinxAtStartPar
Furnished lawyers to defend colored men in lynching cases; secured the acquittal, for example, of 12 of 13 colored men arrested in connection with the lynchings at Duluth, Minn., in June 1920, totally disproving the charge that criminal assault on a young girl had been com­mitted by any of the three men lynched; de­ fended since October, 1919, 12 men sentenced to death and 67 to prison terms, thus preventing a legal lynching in Phillips County, Ark.

\end{enumerate}

\sphinxAtStartPar
In addition to these things we have freely given help and information to other organizations, and when the colored Republican campaign committee asked the aid of our legal department in drawing an anti\sphinxhyphen{}lynching bill, we freely and gladly gave it.

\sphinxAtStartPar
We are surprised and disappointed, therefore, to receive from Perry Howard an appeal for funds to finance a new organization to push an anti\sphinxhyphen{}lynching bill. This move is defensible only if the methods and aims of the N.A.A.C.P. in the anti\sphinxhyphen{}lynching campaign were failing. But this new organization is pushing practically the same bill by identical methods.

\sphinxAtStartPar
Is not this move, therefore, the same futile division of effort and clouding of issues which continually misleads and discourages the Negroes, and gives joy to the Bourbon South? Would not these gentlemen convince the public of their disinterested efforts if they supported established agencies and proven work rather than attempt­ ed to build on new lines and to risk all that has been gained without hope of doing more?

\sphinxAtStartPar
The strength of the N.A.A.C.P. has been its singleness of aim. We did not rescue Haiti in order to annex the \$10,000 salary of the Haitian Minister; we are not fighting lynching in order to become Recorder of Deeds; and our promotion of business enterprise has no string on the Registry of the Treasury. This does not mean that we are unmindful of the right of competent black men to hold office,—indeed we strongly emphasize and maintain this right—but we in­sist that this is subordinate and un­ important beside our fight for freedom, and we refuse to have our most sacred right played with as pawns to official preferment.


\bigskip\hrule\bigskip


\sphinxAtStartPar
\sphinxstyleemphasis{Citation:} “Anti\sphinxhyphen{}Lynching Legislation” Editorial. 1922. \sphinxstyleemphasis{The Crisis}. 22(1): 8\sphinxhyphen{}9.


\section{A University Course in Lynching (1923)}
\label{\detokenize{Volumes/26/02/university_course_in_lynching:a-university-course-in-lynching-1923}}\label{\detokenize{Volumes/26/02/university_course_in_lynching::doc}}


\sphinxAtStartPar
We are glad to note that the University of Missouri has opened a course in Applied Lynching. Many of our American Universities have long defended the institution, but they have not been frank or brave enough actually to arrange a mob murder so that the students could see it in detail.

\begin{sphinxShadowBox}
\sphinxstylesidebartitle{}

\sphinxAtStartPar
\sphinxhref{https://www.redandblack.com/athensnews/georgia-s-past-and-present-100th-anniversary-vigil-of-the-lynching-of-john-lee-eberhart/article\_da652812-732f-11eb-8a92-efb244b64542.html}{John Lee Eberhart} was lynched February 16, 1921 by a mob of 3,000 in Athens, Georgia.
\end{sphinxShadowBox}

\sphinxAtStartPar
The University of Georgia did, to be sure, stage a lynching a few years ago but this was done at night and the girls did not have a fair chance to see it. At the University of Missouri the matter was arranged in broad daylight with ample notice, by five hundred men and boys who were “comparatively orderly”, and it was viewed by some fifty women most of whom we understand were students of the University. We are very much in favor of this method of teaching 100 percent Americanism; as long as mob murder is an approved institution in the United States, students at the universities should have a first hand chance to judge exactly what a lynching is.

\sphinxAtStartPar
In the case of James T. Scott everything was as it should be. He was a janitor at the University who protested his innocence to his last breath. He was charged with having “lured” a fourteen year old girl in broad daylight far from her home and “down the railroad tracks”. He was “positively identified” by the girl, and while the father deprecates violence he has “no doubt” of the murdered man’s guilt.

\sphinxAtStartPar
Here was every element of the modern American lynching. We are glad that the future fathers and mothers of the West saw it, and we are expecting great results from this course of study at one of the most eminent of our State Universities.


\bigskip\hrule\bigskip


\sphinxAtStartPar
\sphinxstyleemphasis{Citation:} Du Bois, W.E.B. 1923. “A University Course in Lynching.” \sphinxstyleemphasis{The Crisis}. 26(2): 55.


\section{Lynching (1926)}
\label{\detokenize{Volumes/32/01/lynching:lynching-1926}}\label{\detokenize{Volumes/32/01/lynching::doc}}
\sphinxAtStartPar
Every 23 days a man is lynched in America and we say lynching is over because once we lynched two men every three days. The situation is still intolerable and uncivilized. The Southern states are still incapable of punishing lynchers. Even men whom the courts have declared innocent have been seized by mobs and lynched and then those same courts were unable to punish the lynchers.

\sphinxAtStartPar
The Dyer Anti\sphinxhyphen{}Lynching Bill is before the Congress. The South is cowering and complaining. The Galveston \sphinxstyleemphasis{News} calls it “legislation by revenge”. The Columbia, S. C. \sphinxstyleemphasis{State} calls it “rotten politics” to “incite race feeling’. The Houston, Texas \sphinxstyleemphasis{Post} sees an “invasion” of state rights. The New Orleans \sphinxstyleemphasis{Picayune} declares it “unconstitutional”, etc.

\sphinxAtStartPar
Is it revenge to stop public murder? Is it politics to stop lynching? Is it unconstitutional to enforce Section 1 of the 14th Amendment which says that no state shall “deny to any person within its jurisdiction the equal protection of the laws”?

\sphinxAtStartPar
If any Southern state does not want its rights “invaded”, let it stop lynching or punish the lynchers. If it cannot do this the United States government must and will. If the men like Dyer and McKinley who have pushed this bill are playing petty politics then the Negro voter answers simply that that is precisely the kind of petty politics he wants played; that he doesn’t care a rap whether the senators and representatives who are pushing this legislation love him or hate him, are actuated by the highest patriotism or the lowest selfishness. What he wants is that they should push the Dyer Anti\sphinxhyphen{}Lynching Bill and he is going to vote for any man who does it and against any man who doesn’t.


\bigskip\hrule\bigskip


\sphinxAtStartPar
\sphinxstyleemphasis{Citation:} Du Bois, W.E.B. 1926. “Lynching.”  32(1):10.


\section{Aiken (1927)}
\label{\detokenize{Volumes/34/01/aiken:aiken-1927}}\label{\detokenize{Volumes/34/01/aiken::doc}}
\sphinxAtStartPar
Does America thoroughly realize Aiken? Citizens declared by South Carolina courts of law to be either entirely innocent of crime or of unproven guilt have been openly murdered by persons well\sphinxhyphen{}known to the public. The authorities so far have refused to indict or arrest these murderers or to make any real inquiry into | their guilt. These red\sphinxhyphen{}handed assassins walk the streets of Aiken today free, impudent and unafraid. And the city of Aiken is advertised in the public press as an “attractive” winter resort!

\sphinxAtStartPar
What shall we do? The Constitution of the United States guarantees each state “A republican form of government”. Blease, the lynchers’ United States Senator, was elected by less than one\sphinxhyphen{}tenth of the qualified voters of the state. The lynchers and the Ku Klux Klan rule the city of Aiken and its county.

\sphinxAtStartPar
Is it possible that we the people of the United States, 120 millions strong, with a great army of 120,000 men and a navy costing \$300,000,000 a year; with 47 million Christians in hundreds of thousands of churches; with millionaire Foundations for Uplift, Art and Charity; with missionaries in China, India and Africa; we who in spotless holiness refuse to recognize Russia and curse the Grand Turk; we who in absolute and impeccable fairness and justice forgive no single foreign debtor a red cent if we can squeeze it from his bankrupt soul; this “Land of the Free and Home of the Brave”, that boasts before God its Fundamental Righteousness—is it possible that in Aiken we can do nothing, nothing, NOTHING?


\bigskip\hrule\bigskip


\sphinxAtStartPar
\sphinxstyleemphasis{Citation:} “Aiken.” 1927. Editorial. \sphinxstyleemphasis{The Crisis} 34(1):34.


\section{Lynching (1927)}
\label{\detokenize{Volumes/33/04/lynching:lynching-1927}}\label{\detokenize{Volumes/33/04/lynching::doc}}
\sphinxAtStartPar
There were thirty\sphinxhyphen{}four lynchings in the United States in 1926, nearly twice as many as in 1925.

\sphinxAtStartPar
There is no doubt as to the reason of the increase. The fear of the Dyer Bill has been removed from the minds of the murderers. This is but a louder call for Federal legislation. There is no civilized country on earth which would allow in one year thirty\sphinxhyphen{}four mob murders to occur without even investigation in most cases and in no case with adequate attempt at punishment. Even the recent imprisonment of Georgia lynchers is not a real case in point, because the man lynched was white. If he had been a Negro these lynchers in jail probability would have gone scot\sphinxhyphen{}free.

\sphinxAtStartPar
The Nation, therefore, is fronted by a situation. Certain parts of the land are so dominated by their uncivilized elements that they cannot punish murder. There never was any excuse for lynching in group hysteria from “unusual” crime but today this is not even alleged. The case is simply, as in Aiken, the inability of the law to function. What shall a nation do in such case? Sit still and recount the certainly encouraging fact that at least the newspapers are “speaking out”? Or see that the strong hand of the Federal government falls upon the community that will not or cannot punish mob murder?

\sphinxAtStartPar
Senator Borah writes to warn us that the Dyer Bill was only a political trick and hopelessly unconstitutional. Very good. Why do not Senator Borah and the other decent men in Congress unite to frame a bill that will be constitutional? It is sheer nonsense to allege that a great country like the United States cannot stop wholesale murder because of petty legal technicalities.

\sphinxAtStartPar
\sphinxstyleemphasis{Citation:} Du Bois, W.E.B. 1927. “Lynching.”  33(4):180\sphinxhyphen{}181.


\section{Causes of Lynching (1931)}
\label{\detokenize{Volumes/38/04/causes_of_lynching:causes-of-lynching-1931}}\label{\detokenize{Volumes/38/04/causes_of_lynching::doc}}
\begin{sphinxShadowBox}
\sphinxstylesidebartitle{}

\sphinxAtStartPar
This group was put together by the Commission on Interracial Cooperation and released their \sphinxhref{https://archive.org/details/lynchingswhatthe00sout}{report} in 1931.
\end{sphinxShadowBox}

\sphinxAtStartPar
We have nothing but praise for the Southern Commission on the study of lynching. It has eleven members: the white members are: George F. Milton, Dr. W. J. McGlothlin, W. P. King, Julian Harris, Dr. Edward W. Odum%
\begin{footnote}[1]\sphinxAtStartFootnote
It should be Howard W. Odum, not Edward W. Odum.
%
\end{footnote}: My footnote text. and Alexander Spence. The colored members are: Dr. R. R. Moton, Dr. John Hope, Dr. Charles S. Johnson, President B. F. Hubert and Professor John Work.

\sphinxAtStartPar
At the same time, we know the temptation that faces such a committee working in the South. They are almost forced to be diplomatic before they are absolutely truthful. We trust, therefore, that they are going to be brave enough to say this plain word about lynching: the cause of lynching lies in ignorance, economic exploitation, religious intolerance, political disfranchisement, and sex prejudice.

\sphinxAtStartPar
The \sphinxstyleemphasis{ignorance} still forced on the colored South and still allowed in the white South is stupendous. \sphinxstyleemphasis{Economic exploitation}, including actual peonage on the plantations of the Gulf States and the Mississippi Valley, is a perfectly well\sphinxhyphen{}known fact. \sphinxstyleemphasis{Political disfranchisement} puts the selection of officials who enforce the law largely in the hands of the white mob. \sphinxstyleemphasis{Religious intolerance} is making hypocrites of Southern white Christians and allowing them to recite the Golden Rule with one side of their mouths and shriek “Kill the N{[}*****{]}!” with the other. And finally, lynching has always been used and is still being used to stop \sphinxstyleemphasis{sexual intercourse} between colored men and white women, whether by consent or not. Every anti intermarriage law of the South is a cause of lynching and an affront to civilization. God only knows how many black men accused of rape have been done to death by mobs simply because they chose a willing white paramour or were chosen by one. There is no use blinking these facts. They are true. But we know that it will take courage to say it.


\bigskip\hrule\bigskip


\sphinxAtStartPar
\sphinxstyleemphasis{Citation:} Du Bois, W.E.B. 1931. “Causes of Lynching.” \sphinxstyleemphasis{The Crisis}. 38(4): 138.


\bigskip\hrule\bigskip



\chapter{Segregation}
\label{\detokenize{Sections/segregation:segregation}}\label{\detokenize{Sections/segregation::doc}}
\sphinxAtStartPar
Editorials on segregation
\begin{itemize}
\item {} 
\sphinxAtStartPar
{\hyperref[\detokenize{Volumes/01/01/Segregation::doc}]{\sphinxcrossref{Segregation (1910)}}}

\item {} 
\sphinxAtStartPar
{\hyperref[\detokenize{Volumes/03/05/homes::doc}]{\sphinxcrossref{Homes (1912)}}}

\item {} 
\sphinxAtStartPar
{\hyperref[\detokenize{Volumes/05/04/blesseddiscrimination::doc}]{\sphinxcrossref{Blessed Discrimination (1913)}}}

\item {} 
\sphinxAtStartPar
{\hyperref[\detokenize{Volumes/13/06/perpetual_dilemma::doc}]{\sphinxcrossref{The Perpetual Dilemma (1917)}}}

\item {} 
\sphinxAtStartPar
{\hyperref[\detokenize{Volumes/17/03/jim_crow::doc}]{\sphinxcrossref{Jim Crow (1919)}}}

\item {} 
\sphinxAtStartPar
{\hyperref[\detokenize{Volumes/18/01/returning_soldiers::doc}]{\sphinxcrossref{Returning Soldiers (1919)}}}

\item {} 
\sphinxAtStartPar
{\hyperref[\detokenize{Volumes/26/02/on_being_crazy::doc}]{\sphinxcrossref{On Being Crazy (1923)}}}

\item {} 
\sphinxAtStartPar
{\hyperref[\detokenize{Volumes/31/01/challenge_of_detroit::doc}]{\sphinxcrossref{The Challenge of Detroit (1925)}}}

\item {} 
\sphinxAtStartPar
{\hyperref[\detokenize{Volumes/41/02/naacp_and_race_segregation::doc}]{\sphinxcrossref{The N.A.A.C.P. and Race Segregation (1934)}}}

\item {} 
\sphinxAtStartPar
{\hyperref[\detokenize{Volumes/41/04/segregation_in_the_north::doc}]{\sphinxcrossref{Segregation in the North (1934)}}}

\item {} 
\sphinxAtStartPar
{\hyperref[\detokenize{Volumes/41/05/segregation::doc}]{\sphinxcrossref{Segregation (1934)}}}

\item {} 
\sphinxAtStartPar
{\hyperref[\detokenize{Volumes/40/03/Color_caste_in_the_united_states::doc}]{\sphinxcrossref{Color Caste in the United States (1933)}}}

\end{itemize}


\section{Segregation (1910)}
\label{\detokenize{Volumes/01/01/Segregation:segregation-1910}}\label{\detokenize{Volumes/01/01/Segregation::doc}}
\sphinxAtStartPar
Some people in Chicago, Philadelphia, Atlantic City, Columbus, O., and other Northern cities are quietly trying to establish separate colored schools. This is wrong, and should be resisted by black men and white. Human contact, human acquaintanceship, human sympathy is the great solvent of human problems. Separate school children by wealth and the result is class misunderstanding and hatred. Separate them by race and the result is war. Separate them by color and they grow up without learning the tremendous truth that it is impossible to judge the mind of a man by the color of his face. Is there any truth that America needs to learn more? Back of this demand for the segregation of black folk in public institutions, or the segregation of Italians, or the segregation of any class, is almost always a shirking of responsibility on the part of the public—a desire to put off on somebody else the work of social uplift, while they themselves enjoy its results. Nobody pretends to deny that probably three\sphinxhyphen{}fourths of the colored children in the public schools of a great Northern city are below the average of their fellow students in some respects. They are, however, capable of improvement, and of rapid improvement. This improvement can be carried on by the community. The community can, however, if it is cowardly and selfish, shirk this responsibility and pile it on the shoulders of the Negroes represented by the one\sphinxhyphen{}fourth of Negro children who are above the average, or equal to it; and they can, if they are persistent, succeed in pushing back and possibly overwhelming a deserving and rising class of colored people.

\sphinxAtStartPar
This is the history of color discrimination in general in Philadelphia, New York and Chicago. When the discrimination comes in various lines of life, it does not bear simply on those who are not hurt by it—who do not feel it, and who by their position naturally fall outside the lines of discrimination, but it comes with crushing weight upon those other Negroes to whom the reasons for discrimination do not apply in the slightest respect, and they are thus made to bear a double burden. Further than this, when the discrimination is once established, immediately the public provisions for the segregated portion become worse. If it is discrimination against poor people, then the schools for the poor people become worse than those for the rich— less well equipped and less well supervised. If it is discrimination against colored people, the colored school becomes poor, with less money and less means of efficiency.

\sphinxAtStartPar
The argument, then, for color discrimination in schools and in public institutions is an argument against democracy and an attempt to shift public responsibility from the shoulders of the public to the shoulders of some class who are unable to defend themselves.


\bigskip\hrule\bigskip


\sphinxAtStartPar
\sphinxstyleemphasis{Citation:} Du Bois, W.E.B. 1910. “Segregation.” \sphinxstyleemphasis{The Crisis}. 1(1): 10\sphinxhyphen{}11.


\section{Homes (1912)}
\label{\detokenize{Volumes/03/05/homes:homes-1912}}\label{\detokenize{Volumes/03/05/homes::doc}}
\sphinxAtStartPar
The injustice toward colored people who want decent living conditions is almost unbelievable unless one comes face to face with the facts. The New York Times, which spares few opportunities to treat black folk unjustly, says in an editorial :

\sphinxAtStartPar
“It is becoming necessary in the upper resident part of New York for the property owners in neighborhoods to enter into agreements to prohibit the occupancy of their dwellings by Negroes. This departure is not due to race prejudice or hatred for the Negro but for the protection of the neighborhood values against designing or ugly white men.”

\sphinxAtStartPar
Not a word for the colored family seeking a decent home; but if that family live in the slums and purlieus and let the surroundings teach their children crime and prostitution, then the holy horror of the Times and its ilk! If black folk rush for decent homes at exorbitant rents is there sense or decency in trying to stop this by frantic appeals to race prejudice? In other and perfectly parallel cases the property owner suffers the inevi­table without thought of appeal to human hatred. If property in Fifth Avenue becomes more valuable for business than for dwellings then the dwellings must go. If people indulge in senseless prejudice against their fellows and find real\sphinxhyphen{}estate men coining this prejudice into gold, they have no right to blame the unhappy victims of their barbarism, but they must blame that barbarism misnamed race pride.


\bigskip\hrule\bigskip


\sphinxAtStartPar
\sphinxstyleemphasis{Citation:} Du Bois, W.E.B. 1912. “Homes.”  \sphinxstyleemphasis{The Crisis}. 3(5): 200\sphinxhyphen{}201.


\section{Blessed Discrimination (1913)}
\label{\detokenize{Volumes/05/04/blesseddiscrimination:blessed-discrimination-1913}}\label{\detokenize{Volumes/05/04/blesseddiscrimination::doc}}
\sphinxAtStartPar
A good friend sends us this word:
\begin{quote}

\sphinxAtStartPar
As an optimist of The Crisis persuasion, I find myself more or less frequently engaged in arguments on the eternal race question. Here is an argument I am often called upon to meet: “Jim Crow” laws make us save money; discrimination makes us appreciate and patronize our own; segregation gives our business men a chance; separate schools give our girls and boys something to work for. Possibly there are many doubtful minds who would be benefited by a word from you on this subject through the columns of The Crisis.
\end{quote}

\sphinxAtStartPar
There is no doubt that colored people travel less than they otherwise would, on account of ” Jim Crow” cars, and thus have this money to spend otherwise.

\sphinxAtStartPar
There is no doubt that thousands of Negro business enterprises have been built up on account of discrimination against colored folks in drug stores, grocery stores, insurance societies and daily papers. In a sense The Crisis is capitalized race prejudice.

\sphinxAtStartPar
There is not the slightest doubt but that separate school systems, by giving colored children their own teachers and a sense of racial pride, are enabled to keep more colored children in school and take them through longer courses than outworn handwork or decadent trades mixed systems. The 100,000 Negroes of Baltimore have 600 pupils in the The result is that our business men are separate high school; New York, with a larger colored population, has less than 200 in its mixed high schools.

\sphinxAtStartPar
Therefore discrimination is a veiled blessing? It is not, save in a few exceptional cases.

\sphinxAtStartPar
Take the “Jim Crow” car; is the money saved or merely diverted? Is it diverted to better things than travel or to worse? A s a matter of fact separate cars and parks and public insult have driven Negro amusements indoors, and the result is tuberculosis and pneumonia; they have deprived colored people of the civilization of public contact, and that is an almost irreparable loss.

\sphinxAtStartPar
Take our business enterprises; they are creditable and promising, but they are compelled to set a lower standard of efficiency than that recognized in the white business world. Our business men must grope in the dark after methods; our buyers do not know how to buy and our clerks do not know how to sell; our banks do not know how to invest, and our insurance societies, with few exceptions, do not know what modern insurance means.

\sphinxAtStartPar
We all know this, but whom do we blame? We blame ourselves. We carp and sneer and criticise among ourselves at “colored” enterprises and declare that we can always tell a “colored” store or a “colored” paper by its very appearance. This is not fair. It is cruel and senseless injustice. Negro enterprises conform to a lower standard not because they want to, but because they must. Color prejudice prevents us from training our children and our men to the same standards as those set for the sur­ rounding white world. The colored boy can learn servility, but he is not allowed to learn business methods; colored men learn how to sweep the floor of a bank, but cannot learn the A B C of modern investments; the colored industrial school does not teach modern machine methods, but old and outworn handwork or decadent trades and medieval conditions.

\sphinxAtStartPar
The result is that our business men are not the travelers of a broad and beaten path, but wanderers in a wilderness. Considering their opportunity, their fifty banks and tens of thousands of business enterprises and hundreds of thousands of dollars in industrial insurance are little short of marvelous. But to call the cruel discrimination that has misdirected effort, discouraged ability, murdered men and sent women to graves of sorrow—to call this an advantage is to misuse language. The open door of opportunity to colored persons, regard­ less of the accident of color, would have given us to\sphinxhyphen{}day \$10 of invested capital where we have \$1; and ten business men trained to the high and exact standard of modern efficiency where now we have one grim and battered survivor clinging to the ragged edge. Thank God for the dollar and the survivor, but do not thank Him for the discrimination. Thank the devil for that. We black people to­ day are succeeding not because of dis­ crimination, but in spite of it. Without it we would succeed better and faster, and they that deny this are either fools or hypocrites.

\sphinxAtStartPar
The same thing is evident in education. Separate school systems give us more pupils but poorer schools. The 200 black high\sphinxhyphen{}school pupils in New York have the best high\sphinxhyphen{}school equipment in the land—beautiful buildings, costly laboratories, scores of the best teachers, books and materials, everything that money and efficiency can furnish; the Baltimore high school has to struggle in a building about half large enough for its work, with too few teachers and those at low salaries, and with a jealous public that grudges every cent the school has and wants to turn the whole machine into a factory for making servants for smart Baltimore. All honor to their teachers for the splendid work they do in spite of discrimination, but do not credit dis­ crimination with the triumph; credit Mason Hawkins.

\sphinxAtStartPar
Turn to our newspapers. They are a sad lot, we grant you. But whose is the fault? How can they get trained men for their work? How can they get capital for their enterprise? How can
they maintain for themselves and their readers a standard even as high as their white contemporaries, not to say higher? Their workers are shut out from the staffs of white magazines and news­ papers ; their readers are deprived of the education of social contact and their very writers are, through no fault of their own, illiterate. There lies on our desk this pitiful letter:
\begin{quote}

\sphinxAtStartPar
Dear Editor of the Crisis
New York

\sphinxAtStartPar
It would confure a great favor upon me. if the nessacery arrangment can be secured that i may constribet to your magazine Some of my origanal MS.S. and Poem, as i have joust Begain to Rite Short M.S.S i awaiteing you Reply
Your truly.
\end{quote}

\sphinxAtStartPar
Shall we laugh at this or weep ? Who knows what this man might have done or said if the State of Florida had let him learn to read and write ? Shall we thank the God of Discrimination for planting literature in such soil or shall we hate it with perfect hatred ?

\sphinxAtStartPar
No. Race discrimination is evil. It forces those discriminated against to a lower standard and then judges them by a higher. It demands that we do more with less opportunity than others do. It denies to present workers the accumu­lated experience of the past and compels them at fearful cost to make again the mistakes of the past. Out of this cruel grilling may and do come strong char­acters, but out of it also come\sphinxhyphen{} the criminal and the stunted, the bitter and the insane. One is just as much the fruit of the tree as the other. If in any place and time race hatred is so un­ reasoning and bitter that separate
schools, cars and churches are inevita­ble, we must accept it. make the best of it and turn even its disadvantages to our advantage. But we must never for­ get that none of its possible advantages can offset its miserable evils, or replace the opportunity, the broad education, the free competition and the generous emul­ation of free men in a free world.


\bigskip\hrule\bigskip


\sphinxAtStartPar
\sphinxstyleemphasis{Citation:} Du Bois, W.E.B. 1913. “Blessed Discrimination.”  \sphinxstyleemphasis{The Crisis}. 5(4): 184\sphinxhyphen{}186.


\section{The Perpetual Dilemma (1917)}
\label{\detokenize{Volumes/13/06/perpetual_dilemma:the-perpetual-dilemma-1917}}\label{\detokenize{Volumes/13/06/perpetual_dilemma::doc}}
\sphinxAtStartPar
We Negroes ever face it.

\sphinxAtStartPar
We cannot escape it.

\sphinxAtStartPar
We must continually choose between insult and injury: no schools or separate schools; no travel or “Jim Crow” travel; homes with disdainful neighbors or homes in slums.

\sphinxAtStartPar
We continually submit to segregated schools, “Jim Crow” cars, and isolation, because it would be suicide to go uneducated, stay at home, and live in the “tenderloin.”

\sphinxAtStartPar
Yet, when a new alternative of such choice faces us it comes with a shock and almost without thinking we rail at the one who advises the lesser of two evils.

\begin{sphinxShadowBox}
\sphinxstylesidebartitle{}

\sphinxAtStartPar
\sphinxhref{https://en.wikipedia.org/wiki/Joel\_Elias\_Spingarn}{Joel Elias Spingarn}, a white liberal Republican, was on the board of the NAACP from 1913 to his death in 1939, serving as chair from 1913\sphinxhyphen{}1919.
\end{sphinxShadowBox}

\sphinxAtStartPar
Thus it was with many hasty edi­tors in the case of the training camp for Negro officers which Dr. J. E. Spingarn is seeking to establish.

\sphinxAtStartPar
Does Dr. Spingarn believe in a “Jim Crow” training camp? Certainly not, and he has done all he could to induce the government to admit Negroes to all training camps.

\sphinxAtStartPar
The government has so far courteously refused.

\sphinxAtStartPar
But war is imminent.

\sphinxAtStartPar
If war comes to\sphinxhyphen{}morrow Negroes will be compelled to enlist under \sphinxstyleemphasis{white} officers because (save in very few cases) no Negroes have had the requisite training.

\sphinxAtStartPar
We must choose the insult of a separate camp and the irreparable injury of strengthening the present custom of putting no black men in positions of authority.

\sphinxAtStartPar
Our choice is as clear as noonday.

\sphinxAtStartPar
Give us the camp.

\sphinxAtStartPar
Let not 200, but 2,000 volunteer.

\sphinxAtStartPar
We did not make the damnable dilemma.

\sphinxAtStartPar
Our enemies made that.

\sphinxAtStartPar
We must make the choice else we play into their very claws.

\sphinxAtStartPar
It is a case of camp or no officers.

\sphinxAtStartPar
Give us the officers.

\sphinxAtStartPar
Give us the camp.

\sphinxAtStartPar
A word to those who object:

\begin{sphinxShadowBox}
\sphinxstylesidebartitle{}

\sphinxAtStartPar
\sphinxhref{https://en.wikipedia.org/wiki/Preparedness\_Movement}{Leonard Wood}, along with  former President Theodore Roosevelt, was a leader in the \sphinxhref{https://en.wikipedia.org/wiki/Preparedness\_Movement}{Preparedness Movement}, which sought to build up US military prior to its involvement in World War I.
\end{sphinxShadowBox}
\begin{enumerate}
\sphinxsetlistlabels{\arabic}{enumi}{enumii}{}{.}%
\item {} 
\sphinxAtStartPar
The army does not wish this camp. It wishes the project to fail. General Wood refuses to name date or place until 200 apply. The reason is obvious. Up to March 8, sixty\sphinxhyphen{}nine men have applied.

\item {} 
\sphinxAtStartPar
The camp is a temporary measure lasting four weeks and designed to FIGHT not encourage discrimina­tion in the army. The New York Negro regiment could not find enough qualified Negroes for its commissions. W e want trained colored officers. This camp will help furnish them.

\item {} 
\sphinxAtStartPar
The South does not want the Negro to receive military training of any sort. For that reason the general staff reduced its estimate from 900,\sphinxhyphen{} 000 to 500,000 soldiers—they expect to EXCLUDE Negroes!

\item {} 
\sphinxAtStartPar
If war comes, conscription will follow. All pretty talk about not vol­unteering will become entirely aca­demic. This is the mistake made by the Baltimore AFRO\sphinxhyphen{}AMERICAN, the Chicago DEFENDER, the New York NEWS, and the Cleveland GAZETTE. They assume a choice between volunteering and not volunteer­ing. The choice will be between con­scription and rebellion.

\end{enumerate}

\sphinxAtStartPar
Can the reader conceive of the pos­sibility of choice? The leaders of the colored race who advise them to add treason and rebellion to the other grounds on which the South urges discrimination against them would hardly be doing a service to those whom they profess to love. No, there is only one thing to do now, and that is to organize the colored people for leadership and service, if war should come. A thousand commissioned officers of colored blood is something to work for.

\sphinxAtStartPar
Give us the camp!


\bigskip\hrule\bigskip


\sphinxAtStartPar
\sphinxstyleemphasis{Citation:} “The Perpetual Dilemma.” Editorial. 1917. \sphinxstyleemphasis{The Crisis}. 13(6): 270\sphinxhyphen{}271.


\section{Jim Crow (1919)}
\label{\detokenize{Volumes/17/03/jim_crow:jim-crow-1919}}\label{\detokenize{Volumes/17/03/jim_crow::doc}}
\sphinxAtStartPar
We  colored folk stand at the parting of ways, and we must take counsel. The objection to segregation and “Jim\sphinxhyphen{}Crowism” was in other days the fact that compelling Ne­groes to associate only with Negroes meant to exclude them from contact with the best culture of the day. How could we learn manners or get knowledge if the heritage of the past was locked away from us?

\sphinxAtStartPar
Gradually, however, conditions have changed. Culture is no longer the monopoly of the white nor is pov­ erty and ignorance the sole heritage of the black. Many a colored man in our day called to conference with his own and rather dreading the con­ tact with uncultivated people even though they were of his own blood has been astonished and deeply grati­fied at the kind of people he has met —at the evidence of good manners and thoughtfulness among his own.

\sphinxAtStartPar
This together with the natural human love of herding like with like has in the last decade set up a tre­ mendous current within the colored race against any contact with whites that can be avoided. They have wel­comed separate racial institutions.

\sphinxAtStartPar
They have voluntarily segregated themselves and asked for more seg­regation. The North is full of in­ stances of practically colored schools which colored people have demanded and, of course, the colored church and social organization of every sort are ubiquitous.

\sphinxAtStartPar
Today both these wings of opinion are getting suspicious of each other and there are plenty of whites to help the feeling along. Whites and Blacks ask the Negro who fights separation: “Are you ashamed of your race?” Blacks and Whites ask the Negro who welcomes and encourages separation: “Do you want to give up your rights? Do you acknowledge your inferi­ority?”

\sphinxAtStartPar
Neither attitude is correct. Segregation is impolitic, because it is impossible. You can not build up a logical scheme of a self\sphinxhyphen{}sufficing, sep­arate Negro America inside America or a Negro world with no close rela­tions to the white world. If there are relations between races they must be based on the knowledge and sym­pathy that come alone from the long and intimate human contact of indi­viduals.

\sphinxAtStartPar
On the other hand, if the Negro is to develop his own power and gifts; if he is not only to fight prejudices and oppression successfully, but also to unite for ideals higher than the world has realized in art and indus­try and social life, then he must unite and work with Negroes and build a new and great Negro ethos.

\sphinxAtStartPar
Here, then, we face the curious paradox and we remember contradictory facts. Unless we had fought segregation with determination, our whole race would have been pushed into an ill\sphinxhyphen{}lighted, unpaved, un\sphinxhyphen{}sewered ghetto. Unless we had built great church organizations and manned our own southern schools, we should be shepherdless sheep. Unless we had welcomed the segregation of Fort Des Moines, we would have had no officers in the National Army. Unless we had beaten open the doors of northern universities, we would have had no men fit to be officers.

\sphinxAtStartPar
Here is a dilemma calling for thought and forbearance. Not every builder of racial co\sphinxhyphen{}operation and solidarity is a “Jim\sphinxhyphen{}Crow” advocate, a hater of white folk. Not every Negro who fights prejudice and segregation is ashamed of his race.


\bigskip\hrule\bigskip


\sphinxAtStartPar
\sphinxstyleemphasis{Citation:} “Jim Crow” Editorial. 1919. \sphinxstyleemphasis{The Crisis}. 17(3): 112\sphinxhyphen{}13.


\section{Returning Soldiers (1919)}
\label{\detokenize{Volumes/18/01/returning_soldiers:returning-soldiers-1919}}\label{\detokenize{Volumes/18/01/returning_soldiers::doc}}
\sphinxAtStartPar
We are returning from war! \sphinxstylestrong{The Crisis} and tens of thousands of black men were drafted into a great struggle. For bleeding France and what she means and has meant and will mean to us and humanity and against the threat of German race arrogance, we fought gladly and to the last drop of blood; for America and her highest ideals, we fought in far\sphinxhyphen{}off hope; for the dominant southern oligarchy entrenched in Washington, we fought in bitter resignation. For the America that represents and gloats in lynching, disfranchisement, caste, brutality and devilish insult—for this in the hateful upturning and mixing of things, we were forced by vindictive fate to fight, also.

\sphinxAtStartPar
But today we return! We return from the slavery of uniform which the world’s madness demanded us to don to the freedom of civil garb. We stand again to look America squarely in the face and call a spade a spade. We sing: This country of ours, despite all its better souls have done and dreamed, is yet a shameful land.

\sphinxAtStartPar
It \sphinxstyleemphasis{lynches}.

\sphinxAtStartPar
And lynching is barbarism of a degree of contemptible nastiness unparalleled in human history. Yet for fifty years we have lynched two Negroes a week, and we have kept this up right through the war.

\sphinxAtStartPar
It \sphinxstyleemphasis{disfranchises} its own citizens.

\sphinxAtStartPar
Disfranchisement is the deliberate theft and robbery of the only protection of poor against rich and black against white. The land that disfranchises its citizens and calls itself a democracy lies and knows it lies.

\sphinxAtStartPar
It encourages \sphinxstyleemphasis{ignorance}.

\sphinxAtStartPar
It has never really tried to educate the Negro. A dominant minority does not want Negroes educated. It wants servants, dogs, whores and monkeys. And when this land allows a reactionary group by its stolen political power to force as many black folk into these categories as it possibly can, it cries in contemptible hypocrisy: “They threaten us with degeneracy; they cannot be educated.”

\sphinxAtStartPar
It \sphinxstyleemphasis{steals} from us.

\sphinxAtStartPar
It organizes industry to cheat us. It cheats us out of our land; it cheats us out of our labor. It confiscates our savings. It reduces our wages. It raises our rent. It steals our profit. It taxes us without representation. It keeps us consistently and universally poor, and then feeds us on charity and derides our poverty.

\sphinxAtStartPar
It \sphinxstyleemphasis{insults} us.

\sphinxAtStartPar
It has organized a nation\sphinxhyphen{}wide and latterly a world\sphinxhyphen{}wide propaganda of deliberate and continuous insult and defamation of black blood wherever found. It decrees that it shall not be possible in travel nor residence, work nor play, education nor instruction for a black man to exist without tacit or open acknowledgment of his inferiority to the dirtiest white dog. And it looks upon any attempt to question or even discuss this dogma as arrogance, unwarranted assumption and treason.

\sphinxAtStartPar
This is the country to which we Soldiers of Democracy return. This is the fatherland for which we fought! But it is or fatherland. It was right for us to fight. The faults of our country are our faults. Under similar circumstances, we would fight again. But by the God of Heaven, we are cowards and jackasses if now that that war is over, we do not marshal every ounce of our brain and brawn to fight a sterner, longer, more unbending battle against the forces of hell in our own land.

\sphinxAtStartPar
We return.

\sphinxAtStartPar
We return from fighting.

\sphinxAtStartPar
We return fighting.

\sphinxAtStartPar
Make way for Democracy! saved it in France, and by the Great Jehovah, we will save it in the United States of America, or know the reason why.


\bigskip\hrule\bigskip


\sphinxAtStartPar
\sphinxstyleemphasis{Citation:} Du Bois, W.E.B. 1918. “Return Soldiers.” \sphinxstyleemphasis{The Crisis}. 18(1): 12\sphinxhyphen{}13.


\section{On Being Crazy (1923)}
\label{\detokenize{Volumes/26/02/on_being_crazy:on-being-crazy-1923}}\label{\detokenize{Volumes/26/02/on_being_crazy::doc}}
\sphinxAtStartPar
I was one o’clock and I was hungry. I walked into a restaurant, seated myself and reached for the bill\sphinxhyphen{}of\sphinxhyphen{}fare. My table companion rose.

\sphinxAtStartPar
“Sir,” said he, “do you wish to force your company on those who do not want you?”

\sphinxAtStartPar
No, said I, I wish to eat.

\sphinxAtStartPar
“Are you aware, Sir, that this is social equality?”

\sphinxAtStartPar
Nothing of the sort, Sir, it is hunger,—and I ate.

\sphinxAtStartPar
The day’s work done, I sought the theatre. As I sank into my seat, the lady shrank and squirmed.

\sphinxAtStartPar
I beg pardon, I said.

\sphinxAtStartPar
“Do you enjoy being where you are not wanted?” she asked coldly.

\sphinxAtStartPar
Oh no, I said.

\sphinxAtStartPar
“Well you are not wanted here.”

\sphinxAtStartPar
I was surprised. I fear you are mistaken, I said. I certainly want the music and I like to think the music wants me to listen to it.

\sphinxAtStartPar
“Usher,” said the lady, “this is social equality.”

\sphinxAtStartPar
No, madame, said the usher, it is the second movement of Beethoven’s

\sphinxAtStartPar
After the theatre, I sought the hotel where I had sent my baggage. The clerk scowled.

\sphinxAtStartPar
“What do you want?” he asked.

\sphinxAtStartPar
Rest, I said.

\sphinxAtStartPar
“This is a white hotel,” he said.

\sphinxAtStartPar
I looked around. Such a color scheme requires a great deal of cleaning, I said, but I don’t know that I object.

\sphinxAtStartPar
“We object,” said he.

\sphinxAtStartPar
Then why—, I began, but he interrupted.

\sphinxAtStartPar
“We don’t keep ‘n{[}******{]}’,” he said, “we don’t want social equality.”

\sphinxAtStartPar
Neither do I. I replied gently, I want a bed.

\sphinxAtStartPar
I walked thoughtfully to the train. I’ll take a sleeper through Texas. I’m a bit dissatisfied with this town.

\sphinxAtStartPar
“Can’t sell you one.”

\sphinxAtStartPar
I only want to hire it, said I, for a couple of nights.

\sphinxAtStartPar
“Can’t sell you a sleeper in Texas,” he maintained. “They consider that social equality.”

\sphinxAtStartPar
I consider it barbarism, I said, and I think I’ll walk.

\sphinxAtStartPar
Walking, I met a wayfarer who immediately walked to the other side of the road where it was muddy. I asked his reasons.

\sphinxAtStartPar
“ ‘N{[}******{]}’ is dirty,” he said.

\sphinxAtStartPar
So is mud, said I. Moreover I added, I am not as dirty as you—at least, not yet.

\sphinxAtStartPar
“But you’re a ‘n{[}*****{]}’, ain’t you?” he asked.

\sphinxAtStartPar
My grandfather was so\sphinxhyphen{}called.

\sphinxAtStartPar
“Well then!” he answered triumphantly.

\sphinxAtStartPar
Do you live in the South? I persisted, pleasantly.

\sphinxAtStartPar
“Sure,” he growled, “and starve there.”

\sphinxAtStartPar
I should think you and the Negroes might get together and vote out starvation.

\sphinxAtStartPar
“We don’t let them vote.”

\sphinxAtStartPar
We? Why not? I said in surprise.

\sphinxAtStartPar
“ `N{[}******{]}’ is too ignorant to vote.”

\sphinxAtStartPar
But, I said, I am not so ignorant as you.

\sphinxAtStartPar
“But you’re a ‘n{[}*****{]}’.”

\sphinxAtStartPar
Yes, I’m certainly what you mean by that.

\sphinxAtStartPar
“Well then!” he returned, with that curiously inconsequential note of triumph. “Moreover,” he said, “I don’t want my sister to marry a n{[}*****{]}.”

\sphinxAtStartPar
I had not seen his sister, so I merely murmured, let her say, no.

\sphinxAtStartPar
“By God you shan’t marry her, even if she said yes.”

\sphinxAtStartPar
But,—but I don’t want to marry her, I answered a little perturbed at the personal turn.

\sphinxAtStartPar
“Why not!” he yelled, angrier than ever.

\sphinxAtStartPar
Because I’m already married and I rather like my wife.

\sphinxAtStartPar
“Is she a ‘n{[}*****{]}’?” he asked suspiciously.

\sphinxAtStartPar
Well, I said again, her grandmother —was called that.

\sphinxAtStartPar
“Well then!” he shouted in that oddly illogical way.

\sphinxAtStartPar
I gave up. Go on, I said, either you are crazy or I am.

\sphinxAtStartPar
“We both are,” he said as he trotted along in the mud.


\bigskip\hrule\bigskip


\sphinxAtStartPar
\sphinxstyleemphasis{Citation:} Du Bois, W.E.B. 1923. “On Being Crazy.” \sphinxstyleemphasis{The Crisis}. 26(2): 56\sphinxhyphen{}57.


\section{The Challenge of Detroit (1925)}
\label{\detokenize{Volumes/31/01/challenge_of_detroit:the-challenge-of-detroit-1925}}\label{\detokenize{Volumes/31/01/challenge_of_detroit::doc}}
\sphinxAtStartPar
In Detroit, Michigan, a black man has shot into a mob which was threatening him, his family, his friends and his home in order to make him move out of the neighborhood. He killed one man and wounded another.



\sphinxAtStartPar
Immediately a red and awful challenge confronts the nation. Must black folk shoot and shoot to kill in order to maintain their rights or is this unnecessary and wanton bloodshed for fancied ill? The answer depends on the facts. The Mayor of Detroit has publicly warned both mob and Negroes. He has repudiated mob law but he adds, turning to his darker audience, that they ought not to invite aggression by going where they are not wanted. There are thus two interpretations:
\begin{enumerate}
\sphinxsetlistlabels{\arabic}{enumi}{enumii}{}{.}%
\item {} 
\sphinxAtStartPar
A prosperous Negro physician of Detroit, seeking to get away from his people, moves into a white residential section where his presence for social reasons is distasteful to his neighbors.

\item {} 
\sphinxAtStartPar
A prosperous Negro physician of Detroit, seeking a better home with more light, air, space and quiet, finds it naturally in the parts of the city where white folk with similar wants have gone rather than in the slums where most of the colored are crowded.

\end{enumerate}

\sphinxAtStartPar
Which version figures: is true? See the figures:

\sphinxAtStartPar
\sphinxincludegraphics{{detroit_pop}.png}

\sphinxAtStartPar
\sphinxstyleemphasis{Negro Population of Detroit}1900 .. 4,1111910 .. 5,7411920 .. 40,8381925 .. 60,000 (estimated)

\sphinxAtStartPar
Two thirds of this population in 1920 were crowded into three wards— the Third, Fifth and Seventh. Meantime the total population of Detroit has more than doubled in ten years and the people have reached out on all sides to new dwelling places. Have the Negroes no right to rush too? Is it not their duty to seek better homes and, if they do, are they not bound to “move into white neighborhoods” which is simply another way of saying “move out of congested slums”?

\sphinxAtStartPar
Why do they not make their own new settlements then? Because no individual can make a modern real estate development; no group of ordinary individuals can compete with organized real estate interests and get a decent deal. When Negroes have tried it they have usually had miserable results; in Birmingham, Alabama, twenty years since, they bought a nice street and lined it with pretty homes; the city took all its prostitutes and stuck them into a segregated vice district right behind the pretty homes! In Macon, Savannah, New Orleans and Atlanta crime and prostitution have been kept and protected in Negro residence districts. In New York City, for years, no Negro could rent or buy a home in Manhattan outside the “Tenderloin”; and white Religion and Respectability far from stretching a helping hand turned and cursed the blacks when by bribery, politics and brute force they broke into the light and air of Harlem. Some great leaders in Negro philanthropy like Clarence Kelsey formed a financial bloc to push the Negroes out of Harlem, to refuse mortgages to landlords renting to them; but only one practical project of furnishing them decent quarters came to fruition.



\sphinxAtStartPar
Dear God! Must we not live? And if we live may we not live somewhere? And when a whole city full of white folk led and helped by banks, Chambers of Commerce, mortgage companies and “realtors” are combing the earth for every decent bit of residential property for whites, where in the name of God can we live and live decently if not by these same whites? If some of the horror\sphinxhyphen{}struck and law\sphinxhyphen{}worshipping white leaders of Detroit instead of winking at the Ku Klux Klan and admonishing the Negroes to allow themselves to be kicked and killed with impunity—if these would finance and administer 2 decent scheme of housing relief for Negroes it would not be necessary for us to kill white mob leaders \sphinxhyphen{}in order to live in peace and decency. These whited sepulchres pulled that trigger and not the man that held the gun.



\sphinxAtStartPar
But, wail the idiots, Negroes depress real estate values! This is a lie—an ancient and bearded lie. Race prejudice decreases values both real estate and human; crime, ignorance and filth decrease values. But a decent, quiet, educated family buying property in a decent neighborhood will not affect values a bit unless the people in that neighborhood hate a colored skin more than they regard the value of their own property. This has been proven in a thousand instances. Sudden fall in values comes through propaganda and hysteria manipulated by real estate agents or by Southern slave drivers who want their labor to return South; or by ignorant gossip mongers. Usually Negroes do not move into new developments but into districts which well\sphinxhyphen{}to\sphinxhyphen{}do whites are deserting. The fall in values is not due to race but to a series of economic readjustments and often, as in Baltimore, real estate values were actually saved and raised, not lowered, when black folk bought Druid Hill Avenue and adjacent streets. Certainly a flood of. noisy dirty black folk will ruin any neighborhood but they ruin black: property as well as white, and the reason is not their color but their condition. And whom, High Heaven, shall we blame for that?



\sphinxAtStartPar
But these facts make no difference to organized American Negro haters. They are using every effort to maintain and increase race friction. In the South time and time again communities have almost forgotten race lines until the bitter, hate\sphinxhyphen{}preaching liar stirred it up again. The whole present “Anglo\sphinxhyphen{}Saxon” and “race purity” agitation in Virginia has arisen because one white family openly acknowledged its colored grandmother! The whole crusade in Detroit has come to a head because, in 1920, 663,000 Southern whites had migrated and were living in Wisconsin, Michigan, Illinois, Indiana and Ohio. Their numbers are swelling. They are largely clerks, artisans and laborers, not illiterate but ignorant of the modern world and forming by habit the lawless material of mobs. They are ruining the finer democracy of the Middle West and using the Negro as an excuse.

\sphinxAtStartPar
What shall we do? I know a black man. He is a professional man and a graduate of a great eastern school. He has studied abroad. His wife was educated in a good western school and is a quiet housewife. His son is a college graduate and a high school teacher. They have never been arrested. They conduct themselves as cultured folk. This man is living in an apartment in Harlem. He would like more air and sunlight and less noise. He would like a new, small, modern house in the further Bronx or in the hills of Westchester or New Jersey or in the higher part of Queens. He sees daily in the papers new homes advertised suitable to his means—\$500, \$1,000 even \$2000 down, the rest as rent. Can he buy one of these? Not without plotting, deception, insult or murder.

\sphinxAtStartPar
\sphinxstyleemphasis{For instance:} A man bought a modest home in Staten Island. He was a mail carrier with a fine record; his wife was a school teacher, educated and well\sphinxhyphen{}bred. They had four sturdy children in school. As a result he has been mobbed and insulted, his property injured, his glass and shrubbery broken, his insurance cancelled, his life threatened, his existence made miserable.  His neighbors do everything to insult him and his, even to crossing the street to avoid passing him. He sticks to his home even though offered a profit to sell, “on principle”. He is “colored”.

\sphinxAtStartPar
Another man in Detroit bought a fine home in a former exclusive district which is now changing. He was a physician with a large practice, the founder of a hospital, public\sphinxhyphen{}spirited and well\sphinxhyphen{}liked. He had married the daughter of perhaps the greatest of the interpreters of Negro folk songs with world\sphinxhyphen{}wide reputation. He moved in. A mob of thousands appeared, yelling and cursing. They broke his windows, threw out his furniture and he and his family escaped under police protection. He gave up his home, made no resistance, moved back whence he came, filed no protest, made no public complaint. He was “colored”.

\sphinxAtStartPar
A little later another physician in Detroit bought another beautiful home and moved in. A mob—almost the same mob—came, cursed, threw stones and ordered him to move. He gathered his family and friends within and locked the door. Five or six thousand people lined the streets. The police set traffic officers to divert the traffic that could not get through. The mob invaded his yard and approached his doors. He shot and shot to kill. His wife and his friends are now in jail charged with \sphinxstyleemphasis{Murder in the first degree!} He was “colored”’.

\sphinxAtStartPar
Gentle Reader, which of these three examples shall my friend of Harlem follow? Which would you follow if you were “free”, black and 21?


\section{The N.A.A.C.P. and Race Segregation (1934)}
\label{\detokenize{Volumes/41/02/naacp_and_race_segregation:the-n-a-a-c-p-and-race-segregation-1934}}\label{\detokenize{Volumes/41/02/naacp_and_race_segregation::doc}}
\begin{sphinxShadowBox}
\sphinxstylesidebartitle{}

\sphinxAtStartPar
This editorial was preceded by a short note called “A Free Forum” where Du Bois writes, “this year we are going to discuss Segregation and seek not dogma but enlightenment. For this purpose, we are earnestly asking not only that our readers read carefully what is going to be said, but also that they contribute their thoughts and experiences for the enlightenment of other
readers.”
\end{sphinxShadowBox}

\sphinxAtStartPar
There is a good deal of misapprehension as to the historic attitude of the National Association for the Advancement of Colored People and race segregation. As a matter of fact, the Association, while it has from time to time discussed the larger aspects of this matter, has taken no general stand and adopted no general philosophy. Of course its action, and often very effective action, has been in specific cases of segregation where the call for a definite stand was clear and decided. For instance, in the preliminary National Negro Convention which met in New York May 31st and June 1st, 1909, segregation was only mentioned in a protest against Jim\sphinxhyphen{}Crow car laws and that because of an amendment by William M. Trotter. In the First Annual Report, January 1, 1911, the Association evolved a statement of its purpose, which said that “it seeks to uplift the colored men and women of this country by securing to them the full enjoyment of their rights as citizens, justice in all courts, and equality of opportunity everywhere.” Later, this general statement was epitomized in the well\sphinxhyphen{}known declaration: “It conceives its mission to be the completion of the work which the great Emancipator began. It proposes to make a group of ten million Americans free from the lingering shackles of past slavery, physically free from peonage, mentally free from ignorance, politically free from disfranchisement, and socially free from insult.” This phrase which I first wrote myself for the Annual Report of 1915 still expresses pregnantly the object of the N.A.A.C.P. and it has my own entire adherence.

\sphinxAtStartPar
It will be noted, however, that here again segregation comes in only by implication. Specifically, it was first spoken of in the Second Report of the Association, January 1, 1912, when the attempt to destroy the property of Negroes in Kansas City because they had moved into a white section was taken up. This began our fight on a specific phase of segregation, namely, the attempt to establish a Negro ghetto by force of law. This phase of segregation we fought vigorously for years and often achieved notable victories in the highest courts of the land.

\sphinxAtStartPar
But it will be noted here that the N.A.A.C.P. expressed no opinion as to whether it might not be a feasible and advisable thing for colored people to establish their own residential sections, or their own towns; and certainly there was nothing expressed or implied that Negroes should not organize for promoting their own interests in industry, literature or art. Manifestly, here was opportunity for considerable difference of opinion, but the matter never was thoroughly threshed out.

\sphinxAtStartPar
The Association moved on to other matters of color discrimination: the “Full Crew” bills which led to dismissal of so many Negro railway employees; the “Jim\sphinxhyphen{}Crow” car laws on railway trains and street cars; the segregation in government departments. In all these matters, the stand of the Association was clear and unequivocal: it held that it was a gross injustice to make special rules which discriminated against the color of employees or patrons.

\sphinxAtStartPar
In the Sixth Annual Report issued in March, 1916, the seven lines of endeavor of the Association included change of unfair laws, better administration of present laws, justice in the courts, stoppage of public slander, the investigation of facts, the encouragement of distinguished work by Negroes, and organizations.

\sphinxAtStartPar
Very soon, however, there came up a more complex question and that was the matter of Negro schools. The Association had avoided from the beginning any thoroughgoing pronouncement on this matter. In the resolutions of 1909, the conference asked: “Equal educational opportunities for all and in all the states, and that public school expenditure be the same for the Negro and white child.” This of course did not touch the real problem of separate schools. Very soon, however, definite problems were presented to the Association: the exclusion of colored girls from the Oberlin dormitories in 1919; the discrimination in the School of Education at the University of Pennsylvania; and the Cincinnati fight against establishing a separate school for colored children, brought the matter squarely to the front. Later, further cases came; the Brooklyn Girls’ High School, the matter of a colored High School in Indianapolis, and the celebrated Gary case.

\sphinxAtStartPar
Gradually, in these cases the attitude of the Association crystalized. It declared that further extension of segregated schools for particular races and especially for Negroes was unwise and dangerous, and the Association undertook in all possible cases to oppose such further segregation. It did not, however, for a moment feel called upon to attack the separate schools where most colored children are educated throughout the United States and it refrained from this not because it approved of separate schools, but because it was faced by a fact and not a theory. It saw no sense in tilting against windmills.

\sphinxAtStartPar
The case at Cheyney was a variation; here was an old and separate private school which became in effect though not in law a separate public normal school; and in the city of Philadelphia a partial system of elementary Negro schools was developed with no definite action on the part of the N.A.A.C.P.

\sphinxAtStartPar
It will be seen that in all these cases the Association was attacking specific instances and not attempting to lay down any general rule as to how far the advancement of the colored race in the United States was going to involve separate racial action and segregated organization of Negroes for certain ends.

\sphinxAtStartPar
To be sure, the overwhelming and underlying thought of the N.A.A.C.P. has always been: that any discrimination based simply on race is fundamentally wrong, and that consequently purely racial organizations must have strong justification to be admissable. On the other hand, they faced certain unfortunate but undeniable facts. For instance, War came. The Negro was being drafted. No Negro officers were being commissioned. The N. A. A. C. P. asked for the admission of Negroes to the officers’ schools. This was denied. There was only one further thing to do and that was to ask for a school for Negro officers. There arose a bitter protest among many Negroes against this movement. Nevertheless, the argument for it was absolutely unanswerable, and Joel E. Spingarn, Chairman of the Board, supported by the students of Howard University, launched a movement which resulted in the commissioning of seven hundred Negro officers if the A. E. F. In all the British Dominions, with their hundreds of millions of colored folk, there was not a single officer of known Negro blood. The American Negro scored a tremendous triumph against the Color Line by their admitted and open policy of segregation. This did not mean that Mr. Spingarn or any of the members of the N. A. A. C. P. thought it right that there should be a separate Negro camp, but they thought a separate Negro camp and Negro officers was infinitely better than no camp and no Negro officers and that was the only practical choice that lay before them.

\sphinxAtStartPar
Similarly, in the question of the Negro vote, the N. A. A. C. P. began in 1920 an attempt to organize the Negro vote and cast it in opposition to open enemies of the Negro race who were running for office. This was without doubt a species of segregation. It was appealing to voters on the grounds of race, and it brought for that reason considerable opposition. Nevertheless, it could be defended on the ground that the election of enemies of the Negro race was not only a blow to that race but to the white race and to all civilization. And while our attitude, even in the Parker case, has been criticized, it has on the whole found abundant justification.

\sphinxAtStartPar
The final problem in segregation presented to us was that of the Harlem Hospital. Here was a hospital in the center of a great Negro population which for years did not have and would not admit a single Negro physician to its staff. Finally, by agitation and by political power, Negroes obtained representation on the staff in considerable numbers and membership on the Board of Control. It was a great triumph. But it was accompanied by reaction on the part of whites and some Negroes who had opposed this movement, and an attempt to change the status of the hospital so that it would become a segregated Negro hospital, and so that presumably the other hospitals of the city would continue to exclude Negroes from their staffs. With this arose a movement to establish Negro hospitals throughout the United States.

\sphinxAtStartPar
Here was an exceedingly difficult problem. On the one hand, there is no doubt of the need of the Negro population for wider and better hospitalization; and of the demand or the part of Negro physicians for opportunities of hospital practice. This was illustrated by the celebrated Tuskegee hospital where nearly all the Negro veterans are segregated but where an efficient Negro staff has been installed. Perhaps nothing illustrates better than this the contradiction and paradox of the problem of race segregation in the United States, and the problem which the N.A.A.C.P. faced and still faces.

\sphinxAtStartPar
The N.A.A.C.P. opposed the initial establishment of the hospital at Tuskegee although it is doubtful if it would have opposed such a hospital in the North. On the other hand, once established, we fought to defend the Tuskegee hospital and give it widest opportunity.

\sphinxAtStartPar
In other words, the N.A.A.C.P. has never officially opposed separate Negro organizations—such as churches, schools and business and cultural organizations. It has never denied the recurrent necessity of united separate action on the part of Negroes for self\sphinxhyphen{}defense and self\sphinxhyphen{}development; but it has insistently and continually pointed out that such action is in any case a necessary evil involving often a recognition from within of the very color line which we are fighting without. That race pride and race loyalty, Negro ideals and Negro unity, have a place and function today, the N.A.A.C.P. never has denied and never can deny.

\sphinxAtStartPar
But all this simply touches the whole question of racial organization and initiative. No matter what we may wish or say, the vast majority of the Negroes in the United States are born in colored homes, educated in separate colored schools, attend separate colored churches, marry colored mates, and find their amusement in colored Y.M.C.A.’s and Y.W.C.A.’s. Even in their economic life, they are gradually being forced out of the place in industry which they occupied in the white world and are being compelled to seek their living among themselves. Here is segregation with a vengeance, and its problems must be met and its course guided. It would be idiotic simply to sit on the side lines and yell: “No segregation’’ in an increasingly segregated world.

\sphinxAtStartPar
On the other hand, the danger of easily and eagerly yielding to suggested racial segregation without reason or pressure stares us ever in the face. We segregate ourselves. We herd together. We do things such as this clipping from the Atlanta Constitution indicates:
\begin{quote}

\sphinxAtStartPar
“A lecture on the raising of Lazarus from the dead will be delivered at the city auditorium on Friday night. The Big Bethel choir will sing and the Graham Jackson band will give additional music. Space has been set aside for white people.”
\end{quote}

\sphinxAtStartPar
The “Jim Crow” galleries of Southern moving picture houses are filled with some of the best Negro citizens. Separate schools and other institutions have been asked by Negroes in the north when the whites had made no real demand.

\sphinxAtStartPar
Such are the flat and undeniable facts. What are we going to do about them? We can neither yell them down nor make them disappear by resolutions. We must think and act. It is this problem which THE Crisis dgsires to discuss during the present year in all its phases and with ample and fair representation to all shades of opinion.


\bigskip\hrule\bigskip


\sphinxAtStartPar
\sphinxstyleemphasis{Citation:} Du Bois, W.E.B. 1934. “The N.A.A.C.P. and Race Segregation.” 41(2):52\sphinxhyphen{}53.


\section{Segregation in the North (1934)}
\label{\detokenize{Volumes/41/04/segregation_in_the_north:segregation-in-the-north-1934}}\label{\detokenize{Volumes/41/04/segregation_in_the_north::doc}}
\sphinxAtStartPar
I have with interest the various criticisms on my recent discussions of segregation. Those like that of Mr. Pierce of Cleveland, do not impress me. I am not worried about being inconsistent. What worries me is the Truth. I am talking about conditions in 1934 and not in 1910. I do not care what I said in 1910 or 1810 or in B.C. 700.

\sphinxAtStartPar
The arguments of Walter White, George Schuyler and Kelly Miller have logic, but they seem to me quite beside the point. In the first place, Walter White is white. He has more white companions and friends than colored. He goes where he will in New York City and naturally meets no Color Line, for the simple and sufficient reason that he isn’t “colored”; he feels his new freedom in bitter contrast to what he was born to in Georgia. This is perfectly natural and he does what anyone else of his complexion would do.

\sphinxAtStartPar
But it is fantastic to assume that this has anything to do with the color problem in the United States. It naturally makes Mr. White an extreme opponent of any segregation based on a myth of race. But this argument does not apply to Schuyler or Miller or me. Moreover, Mr. White knows this. He moved once into a white apartment house and it went black on him. He now lives in a colored apartment house with attendant limitations. He once took a friend to dine with him at the celebrated café of the Lafayette Hotel, where he had often been welcomed. The management humiliated him by refusing to serve Roland Hayes.

\sphinxAtStartPar
The attitudes of Schuyler and Kelly Miller are historically based on the amiable assumption that there is little or no segregation in the North, and that agitation and a firm stand is making this disappear; that obvious desert and accomplishment by Negroes can break down prejudice. This is a fable. I once believed it passionately. It may become true in 250 or 1,000 years. Now it is not true. No black man whatever his culture or ability is today in America regarded as a man by any considerable number of white Americans. The difference between North and South in the matter of segregation is largely a difference of degree; of wide degree certainly, but still of degree.

\sphinxAtStartPar
In the North, neither Schuyler nor Kelly Miller nor anyone with a visible admixture of Negro blood can frequent hotels or restaurants. They have difficulty in finding dwelling places in better class neighborhoods. They occupy “Lower 1” on Pullmans, and if they are wise, they do not go into dining cars when any large number of white people is there. Their children either go to colored schools or to schools nominally for both races, but actually attended almost exclusively by colored children. In other words, they are confined by unyielding public opinion to a Negro world. They earn a living on colored newspapers or in colored colleges, or other racial institutions. They treat colored patients and preach to colored pews. Not one of the 12 colored Ph.D.’s of last year, trained by highest American and European standards, is going to get a job in any white university. Even when Negroes in the North work side by side with whites, they are Segregated. like the postal clerks, or refused by white unions or denied merited promotion.

\sphinxAtStartPar
No matter how much we may fulminate about “No segregation,” there stand the flat facts. Moreover, this situation has in the last quarter century been steadily growing worse. Mr. Spingarn may ask judicially as to whether or not the N.A.A.C.P. should change its attitude toward segregation. The point that he does not realize is that segregation has changed its attitude toward the N.A.A.C.P. The higher the Negro climbs or tries to climb, the more pitiless and unyielding the color ban. Segregation may be just as evil today as it was in 1910, but it is more insistent, more prevalent and more unassailable by appeal or argument. The pressing problem is: What are we going to do about it?

\sphinxAtStartPar
In 1910, colored men could be entertained in the best hotels in Cleveland, Detroit and Chicago. Today, there is not a single Northern city, except New York, where a Negro can be a guest at a first\sphinxhyphen{}class hotel. Not even in Boston is he welcome; and in New York, the number of hotels where he can go is very small. Roland Hayes was unable to get regular hotel accommodations, and Dr. Moton only succeeds by powerful white influence and by refraining from use of the public dining room or the public lobbies.

\sphinxAtStartPar
If as Spingarn asserts, the N.A.A.C.P. has conducted a quarter\sphinxhyphen{}century campaign against segregation, the net result has been a little less than nothing. We have by legal action steadied the foundation so that in the future, segregation must be by wish and will and not law, but beyond that we have not made the slightest impress on the determination of the overwhelming mass of white Americans not to treat Negroes as men.

\sphinxAtStartPar
These are unpleasant facts. We do not like to voice them. The theory is that by maintaining certain fictions of law and administration, by whistling and keeping our courage up, we can stand on the “principle” of no segregation and wait until public opinion meets our position. But can we do this? When we were living in times of prosperity; when we were making post\sphinxhyphen{}war incomes; when our labor was in demand, we perhaps could afford to wait. But today, faced by starvation and economic upheaval, and by the question of being able to survive at all in this land in the reconstruction that is upon us, it is ridiculous not to see, and criminal not to tell, the colored people that they can not base their salvation upon the empty reiteration of a slogan.

\sphinxAtStartPar
What then can we do? The only thing that we not only can, but must do, is voluntarily and insistently to organize our economic and social power, no matter how much segregation it involves. Learn to associate with ourselves and to train ourselves for effective association. Organize our strength as consumers; learn to co\sphinxhyphen{}operate and use machines and power as producers; train ourselves in methods of democratic control within our own group. Run and support our own institutions.

\sphinxAtStartPar
We are doing this partially now, only we are doing it under a peculiar attitude of protest, and with only transient and distracted interest. A number of excellent young gentlemen in Washington, having formed a Negro Alliance, proceed to read me out of the congregation of the righteous use I dare even discuss segregation. But who are these young men? The products of a segregated school system; the talent selected by Negro teachers; the persons who can today, in nine cases out of ten, earn only a living through segregated Negro social institutions. These are the men who are yelling against segregation. If most of them had been educated in the mixed schools in New York instead of the segregated schools of Washington, they never would have seen college, because Washington picks out and sends ten times as many Negroes to college as New York does.

\sphinxAtStartPar
It would, of course, be full easy to deny that this voluntary association for great social and economic ends is segregation; and if I had done this in the beginning of this debate, many people would have been easily deceived, and would have yelled “No segregation” with one side of their mouths and “Race pride and Race initiative’ with the other side. No such distinction can possibly be drawn. Segregation may be compulsory by law or it may be compulsory by economic or social condition, or it may be a matter of free choice. At any rate, it is the separation of human beings and separation despite the will to humanity. Such separation is evil; it leads to jealousy, greed, nationalism and war; and yet it is today and in this world inevitable; inevitable to Jews because of Hitler; inevitable to Japanese because of white Europe; inevitable to Russia because of organized greed over all the white world; inevitable to Ethiopia because of white armies and navies; inevitable, because without it, the American Negro will suffer evils greater than any possible evil of separation: we would suffer the loss of self\sphinxhyphen{}respect, the lack of faith in ourselves, the lack of knowledge about ourselves, the lack of ability to make a decent living by our own efforts and not by philanthropy.

\sphinxAtStartPar
This situation has been plunged into crisis and precipitated to an open demand for thought and action by the Depression and the New Deal. The government, national and state, is helping and guiding the individual. It has entered and entered for good into the social and economic organization of life. We could wish, we could pray, that this entrance could absolutely ignore lines of race and color, but we know perfectly well it does not and will not, and with the present American opinion, it cannot. The question is then, are we going to stand out and refuse the inevitable and inescapable government aid because we first wish to abolish the Color Line? This is not simply tilting at windmills; it is, if we are not careful, committing race suicide.


\subsection{No Segregation}
\label{\detokenize{Volumes/41/04/segregation_in_the_north:no-segregation}}
\sphinxAtStartPar
Back of all slogans lies the difficulty that the meanings may change without changing the words. For instance, “no segregation” may mean two very different things:
\begin{enumerate}
\sphinxsetlistlabels{\arabic}{enumi}{enumii}{}{.}%
\item {} 
\sphinxAtStartPar
A chance for the Negro to advance without the hindrances which arise when he is segregated from the main group, and the main social institutions upon which society depends. He becomes, thus, an outsider, a hanger on, with no chance to function properly as a man.

\item {} 
\sphinxAtStartPar
It may mean utter lack of faith of Negroes in Negroes, and the desire to escape into another group, shirking, on the other hand, all responsibility for ignorance, degradation and lack of experience among Negroes, while asking admission into the other group on terms of full equality and with full chance for individual development.

\end{enumerate}

\sphinxAtStartPar
It is in the first sense that I have always believed and used the slogan: “No Segregation.” On the other hand, in the second sense, I have no desire or right to hinder or estop those persons who do not want to be Negroes. But I am compelled to ask the very plain and pertinent question: Assuming for the moment that the group into which you demand admission does not want you, what are you going to do about it? Can you demand that they want you? Can you make them by law or public opinion admit you when they are supreme over this same public opinion and make these laws? Manifestly, you cannot. Manifestly your admission to the other group on the basis of your individual desert and wish, can only be accomplished if they, too, join in the wish to have you. If they do so join, all problems based mostly on race and color disappear, and there remains only the human problems of social uplift and intelligence and group action. But there is in the United States today no sign that this objection to the social and even civic recognition of persons of Negro blood is going to occur during the life of persons now living. In which case there can be only one meaning to the slogan “No Segregation;” and that is, no hindrance to my effort to be a man. If you do not wish to associate with me, I am more than willing to associate with myself. Indeed, I deem it a privilege to work with and for Negroes, only asking that my hands be not tied nor my feet hobbled.


\subsection{Objects of Segregation}
\label{\detokenize{Volumes/41/04/segregation_in_the_north:objects-of-segregation}}
\sphinxAtStartPar
What is the object of those persons who insist by law, custom and propaganda to keep the American Negro separate in rights and privileges from other citizens of the United States? The  real object, confessed or semiconscious, is to so isolate the Negro that he will be spiritually bankrupt, physically degenrate, and economically dependent.

\sphinxAtStartPar
Against this it is the bounden duty of every Negro and every enlightened American to protest; to oppose the policy so far as it is manifest by laws; to agitate against customs by revealing facts; and to appeal to the sense of decency and justice in all American citizens.

\sphinxAtStartPar
I have never known an American Negro who did not agree that this was a proper program. Some have disagreed as to the emphasis to be put on this and that method of protest; on the efficacy of any appeal against American prejudice; but all Negroes have agreed that segregation is bad and should be opposed.

\sphinxAtStartPar
Suppose, however, that this appeal is ineffective or nearly so? What is the Negro going to do? There is one thing that he can or must do, and that is to see to it that segregation does not undermine his health; does not leave him spiritually bankrupt; and does not make him an economic slave; and he must do this at any cost.

\sphinxAtStartPar
If he cannot live in sanitary and decent sections of a city, he must build his own residential quarters, and raise and keep them on a plane fit for living. If he cannot educate his children in decent schools with other children, he must, nevertheless, educate his children in decent Negro schools and arrange and conduct and oversee such schools. If he cannot enter American industry at a living wage, or find work suited to his education and talent, or receive promotion and advancement according to his desserts, he must organize his own economic life so that just as far as possible these discriminations will not reduce him to abject exploitation.

\sphinxAtStartPar
Everyone of these movements on the part of colored people are not only necessary, but inevitable. And at the same time, they involve more or less active segregation and acquiescence in segregation.

\sphinxAtStartPar
Here again, if there be any number of American Negroes who have not in practical life made this fight of self\sphinxhyphen{}segregation and self\sphinxhyphen{}association against the compulsory segregation forced upon them, I am unacquainted with such persons. They may, of course, explain their compulsory retreat from a great ideal, by calling segregation by some other name. They may affirm with fierce insistency that they will never, no never, under any circumstances acquiesce in segregation. But if they live in the United States in the year of our Lord 1934, or in any previous year since the foundation of the government, they are segregated; they accept segregation, and they segregate themselves, because they must. From this dilemma I see no issue.


\subsection{Boycott}
\label{\detokenize{Volumes/41/04/segregation_in_the_north:boycott}}
\sphinxAtStartPar
Whither does all this sudden talk of segregation lead? May I illustrate by an appositive example. Several times \sphinxstylestrong{The Crisis} has commended what seemed to us the epoch\sphinxhyphen{}making work of The \sphinxstyleemphasis{Chicago Whip} when it instituted boycotts against stores in the black belt which refused to employ Negro clerks. Recently, in Washington, a group of young intellectuals sought to do the same thing but fell afoul of the ordinances against picketing. These efforts illustrate the use of mass action by Negroes who take advantage of segregation in order to strengthen their economic foundation. The Chicago success was applauded by every Negro in the land and the Washington failure deserved success. Today the same sort of move is being made in Richmond.

\sphinxAtStartPar
Yet, mind you, both these efforts were efforts toward segregation. The movement meant, in essence, Negro clerks for Negro customers. Of course, this was not directly said but this is what it amounted to. The proponents knew that Negro clerks would only be hired if Negro customers demanded it, and if the Negro customers, as happened in some cases, did not want to be waited on by Negro clerks, or even felt insulted if the Negro clerk came to them, then the proprietors had a perfect right to refuse to employ Negro clerks. Indeed, this happened in several cases in Harlem, New York.

\sphinxAtStartPar
And yet given the practically compulsory segregation of residence, and the Negro race is not only justified but compelled to invoke the additional gesture which involves segregation by asking Negro clerks for Negro customers. Of course, the logical demand of those who refuse to contemplate any measure of segregation, would be to demand the employment of Negro clerks everywhere in the city, and in all stores, at least in the same proportion that the Negro population bears to the total population. This was not demanded because such a demand would be futile and have no implement for its enforcement. But you can enforce the employment of Negroes by commercial houses in a Negro community and this ought to be done and must be done, and this use of the boycott by American Negroes must be widened and systematized, with care, of course, to avoid the ridiculous laws which make boycotts in so many cases illegal.

\sphinxAtStartPar
The funny postscript to all this, is that the same group of young Negroes who sought in Washington to fight segregation with segregation, or better to build a decent living on compulsory segregation, immediately set up a yell of “No Segregation,” when they read \sphinxstyleemphasis{The Crisis.}


\subsection{Integration}
\label{\detokenize{Volumes/41/04/segregation_in_the_north:integration}}
\sphinxAtStartPar
Extreme opponents of segregation act as though there was but one solution of the race problem, and that, complete integration of the black race with the white race in America, with no distinction of color in political, civil or social life. There is no doubt but what this is the great end toward which humanity is tending, and that so long as there are artificially emphasized differences of nationality, race and color, not to mention the fundamental discriminations of economic class, there will be no real Humanity.

\sphinxAtStartPar
On the other hand, it is just as clear, that not for a century and more probably not for ten centuries, will any such consummation be reached. No person born will ever live to see national and racial distinctions, altogether abolished, and economic distinctions will last many a day.

\sphinxAtStartPar
Since this is true, the practical problem that faces us is not a choice between segregation and no segregation, between compulsory interferences with human intercourse and complete liberty of contact; the thing that faces us is given varying degrees of segregation. How shall we conduct ourselves so that in the end human differences will not be emphasized at the expense of human advance.

\sphinxAtStartPar
It is perfectly certain that, not only shall we be compelled to submit to much segregation, but that sometimes it will be necessary to our survival and a step toward the ultimate breaking down of barriers, to increase by voluntary action our separation from our fellowmen.

\sphinxAtStartPar
When my room\sphinxhyphen{}mate gets too noisy and dirty, I leave him; when my neighbors get too annoying and insulting I seek another home; when white Americans refuse to treat me as a man, I will cut my intercourse with white Americans to the minimum demanded by decent living.

\sphinxAtStartPar
It may be and often has been true that oppression and insult have become so intense and so unremitting that there is no alternative left to self\sphinxhyphen{}respecting men but to herd by themselves in self\sphinxhyphen{}defense, until the attitude of the world changes. It happens that today is peculiarly a day when such voluntary union for self\sphinxhyphen{}expression and self\sphinxhyphen{}defense is forced upon large numbers of people. We may rail against this. We may say that it is not our fault, and it certainly is not. Nevertheless, to do nothing in the face of it: to accept opposition without united counter opposition is the program of fools.

\sphinxAtStartPar
Moreover if association and contact with Negroes is distasteful to you, what is it to white people? Remember that the white people of America will certainly never want us until we want ourselves. We excuse ourselves in this case and say we do not hate Negroes but we do hate their condition, and immediately the answer is thrown back on us in the very words. Whose job is it to change that condition? The job of the white people or the job of the black people themselves, and especially of their uplifted classes?


\bigskip\hrule\bigskip


\sphinxAtStartPar
\sphinxstyleemphasis{Citation:} Du Bois, W.E.B. 1934. “Segregation in the North.”  41(4):115\sphinxhyphen{}117.


\section{Segregation (1934)}
\label{\detokenize{Volumes/41/05/segregation:segregation-1934}}\label{\detokenize{Volumes/41/05/segregation::doc}}
\sphinxAtStartPar
A girl with brown and serious face, came to me after a lecture. She was not satisfied with what I had said, nor to my answer to her questions from the floor. She said: “It seems to me you used to fight Segregation, and that now you are ready to compromise.” I answered: If you mean by fighting Segregation, fighting with fists, this is a thing I have never advocated, because it seems to me that such a policy always loses more than it gains. I have fought Segregation and other evils with reason, with facts, and with agitation. In this way, I’m still fighting. I said in the past that Segregation is wrong. I am still saying it.

\sphinxAtStartPar
As to my willingness to compromise, that depends upon what you mean by compromise. If you mean by compromise, taking less than you want, but not wanting less, then I do compromise. I take what I can get, as I always have in the past. Yet I want all. And if in this matter of taking what I can get, I compromise, so do you, and so must you.

\sphinxAtStartPar
Moreover, and beyond this, I fight Segregation with Segregation, and I do not consider this compromise. I consider it common sense.

\sphinxAtStartPar
Out beyond me, where I write, lies a slum, Beaver Slide; named after an Atlanta Chief of Police, who went that way hurriedly one night because of certain dark dangers. I have seen this slum now and again for thirty years: Its drab and crowded houses; its mud, dust and unpaved streets; its lack of water, light and sewage; its crowded and unpoliced gloom. Just now, it seems certain, that the United States Government is going to spend \$2,000,000 to erase this slum from the face of the earth, and put in its place, beautiful, simple, clean homes, for poor colored people, with all modern conveniences.

\sphinxAtStartPar
This is Segregation. It is Segregation by the United States Government. These homes are going to be for Negroes, and only for Negroes; and yet I am a strong advocate for this development. If this is compromise with Segregation; I am compromising. Not only that, but strange enough, I hear no opposition even from the Washington Brain Trust. Dr. George Haynes has uttered no word of protest. The embattled Youth of Washington have not gone on record, and why? Is it because they or I like the necessity of having public development for social uplift divided along an artificial Color Line?’ Certainly not. But it’s because I have sense enough to know, and I hope they have, that either we get a segregated development here, or we get none at all; and the advantage of decent homes for five thousand colored people, outweighs any disadvantage which will come from this development.

\sphinxAtStartPar
I say again, if this is compromise; if this is giving up what I have advocated for many years; the change, the reversal, bothers me not at all. But Negro poverty and idleness, and distress, they bother me, and always will.


\bigskip\hrule\bigskip


\sphinxAtStartPar
\sphinxstyleemphasis{Citation:} “Segregation”.” 1934. Editorial.  41(5):147.


\section{Color Caste in the United States (1933)}
\label{\detokenize{Volumes/40/03/Color_caste_in_the_united_states:color-caste-in-the-united-states-1933}}\label{\detokenize{Volumes/40/03/Color_caste_in_the_united_states::doc}}
\begin{sphinxShadowBox}
\sphinxstylesidebartitle{}

\sphinxAtStartPar
This ran as an article in \sphinxstylestrong{The Crisis} but in form and content is similar to the editorial content he routinely produced.
\end{sphinxShadowBox}

\sphinxAtStartPar
There are a large number of well meaning citizens of this country who are under the impression that the main lines of the American Negro problem are settled, and that while there is a deal more advance to be made, nevertheless, the average Negro, who is not too impatient, should be willing from now on to proceed toward his ultimate goal by quiet progress and unemotional appeal.

\sphinxAtStartPar
This is not true. It is so far from the truth that it is probably a fact that if inhabitants of a modern country, like France, England or Germany, who know the meaning of freedom, were subjected to the caste restrictions which surround American Negroes, they would without hesitation burst into flaming revolution.

\sphinxAtStartPar
Let us consider the facts.

\sphinxAtStartPar
We may begin with marriage. A Negro in this country may not, in 26 of the 38 states, marry the person whom he wishes to marry, unless the partner is of Negro descent. Even if he has been married legally elsewhere, he may not in most Southern states live with his wife on pain of fine or imprisonment, unless she is also of Negro descent. A colored girl who is with a child by a white man in the South has no legal way of making her child legitimate, and in most Southern states and many Northern could get no standing in court. The very fact of having Negro blood is regarded in most Southern states as being such a stigma that the false allegation of it may be basis of an action for damages.

\sphinxAtStartPar
The Negro married couple may not live where they wish or in a home that they are able to buy. By law or custom, covenant or contract, or by mob violence, they are everywhere in the United States restricted in their right of domicile and for the most part must live in the worst parts of the city, and on the poorest land in the country; their sections receive from the local government the least attention and they are peculiarly exposed to crime and disease.

\sphinxAtStartPar
Negroes are especially restricted in the chance to earn a living. If they are farmers in the South, the quality and situation of the land they can buy, their access to market, their freedom to plant and do business is seriously curtailed. They cannot in the South take part in co\sphinxhyphen{}operative farming or farmers’ organizations, except in organizations which are composed entirely of Negroes. They are restricted and systematically cheated in the selling of their crop. They have little chance for the education of their children and they have no voice in their own government and taxation, and over wide areas in Louisiana, Mississippi, Alabama and Georgia Negro farm labor is held in actual peonage.

\sphinxAtStartPar
In general, Negroes are “segregated,” which means that their normal, social development in all lines is narrowed, curtailed or stopped. They cannot develop as a separate group without police power and inner social sanctions, economic protection and the power of public opinion. Where, for instance, schools are separated by race, the colored race has no power to select teachers,  choose text books or administer its own schools, nor does it have voice in spending the school funds. It cannot develop as a part of the larger group because the developing and differentiating individual who by ability, education, wealth or character seeks to rise from the average level of weakness, ignorance, poverty, and delinquency of his group is clubbed back by the color bar and condemned to submersion or fruitless revolt.

\sphinxAtStartPar
This is especially illustrated in the Negroes’ efforts to earn a living. In manufacturing, business and industry, they cannot get capital to carry on enterprises unless they raise it in small sums from their own poverty. In case they do, their small capital, their inability to gain practical experience by contact, and their lack of credit dooms them from the start. In trade, Negroes are greatly restricted. Their small banks are entirely at the mercy of the big credit organizations. The retail stores stand little chance in competition with the chain stores, and in the chain stores for the most part no Negroes are hired as managers and few as clerks. In insurance, their opportunity is restricted by the white state insurance officials and by limited knowledge and opportunity for investment of funds.

\sphinxAtStartPar
Negroes forming one\sphinxhyphen{}tenth of the population own but 1/140 per cent of the wealth. Their per capita wealth is \$215 as compared with over \$3,000 for the average American. They can command certainly not more than 1/10,000 of current bank credit.

\sphinxAtStartPar
If they learn mechanical trades they are restricted by the unions, most of which either by actual legislation or by vote of local unions will not allow them to join. If they do not join the Union, they only get a chance to work as scabs, in which case they are in danger of mob violence by their fellow workers and of a widespread propaganda which represents them as scabbing by choice or ignorance. If they attempt to form their own unions, they can seldom find enough Negroes in a given locality skilled to take over whole units of work. Negroes must, therefore, compete mostly for unskilled and semi\sphinxhyphen{}skilled labor below the current rate of wages. They cannot expect promotion. They have, therefore, no voice in the conduct of industry, and are liable to dismissal with the least consideration of any workers. Particularly, when it comes to new kinds of work and new machines and new methods, they are the last to be given an opportunity to learn and they are admitted to a minimum of wage and consideration.

\sphinxAtStartPar
In transportation, they can only work as common laborers and porters. The railway unions of engineers, conductors and firemen by constitutional enactment will not admit them. There is a colored Firemen’s Union but it has been fought in every way by the white firemen, who in the last few years have resorted to open murder. The union of Pullman Porters is excluded from the effective councils of the American Federation of Labor.

\sphinxAtStartPar
In public and civil service they are especially restricted, even under Civil Service rules. The United States Civil Service Commission requires the filing of a photograph with each application which makes and was designed to make the systematic exclusion of successful Negro applicants easier. Government trade\sphinxhyphen{}unions, like the postal clerks, segregate Negroes and thus disfranchise them in negotiations. In state, city and county civil service a few Negroes get in in the North; but in all the states of the Southern South any black man, whether he can read or write, or owns property, or whatever his qualifications, can be kept from effectively exercising the right to vote. In those Southern States where eight of the twelve million Negroes live there is not a single Negro member of the legislature, not a single Negro who holds a county office, not a single member of a city council, and of the 1,700 Southern cities and towns of 1,000 and more inhabitants, not 50 have a single Negro policeman. Negroes are thus taxed without representation and receive scant consideration from the officers of the administration.

\sphinxAtStartPar
On the stage and in literature and art, the Negro has some opportunity but his genius is limited by a public who will not endure any portrayal of a Negro save as a fun\sphinxhyphen{}maker, a moron or criminal. There have been some few exceptions to this but they emphasize the rule. Neither Paul Robeson nor Jules Bledsoe was allowed to sing the title role in “Emperor Jones” at the Metropolitan Opera; it was given to a white man blacked\sphinxhyphen{}up. Colored artists and writers who portray what they wish and feel and know get but a restricted and cool white audience and their colored audience is not only limited by poverty and ignorance, but its enlightened elements get their standards of judgment indirectly from the whites.

\sphinxAtStartPar
The Negro is forced into crime. His lawyers stand small opportunity in the courts of the South and restricted opportunity in the North. For the most part, the Negro is arrested by an ignorant, prejudiced and venal white policeman and his mere arrest usually means conviction. He gets little to no legal defense, and even if innocent, is apt to receive the “limit of the law.” His crime in the South is traded in so that many states actually make a surplus income by selling the work of criminals to private profiteers, The courts for years in the South have been made instruments for reducing the Negro to peonage and slavery. Any attempt to measure the real amount of Negro crime by counting the persons arrested and convicted is nonsense, and the treatment of Negroes in confinement can be read in Spivak’s “Georgia N{[}*****{]}.”

\sphinxAtStartPar
In the professions, law, medicine and the ministry, the Negro must serve mainly his own race; for this reason, he is tempted or compelled to get cheaper or poorer preparation, to lack contact with the best and newest thought and method, to receive lower compensation and even if he surmounts these obstacles, to have his skill discounted and his opportunity curtailed on account of race.

\sphinxAtStartPar
There are (1926) in the United States, 232,154 churches, of which 42,585 are confined to Negroes, This leaves 189,569 white churches. Of at least 175,000 of these, no Negro can be a member. There are 5,535 Y.M.C.A. and Y.W.C.A. organizations, of which 200 are for Negroes. No Negro can join at least 5,000 of the other associations,

\sphinxAtStartPar
In domestic and personal service, the lower the grade of work, the more unprotected the women and children are, the larger opportunity is given to Negroes. Wherever the service is standardized and given a decent wage as in most first class hotels they are excluded.

\sphinxAtStartPar
The opportunity of the Negro for education is limited. In the sixteen former slave states in 1930, over a million Negro children of school age were not in school a single day in the year; and half the Negro children are not in regular attendance. Southern Negro children, forming a third of the school population, received 1/10 of the school funds, and the million and a half who attended school had an average term of only six months.

\sphinxAtStartPar
In the mixed schools of the North, colored children are often neglected and discouraged and if they are migrants from the South, suffer retardation because of lack of educational opportunity in early years.

\sphinxAtStartPar
The little money which the Negro earns is spent under equally difficult circumstances. In the South, he cannot travel without the insult of separate and inferior cars, for which he pays the standard price. On many express trains he cannot travel at all. In Texas, Oklahoma and Arkansas he cannot use a sleeping car at all and in other Southern States he can hire a berth only under humiliating difficulties. In Southern railway stations he must often use side entrances. He cannot take recreation in public parks and museums, or in public buildings, at public lectures, concerts and entertainments, He is not admitted to public competitions on equal terms. He can attend most Northern colleges, but cannot instruct or teach or take advantage of scientific foundations, so that the whole field of graduate training and scientific research is thus seriously narrowed for him.

\sphinxAtStartPar
He lives under a stigma which is increased by deliberate propaganda: by the teaching of professors and school teachers, by the words of textbooks, by the distorted message of history, and by the deliberate misinterpretation of science. And above all, it is practically impossible for any Negro in the United States, no matter how small his heritage of Negro blood may be, to meet his fellow citizens on terms of social equality without being made the subject of all sorts of discriminations, embarassments and insults. Finally, the persons thus called and treated and measured as “Negroes” may be predominantly of Nordic blood with perhaps one Negro great\sphinxhyphen{}great grandparent in 16.

\sphinxAtStartPar
It would be untrue to say that all these restrictions happen to all Negroes at all times. There are innumerable exceptions, personal and geographical. Nevertheless, by and large, this is a true picture of the caste situation in the United States. today. On the other hand, those people who insist that color discrimination has decreased are right. There was a time 100 years ago, when a Negro had no rights which a white man was bound to respect. Even 50 years ago, a white man had a right to knock a Negro down for any offense, real or fancied and walk away without danger or arrest. In certain districts of Mississippi, Alabama, Louisiana, Georgia and Texas this is still true. In every Southern state and in some Northern states, the case of a white man who kills a Negro and is punished for it is regarded by the newspapers as unusual news. Today, in any altercation between a white and black man the burden of proof is on the black man and the chances of his getting the worst of it at the hands of the law are ten times as great as those of the white man.

\sphinxAtStartPar
This, then, is the situation, and the question is, what are modern, educated people going to do about, whether they are white or black?


\bigskip\hrule\bigskip


\sphinxAtStartPar
\sphinxstyleemphasis{Citation:} “Color Caste in the United States.” 1933. Editorial.  40(3):59\sphinxhyphen{}60, 70.


\chapter{Racism and Discrimination}
\label{\detokenize{Sections/racism:racism-and-discrimination}}\label{\detokenize{Sections/racism::doc}}
\sphinxAtStartPar
Editorials on prejudice and discrimination
\begin{itemize}
\item {} 
\sphinxAtStartPar
{\hyperref[\detokenize{Volumes/01/03/social_equality::doc}]{\sphinxcrossref{“Social Equality” (1911)}}}

\item {} 
\sphinxAtStartPar
{\hyperref[\detokenize{Volumes/01/03/ashamed::doc}]{\sphinxcrossref{“Ashamed” (1911)}}}

\item {} 
\sphinxAtStartPar
{\hyperref[\detokenize{Volumes/03/04/light::doc}]{\sphinxcrossref{Light (1912)}}}

\item {} 
\sphinxAtStartPar
{\hyperref[\detokenize{Volumes/06/02/logic::doc}]{\sphinxcrossref{Logic (1913)}}}

\item {} 
\sphinxAtStartPar
{\hyperref[\detokenize{Volumes/10/01/clansman::doc}]{\sphinxcrossref{The Clansman (1915)}}}

\item {} 
\sphinxAtStartPar
{\hyperref[\detokenize{Volumes/12/05/conduct_not_color::doc}]{\sphinxcrossref{Conduct, Not Color (1916)}}}

\item {} 
\sphinxAtStartPar
{\hyperref[\detokenize{Volumes/12/06/migration::doc}]{\sphinxcrossref{Migration (1916)}}}

\item {} 
\sphinxAtStartPar
{\hyperref[\detokenize{Volumes/24/01/slavery::doc}]{\sphinxcrossref{Slavery (1921)}}}

\item {} 
\sphinxAtStartPar
{\hyperref[\detokenize{Volumes/31/04/newer_south::doc}]{\sphinxcrossref{The Newer South (1926)}}}

\item {} 
\sphinxAtStartPar
{\hyperref[\detokenize{Volumes/34/02/higher_friction::doc}]{\sphinxcrossref{The Higher Friction (1927)}}}

\item {} 
\sphinxAtStartPar
{\hyperref[\detokenize{Volumes/34/09/prejudice::doc}]{\sphinxcrossref{Prejudice (1927)}}}

\item {} 
\sphinxAtStartPar
{\hyperref[\detokenize{Volumes/40/07/protest::doc}]{\sphinxcrossref{A Protest (1933)}}}

\end{itemize}


\section{“Social Equality” (1911)}
\label{\detokenize{Volumes/01/03/social_equality:social-equality-1911}}\label{\detokenize{Volumes/01/03/social_equality::doc}}
\sphinxAtStartPar
At last we have a definition of the very elusive phrase “Social Equality” as applied to the Negro problem. In stating their grievances colored people have recently specified these points:
\begin{enumerate}
\sphinxsetlistlabels{\arabic}{enumi}{enumii}{}{.}%
\item {} 
\sphinxAtStartPar
Disfranchisement, even of educated Negroes.

\item {} 
\sphinxAtStartPar
Curtailment of common school training.

\item {} 
\sphinxAtStartPar
Confinement to “Ghettos.”

\item {} 
\sphinxAtStartPar
Discrimination in wages.

\item {} 
\sphinxAtStartPar
Confinement to menial employment.

\item {} 
\sphinxAtStartPar
Systematic insult of their women.

\item {} 
\sphinxAtStartPar
Lynching and miscarriage of justice.

\item {} 
\sphinxAtStartPar
Refusal to recognize fitness “in political or industrial life.”

\item {} 
\sphinxAtStartPar
Personal discourtesy.

\end{enumerate}

\sphinxAtStartPar
Southern papers in Charlotte, Richmond, New Orleans and Nashville have with singular unanimity hastened to call this complaint an unequivocal demand for “social equality,” and as such absolutely inadmissible. We are glad to have a frank definition, because we have always suspected this smooth phrase. We recommend on this showing that hereafter colored men who hasten to disavow any desire for “social equality” should carefully read the above list of disabilities which social inequality would seem to prescribe.


\bigskip\hrule\bigskip


\sphinxAtStartPar
\sphinxstyleemphasis{Citation:} Du Bois, W.E.B. 1911. “Social Equality.”  1(3)20\sphinxhyphen{}21.


\section{“Ashamed” (1911)}
\label{\detokenize{Volumes/01/03/ashamed:ashamed-1911}}\label{\detokenize{Volumes/01/03/ashamed::doc}}
\sphinxAtStartPar
ASHAMED.

\sphinxAtStartPar
Any colored man who complains of the treatment he receives in America is apt to be faced sooner or later by the statement that he is ashamed of his race.

\sphinxAtStartPar
The statement usually strikes him as a most astounding piece of illogical reasoning, to which a hot reply is appropriate.

\sphinxAtStartPar
And yet notice the curious logic of  the persons who say such things. They argue:

\sphinxAtStartPar
White men alone are men. This Negro wants to be a man. Ergo he wants to be a white man.

\sphinxAtStartPar
Their attention is drawn to the efforts of colored people to be treated decently. This minor premise therefore attracts them. But the major premise—the question as to treating black men like white men—never enters their heads, nor can they conceive it entering the black man’s head. If he wants to be a man he must want to be white, and therefore it is with peculiar complacency that a Tennessee paper says of a dark champion of Negro equality: “He bitterly resents his Negro blood.”

\sphinxAtStartPar
Not so, O Blind Man. He bitterly resents your treatment of Negro blood. The prouder he is, or has a right to be, of the blood of his black fathers, the more doggedly he resists the attempt to load men of that blood with ignominy and chains. It is race pride that fights for freedom; it is the man ashamed of his blood who weakly submits and smiles.


\bigskip\hrule\bigskip


\sphinxAtStartPar
\sphinxstyleemphasis{Citation:} Du Bois, W.E.B. 1911. ” Ashamed.”  1(3)21.


\section{Light (1912)}
\label{\detokenize{Volumes/03/04/light:light-1912}}\label{\detokenize{Volumes/03/04/light::doc}}
\sphinxAtStartPar
When the trustees of the Phelps\sphinxhyphen{}Stokes fund gave two Southern universities \$12,500 each to endow a fellowship for the study of the Negro they did well. For many decades there has been a venerable tradition that the South “knows” the Negro better tan others. Gradually, however, it is dawning even on the white South that	there	is nothing in mere physical distance half so separating as the artificial social, economic and racial barriers erected in the South since the war, and that the ignorance of the white South as to the life, hurts and dreams of the darker half of their world is, in some respects, both phenomenal and disgraceful. Take, for instance, this letter from a Southern woman who has read The Crisis:
\begin{quote}

\sphinxAtStartPar
“The Negro is a child, incompetent to right his own wrongs, but wonderfully susceptible to inspirational teaching. As a race he has a childlike conceit and thoroughly enjoys being ‘in the limelight.’ He has the untutored’s love of the morbid, revels in the sensational, and under praise wisely ad­ministered gives forth his best efforts. Some of the crime committed by the Negro is undoubtedly due to the de­sire to attract public attention. ‘If fame cannot be won, infamy can,’ is the subconscious conclusion of some Negro criminals—as it is with some white criminals. It seems to me that the best and most practical philan­thropy that can be performed for the race is to cease discussing him as a problem and consider him as child whose future career is to be shaped and molded by wise disciplin­ary educational methods.”
\end{quote}

\sphinxAtStartPar
Consider for a moment this extraordinary judgment: “The Negro” and “A Child!” Ten million people tossed nonchalantly into one mold with one estimate, one final and eternal judg­ment. One could not find ten million dogs, much less ten million men, whom one definition would fit.

\sphinxAtStartPar
The difficulty is, of course, that this honest woman knows and can know but one or few types of Negro. Her observation is confined to her kitchen, the almshouse and the chances of the street. Of the black man as a· man, of the black woman as a woman, she has almost no experience, and by grace of the color line can have no experience. Her ignorance is all the greater because it is not known to be ignorance, but parades as deep and subtle knowl­edge. · The world\sphinxhyphen{}old phenomenon of the childishness, laziness and crimi­nality of the ignorant and oppressed becomes in her blindness purely a racial, a “Negro” trait. If the gift to the University of Virginia will do something to shake the appalling con­fidence of such wild judgments the money will be well spent.

\sphinxAtStartPar
Of course, the scientific result will be small. For many years these young students will record not the observed facts, but their preconceived prejudices. This is inevitable with persons who start despising and not revering human souls simply because of their humanity. Gradually, however, truth will triumph.	Gradually it will not be possible to assert unchallenged in the University of Geor­gia that “{[}****{]} are lazy.” It will be explained by some perverse per­son that this laziness has somehow accumulated a thousand millions in fifty years—although, of course, those who did this are “exceptions.” In time this center of learning will cease to talk of “the” Negro and begin to talk of men—some rich, some poor; some good, some bad; some undeveloped “children” and some children of the Kingdom of God.


\bigskip\hrule\bigskip


\sphinxAtStartPar
\sphinxstyleemphasis{Citation:} Du Bois, W.E.B. 1912. “Light.” \sphinxstyleemphasis{The Crisis}. 3(4): 152\sphinxhyphen{}153.


\section{Logic (1913)}
\label{\detokenize{Volumes/06/02/logic:logic-1913}}\label{\detokenize{Volumes/06/02/logic::doc}}
\sphinxAtStartPar
The logical end of hatred is murder. Race prejudice is traditional hatred of human beings. Its end is lynching, war and extermination.

\sphinxAtStartPar
To say this thus bluntly and brutally is to invite strong denial. Race prejudice has often been professed by men of highest ideal and motive who would shrink at violence of any kind. But this is because such men are deliberately illogical, and their followers in the long run are not illogical, but carry their leaders’ doctrine to the bitter end. For instance, it is said this group of people are inferior to my group. Therefore, they are not entitled to the same privileges. But suppose they demand rights beyond their desert; then refuse them; if they keep demanding, silence them by law; if legal means do not keep them in their place, mob law is justifiable.

\sphinxAtStartPar
Thus the doctrine of race inferiority runs down to murder. Let us trace it in this country since the war. Negroes, being inferior, ought not to vote, said the reconstruction protesters. The nation, therefore, consented to their disfranchisement with the distinct understanding that all their other rights and privileges were to be preserved.

\sphinxAtStartPar
But if a man is not fit to vote why educate him and make him discontented? Consequently there was a movement against education which was so successful that to\sphinxhyphen{}day there are 2,000,000 Negroes not even enrolled, and practically half the Negro children in the land are not being decently trained in elementary schooling.

\sphinxAtStartPar
True, but one will give them good industrial training, make them skilled workmen, so that they can save their money and buy property. No, answers the white workman, they will compete with me and lower my wages. No, cries the white home owner, I don’t want Negroes in my block.

\sphinxAtStartPar
Very well, says the compromiser, segregate Negroes in a Ghetto. But, answers the Negro, the Ghetto is in the worst part of the city, is unhealthy, ill\sphinxhyphen{}cared for, filled with prostitutes whom you segregate with us, and we can’t better our condition because we cannot vote.

\sphinxAtStartPar
What then is the next step? Are we not harking right back to slavery? Is there any logical resting place on this downward path between a theory of inferiority and a theory of mob violence and extinction?

\sphinxAtStartPar
No. The man who begins by saying “This man is not entitled to equal rights with me,” ends by either himself saying or letting others say ‘‘Lynch the N{[}*****{]}.”

\sphinxAtStartPar
The new step which attacks the property of Negroes comes at this time because of the advance of the Negro in economic lines. Let us note this advance in a single State like Virginia with 670,000 Negro inhabitants. The Negroes cultivate 48,114 farms and the value of the farms which they own and rent increased from \$24,529,016 in 1900 to \$54,748,907 in 1910, or 123 per cent. Or if we would have figures covering simply ownership we find that

\sphinxAtStartPar
In 1891 Negroes owned \$12,089,965
In 1900 Negroes owned 15,856,570
In 1911 Negroes owned 32,944,336

\sphinxAtStartPar
This astounding advance of over 100 per cent. in property holding in a decade is the real reason for the attack on Negro property rights in Virginia, where three cities have tried to erect Negro Ghettos.

\sphinxAtStartPar
What lies beyond if the nation allows this last attack to succeed?
\begin{quote}

\sphinxAtStartPar
“The ray of hope for justice to the Negro in the South is like the shadow of the dawn. We have caught such glimpses of it as to indicate to us that the morning of our future has not yet appeared.’’
– Harrisburg, Pa., \sphinxstyleemphasis{Advocate Verdict.}
\end{quote}


\bigskip\hrule\bigskip


\sphinxAtStartPar
\sphinxstyleemphasis{Citation:} Du Bois, W.E.B. 1913. “Logic.” \sphinxstyleemphasis{The Crisis}. 6(2): 81.


\section{The Clansman (1915)}
\label{\detokenize{Volumes/10/01/clansman:the-clansman-1915}}\label{\detokenize{Volumes/10/01/clansman::doc}}
\begin{sphinxShadowBox}
\sphinxstylesidebartitle{}

\sphinxAtStartPar
The film, based on \sphinxhref{https://en.wikipedia.org/wiki/Thomas\_Dixon\_Jr}{Thomas Dixon Jr.’s} novel \sphinxhref{https://en.wikipedia.org/wiki/The\_Clansman:\_A\_Historical\_Romance\_of\_the\_Ku\_Klux\_Klan}{The Clansman: A Historical Romance of the Ku Klux Klan} was renamed \sphinxhref{https://en.wikipedia.org/wiki/The\_Birth\_of\_a\_Nation}{The Birth of a Nation} in March, 1915.
\end{sphinxShadowBox}

\sphinxAtStartPar
Several years ago a “professional southerner” named Dixon wrote a sensational and melodramatic novel which has been widely read. Eight years ago Dixon brought out his novel as a sordid and lurid melodrama. In several cities the performance of this play was prohibited because of its indecency or incitement to riot. Recently this vicious play has been put into moving pictures. With great adroitness the real play is preceded by a number of marvelously good war pictures; then in the second part comes the real “Clansman” with the Negro represented either as an ignorant fool, a vicious rapist, a venal and unscrupulous politician or a faithful but doddering idiot. By curious procedure this film received the preliminary approval of the National Board of Censors. It was put on in Los Angeles and immediately the fine organization of the N.A.A.C.P. was manifest. The facts were telegraphed to us from our Los Angeles Branch. We started at the Board of Censors. The proprietors of the film fought madly but the Censors met, viewed the film and immediately withdrew their sanction. Many of them were astonished that any committee of their board had ever passed it. The owners of the film promised to modify it but the modifications were unimportant. Yet this remarkable Board of Censors met a third time and passed the film over the protests of a minority of nine persons. Among these nine, however, was the chairman and founder of the board, Frederick C. Howe, Commissioner of Immigration at the Port of New York and several others equally influential. In other words, the board of censorship is. now practically split in two. The Association was not discouraged but immediately took steps on the one hand to bring the matter into court and on the other hand to interview the mayor. ‘The interview with the mayor was after some difficulty arranged. The delegation of five hundred of the most prominent white and colored people in the city filled the Council Chamber at City Hall and for an hour in terse, tense speeches urged the mayor to act. The mayor promised to have the two rape scenes cut out immediately and to go further than this if the play still seemed objectionable from the point of view of public peace and decency.

\sphinxAtStartPar
This action while commendable is not sufficient. The whole second half of the play ought to be suppressed and the Association will continue to work to this end. It is gratifying to know that in this work we have the cordial co\sphinxhyphen{}operation of all elements of colored people. The New York \sphinxstyleemphasis{Age} and the \sphinxstylestrong{Crisis} worked hand in hand with Harlem, Brooklyn and Jersey City. We know no factions in the righting of this great wrong.

\sphinxAtStartPar
It is sufficient to add that the main incident in the “Clansman” turns on a thinly veiled charge that Thaddeus Stephens, the great abolition statesman, was induced to give the Negroes the right to vote and secretly rejoice in Lincoln’s assassination because of his infatuation for a mulatto mistress. Small wonder that a man who can thus brutally falsify history has never been able to do a single piece of literary work that has brought the slightest attention, except when he seeks to capitalize burning race antagonisms.


\bigskip\hrule\bigskip


\sphinxAtStartPar
\sphinxstyleemphasis{Citation:} Du Bois, W.E.B. 1913. “The Clansman.”” \sphinxstyleemphasis{The Crisis}. 10(1): 33.


\section{Conduct, Not Color (1916)}
\label{\detokenize{Volumes/12/05/conduct_not_color:conduct-not-color-1916}}\label{\detokenize{Volumes/12/05/conduct_not_color::doc}}
\sphinxAtStartPar
A number of papers have been repeating a recent dictum that it is conduct, not color, that counts in the advance  of the Negro race. We wish this were the truth; but it is not the truth and those who say it know that it is not the truth. Conduct counts, but color counts more. It is this that constitutes the Negro problem.


\bigskip\hrule\bigskip


\sphinxAtStartPar
\sphinxstyleemphasis{Citation:} “Conduct, Not Color.” Editorial. 1916. \sphinxstyleemphasis{The Crisis}. 12(5): 217\sphinxhyphen{}218.


\section{Migration (1916)}
\label{\detokenize{Volumes/12/06/migration:migration-1916}}\label{\detokenize{Volumes/12/06/migration::doc}}
\sphinxAtStartPar
It has long been the custom of colored leaders to advise the colored people to stay in the South. This has been supplemented by the startling information on the part of southern whites
that they are the “best friends of the colored people”, etc.

\sphinxAtStartPar
We might as well face the facts squarely: If there is any colored man in the South who wishes to have his children educated and who wishes to be in close touch with civilization and who has any chance or ghost of a chance of making a living in the North it is his business to get out of the South as soon as possible. He need not seek for reasons for so doing. The same reasons that drive the Jew from Russia, the peasants from Austria, the Armenians from Turkey and the oppressed from tyranny everywhere should drive the colored man out of the land of lynching, lawlessness and industrial oppression.

\sphinxAtStartPar
It would, of course, be foolish for a man to give up a good chance of making a living and migrate to a country where he had little or no chance. But we are speaking of the case where men have an opportunity. These opportunities at present are widening in the North. Every single colored man who can should take advantage of it. The only effective protest that the Negroes en masse can make against lynching and dis­franchisement is through leaving the devilish country where these things take place.

\sphinxAtStartPar
The colored people of the North on the other hand, should welcome their escaping fellows. It means, undoubtedly, increased hardships for them; it will be proscription and temporary difficulties, but anything that means freedom to black slaves should be welcomed by their free northern brothers.


\bigskip\hrule\bigskip


\sphinxAtStartPar
\sphinxstyleemphasis{Citation:} “Migration.” Editorial. 1916. \sphinxstyleemphasis{The Crisis}. 12(6): 270.


\section{Slavery (1921)}
\label{\detokenize{Volumes/24/01/slavery:slavery-1921}}\label{\detokenize{Volumes/24/01/slavery::doc}}
\begin{sphinxShadowBox}
\sphinxstylesidebartitle{}

\sphinxAtStartPar
\sphinxhref{https://law.jrank.org/pages/2818/John-S-Williams-Clyde-Manning-Trials-1921-Murdering-Evidence-Peonage.html}{John S. Williams} owned a Georgia plantation and when a investigation began into the conditions of his workers, he had nine of his \sphinxstyleemphasis{peons} murdered so they could not testify against him.
\end{sphinxShadowBox}

\sphinxAtStartPar
Slavery still exists in the United States. In the courts of Carolina, Georgia, Alabama, Mississippi, Louisiana, Arkansas and Texas, human beings are daily sold into slavery to men like the murderer Williams of Jasper County, Georgia.

\sphinxAtStartPar
Throughout the South—but especially in the Mississippi and Red River bottoms, from Memphis south; in middle and south Georgia and Ala­bama; and in the Brazos bottoms of Texas—Negroes are held today in as complete and awful and soul destroying slavery as they were in 1860. Their overseers ride with guns and whips; their women are prostitutes to white owners and drivers; their children are trained in ignorance, im­morality and crime.

\sphinxAtStartPar
Every Southerner knows this. The States know it. The Government knows it. Distinguished Southerners are getting wealthy on the system; the Southern White Church is sending missionaries to Africa on its proceeds; lovely young white ladies are being finished in exclusive schools on its dirty blood profits; and yet it goes on and on and on, and it will go on until one day its red upheaval will shake the civilized world.

\sphinxAtStartPar
But there comes a ray of hope. Georgia is to be congratulated on the conviction of Williams for murder, and for the fine spirit of press and people that stand behind it. The Atlanta Committee on Church Cooperation calls the conviction “Georgia’s message to the Negro and to the world that this Christian State, whatever may have been conditions in the past in sore spots within her borders, will in the future do justice to the Negro”; and the Atlanta Georgian adds “Georgia has renewed her sacred pledge to civilization.”

\sphinxAtStartPar
Amen! And the Negro stands ready to recognize and appreciate every act of Justice done by the white South.


\bigskip\hrule\bigskip


\sphinxAtStartPar
\sphinxstyleemphasis{Citation:} “Slavery” Editorial. 1922. \sphinxstyleemphasis{The Crisis}. 24(1): 6.


\section{The Newer South (1926)}
\label{\detokenize{Volumes/31/04/newer_south:the-newer-south-1926}}\label{\detokenize{Volumes/31/04/newer_south::doc}}
\sphinxAtStartPar
The New South of Henry Grady had nothing new for the Negro. And since that time thoughtful Negroes have received professions of friendship on the part of the white South with much salt. Nor can they be blamed for this: lynching, “Jim\sphinxhyphen{}Crow” cars, poor schools, segregation and insult form a difficult atmosphere in which to breathe the air of freedom, friendship and hope. But there can be no doubt but that the white South is changing; there is nothing revolutionary as yet, but leaven is working. Today as never before since 1863 there can be found in the white South a few intelligent and determined people who are willing to recognize black men as men— not as Super\sphinxhyphen{}men nor as morons, but as men. This group is not large; in no community is it in majority; only here and there is it self\sphinxhyphen{}conscious and vocal. But it exists and it is slowly growing in numbers and courage.

\sphinxAtStartPar
May we note a few evidences? Most of the circular matter sent out from the Atlanta headquarters of the Inter\sphinxhyphen{}racial movement is pure pro\sphinxhyphen{}Southern propaganda; but not all of it. Recently a resumé of Negro progress by Robert B. Eleazer was issued which was complete, sympathetic and beyond criticism. A Negro, Silas Parmore, extradited from New Jersey to Georgia over our protest, was tried and acquitted; and not only this but the Governor Walker, of Georgia, boasted of the fact. Mississippi is the nadir of the South; she murders, disfranchises and enslaves her labor; she has neither literature, science nor art; no actor, singer or lecturer of note thinks of stopping there; only 90 of the 2 million residents of the state are in “Who’s Who” and 26 of these because of positions they were elected to; the state has lynched and burned alive over 530 human beings in the last generation. Yet Mississippi this year for the first time in her history has issued a protest against lynching signed
\begin{quote}

\sphinxAtStartPar
“by Governor H. L. Whitfield, Speaker Thomas L. Bailey, of the House of Representatives, President J. N. Flowers, of the State Bar Association, a number of judges of the Supreme Court, members of Congress, prominent lawyers, educators, churchmen, and club women. Prominent place is given also to the recent anti\sphinxhyphen{}lynching statement made by the Mississippi Woman’s Committee on Interracial Cooperation, which has since been affirmed by hundreds of Mississippi women at meetings throughout the State.
\end{quote}
\begin{quote}

\sphinxAtStartPar
“An important section of the pamphlet is given to suggestions for the prevention of lynching, Sheriffs are urged to announce in advance that they expect to do their duty in every case, even at the risk of their own lives; to employ as deputies only those persons who agree to go to the same length in upholding the law; to ascertain the names of men who are opposed to mob violence and to swear these in as special deputies at the first sign of trouble; to remove to the jails of other counties prisoners threatened with mob violence; and to call upon the Governor to order out the National Guard if needed;
\end{quote}
\begin{quote}

\sphinxAtStartPar
“The popular fallacy regarding the ‘usual cause’ of lynchings is also mercilessly exposed. Photographs of a recent lynching are shown and ‘respectfully referred to the next Grand Jury”.
\end{quote}
\begin{quote}

\sphinxAtStartPar
“State officials, members of the Bar Association, and other prominent people are distributing the pamphlet widely and are offering medals in each congressional district for the best essays on the subject by high school students”.
\end{quote}

\sphinxAtStartPar
In Kentucky the Inter\sphinxhyphen{}racial movement has ceased to be simply a method of stopping agitation by encouraging “white folk’s n*****” and seems to be trying really to attack certain pressing problems of race contact; North Carolina is resolutely facing the problem of Negro education and has established a class A college. Roland Hayes has been heard by mixed audiences in Richmond, Louisville and Atlanta. A colored girl elected to represent the South in a national student organization was not displaced when the fact of her race was known.

\sphinxAtStartPar
But all those symptoms are of but passing significance except as they indicate these fundamental changes:

\sphinxAtStartPar
\sphinxstyleemphasis{First}, the definite breaking up of the effort of the South to present morally and socially a solid front to the world. The South is beginning to realize that the fight for righteousness in its borders as elsewhere in the world cannot conceal itself behind the apparent absolute agreement of all southern whites on the Negro problems. Until the Better South is willing openly and flatly to take a stand and to fight the Bourbon race reactionaries, they will find themselves circumvented and represented by the Worst South. There are signs that a few Southerners, and especially the younger men and women, are realizing this and are prepared to pay the heavy price.

\sphinxAtStartPar
\sphinxstyleemphasis{Secondly}, just as the South has hitherto heard with sympathetic and even exaggerated patience and respect the demands of extreme white Southern Negro haters, so too they must be willing now to listen to Negro “radicals”. To read out of the congregation of decent, reasonable and law\sphinxhyphen{}abiding people those black folk who demand the ballot, equal education, the abolition of “Jim Crow” legislation, the abrogation of laws and customs which protect and encourage bastardy and prostitution, and right of social equals to social equality with those who wish it—to lynch such men morally is a coward’s trick and a scoundrel’s subterfuge and there are southern white men today who realize this as never before. To such men and to such women in the dawn of the nineteen hundred and twenty\sphinxhyphen{}sixth year of the Prince of Peace, our hand and heart, comrades.


\bigskip\hrule\bigskip


\sphinxAtStartPar
\sphinxstyleemphasis{Citation:} Du Bois, W.E.B. 1926. “The Newer South.”  31(4):163\sphinxhyphen{}165.


\section{The Higher Friction (1927)}
\label{\detokenize{Volumes/34/02/higher_friction:the-higher-friction-1927}}\label{\detokenize{Volumes/34/02/higher_friction::doc}}
\sphinxAtStartPar
Let us take courage from certain present aspects of the Negro problem. Friction there has always been between black and white since 1619. Friction there will probably be still in 2019. But the friction rises in the scale; it touches, decade by decade, higher levels—higher interests, higher sensibilities, even while the lower friction persists. To illustrate our meaning consider this table.

\sphinxAtStartPar
1860—Physical freedom1870—Crime and poverty1880—Right to hold property1890—Reading and writing1900—Voting1910—Lynching1920—Homes

\sphinxAtStartPar
There was freedom for some Negroes before 1860 but that year it became a problem for all. The crime and degradation incident to emancipation was critical in 1870. By 1880 we had to answer the query if Negroes could own property. By 1890 the Negroes’ right to some education was won. By 1900 the Negro had been disfranchised in fact but in law he was a legal voter. Before 1900, lynching was defensible and met with little opposition and until this decade there was almost no wide\sphinxhyphen{}spread problem of Negroes living in desirable homes next to whites because black folk were too poor to buy such homes. Thus even in the record of discrimination we are pressing on and up. The founding stones still waver, far from fast, but the trembling walls reach up to higher friction.


\bigskip\hrule\bigskip


\sphinxAtStartPar
\sphinxstyleemphasis{Citation:} Du Bois, W.E.B. 1927. “The Higher Friction.”  34(2):70.


\section{Prejudice (1927)}
\label{\detokenize{Volumes/34/09/prejudice:prejudice-1927}}\label{\detokenize{Volumes/34/09/prejudice::doc}}
\sphinxAtStartPar
We have received from Mrs. Helen D. Pecu of Vashon Islands, Washington, the following letter:
\begin{quote}

\sphinxAtStartPar
“It have hesitated many months in writing you but each time I read your editorials the impulse comes to protest, not because I am white, but for the sake of those whose skin pigment only has set them apart in an unjust world—a world of unreasoning prejudice—that men like you have built up —for you are as violently prejudiced against all whites as the most intolerant white man is against you. It is deplorable that so excellent a magazine as \sphinxstylestrong{The Crisis}  should have at its head a man who sneers at all forms of government—save the Soviet government in Russia—who can but intimidate the budding courage and genius of his people by his own big stick of Prejudice. In strong contrast, how fine is the character of Allison Davis, whose article ‘On Misgivings’ appears in the August Crisis. What influence could he not wield between the two races! What respect would his cause not command. When we who regard the colored race with the same respect we feel for our own are assailed and derided because we are white, we begin to wonder if there might not have been in some age past a basic reason for this separation of the races, that reason now continuing in the form of prejudice on both sides between minds incapable of growth.
\end{quote}
\begin{quote}

\sphinxAtStartPar
“I do not refer to the publication of crimes or acts of injustice. They are facts that both races should gravely consider, but as any friend, white or colored, who has the interest of your race at heart, I ask you to withhold your private political and prejudiced rancor. It is improper material in your magazine and detracts from its constructive purpose.”
\end{quote}

\sphinxAtStartPar
There is no doubt but that colored people are prejudiced against white people, and that the Editor of \sphinxstylestrong{The Crisis} is one of the greatest of sinners in this respect. From long experience he has gotten into the habit of expecting certain actions, certain thoughts, certain treatment from the majority of white people. He is sometimes pleasantly disappointed. In most cases he is not. In most cases he gets just what he has been looking for, and it is quite possible that, in some of these instances, he gets it because he has been expecting it!

\sphinxAtStartPar
But in any case, the worse fruit of prejudice is retaliatory prejudice; because white Americans have reasons based on slavery, poverty and ignorance in the past and on thoughtlessness and lack of information in our own day, they have gotten into the habit of treating black folk in certain ways. Black folk have gradually adopted the reciprocal habit of hating white skins, of being suspicious of every white action, and particularly of talking and acting as though even those white people who are not prejudiced, or who earnestly desire not to be, belonged to the unfortunate majority.

\sphinxAtStartPar
What Mrs. Pecu and others must learn is that this is the natural fruit of race prejudice. Just as no ordinary white man born and bred in the South can be expected to treat Negroes decently, in the same way, no Negro born in America can be expected to be sweet\sphinxhyphen{}tempered, charitable and broadminded toward white people.

\sphinxAtStartPar
The Editor of a magazine like \sphinxstylestrong{The Crisis} should nevertheless try to achieve such an attitude. He does try. If he fails, do not lay the fault entirely at his door. Lay it to the last lynching, or to the last time he was insulted in the theater, or to the last time he went hungry because all available hotels and restaurants were closed against him. It is all a mess, he admits and that is precisely what \sphinxstylestrong{The Crisis} has been trying to say for many years.


\bigskip\hrule\bigskip


\sphinxAtStartPar
\sphinxstyleemphasis{Citation:} “Prejudice.” 1927. Editorial. \sphinxstyleemphasis{The Crisis} 34(9):311.


\section{A Protest (1933)}
\label{\detokenize{Volumes/40/07/protest:a-protest-1933}}\label{\detokenize{Volumes/40/07/protest::doc}}
\begin{sphinxuseclass}{sphinx-bs}
\begin{sphinxuseclass}{container}
\begin{sphinxuseclass}{pb-4}
\begin{sphinxuseclass}{row}
\begin{sphinxuseclass}{d-flex}
\begin{sphinxuseclass}{col-lg-6}
\begin{sphinxuseclass}{col-md-6}
\begin{sphinxuseclass}{col-sm-6}
\begin{sphinxuseclass}{col-xs-12}
\begin{sphinxuseclass}{p-2}
\begin{sphinxuseclass}{card}
\begin{sphinxuseclass}{w-100}
\begin{sphinxuseclass}{shadow}
\begin{sphinxuseclass}{card-body}
\sphinxAtStartPar
\sphinxstyleemphasis{The following speech was delivered in the House on May 11 by Mrs. Rogers, Representative from Massachusetts.}

\end{sphinxuseclass}
\end{sphinxuseclass}
\end{sphinxuseclass}
\end{sphinxuseclass}
\end{sphinxuseclass}
\end{sphinxuseclass}
\end{sphinxuseclass}
\end{sphinxuseclass}
\end{sphinxuseclass}
\end{sphinxuseclass}
\begin{sphinxuseclass}{d-flex}
\begin{sphinxuseclass}{col-lg-6}
\begin{sphinxuseclass}{col-md-6}
\begin{sphinxuseclass}{col-sm-6}
\begin{sphinxuseclass}{col-xs-12}
\begin{sphinxuseclass}{p-2}
\begin{sphinxuseclass}{card}
\begin{sphinxuseclass}{w-100}
\begin{sphinxuseclass}{shadow}
\begin{sphinxuseclass}{card-body}
\sphinxAtStartPar
\sphinxstyleemphasis{The following speech was not delivered in the Reichstag on May 11 by Adolf Hitler, Chancellor of the Third Empire.}

\end{sphinxuseclass}
\end{sphinxuseclass}
\end{sphinxuseclass}
\end{sphinxuseclass}
\end{sphinxuseclass}
\end{sphinxuseclass}
\end{sphinxuseclass}
\end{sphinxuseclass}
\end{sphinxuseclass}
\end{sphinxuseclass}
\end{sphinxuseclass}
\end{sphinxuseclass}
\end{sphinxuseclass}
\end{sphinxuseclass}
\begin{sphinxuseclass}{sphinx-bs}
\begin{sphinxuseclass}{container}
\begin{sphinxuseclass}{pb-4}
\begin{sphinxuseclass}{row}
\begin{sphinxuseclass}{d-flex}
\begin{sphinxuseclass}{col-lg-6}
\begin{sphinxuseclass}{col-md-6}
\begin{sphinxuseclass}{col-sm-6}
\begin{sphinxuseclass}{col-xs-12}
\begin{sphinxuseclass}{p-2}
\begin{sphinxuseclass}{card}
\begin{sphinxuseclass}{w-100}
\begin{sphinxuseclass}{shadow}
\begin{sphinxuseclass}{card-body}
\sphinxAtStartPar
I take the floor to protest against the brutal and unwarranted treatment of the nationals of Jewish extraction in Germany by Adolf Hitler.

\end{sphinxuseclass}
\end{sphinxuseclass}
\end{sphinxuseclass}
\end{sphinxuseclass}
\end{sphinxuseclass}
\end{sphinxuseclass}
\end{sphinxuseclass}
\end{sphinxuseclass}
\end{sphinxuseclass}
\end{sphinxuseclass}
\begin{sphinxuseclass}{d-flex}
\begin{sphinxuseclass}{col-lg-6}
\begin{sphinxuseclass}{col-md-6}
\begin{sphinxuseclass}{col-sm-6}
\begin{sphinxuseclass}{col-xs-12}
\begin{sphinxuseclass}{p-2}
\begin{sphinxuseclass}{card}
\begin{sphinxuseclass}{w-100}
\begin{sphinxuseclass}{shadow}
\begin{sphinxuseclass}{card-body}
\sphinxAtStartPar
I take the floor to protest against the brutal and unwarranted treatment of the Nationals of Negro extraction in America by the whole nation, and more especially by the South.

\end{sphinxuseclass}
\end{sphinxuseclass}
\end{sphinxuseclass}
\end{sphinxuseclass}
\end{sphinxuseclass}
\end{sphinxuseclass}
\end{sphinxuseclass}
\end{sphinxuseclass}
\end{sphinxuseclass}
\end{sphinxuseclass}
\end{sphinxuseclass}
\end{sphinxuseclass}
\end{sphinxuseclass}
\end{sphinxuseclass}

\begin{savenotes}\sphinxattablestart
\centering
\begin{tabulary}{\linewidth}[t]{|T|T|}
\hline
\sphinxstyletheadfamily 
\sphinxAtStartPar
\sphinxstyleemphasis{The following speech was delivered in the House on May 11 by Mrs. Rogers, Representative from Massachusetts.}
&\sphinxstyletheadfamily 
\sphinxAtStartPar
\sphinxstyleemphasis{The following speech was not delivered in the Reichstag on May 11 by Adolf Hitler, Chancellor of the Third Empire.}
\\
\hline
\sphinxAtStartPar
I take the floor to protest against the brutal and unwarranted treatment of the nationals of Jewish extraction in Germany by Adolf Hitler.
&
\sphinxAtStartPar
I take the floor to protest against the brutal and unwarranted treatment of the Nationals of Negro extraction in America by the whole nation, and more especially by the South.
\\
\hline
\sphinxAtStartPar
Our forefathers fled from oppression to New England. We from that section especially sympathize with any persecuted race. Our heritage demands that a protest be made. Some will say that we should not interfere with the private affairs of the German people or with the internal affairs of that country. We must take note of such unjust and inhuman treatment as has been dealt out in Germany of late.
&
\sphinxAtStartPar
Our forefathers resented oppression. We Germans especially sympathize with any persecuted race. Our heritage demands that a protest be made. Some will say that we should not interfere with the private affairs of the American people, or with the internal affairs of the United States. We must take note of such unjust and inhuman treatment as has been dealt out by America for many years.
\\
\hline
\end{tabulary}
\par
\sphinxattableend\end{savenotes}


\bigskip\hrule\bigskip


\sphinxAtStartPar
\sphinxstyleemphasis{Citation:} Du Bois, W.E.B. 1933. “A Protest.” \sphinxstyleemphasis{The Crisis} 40(7):165.


\chapter{Economics}
\label{\detokenize{Sections/economics:economics}}\label{\detokenize{Sections/economics::doc}}
\sphinxAtStartPar
Editorials on labor, unions, socialism and communism.
\begin{itemize}
\item {} 
\sphinxAtStartPar
{\hyperref[\detokenize{Sections/labor::doc}]{\sphinxcrossref{Labor}}}
\begin{itemize}
\item {} 
\sphinxAtStartPar
{\hyperref[\detokenize{Volumes/03/06/servant_in_the_south::doc}]{\sphinxcrossref{The Servant in the South (1912)}}}

\item {} 
\sphinxAtStartPar
{\hyperref[\detokenize{Volumes/15/05/the_black_man_and_the_unions::doc}]{\sphinxcrossref{The Black Man and the Unions (1918)}}}

\item {} 
\sphinxAtStartPar
{\hyperref[\detokenize{Volumes/18/05/labor_omnia_vincit::doc}]{\sphinxcrossref{Labor Omnia Vincit (1919)}}}

\item {} 
\sphinxAtStartPar
{\hyperref[\detokenize{Volumes/31/02/black_man_and_labor::doc}]{\sphinxcrossref{The Black Man and Labor (1925)}}}

\item {} 
\sphinxAtStartPar
{\hyperref[\detokenize{Volumes/31/06/again_pullman_porters::doc}]{\sphinxcrossref{Again, Pullman Porters (1926)}}}

\end{itemize}

\item {} 
\sphinxAtStartPar
{\hyperref[\detokenize{Sections/socialism::doc}]{\sphinxcrossref{Socialism and Communism}}}
\begin{itemize}
\item {} 
\sphinxAtStartPar
{\hyperref[\detokenize{Volumes/19/04/cooperation::doc}]{\sphinxcrossref{Coöperation (1920)}}}

\item {} 
\sphinxAtStartPar
{\hyperref[\detokenize{Volumes/22/03/negro_and_radical_thought::doc}]{\sphinxcrossref{The Negro and Radical Thought (1921)}}}

\item {} 
\sphinxAtStartPar
{\hyperref[\detokenize{Volumes/22/04/class_struggle::doc}]{\sphinxcrossref{The Class Struggle (1921)}}}

\item {} 
\sphinxAtStartPar
{\hyperref[\detokenize{Volumes/22/06/socialism_and_the_negro::doc}]{\sphinxcrossref{Socialism and the Negro (1921)}}}

\item {} 
\sphinxAtStartPar
{\hyperref[\detokenize{Volumes/22/06/single_tax::doc}]{\sphinxcrossref{The Single Tax (1921)}}}

\item {} 
\sphinxAtStartPar
{\hyperref[\detokenize{Volumes/27/03/black_man_and_the_wounded_world::doc}]{\sphinxcrossref{The Black Man and the Wounded World (1923)}}}

\item {} 
\sphinxAtStartPar
{\hyperref[\detokenize{Volumes/33/01/russia_1926::doc}]{\sphinxcrossref{Russia, 1926 (1926)}}}

\item {} 
\sphinxAtStartPar
{\hyperref[\detokenize{Volumes/33/04/judging_russia::doc}]{\sphinxcrossref{Judging Russia (1927)}}}

\item {} 
\sphinxAtStartPar
{\hyperref[\detokenize{Volumes/35/11/dunbar_national_bank::doc}]{\sphinxcrossref{The Dunbar National Bank (1928)}}}

\item {} 
\sphinxAtStartPar
{\hyperref[\detokenize{Volumes/38/09/negro_and_communism::doc}]{\sphinxcrossref{The Negro and Communism (1931)}}}

\item {} 
\sphinxAtStartPar
{\hyperref[\detokenize{Volumes/40/07/our_class_struggle::doc}]{\sphinxcrossref{Our Class Struggle}}}

\end{itemize}

\end{itemize}


\section{Labor}
\label{\detokenize{Sections/labor:labor}}\label{\detokenize{Sections/labor::doc}}
\sphinxAtStartPar
Editorials on labor and unions.
\begin{itemize}
\item {} 
\sphinxAtStartPar
{\hyperref[\detokenize{Volumes/03/06/servant_in_the_south::doc}]{\sphinxcrossref{The Servant in the South (1912)}}}

\item {} 
\sphinxAtStartPar
{\hyperref[\detokenize{Volumes/15/05/the_black_man_and_the_unions::doc}]{\sphinxcrossref{The Black Man and the Unions (1918)}}}

\item {} 
\sphinxAtStartPar
{\hyperref[\detokenize{Volumes/18/05/labor_omnia_vincit::doc}]{\sphinxcrossref{Labor Omnia Vincit (1919)}}}

\item {} 
\sphinxAtStartPar
{\hyperref[\detokenize{Volumes/31/02/black_man_and_labor::doc}]{\sphinxcrossref{The Black Man and Labor (1925)}}}

\item {} 
\sphinxAtStartPar
{\hyperref[\detokenize{Volumes/31/06/again_pullman_porters::doc}]{\sphinxcrossref{Again, Pullman Porters (1926)}}}

\end{itemize}


\subsection{The Servant in the South (1912)}
\label{\detokenize{Volumes/03/06/servant_in_the_south:the-servant-in-the-south-1912}}\label{\detokenize{Volumes/03/06/servant_in_the_south::doc}}
\sphinxAtStartPar
During slavery days the house servants were rewarded with extra privileges, among which were the left\sphinxhyphen{}over food and cast\sphinxhyphen{}off clothing of the “big house.” This easily became, under the less rigorous forms of serfdom, a sort of payment in kind for personal service, and now and then “tips” in actual money were given. When formal emancipation came the servants were promised wages, but as a matter of fact the wages were seldom paid in cash, while a money value was often given to the food and old clothes. This old custom could easily degenerate into something very like stealing, and yet the custom could seem justifiable in the eyes of the ignorant, especially when their wages were low and often unpaid, and when they saw mistresses wink at and even expect peculations of this sort. On the other hand, colored servants are not dishonest; money, jewelry and the like are safe in their hands with few exceptions.

\sphinxAtStartPar
The result of the old system was unrest among servants, and the more intelligent and thrifty escaped from domestic service into the care of their own homes or day’s work or other industrial avenues. Or if they continued in service they went North, where instead of receiving \$1.50 a week in old clothes and cold victuals, they could earn \$5 and \$6 a week in cash.

\sphinxAtStartPar
Moreover, the conditions under which a colored servant in the South must work are the worst in the civilized world. The hours are endless, the quarters are poor, the deference demanded is unbearable to people of the least spirit, and the assumption of the natural inferiority of the servant is almost universal.

\sphinxAtStartPar
Not only this but there is in the majority of cases in the South absolutely no protection for the black girl’s virtue in the white man’s home. Everybody knows that the mulatto both before and since slavery was the outcome of house service.

\sphinxAtStartPar
What is the result? Poor and un­willing service. The best Negroes are withdrawing their sons and daughters from house service just as quickly as they can, and they deserve commendation for so doing. Even those Negroes who publicly commend house service are curiously careful to keep their children out of it. Those who cannot escape are demanding shorter hours, proper wages and better treatment. And those Southern families who can keep their black servants but three weeks would better ask advice of their neighbors who keep good and faithful black servants for ten and twenty years.

\sphinxAtStartPar
Instead of responding to a legiti­mate demand for change in working conditions, the majority of South­erners take their usual refuge in whining and shrieking “Negro” problem. Every time that the white South runs head foremost into the inevitable laws of nature by trying to keep slavery, establish peonage, deny manhood rights to men and de­grade decent women—every time the
South tries this there is a mawkish sentimentality throughout the North to encourage the idea that these laws are not human but peculiar or racial.

\sphinxAtStartPar
If people pay their laborers low wages and cheat them out of even these, they will get cheap labor, whether that labor be black, white or blue.

\sphinxAtStartPar
If the South or the North wants de­cent domestic service it must
\begin{enumerate}
\sphinxsetlistlabels{\arabic}{enumi}{enumii}{}{.}%
\item {} 
\sphinxAtStartPar
Pay decent wages.

\item {} 
\sphinxAtStartPar
Give shorter hours and more definite duties.

\item {} 
\sphinxAtStartPar
Treat servants as men and women and not as cattle.

\end{enumerate}

\sphinxAtStartPar
The people that are unwilling to do this will find the “servant problem” always with them, even though they nickname it a “Negro” problem.


\bigskip\hrule\bigskip


\sphinxAtStartPar
\sphinxstyleemphasis{Citation:} Du Bois, W.E.B. 1912. “Servant in the South.”  \sphinxstyleemphasis{The Crisis}. 3(5): 200\sphinxhyphen{}201.


\subsection{The Black Man and the Unions (1918)}
\label{\detokenize{Volumes/15/05/the_black_man_and_the_unions:the-black-man-and-the-unions-1918}}\label{\detokenize{Volumes/15/05/the_black_man_and_the_unions::doc}}
\sphinxAtStartPar
I am among the few colored men who have tried conscientiously to bring about understanding and co\sphinxhyphen{}operation between American Negroes and the Labor Unions. I have sought to look upon the Sons of Freedom as simply a part of the great mass of the earth’s Disinherited, and to realize that world movements which have lifted the lowly in the past and are opening the gates of opportunity to them today are of equal value for all men, white and black, then and now.

\sphinxAtStartPar
I carry on the title page, for instance, of this magazine the Union label, and yet I know, and everyone of my Negro readers knows, that the very fact that this label is there is an advertisement that no Negro’s hand is engaged in the printing of this magazine, since the International Typographical Union systematically and deliberately excludes every Negro that it dares from membership, no matter what his qualifications.

\sphinxAtStartPar
Even here, however, and beyond the hurt of mine own, I have always striven to recognize the real cogency of the Union argument. Collective bargaining has, undoubtedly, raised modern labor from something like chattel slavery to the threshold of industrial freedom, and in this advance of labor white and black have shared.

\sphinxAtStartPar
I have tried, therefore, to see a vision of vast union between the laboring forces, particularly in the South, and hoped for no distant day when the black laborer and the white laborer, instead of being used against each other as helpless pawns, should unite to bring real democracy in the South.

\sphinxAtStartPar
On the other hand, the whole scheme of settling the Negro problem, inaugurated by philanthropists and carried out during the last twenty years, has been based upon the idea of playing off black workers against white. That it is essentially a mischievous and dangerous pro­ gram no sane thinker can deny, but it is peculiarly disheartening to realize that it is the Labor Unions themselves that have given this movement its greatest impulse and that today, at last, in East St. Louis have brought the most unwilling of us to acknowledge that in the present Union movement, as represented by the American Federation of Labor, there is absolutely no hope of justice for an American of Negro descent.

\sphinxAtStartPar
Personally, I have come to this decision reluctantly and in the past have written and spoken little of the closed door of opportunity, shut impudently in the faces of black men by organized white workingmen. I realize that by heredity and century\sphinxhyphen{}long lack of opportunity one cannot expect in the laborer that larger sense of justice and duty which we ought to demand of the privileged classes. I have, therefore, inveighed against color discrimination by employers and by the rich and well\sphinxhyphen{}to\sphinxhyphen{}do, knowing at the same time in silence that it is practically impossible for any colored man or woman to become a boiler maker or book binder, an electrical worker or glass maker, a worker in jewelry or leather, a machinist or metal polisher, a paper maker or piano builder, a plumber or a potter, a printer or a pressman, a telegrapher or a railway trackman, an electrotyper or stove mounter, a textile worker or tile layer, a trunk maker, upholsterer, carpenter, loco­ motive engineer, switchman, stone cutter, baker, blacksmith, boot and shoemaker, tailor, or any of a dozen other important well\sphinxhyphen{}paid employments, without encountering the open determination and unscrupulous opposition of the whole united labor movement of America. That further than this, if he should want to be­ come a painter, mason, carpenter, plasterer, brickmaker or fireman he would be subject to humiliating dis­criminations by his fellow Union workers and be deprived of work at every possible opportunity, even in defiance of their own Union laws. If, braving this outrageous attitude of the Unions, he succeeds in some small establishment or at some exceptional time at gaining employment, he must be labeled as a “scab” throughout the length and breadth of the land and written down as one who, for his selfish advantage, seeks to over­ throw the labor uplift of a century.

\sphinxAtStartPar
The recent convention of the American Federation of Labor, at Buffalo, is no proof of change of heart. Grudgingly, unwillingly, almost insultingly, this Federation yields to us inch by inch the status of half\sphinxhyphen{}a\sphinxhyphen{}man, denying and withholding every privilege it dares at all times.


\bigskip\hrule\bigskip


\sphinxAtStartPar
\sphinxstyleemphasis{Citation:} “The Black Man and the Unions” Editorial. 1918. \sphinxstyleemphasis{The Crisis}. 15(5): 114.


\subsection{Labor Omnia Vincit (1919)}
\label{\detokenize{Volumes/18/05/labor_omnia_vincit:labor-omnia-vincit-1919}}\label{\detokenize{Volumes/18/05/labor_omnia_vincit::doc}}
\sphinxAtStartPar
Labor conquers all things— but slowly, O, so slowly. Ever the weary worldlings seek some easier, quicker way— the Way of Wealth, of Privilege, of Chance, of Power; but in the end all that they get—Food, Raiment, Palace and Pleasure—is the result of Toil, but not always of their own toil. The great cry of world Justice today is that the fruit of toil go to the Laborer who produces it. In this labor of Production we recognize effort of all sorts—lifting, digging, carrying, measuring, thinking, foreseeing; but we are refusing to recognize Chance, Birth or Monopoly as just grounds for compelling men to serve men.

\sphinxAtStartPar
In this fight for Justice to Labor the Negro looms large. In Africa and the South Seas, in all the Americas and dimly in Asia he is a mighty worker and potentially, perhaps, the mightiest. But of all laborers cheated of their just wage from the world’s dawn to today, he is the poorest and bloodiest.

\sphinxAtStartPar
In the United States he has taken his fastest forward step, rising from owned slave to tied serf, from servant to day laborer, from scab to half\sphinxhyphen{}recognized union man in less than a century. Armies, mobs, lynchers, laws and customs have opposed him, yet he lurches forward. His very so called indolence is his dimly\sphinxhyphen{}conceived independence; his singing soul is his far\sphinxhyphen{}flaming ideal; and nothing but organized and persistent murder and violence can prevent him from becoming in time the most efficient laboring force in the modern world.

\sphinxAtStartPar
Meantime, in the world round him, the battle of Industrial Democracy is being fought and the white laborers who are fighting it are not sure whether they want their black fellow laborer as ally or slave. If they could make him a slave, they probably would; but since he can underbid their wage, they slowly and reluctantly invite him into the union. But can they bring themselves inside the Union to regard him as a man—a fellow\sphinxhyphen{}voter, a brother?

\sphinxAtStartPar
No—not yet. And there lies the most stupendous labor problem of the twentieth century—transcending the problem of Labor and Capital, of Democracy, of the Equality of Women—for it is the problem of the Equality of Humanity in the world as against white domination of black and brown and yellow serfs.


\bigskip\hrule\bigskip


\sphinxAtStartPar
\sphinxstyleemphasis{Citation:} Du Bois, W.E.B. 1919. “Labor Omnia Vincit.” \sphinxstyleemphasis{The Crisis}. 18(5): 231\sphinxhyphen{}232.


\subsection{The Black Man and Labor (1925)}
\label{\detokenize{Volumes/31/02/black_man_and_labor:the-black-man-and-labor-1925}}\label{\detokenize{Volumes/31/02/black_man_and_labor::doc}}
\sphinxAtStartPar
Two significant movements have recently taken place among us. The Pullman Porters are organizing a trade union and the colored Communists have met in Chicago. Both movements have found opposition and from the same source. So intelligent and skilled a group of men as the Pullman porters ought to have long since adopted collective bargaining with their employers. But the Pullman company and other white capitalists have discouraged this with threats and propaganda and lately they have hired a silly black lawyer, who holds a small job in. Washington, to help them.

\sphinxAtStartPar
Why is this? It is because since the days of slavery the black laborer has been allied with the white capitalist. Since emancipation he has been bribed by philanthropy so that he thinks the thoughts of the rich, the powerful, the employing class. At the same time he has been kicked out of the major part of the white labor movement and his resultant resentment has helped his alignment with capital. Today he is beginning to awake. The Pullman porters are going to have a union.

\sphinxAtStartPar
But beyond that if black men wish to meet and learn what laborers are doing in England or in Russia and sympathize with these movements they have a perfect right to do so; it is unjust of white men and idiotic of colored men to criticize the attempt. We should stand before the astounding effort of Soviet Russia to reorganize the industrial world with open mind and listening ears. Russia has not yet failed and Negroes must not swallow all the lies told about her. She may yet show the world the Upward Path. \sphinxstylestrong{The Crisis} does not know the persons back of the Chicago convention, but it asserts the right of any set of American Negroes to investigate and sympathize with any industrial reform whether it springs from Russia, China or the South Seas.


\bigskip\hrule\bigskip


\sphinxAtStartPar
\sphinxstyleemphasis{Citation:} Du Bois, W.E.B. 1925. “The Black Man and Labor.”  31(2):60.


\subsection{Again, Pullman Porters (1926)}
\label{\detokenize{Volumes/31/06/again_pullman_porters:again-pullman-porters-1926}}\label{\detokenize{Volumes/31/06/again_pullman_porters::doc}}
\sphinxAtStartPar
As a class Pullman porters are gentlemen in the best sense of that overworked word. They are courteous, silent and of infinite patience. Nowhere in the traveling world can one find a set of public servants who do their work so thoroughly and so well.

\sphinxAtStartPar
The porters are men of unusual skill. Let the doubter try to keep house and make and unmake beds and even serve meals, and at the same time satisfy the exacting and querulous tastes of two or three dozen persons, in a room 36 by 10 feet. In addition to that they have the most delicate duties and responsibilities. The womanhood of America rides undressed under their care and service and not in one case in a million have the porters even been impolite much less impertinent. The porters must decide difficult problems as to men and women, young and old, rich and poor, noisy and nervous, gamblers and prohibitionists, white and black.

\sphinxAtStartPar
Particularly in the service of their own race have the porters done an unforgetable service. Without stirring racial animosities, with infinite tact and with sympathetic courtesy they have made it possible for twelve million insulted people to travel with a minimum of insult and inconvenience. I have travelled 50,000 miles in every state in the union and without the ministrations of the Pullman Porter I should today be dead of exhaustion and shame.

\sphinxAtStartPar
As it is the Pullman Company, relying on indifferent public opinion, can buy directly and indirectly the silence of the press black and white, the connivance of the United States Department of Justice and the halfhearted slobbering of white union labor so as to block the belated effort of Pullman porters to form a real and effective labor union. And in order more completely to befuddle the men who are at their economic mercy, the company is offering them, with wide gestures of benevolence, a “company union” where hand\sphinxhyphen{}picked lackeys “representing” the porters will smother complaints and take orders meekly. And, says this rich corporation, take this, shut up or lose your jobs.

\sphinxAtStartPar
Well, perhaps it is better to lose this job. Perhaps we have served as porters long enough. We were good slaves; but we outgrew the job. We were good cheap servants; we are outgrowing that job. We are good porters. But if being porters means being driven slaves and alms\sphinxhyphen{}taking servants, then God haste the day when we outgrow that job.


\bigskip\hrule\bigskip


\sphinxAtStartPar
\sphinxstyleemphasis{Citation:} Du Bois, W.E.B. 1926. “Again, Pullman Porter.”  31(6):271.


\section{Socialism and Communism}
\label{\detokenize{Sections/socialism:socialism-and-communism}}\label{\detokenize{Sections/socialism::doc}}
\sphinxAtStartPar
Editorials on socialism and communism.
\begin{itemize}
\item {} 
\sphinxAtStartPar
{\hyperref[\detokenize{Volumes/19/04/cooperation::doc}]{\sphinxcrossref{Coöperation (1920)}}}

\item {} 
\sphinxAtStartPar
{\hyperref[\detokenize{Volumes/22/03/negro_and_radical_thought::doc}]{\sphinxcrossref{The Negro and Radical Thought (1921)}}}

\item {} 
\sphinxAtStartPar
{\hyperref[\detokenize{Volumes/22/04/class_struggle::doc}]{\sphinxcrossref{The Class Struggle (1921)}}}

\item {} 
\sphinxAtStartPar
{\hyperref[\detokenize{Volumes/22/06/socialism_and_the_negro::doc}]{\sphinxcrossref{Socialism and the Negro (1921)}}}

\item {} 
\sphinxAtStartPar
{\hyperref[\detokenize{Volumes/22/06/single_tax::doc}]{\sphinxcrossref{The Single Tax (1921)}}}

\item {} 
\sphinxAtStartPar
{\hyperref[\detokenize{Volumes/27/03/black_man_and_the_wounded_world::doc}]{\sphinxcrossref{The Black Man and the Wounded World (1923)}}}

\item {} 
\sphinxAtStartPar
{\hyperref[\detokenize{Volumes/33/01/russia_1926::doc}]{\sphinxcrossref{Russia, 1926 (1926)}}}

\item {} 
\sphinxAtStartPar
{\hyperref[\detokenize{Volumes/33/04/judging_russia::doc}]{\sphinxcrossref{Judging Russia (1927)}}}

\item {} 
\sphinxAtStartPar
{\hyperref[\detokenize{Volumes/35/11/dunbar_national_bank::doc}]{\sphinxcrossref{The Dunbar National Bank (1928)}}}

\item {} 
\sphinxAtStartPar
{\hyperref[\detokenize{Volumes/38/09/negro_and_communism::doc}]{\sphinxcrossref{The Negro and Communism (1931)}}}

\item {} 
\sphinxAtStartPar
{\hyperref[\detokenize{Volumes/40/07/our_class_struggle::doc}]{\sphinxcrossref{Our Class Struggle}}}

\end{itemize}


\subsection{Coöperation (1920)}
\label{\detokenize{Volumes/19/04/cooperation:cooperation-1920}}\label{\detokenize{Volumes/19/04/cooperation::doc}}
\sphinxAtStartPar
Several coöperative efforts are starting among colored people. Probably today, there are fifty or more local efforts. Most of them are sporadic, and will fail. Some few are the efforts of individuals who use the magic word coöperation for stores in which there is not a trace of the coöperation principle.

\sphinxAtStartPar
There are a dozen or more which are largely coöperative, but not entirely—for instance, they have shares, and the number which one man may own is limited. The shareholders are obliged to buy a certain minimum amount of goods before they can share in the profits.

\sphinxAtStartPar
This is only partially coöperative. Full coöperation requires: cheap shares, of which anyone can own any number; \sphinxstylestrong{but} there is no temptation to own large numbers of shares, because \sphinxstylestrong{profits are divided according to the amount the person buys.}

\sphinxAtStartPar
Why, now, do beginners hesitate to make this last provision? Because having stirred up the people by the argument of race loyalty and opened the store, they say: “Why should I surrender the coming profits to a mass of people whom the driblets will not greatly benefit? Why not keep them and \sphinxstylestrong{grow rich}!”

\sphinxAtStartPar
Hesitate, brother, hesitate, \sphinxstylestrong{right there}! Remember that with the present chain grocery store and trust system, your individual grocery has a small chance to succeed, because the Trust can and will undersell you.

\sphinxAtStartPar
But with the true coöperative principle, your clientele is nailed down. Your shareholders are pledged by their own interests to trade with you, and to trade often and much. The more they spend the more they make. Your business is no guesswork. You know just how much to buy. If the chain store cuts prices below cost, your people will buy of you at the higher price, because they know that the low price is a temporary trick for which they themselves will eventually pay. Whatever happens, you \sphinxstylestrong{cannot} fail as long as your shareholders are true, and they will be true as long as. they share in the profits according to their purchases.

\sphinxAtStartPar
Don’t be afraid. Try the whole coöperative program. Write us.


\bigskip\hrule\bigskip


\sphinxAtStartPar
\sphinxstyleemphasis{Citation:} Du Bois, W.E.B. 1920. “Coöperation.” \sphinxstyleemphasis{The Crisis}. 19(4): 171\sphinxhyphen{}172.


\subsection{The Negro and Radical Thought (1921)}
\label{\detokenize{Volumes/22/03/negro_and_radical_thought:the-negro-and-radical-thought-1921}}\label{\detokenize{Volumes/22/03/negro_and_radical_thought::doc}}
\begin{sphinxShadowBox}
\sphinxstylesidebartitle{}

\sphinxAtStartPar
\sphinxhref{https://en.wikipedia.org/wiki/Claude\_McKay}{Claude McKay} was a prominent African American poet, socialist, and black nationalist who helped form the \sphinxhref{https://en.wikipedia.org/wiki/African\_Blood\_Brotherhood}{African Blood Brotherhood}.
\end{sphinxShadowBox}

\sphinxAtStartPar
Mr. \sphinxstylestrong{Claude McKay}, one of the editors of \sphinxstyleemphasis{The Liberator} and a Negro poet of distinction, writes us as follows: “I am surprised and sorry that in your editorial, ‘The Drive’, published in \sphinxstylestrong{The Crisis} for May, you should leap out of your sphere to sneer at the Russian Revolution, the greatest event in the history of humanity; much greater than the French Revolution, which is held up as a\_ wonderful achievement to Negro children and students in white and black schools. For American Negroes the indisputable and outstanding fact of the Russian Revolution is that a mere handful of Jews, much less in ratio to the number of Negroes in the American population, have attained, through the Revolution, all the political and social rights that were denied to them under the regime of the Czar.
\begin{quote}

\sphinxAtStartPar
“Although no thinking Negro can deny the great work that the N.A.A.C.P. is doing, it must yet be admitted that from its platform and personnel the Association cannot function as a revolutionary working class organization. Aid the overwhelming majority of American Negroes belong by birth, condition and repression to the working class. Your aim is to get for the American Negro the political and social rights that are his by virtue of the Constitution, the rights which are denied him by the Southern oligarchy with the active cooperation of the state governments and the tacit support of northern business interests. And your ‘aim is a noble one, which deserves the support of all progressive Negroes. \_ “But the Negro in politics and social life 1s ostracized only technically by the distinction of color; in’ reality the Negro is discriminated against because he is of the lowest type of worker.
\end{quote}
\begin{quote}

\sphinxAtStartPar
“Obviously, this economic difference between the white and black workers manifests itself in various forms, in color prejudice, race hatred, political and social boycotting and lynching of Negroes. And all the entrenched institutions of white America,—law courts, churches, schools, the fighting forces and the Press,—condone these iniquities perpetrated upon black men; iniquities that are dismissed indifferently as the inevitable result of the social system. Still, whenever it suits the business interests controlling these institutions to mitigate the persecutions against Negroes, they do so with impunity. When organized white workers quit their jobs, Negroes, who are discouraged by the whites to organize, are sought to take their places. And these strike\sphinxhyphen{}breaking Negroes work under the protection of the military and the police. But as ordinary citizens and workers, Negroes are not protected by the military and the police from the mob. The ruling classes will not grant Negroes those rights which, on a lesser scale and more plausibly, are withheld from the white proletariat. The concession of these rights would immediately cause a Revolution in the economic life of this country.”
\end{quote}

\sphinxAtStartPar
We are aware that some of our friends have been disappointed with \sphinxstylestrong{The Crisis} during and since the war. Some have assumed that we aimed chiefly at mounting the band wagon with our cause during the madness of war; others thought that we were playing safe so as to avoid the Department of Justice; and still a third class found us curiously stupid in our attitude toward the broader matters of human reform. Such critics, and Mr. McKay is among them, must give us credit for standing to our guns in the past at no little cost in many influential quarters, and they must also remember that we have one chief cause,—the emancipation of the Negro, and to this all else must be subordinated—not because other questions are not important but because to our mind the most important social question today is recognition of the darker races.

\sphinxAtStartPar
Turning now to that marvelous set of phenomena known as the Russian Revolution, Mr. McKay is wrong in thinking that we have ever intentionally sneered at it. On the contrary, time may prove, as he believes, that the Russian Revolution is the greatest event of the nineteenth and twentieth centuries, and its leaders the most unselfish prophets. At the same time \sphinxstylestrong{The Crisis} does not know this to be true. Russia is incredibly vast, and the happenings there in the last five years have been intricate to a degree that must make any student pause. We sit, therefore, with waiting hands and listening ears, seeing some splendid results from Russia, like the cartoons for public education recently exhibited in America, and hearing of other things which frighten us.

\sphinxAtStartPar
We are moved neither by the superficial omniscience of Wells nor the reports in the New York \sphinxstyleemphasis{Times}; but this alone we do know: that the immediate work for the American Negro lies in America and not in Russia, and this, too, in spite of the fact that the Third Internationale has made a pronouncement which cannot but have our entire sympathy:
\begin{quote}

\sphinxAtStartPar
“The Communist Internationale once forever breaks with the traditions of the Second Internationale which in reality only recognized the white race. The Communist Internationale makes it its task to emancipate the workers of the entire world. The ranks of the Communist Internationale fraternally unite men of all colors: white, yellow and black—the toilers of the entire world.”
\end{quote}

\sphinxAtStartPar
Despite this there come to us black men two insistent questions: What is today the right program of socialism? The editor of \sphinxstylestrong{The Crisis} considers himself a Socialist but he does not believe that German State Socialism or the dictatorship of the proletariat are perfect panaceas. He believes with most thinking men that the present method of creating, controlling and distributing wealth is desperately wrong; that there must come and is coming a social control of wealth; but he does not know just what form that control is going to take, and he is not prepared to dogmatize with Marx or Lenin. Further than that, and more fundamental to the duty and outlook of \sphinxstylestrong{The Crisis}, is this question: How far can the colored people of the world, and particularly the Negroes of the United States, trust the working classes?

\sphinxAtStartPar
Many honest thinking Negroes assume, and Mr. McKay seems to be one of these, that we have only to embrace the working class program to have the working class embrace ours; that we have only to join trade Unionism and Socialism or even Communism, as they are today expounded, to have Union Labor and Socialists and Communists believe and act on the equality of mankind and the abolition of the color line. \sphinxstylestrong{The Crisis} wishes that this were true, but it is forced to the conclusion that it is not.

\sphinxAtStartPar
The American Federation of Labor, as representing the trade unions in America, has been grossly unfair and discriminatory toward Negroes and still is. American Socialism has discriminated against black folk and before the war was prepared to go further with this discrimination. European Socialism has openly discriminated against Asiatics. Nor is this surprising. Why should we assume on the part of unlettered and suppressed masses of white workers, a clearness of thought, a sense of human brotherhood, that is sadly lacking in the most educated classes?

\sphinxAtStartPar
Our task, therefore, as it seems to \sphinxstylestrong{The Crisis}, is clear: We have to convince the working classes of the world that black men, brown men, and yellow men are human beings and suffer the same discrimination that white workers suffer. We have in addition to this to espouse the cause of the white workers, only being careful that we do not in this way allow them to jeopardize our cause. We must, for instance, have bread. If our white fellow workers drive us out of decent jobs, we are compelled to accept indecent wages even at the price of “scabbing”. It is a hard choice, but whose is the blame? Finally despite public prejudice and clamour, we should examine with open mind in literature, debate and in real life the great programs of social reform that are day by day being put forward.

\sphinxAtStartPar
This was the true thought and meaning back of our May editorial. We have an immediate program for Negro emancipation laid down and thought out by the N.A.A.C.P. It is foolish for us to give up this practical program for mirage in Africa or by seeking to join a revolution which we do not at present understand. On the other hand, as Mr. McKay says, it would be just as foolish for us to sneer or even seem to sneer at the blood\sphinxhyphen{}entwined writhing of hundreds of millions of our whiter human brothers.


\bigskip\hrule\bigskip


\sphinxAtStartPar
\sphinxstyleemphasis{Citation:} “Socialism and the Negro.” Editorial. 1921. 22(3):102\sphinxhyphen{}104.


\subsection{The Class Struggle (1921)}
\label{\detokenize{Volumes/22/04/class_struggle:the-class-struggle-1921}}\label{\detokenize{Volumes/22/04/class_struggle::doc}}
\sphinxAtStartPar
The N.A.A.C.P. hag been accused of not being a ‘“revolutionary” body. This is quite true. We do not believe in revolution. We expect revolutionary changes in many parts of this life and this world, but we expect these changes to come mainly through reason, human sympathy and the education of children, and not by murder. We know that there have been times when organized murder seemed the only way out of wrong, but we believe those times have been very few, the cost of the remedy excessive, the results as terrible as beneficent, and we gravely doubt if in the future there will be any real recurrent necessity for such upheaval.

\sphinxAtStartPar
Whether this is true or not, the N.A.A.C.P. is organized to agitate, to investigate, to expose, to defend, to reason, to appeal. This is our program and this is the whole of our program. What human reform demands today is light, more light; clear thought, accurate knowledge, careful distinctions.

\sphinxAtStartPar
How far, for instance, does the dogma of the “class struggle” apply to black folk in the United States today? Theoretically we are a part of the world proletariat in the sense that we are mainly an exploited class of cheap laborers; but practically we are not a part of the white proletariat and are not recognized by that proletariat to any great extent. We are the victims of their physical oppression, social ostracism, economic exclusion and personal hatred; and when in self defense we seek sheer subsistence we are howled down as “scabs”.

\sphinxAtStartPar
Then consider another thing: the colored group is not yet divided into capitalists and laborers. There are only the beginnings of such a division. In one hundred years if we develop along conventional lines we would have such fully separated classes, but today to a very large extent our laborers are our capitalists and our capitalists are our laborers. Our small class of well\sphinxhyphen{}to\sphinxhyphen{}do men have come to affluence largely through manual toil and have never been physically or mentally separated from the toilers. Our professional classes are sons and daughters of porters, washerwomen and laborers.

\sphinxAtStartPar
Under these circumstances how silly it would be for us to try to apply the doctrine of the class struggle without modification or thought. Let us take a particular instance. Ten years ago the Negroes of New York City lived in hired tenement houses in Harlem, having gotten possession of them by paying higher rents than If they had tried to white tenants. If they had tried to escape these high rents and move into quarters where white laborers lived, the white laborers would have mobbed and murdered them. On the other hand, the white capitalists raised heaven and earth either to drive them out of Harlem or keep their rents high. Now between this devil and deep sea, what ought the Negro socialist or the Negro radical or, for that matter, the Negro conservative do?

\sphinxAtStartPar
Manifestly there was only one thing for him to do, and that was to buy Harlem; but the buying of real estate calls for capital and credit, and the institutions that deal in capital and credit are capitalistic institutions. If now, the Negro had begun to fight capital in Harlem, what capital was he fighting? If he fought capital as represented by white big real estate interests, he was wise; but he was also just as wise when he fought labor which insisted on segregating him in work and in residence.

\sphinxAtStartPar
If, on the other hand, he fought the accumulating capital in his own group, which was destined in the years 1915 to 1920 to pay down \$5,000,000 for real estate in Harlem, then he was slapping himself in his own face. Because either he must furnish capital for the buying of his own home, or rest naked in the slums and swamps. It is for this reason that there is today a strong movement in Harlem for a Negro bank, and a movement which is going soon to be successful. This Negro bank eventually is going to bring into cooperation and concentration the resources of fifty or sixty other Negro banks in the United States, and this aggregation of capital is going to be used to break the power of white capital in enslaving and exploiting the darker world.

\sphinxAtStartPar
Whether this is a program of socialism or capitalism does not concern us. It is the only program that means salvation to the Negro race. The main danger and the central question of the capitalistic development through which the Negro American group is forced to go is the question of the ultimate control of the capital which they must raise and use. If this capital is going to be controlled by a few men for their own benefit, then we are destined to suffer from our own capitalists exactly what we are suffering from white capitalists today. And while this is not a pleasant prospect, it is certainly no worse than the present actuality. If, on the other hand, because of our more democratic organization and our widespread inter\sphinxhyphen{}class sympathy we can introduce a more democratic control, taking advantage of what the white world is itself doing to introduce industrial democracy, then we may not only escape our present economic slavery but even guide and lead a distrait economic world.


\bigskip\hrule\bigskip


\sphinxAtStartPar
\sphinxstyleemphasis{Citation:} “The Class Struggle.” Editorial. 1921. \sphinxstyleemphasis{The Crisis}. 22(4): 151\sphinxhyphen{}152.


\subsection{Socialism and the Negro (1921)}
\label{\detokenize{Volumes/22/06/socialism_and_the_negro:socialism-and-the-negro-1921}}\label{\detokenize{Volumes/22/06/socialism_and_the_negro::doc}}
\sphinxAtStartPar
We have an interesting letter from John H. Owens of Washington, which we would like to publish in full but can only note certain extracts. Mr. Owens says, in answer to the editorial in the July \sphinxstylestrong{Crisis} on “{\hyperref[\detokenize{Volumes/22/03/negro_and_radical_thought::doc}]{\sphinxcrossref{\DUrole{doc,std,std-doc}{The Negro and Radical Thought}}}}”:
\begin{quote}

\sphinxAtStartPar
“Is there not just the bare possibility that some of the issues which you consider subordinate to your central idea (of the emancipation of the Negro) might possess the neucleus of a tangible and definite solution?”
\end{quote}

\sphinxAtStartPar
There is more than a bare possibility, and the Negro must study proposals and reforms with great care to see if they do not carry with them some help in the solution of his problem. But he must not assume that because a proposed solution settles many important human problems, for this reason it is necessarily going to settle his.

\sphinxAtStartPar
Mr. Owens continues:
\begin{quote}

\sphinxAtStartPar
“The Negro group is almost a pure proletarian group,—this fact admits of no denial. Above 90 per cent. of the Negroes are unskilled, untrained workers, and unorganized. Thus it would seem that the race as a whole has less reason to be suspicious of any movement of a proletarian nature than of some scheme which offers a questionable solution for the ills of the talented minority.”
\end{quote}

\sphinxAtStartPar
The Negro has little reason to be suspicious of a proletarian movement if that movement is for the good of the proletariat; but it does not follow that all movements proposed by the proletariat themselves are for their own good. The workers of the world are, through no fault of their own, ignorant, inexperienced men. It is not for a moment to be assumed that movements into which they are drawn or which they themselves initiate are necessarily the best for them. If, however, the Negro sees a movement for the proletariat which, after careful thought and experience, he is convinced is for the good of the working class, then as a worker he is bound to give every aid to such a movement.
\begin{quote}

\sphinxAtStartPar
“Universal political enfranchisement would offer no positive relief. This the Northern Negro already enjoys; yet he suffers under the burden of social, political, and economic injustices. His condition is little more to be envied than that of his Southern brother.”
\end{quote}

\sphinxAtStartPar
The vote is not a panacea. It is a means to an end. The condition of the Negro in the North because of his political power is a great deal better than the condition of the Negro in the South. He is, of course, hindered in the North by greater competition for work, while in the South certain fields are open to him. The voter, white and black, has not yet learned to control industry through his vote, but he is learning, and only through the use of the ballot is real reform in industry and industrial relations coming.
\begin{quote}

\sphinxAtStartPar
“Does not the editor think that State Socialism, Communism, or even the dread dictatorship of the proletariat, offers a better solution to the problems of the proletariat than any scheme suggested by the exploiting classes,—those who profit by the present system? And since the Negro is over 90 per cent. proletarian, is it not almost logical to assume that this would also offer a better solution to this problem than anything heretofore proposed?”
\end{quote}

\sphinxAtStartPar
I do decidedly think that many proposals made by Socialists and Communists and even by the present rulers of Russia would improve the world if they could be adopted; but I do not believe that such adoption can successfully come through war or force or murder, and I do not believe that the sudden attempt to impose a new industrial system and new ideas of industrial life can be successful without the long training of human beings. I believe that Socialism must be evolutionary, not in the sense that it must take 50,000 years, but in the sense that it does mean hard work for many generations. Beginnings can and should be made this minute or tomorrow or next year. It is precisely because of our present ignorance and our widespread  assumptions as to profit and business that we cannot immediately change the world. It is true that those who today are sucking the industrial life blood of the nations get their chance to keep on by simply asserting that no better way is offered and present methods suit present human nature. We who suffer and believe in reform must not think that we can answer such persons. successfully simply by saying that present industrial society is not in accordance with human nature. It is in accordance with human nature today, but human nature can and must and will be changed.
\begin{quote}

\sphinxAtStartPar
“We are both of the opinion that the present method of control and distribution of wealth is desperately wrong. We are en rapport on the conclusion that a form of social control is inevitable. We hold this particular truth to be self\sphinxhyphen{}evident,—that a change must come about. But how? I think that we both may be safe in assuming that any initiative in bringing about a better distribution of wealth must be taken by those who benefit least by the present system.”
\end{quote}

\sphinxAtStartPar
The change in industrial organization must come from those who think and believe. We cannot assume that necessarily redemption is coming from those who suffer. It may come from those who enjoy the fruit of suffering, but who come to see that such enjoyment is wrong. The point that we must hold clearly is that a proposal for reform is not necessarily good and feasible simply because it comes from a laboring man, and it is not wrong and unjustifiable simply because it comes from a millionaire. It must be judged by itself and not by its source.
\begin{quote}

\sphinxAtStartPar
“You ask the question: ‘How far can the colored people of the world, and the Negroes of the United States in particular, trust the working classes?’ This is a good question, and easier asked than answered. But I would like to ask further: How far can the Negroes and other dark peoples trust the exploiting Nationalists and Imperialists? Is it the English working classes that are exploiting India, sucking the very life\sphinxhyphen{}blood from a starving population and grinding the natives down into the desert dust in order to support English ‘gentlemen’ in idleness and luxury? Are the English, French and Belgian working classes raping Africa, taking ill\sphinxhyphen{}gotten gains from a trusting population? Are the working classes of America attempting to fasten the yoke of subjugation upon the neck of Santo Domingo, and stifle liberty and freedom of speech and press in Haiti? If we have cause to distrust the working classes, by what precept of example should we put faith in the specious promises of the masters?”
\end{quote}

\sphinxAtStartPar
I think these questions touch the center of much modern effort and reform. I maintain that English working classes are exploiting India; that the English, French and Belgian laborers are raping Africa; that the working classes of America are subjugating Santo Domingo and Haiti. They may not be as conscious of all they are doing as their more educated masters, called Nationalists and Imperialists, but they are consciously submitting themselves to the leadership of these men; they are voluntarily refusing to know; they are systematically refusing to listen; they are blindly voting armies and navies and hidden diplomacy, regardless of the result, and while the individual white employee in Europe and America is less to be condemned than the individual capitalist for the way in which the darker nations have been treated, he can not escape his responsibility. He is co\sphinxhyphen{}worker in the miserable modern subjugation of over half the world.


\bigskip\hrule\bigskip


\sphinxAtStartPar
\sphinxstyleemphasis{Citation:} “Socialism and the Negro (1921)” Editorial.  22(6):245\sphinxhyphen{}247.


\subsection{The Single Tax (1921)}
\label{\detokenize{Volumes/22/06/single_tax:the-single-tax-1921}}\label{\detokenize{Volumes/22/06/single_tax::doc}}
\sphinxAtStartPar
Negro radicals com­paratively little has been said of the Single Tax, and we take it that few of our colored readers know much of Henry George’s “Progress and Poverty” and its after­ math. There are, however, many signs that the economic thought of the world is turning increasingly to­ ward this rather awkwardly named method of righting economic wrong. The basic thought which Henry George and his followers laid for the world was that the sum of economic villanies is monopoly and that monop­oly always in the last analysis rests on the land and the produce of land; that so long as it is possible and legal to own land, to own mines, to own oil wells, to own “rights of way,” it will be possible to lay upon the pub­ lic an enormous tax which labor must pay and which will in the end defeat democracy.

\sphinxAtStartPar
None can doubt but that this is true. They may doubt if the single and simple expedient of a tax on land values will remedy the growing difficulty, but even this is arguable. At any rate, monopoly of land and its products is the most sinister thing that faces modern industrial progress the rise of laboring classes, and the emancipation of the Darker World.


\bigskip\hrule\bigskip


\sphinxAtStartPar
\sphinxstyleemphasis{Citation:} “The Single Tax.” (1921) Editorial. 22(6):248.


\subsection{The Black Man and the Wounded World \sphinxhyphen{} A History of the Negro Race in the World War and After (1923)}
\label{\detokenize{Volumes/27/03/black_man_and_the_wounded_world:the-black-man-and-the-wounded-world-a-history-of-the-negro-race-in-the-world-war-and-after-1923}}\label{\detokenize{Volumes/27/03/black_man_and_the_wounded_world::doc}}
\sphinxAtStartPar
\sphinxstyleemphasis{Chapter 1. Interpretations}

\sphinxAtStartPar
What is the ruling power in any given country? Speaking modernwise most would say Public Opinion. But this of course is a loose and inaccurate term. Opinion is individual. No “Public” can have an “opinion”. The figure of speech is permissible but easily and crassly misleading. It is the power, wishes, opinion of certain persons which rule the world. These Dominant Wills may rule by physical force, or superior intelligence or greater wealth or logical persuasion, and consequently may be regarded as Dominant Powers or Dominant Wishes or Dominant Intelligence or Highest Good—but always whatever rules exhibits itself as Will—action, effective deed. To these Dominant Wills, be it the Will of One, or the Agreements of Many—of a Minority or of a Majority, and be it put in power by chance, force or reason—there must be, as long as it rules, the Submission of all individuals to its mandates. In these Current Submissions of individual men lies the core and kernel of modern ethical judgment of group action.

\sphinxAtStartPar
The effort to make these acts of submission free individual judgments is the movement toward Freedom. But Freedom is always restrained by the fear that the dethroning of the Dominant Wills at any time —that is, the refusal of a large number of persons to submit to a particular opinion or set of opinions—will result in the partial or total overthrow of civilized society, before enough submissions acknowledge, and thus enthrone, another Dominant Will. It is this fear of anarchy that leads to the persistent opposition to the right to challenge the Dominant Wills. The Right of Challenge is Democracy, and to Democracy the momentarily Dominant Wills are almost always opposed, particularly if dominion is based on force or bribery.

\sphinxAtStartPar
If the Dominant Wills are based on reason why should they fear universal Challenge—universal Democracy? Because most people are too inexperienced to get at the truth and too ignorant to reason correctly  on given data. This ignorance can be corrected by universal education, but the Dominant Wills sometimes (1) do not believe in the possibility of educating all folk; (2) have desires and ambitions which can be satisfied only by the persistence of ignorance among the mass.

\sphinxAtStartPar
Thus the Dominant Wills in most periods of history have opposed the Challenge of Democracy because they desire the ignorance of most men. And they defend this desire by the assertion or even passionate belief that most men must and should be ignorant if civilization is to prevail.

\sphinxAtStartPar
Here then lies the heart and kernel of all social and political problems at any time. First we must ask whose is this Dominant Will? Then, is there any right of challenge and who can and does exercise the right? What is the attitude of the Dominant Will toward the increased intelligence and efficiency of men?

\sphinxAtStartPar
In the first quarter of the 20th century, the Dominant Wills in most lands are the wills of those persons who are seeking Incomes as distinguished from Wages and who are, by training, masters of the intricate organization of modern commerce and industry. The distinction between Income and Wage is of course not absolute, but Wage usually means a direct return for personal effort, while Income is the return which one commands by reason of his property rights or influence or social power. It is the almost universal ambition of men today to receive sufficient Income so as to make personal exertion on their part unnecessary—in other words, as we say, they desire to be “independent”.

\sphinxAtStartPar
The income\sphinxhyphen{}receiving persons form a small but intelligent and highly specialized minority of men, while the mass of men are wage\sphinxhyphen{}earners or community workers in unorganized industry. So powerful and persuasive is this ruling class that most people identify its will with civilization and its industrial aims with life itself. Industry is life—commerce is government, they say openly or silently. Now modern industry  requires (1) large accumulations of tools, machines, materials and transition goods and (2) regular skilled labor working over large areas of time and space synchronized with machine\sphinxhyphen{}like co\sphinxhyphen{}operation. The result is great income in goods and services which the Dominant Wills may allocate as they wish; and since the \sphinxstyleemphasis{raison d’étre} of the present supporters of the Dominant Wills is the desire to share largely in this income, present government tends to support and develop the rich.

\sphinxAtStartPar
To this tendency is opposed the interest of the majority of men who are wage earners or in unorganized or primitive industry. What Right of Challenge have they before the Dominant Will? The democratic movement of the 19th century gave a few of them (the men in organized industry) a nominal right to challenge legally and at regular intervals the Dominant Wills. This Right to Vote—a mighty landmark in the advance of Man, and one which every group achieves sooner or later, or dissolves —is the beginning and not the end of democracy and meets at the outset baffling difficulties and limitations. These are chiefly (1) Ignorance (2) Propaganda (3) Law and Custom.
\begin{enumerate}
\sphinxsetlistlabels{\arabic}{enumi}{enumii}{}{.}%
\item {} 
\sphinxAtStartPar
Human society in its industrial, religious, aesthetic and other aspects is a tremendously intricate mechanism. Few even of the most intelligent grasp it thoroughly and most men have no adequate conception of it. When now the Dominant Wills of a society form a trained group led by their selfish interests as well as their intelligence and ideals to fasten themselves in power, the ignorant mass has small chance of using their vote with enough intelligence to dislodge them without catastrophe to the State. The evident remedy for this is Education, the formal training of Children, the higher training of Youth, and the broader training of Citizens by experience, information, contacts and art—in other words the spread of Truth.

\item {} 
\sphinxAtStartPar
But the spread of Truth is undertaken today by Propaganda. Now the dissemination of Truth presupposes normally a group of absolutely impartial Truth Bearers or Teachers or Priests or Prophets who know the Truth and who quite impartially and persistently make all free of it who will listen. But Propaganda is the effort not necessarily to spread the Truth, but to make people BELIEVE that what they hear is true; and to the propagandist any means which will accomplish this end of passionate, of unwavering and of forcible uncritical belief is justifiable. This is a dangerous but a very widespread method of public teaching today and what makes it most dangerous is the use which it makes of the Lie.

\sphinxAtStartPar
Lying is so dangerous an enemy to organized human life that usually it is regarded as an absolutely unjustifiable instrument of human advance. Yet manifestly everyone admits certain extreme cases when a deliberate Lie can be defended; and many are willing to use a partial truth to gain a good end while millions are willing without any attempt at investigation or corroboration to assume as true anything that they passionately wish to be true. Propaganda then, with large use of the deliberate Lie, the Half Truth and the Unproven Wish, has become a tremendous weapon in our day and is used particularly by the Dominant Wills to establish themselves in power by voluntary limitation of the Right of Challenge, or in other words by limiting the right to vote or the votable questions or the general field of democratic government. By this means most people today are convinced that the matters of work, wages and organized industry are quite beyond the possibility of democratic control and always will be; while a goodly number believe that the inter\sphinxhyphen{}relations of great nations can never be matters of open popular decision and many think that the making and interpreting of laws is not a matter for the average voter to have a voice in.

\item {} 
\sphinxAtStartPar
Finally Law systems greatly impede democracy. Law is the attempt to reduce to logical statement the Dominant Will of the day. This is an exceedingly difficult task in itself but it is made more difficult because both the statement and the interpretation of the statement’s meaning in particular cases are in the hands of technicians. These technically trained lawyers are dominated on the one hand by a mediaeval desire for perfections and consistency which makes them slaves to the precedents of dead centuries, and on the other hand they drift largely into the pay and control of the dominant income\sphinxhyphen{}seeking classes.

\end{enumerate}

\sphinxAtStartPar
Thus ignorance, propaganda and customary law have so delimited the field of practical political democracy that it has become a very ineffective method of challenging the rule of the Dominant Wills. At the same time the rule of the Income\sphinxhyphen{}seekers has become peculiarly oppressive and dangerous, and for this reason the call for democratic control becomes more and more insistent. To repeat: In order to understand modern civilized life one must realize the conflict which has arisen between the Income Seekers and the Wage Earners. The great accomplishment of the 19th century was the organization of work—the far gathering of raw materials, the making of tools, machines and production goods and the synchronizing of effort. The result is a marvelous triumph of human skill and efficiency in making available a miraculous amount of consumptive goods and human services. If these goods and services had been designed for and applied to satisfying the highest wants of the mass of men our advance in culture would have been tremendous. It has been great despite the fact that the annual output of goods and services is arranged mainly to satisfy the wants of a small but powerful minority of the civilized world.

\sphinxAtStartPar
The power of this minority arises from their monopoly of finished goods, materials and production goods, which enables them largely to determine what goods shall be produced and what services paid for and at what rate, and also the ownership of the goods and services. This tremendous power —by far the greatest of modern days, and overshadowing most political power—has been successfully challenged with very great difficulty. The hindrances are: the widespread ignorance of the industrial process; the desire of most men to share this vast power rather than curb it; the use of widespread propaganda to prove the impossibility of any fundamental change in the control of industry. In this way Democratic control has been largely kept out of industry and the owners of goods and materials have become the almost unchallenged Dominant Wills of the World.

\sphinxAtStartPar
Almost, but not quite, unchallenged, for the wage\sphinxhyphen{}earners have begun the challenge. The wage\sphinxhyphen{}earners are those whose work is determined and wages fixed by the Income Owners. There are among them a large number of Income\sphinxhyphen{}Seekers—i. e. those who wish not so much to curb the power of the Income Owners as to share their spoils. But gradually there has grown up among them an opinion that the wage system is right and the income system wrong, that every one should work and be paid for the work, and that the ownership of materials and productive goods should vest in the democratically controlled state. Meantime, however, before this thought became clear in their minds, their practical protest was against the amount of wages allotted them. It was too small for decent living or the rearing of children, especially when contrasted with the riches and power of the industrial world. At first they were answered that the rate of wages was not a matter of will but of natural law. Wages could not be increased save by reducing the number of laborers, by starvation, cataclysm, or voluntary restraints. This the laborers refused to believe. They tried to use political power, but were baffled by ignorance, law and propaganda.

\sphinxAtStartPar
Waters dammed in one direction burst their bounds in other and unexpected places. Democratic control, baffled in electing officials and law\sphinxhyphen{}making, found a new weapon in the Strike and Boycott. That is, realizing that the heart and centre of the Income\sphinxhyphen{}Seeker’s power was his synchronizing of the industrial process over wide areas coupled with control of materials and machines, the hand workers sought to stop their coöperation and to refuse to work or refuse to buy at such critical times and places as would compel consideration of their wishes.

\sphinxAtStartPar
Propaganda and legal obstacles were for years used against the Strike, but after a century it has become a recognized weapon of offense by Wage\sphinxhyphen{}Earners against Employers. The open warfare of the Strike gradually softened into the parleys of the Labor Union and the Corporations; then came the shop committees and Coöperative buying, and there was foreshadowed the syndicalist control of factories and Coöperative production.

\sphinxAtStartPar
This development stimulated political democracy by educating the voters in the intricacies of industrial organization and giving them experience in group work. More political activity and more effective voting appeared so that the State itself was forced not only into some general control of Industry but even into undertaking certain lines of industrial activity. Industrial Democracy or rather Democracy in Industry seemed the swiftly approaching goal of civilization at the opening of the 20th century.

\sphinxAtStartPar
But the Dominant Wills of the Income Seekers were moving to wider conquests and had been for nearly fifty years, and the very triumphs of Industrial Democracy furnished an opportunity. Beginning with the African Slave trade a world commerce had grown up. Like national industry it began haphazard and was gradually organized and systematized. Gradually the local and national industrial systems tended to become cogs in the wheel of an international industrial organization. The basic foundations of this vaster set of enterprises were: (A) The ownership of vast areas of “colonies” inhabited by semi\sphinxhyphen{}civilized people; (B) Slave labor or peons without wage; (C) A monopoly of valuable raw material; (D) A monopoly of transportation facilities. On this foundation it was proposed to build a national set of industries, and in these industries the wage earner would be pacified by high wages and even allowed some measure of democratic control. In other words the Dominant Wills proposed to share some of their economic power with the laborers in return for the political consent of the Laborer to the policy of conquest, slavery, monopoly and theft in Eastern Europe, Asia, Africa, and Central and South America.

\sphinxAtStartPar
This New Imperialism has widely prevailed and its way has been cleared by a new Propaganda. This Propaganda bases itself mainly on Race and Color—human distinctions long since discarded by Science as of little or no real significance. But this false scientific dogma which the 18th century rejected with avidity making freedom the basis of a new and world wide Humanity has been revamped by 20th century Industrialism as an Eternal Truth, so that most modern men of the masses believe the advancement of civilization necessarily involves slavery, lust and rapine in Africa.

\sphinxAtStartPar
With scarce an articulate word of protest then the world in the late 19th and early 20th centuries was hurriedly divided up among European Countries and the United States into colonies owned or controlled by white civilized nations, or “spheres of influence” dominated by them. To the casual glance of most folk this was simply a process of civilizing barbarians, “protecting” them and “developing” their resources. But its real nature is manifest when we ask, “For whose benefit is this New Imperialism of the white over the darker world?” Before 1914 the world answered with shrill accord, “For the benefit of the whites!” And they believed, thanks to organized Propaganda, that the salvation of Civilized Europe lay in the degradation of Uncivilized Africa and the subjection of the Balkans, Asia and the islands of the sea.

\sphinxAtStartPar
Since 1914, we are less assured. Since 1914 we have begun to fear lest our theory of exploitation of the semi\sphinxhyphen{}barbarians may not necessarily involve our own glorification. And this because in allocating the spoils of the Earth, Europe fell into a jealous quarrel that nearly overthrew Civilization and left it mortally wounded. Some there are still who see in this greatest catastrophe which the world ever knew simply a failure to agree. They argue that if Germany had not been so greedy and had been satisfied with the domination of Asia Minor, half of Portuguese Africa, and part of the Belgian Congo; and if Austria had been content with Bosnia and Herzegovina, and had not coveted Serbia, Roumania and most of the Balkans—that in this case the world industrial dominion under England, Germany, France, Italy and the United States could have been established and maintained. But is this true? On the contrary, it is a very doubtful truth. This God\sphinxhyphen{}defying dream had a thousand seeds of disaster: not simply a hundred recurring points of disagreement in colonial expansion and development, but the inevitable future reaction of the wage earners of Europe and the natives of the colonies.

\sphinxAtStartPar
Sooner or later Europe would learn two facts: (1) The dullest European wage\sphinxhyphen{}earner will gradually come to see that by upholding Imperial Aggression over the darker peoples by his political vote and his growing economic power he is but fastening tighter on himself the rule of the Rich; (2) Not even the most successful Propaganda, aided by Pseudo\sphinxhyphen{}Science and human hatred, can forever keep the white wage\sphinxhyphen{}earner from realizing that the victims of imperial greed in Asia and Africa are human beings like himself—suffering like him and from like causes, held in degradation and ignorance and like him, too, capable of infinite uplift and of ruling themselves and the world.

\sphinxAtStartPar
The Crisis then was bound to come. It did come in 1914\sphinxhyphen{}18. The Great War was a Scourge, an Evil, a retrogression to Barbarism, a waste, a wholesale murder. It was not necessary—it was precipitated by the will of men.

\sphinxAtStartPar
Who was to blame? Not Germany but certain Germans. Not England but certain Englishmen. Not France but certain Frenchmen. All those modern civilized citizens who submitted voluntarily to the Dominant Wills of those who ruled the leading lands in 1914 were blood guilty of the murder of the men who fell in the war. More guilty were those whose acts and thoughts made up the Dominant Wills and who were willing to increase their incomes at the expense of those who suffer in Europe and out, under the present industrial system. There is no dodging the issue. Guilt is personal. Deed is personal, Opinion and Will are personal. Systems and Nations are not to blame—individuals are to blame. Individuals caused the Great War, did its deviltry and are guilty of its endless Crime.
\begin{quote}

\sphinxAtStartPar
On account of its length and its frankly pro\sphinxhyphen{}Negro attitude, it is possible that Dr. Du Bois’ history of the Negro in the World War will have to be published by subscription. In this case the possibility of publication will depend on the number of persons willing to subscribe. If you are interested will you sign and return the appended blank or one similar to it?

\sphinxAtStartPar
The undersigned is interested in the publication of “The Black Man in the Wounded World” by Dr. W. E. Burghardt Du Bois and would like details as to its size, cost and date of issue when these matters have been determined on.
\end{quote}


\bigskip\hrule\bigskip


\sphinxAtStartPar
\sphinxstyleemphasis{Citation:} Du Bois, W.E.B. 1924. “The Black Man and the Wounded World: A History of the Negro Race in the World War and After” 27(3):110\sphinxhyphen{}114.


\subsection{Russia, 1926 (1926)}
\label{\detokenize{Volumes/33/01/russia_1926:russia-1926-1926}}\label{\detokenize{Volumes/33/01/russia_1926::doc}}
\sphinxAtStartPar
I am writing this in Russia. I am sitting in Revolution Square opposite the Second House of the Moscow Soviets and in a hotel run by the Soviet Government. Yonder the sun pours into my window over the domes and eagles and pointed towers of the Kremlin. Here is the old Chinese wall of the inner city; there is the gilded glory of the Cathedral of Christ, the Savior. Thro’ yonder gate on the vast Red Square, Lenin sleeps his last sleep, with long lines of people peering each day into his dead and speaking face. Around me roars a city of two millions—Holy Moscow.

\sphinxAtStartPar
I have been in Russia something less than two months. I did not see the Russia of war and blood and rapine. I know nothing of political prisoners, secret police and underground propaganda. My knowledge of the Russian language is sketchy and of this vast land, the largest single country on earth, I have traveled over only a small, a very small part.

\sphinxAtStartPar
But I have had certain advantages; I have seen something of Russia. I have traveled over two thousand miles and visited four of its largest cities, many of its towns, the Neva, Dneiper, Moscow and Volga of its rivers, and stretches of land and village. I have looked into the faces of its races—Jews, Tartars, Gypsies, Caucasians, Armenians and Chinese. To help my lack of language I have had personal friends, whom I knew before I came to Russia, as interpreters. They were born in Russia and speak English, French and German. This, with my English, German and French, has helped the language difficulty, but did not, of course, solve it.

\sphinxAtStartPar
I have not done my sight seeing and investigation in gangs and crowds nor according to the program of the official Foreign Bureau; but have in nearly all cases gone alone with one Russian speaking friend. In this way I have seen schools, universities, factories, stores, printing establishments, government offices, palaces, museums, summer colonies of children, libraries, churches, monasteries, boyar houses, theatres, moving\sphinxhyphen{}picture houses, day nurseries and co\sphinxhyphen{}operatives. I have seen some celebrations—self\sphinxhyphen{}governing children in a school house of an evening and 200,000 children and youths marching on Youth Day. I have talked with peasants and laborers, Commissars of the Republic, teachers and children.

\sphinxAtStartPar
Alone and unaccompanied I have walked the miles of streets in Leningrad, Moskow, Nijni Novgorod and Kiev at morning, noon and night; I have trafficked on the curb and in the stores; I have watched crowds and audiences and groups. I have gathered some documents and figures, plied officials and teachers with questions and sat still and gazed at this Russia, that the spirit of its life and people might enter my veins.

\sphinxAtStartPar
I stand in astonishment and wonder at the revelation of Russia that has come to me. I may be partially deceived and half\sphinxhyphen{}informed. But if what I have seen with my eyes and heard with my ears in Russia is Bolshevism, I am a Bolshevik.


\bigskip\hrule\bigskip


\sphinxAtStartPar
\sphinxstyleemphasis{Citation:} Du Bois, W.E.B. 1926. ” Russia, 1926.”  33(1):8.


\subsection{Judging Russia (1927)}
\label{\detokenize{Volumes/33/04/judging_russia:judging-russia-1927}}\label{\detokenize{Volumes/33/04/judging_russia::doc}}
\sphinxAtStartPar
There is no question but that a government can carry on business. Every government does. Whether governmental industry compares in efficiency with private industry depends entirely upon what we call efficiency. And here it is and not elsewhere that the Russian experiment is astonishing and new and of fateful importance to the future civilization. What we call efficiency in America is judged primarily by the resultant profit to the rich and only secondarily by the results to the workers. The face of industrial Europe and America is set toward private wealth; that is, toward the people who have large incomes. We recognize the economic value of small incomes mainly as a means of profit for great incomes. Russia seeks another psychology. Russia is trying to make the workingman the main object of industry. His well\sphinxhyphen{}being and his income are deliberately set as the chief ends of organized industry directed by the state.

\sphinxAtStartPar
One can stand on the streets of Moscow or Kiev and see clearly that Russia has struck at the citadels of the power that  rules modern countries. Not manhood suffrage, woman’s suffrage, state regulation of industry, social reform nor religious and moral teaching in any modern country have shorn organized wealth of its power as the Bolshevik Revolution has done in Russia. Is it possible to conduct a great modern government without the autocratic leadership of the rich? The answer is: this is exactly what Russia is doing today. But can she continue to do this? This is not a question of ethics or economics; it is a question of psychology. Can Russia continue to think of the State in terms of the worker? This can happen only if the Russian people believe and idealize the workingman as the chief citizen. In America we do not. The ideal of every American is the millionaire—or at least the man of “indepentent” income. We regard the laborers as the unfortunate part of the community and even liberal thought is directed toward “emancipating” the workingman by relieving him in part if not entirely of the necessity of work. Russia, on the contrary, is seeking to make a nation believe that work and work that is hard and in some respects disagreeable and work which is to a large extent physical is a necessity of human life at present and likely to be in any conceivable future world; that the people who do this work are the ones who should determine how the national income from their combined efforts should be distributed; in fine, that the Workingman is the State; that he makes civilization possible and should determine what civilization is to be.

\sphinxAtStartPar
For this purpose he must be a workingman of skill and intelligence and to this combined end Russian education is being organized. This is what the Russian Dictatorship of the Proletariat means. This dictatorship does not stop there. As the workingman is today neither skilled nor intelligent to any such extent as his responsibilities demand, there is within his ranks the Communist party, directing the proletariat toward their future dictatorship. This is nothing new. In this government “of the people” we have elaborate and many\sphinxhyphen{}sided arrangements for ruling the rulers. The test is, are we and Russia really preparing future rulers? In so far as I could see, in shop and school, in the press and on the radio, in books and lectures, in trades unions and National Congresses, Russia is. We are not.

\sphinxAtStartPar
Visioning now a real Dictatorship of the Proletariat, two questions follow. Is it possible today for a great nation to achieve such a workers’ psychology? And secondly, if it does achieve it what will be its effect upon the world? The achievement of such a psychology depends partly upon Russia and partly upon Western Europe and the United States. In Russia one feels today, even on a casual visit, the beginning of a workingman’s psychology. Workers are the people that fill the streets and live in the best houses, even though these houses are dilapidated; workers crowd (literally crowd) the museums and theaters, hold the high offices, do the public talking, travel in the trains.

\sphinxAtStartPar
Nowhere in modern lands can one see less of the spender and the consumer, the rich owners and buyers of luxuries, the institutions which cater to the idle rich. One sees in Moscow, Leningrad and Kiev neither first\sphinxhyphen{}class hotels, nor luxurious restaurants, nor private motor cars, nor silk stockings, nor prostitutes. All these insignia of the great modern city are lacking. On the other hand, the traveler misses the courtesy and savoir faire which one meets in the hotel corridors of London and Paris; one misses the smart shops and well\sphinxhyphen{}groomed men and women who are so plentiful in Constantinople and Berlin. Does this mean that Russia has “put over” her new psychology? Not by any means. She is trying and trying hard, but there are plenty of people in Russia who still hate and despise the workingman’s blouses and the peasant’s straw shoes; and plenty of workers who regret the passing of the free\sphinxhyphen{}handed Russian nobility; who miss the splendid pageantry of the Czars and who cling doggedly to religious dogma and superstition. There must be in Russia dishonest officials and inefficient statesmen. But here Russia has no monopoly. There are those in Russia and out who say that the present effort cannot succeed for exactly the same reasons that men said the Bourgeoisie could never rule France.

\sphinxAtStartPar
But it is the organized capital of America, England, France and Germany which is chiefly instrumental in preventing the realization of the Russian workingman’s psychology. It has used every modern weapon to crush Russia. It sent against Russia every scoundrel who could lead a mob and gave him money, guns and ammunition; and when Russia nearly committed suicide in crushing this civil war, modern industry began the industrial boycott, the refusal of capital and credit which is being carried on today just as far as international jealousy and greed will allow. And can we wonder? If modern capital is owned by the rich and handled for their power and benefit, can the rich be expected to hand it over to their avowed and actual enemies? On the contrary, if modern industry is really for the benefit of the people and if there is an effort to make the people the chief beneficiaries of industry, why is it that this same people is powerless today to help this experiment or at least to give it a clear way? On the other hand, so long as the most powerful nations in the world are determined that Russia must fail, there can be but a minimum of free discussion and democratic difference of opinion in Russia.

\sphinxAtStartPar
There is world struggle then in and about Russia; but it is not simply an ethical problem as to whether or not the Russian Revolution was morally right; that is a question which only history will settle. It is not simply the economic question as to whether or not Russia can conduct industry on a national scale. She is doing it today and in so doing she differs only in quantity, not in quality from every other modern country. It is not a question merely of “dictatorship”. We are all subject to this form of government. The real Russian question is: Can you make the worker and not the millionaire the center of modern power and culture? If you can, the Russian Revolution will sweep the world.

\sphinxAtStartPar
\sphinxstyleemphasis{Citation:} Du Bois, W.E.B. 1927. “Judging Russia.”  33(4):189\sphinxhyphen{}190.


\subsection{The Dunbar National Bank (1928)}
\label{\detokenize{Volumes/35/11/dunbar_national_bank:the-dunbar-national-bank-1928}}\label{\detokenize{Volumes/35/11/dunbar_national_bank::doc}}
\sphinxAtStartPar
The establishment of the Dunbar National Bank in New York City, may be simply another bank; but it might prove to be an epoch making event for the darker races of the world. Here is a bank with over a million dollars of capital and surplus, with a colored and white directorate and a colored and white personnel. Even though white business men and capital predominate in numbers and authority, yet the possibilities of such an organization are tremendous.

\sphinxAtStartPar
In the present organization of the world in politics and industry the line between capital and labor coincides roughly to the line between the white and darker races. Whatever salvation the darker races seek under present conditions must be attained through their admission to the ranks of capitalists.

\sphinxAtStartPar
Failing this, they must fight capital and modern industry and their fight must be primarily racial and not based on the intrinsic merits or demerits of capitalistic industry.

\sphinxAtStartPar
Thus two questions face Negroes and Chinese and Indians.

\sphinxAtStartPar
First, is capitalism, as at present organized, the best director of work and income? Second, can the darker peoples secure voice and influence in the governing councils of modern organized capital? These are separate questions, but they tend to be one today, because organized capital today is almost exclusively in white control. The control of capital and credit enables the white people of the world to rule the world for their own benefit and to ignore whenever they so wish, the best interests of the colored peoples.

\sphinxAtStartPar
The leaders of colored thought, therefore, are faced by this problem: is the whole capitalistic system wrong or is the color problem merely the problem of securing for the darker people proper representation in the centers of capitalistic control? This question has been variously answered.

\sphinxAtStartPar
Booker T. Washington in the United States, most of the Negro leaders of West Africa, and some of the leaders of India, have seen salvation in a chance to share the capitalistic control of industry with white Europe and America. Others, including the Editor of \sphinxstylestrong{The Crisis}, believe that industrial reform must be far more radical than this.

\sphinxAtStartPar
But no matter what differences of opinion arise on this point, so long as organized capital excludes Negroes and other darker folk from its counsels and official positions, just so long will this people be forced toward radical industrial reform.

\sphinxAtStartPar
It is useless to reply that the capitalistic system is always open to individual merit. That is not true so far as white boys are concerned, and it is a fiat lie in the case of black boys. No black boy today, no matter what his education or ability, has any chance of admission and promotion in a white bank, insurance company, corporation or manufacturing concern. And no bank organized by black folk has a ghost of a chance to grow to real power in a financial world dominated by white banks and white captains of industry. Indeed, it has become almost axiomatic in England and America to put no real financial or industrial power in the hands of black folk.

\sphinxAtStartPar
Repeated attempts have been made to break over this industrial dead line: on the Gold Coast in West Africa in connection with the cocoa trade; in the establishment of some fifty small Negro banks in the United States; in various co\sphinxhyphen{}operative movements in Asia and Africa ;—none of these movements have had any real and conspicuous success. Each one has found itself eventually in the masterful grasp of the great white capitalistic monopoly.

\sphinxAtStartPar
It is not too much to say that the Dunbar National Bank offers the greatest opportunity of modern days for something different. One cannot think that Mr. John Rockefeller, Jr., and his associates have gone into this enterprise merely for profit. They must have a vision. How wide is that vision? It may, of course, be narrow and conventional: the training of colored bank officials, the extension of credit to promising small colored enterprises. This would be of value. Eventually, it would lead to better banking among Negroes and more adventure in business. But ultimately, it would do little more than to emphasize the division among colored people into rich and poor, exploiter and exploited, landlord and tenant, employer and employee.

\sphinxAtStartPar
Beyond this there should be, and we sincerely trust there is, a wider and broader dream. This dream would be to break up the controlling caste in organized capitalistic industry; to say to the world that the use of capital is one of the greatest of modern inventions and it must no longer be monopolized by white people. We are going to train colored people in its use and proper control, and through these trained men, we are going to see how far it is possible in the United States, in the West Indies, and in Africa, to put colored men in control of capital and credit, and to let them develop it, not simply for the profit of white people, but for the advancement of darker peoples.

\sphinxAtStartPar
And this is no idle dream. A proper use of capital and credit in the cocoa raising regions of the Gold Coast of Africa would do more to emancipate black West Africa and educate and uplift Negroes, than any other single movement. The West Indies, by far the most beautiful part of the new world, are today prostrate and enslaved under the heels of absentee white capitalists and landlords. They could be redeemed if the power of capital and credit was put into the hands of trained colored men who believe or could be led to believe in the possibilities of Negro blood. American imperialism in Haiti and Central America, instead of being carried out by “N{[}****{]}”\sphinxhyphen{}hating Louisianians, could be put in the hands of black Americans who believe in Haitian freedom and independence.

\sphinxAtStartPar
The seemingly insoluble problems of South Africa and of East Africa could be mitigated in the same way. While in the United States the only thing that is going to save organized labor and bring true industrial democracy, is the abolition of the color line in capital and credit.

\sphinxAtStartPar
We shall look forward then with interest to the development of this bank. If it adds simply one more bank in Harlem to the banks of New York, we shall be profoundly disappointed, no matter how large its capital and how great its dividends. If it takes a real step toward an industrial democracy which includes the darker races, we shall hail it as one of the great steps of the 20th Century.


\bigskip\hrule\bigskip


\sphinxAtStartPar
\sphinxstyleemphasis{Citation:} “The Dunbar National Bank.” 1928. Editorial.  35(11):382.


\subsection{The Negro and Communism (1931)}
\label{\detokenize{Volumes/38/09/negro_and_communism:the-negro-and-communism-1931}}\label{\detokenize{Volumes/38/09/negro_and_communism::doc}}
\sphinxAtStartPar
The Scottsboro, Alabama, cases have brought squarely before the American Negro the question of his attitude toward Communism.

\sphinxAtStartPar
The importance of the Russian Revolution can not be gainsaid. It is easily the greatest event in the world since the French Revolution and possibly since the fall of Rome. The experiment is increasingly successful. Russia occupies the center of the world’s attention today and as a state it is recognized by every civilized nation, except the United States, Spain, Portugal and some countries of South America.

\sphinxAtStartPar
The challenge to the capitalistic form of industry and to the governments which this form dominates, is more and more tremendous because of the present depression. If Socialism as a form of government and industry is on trial in Russia, capitalism as a form of industry and government is just as surely on trial throughout the world and is more and more clearly recognizing the fact.


\subsubsection{The American Worker}
\label{\detokenize{Volumes/38/09/negro_and_communism:the-american-worker}}
\sphinxAtStartPar
It has always been felt that the United States was an example of the extraordinary success of capitalistic industry, and that this was proven by the high wage paid labor and the high standard of intelligence and comfort prevalent in this country. Moreover, for many years, democratic political control of our government by the masses of the people made it possible to envisage without violence any kind of reform in government or industry which appealed to the people. Recently, however, the people of the United States have begun to recognize that their political power is curtailed by organized capital in industry and that in this industry, democracy does not prevail; and that until wider democracy does prevail in industry, democracy in government is seriously curtailed and often quite ineffective. Also, because of recurring depressions the high wage is in part illusory.


\subsubsection{The Amerian Negro}
\label{\detokenize{Volumes/38/09/negro_and_communism:the-amerian-negro}}
\sphinxAtStartPar
Moreover, there is in the United States one class of people who more than any other suffer under Present conditions. Because of wholesale disfranchisement and a system of color caste, discriminatory legislation and widespread propaganda, 12,000,000 American Negroes have only a minimum of that curtailed freedom which the right to vote and influence on public opinion gives to white Americans. And in industry Negroes are for historic and social reasons upon the lowest round.


\subsubsection{Proposed Reform}
\label{\detokenize{Volumes/38/09/negro_and_communism:proposed-reform}}
\sphinxAtStartPar
The proposals to remedy the economic and political situation in America range from new legislation, better administration and government aid, offered by the Republican and Democratic parties, on to liberal movements fathered by Progressives, the Farmer\sphinxhyphen{}Labor movement and the Socialists, and finally to the revolutionary proposals of the Communists. The Progressives and Socialists propose in general increased government ownership of land and natural resources, state control of the larger public services and such progressive taxation of incomes and inheritance as shall decrease the number and power of the rich. The Communists, on the other hand, propose an entire sweeping away of the present organization of industry; the ownership of land, resources, machines and tools by the state, the conducting of business by the state under incomes which the state limits. And in order to introduce this complete Socialistic regime, Communists propose a revolutionary dictatorship by the working class, as the only sure, quick and effective path.


\subsubsection{Advice to Negroes}
\label{\detokenize{Volumes/38/09/negro_and_communism:advice-to-negroes}}
\sphinxAtStartPar
With these appeals in his ears, what shall the American Negro do? In the letters from United States Senators published in this issue of \sphinxstylestrong{The Crisis}, we find, with all the sympathy and good\sphinxhyphen{}will expressed, a prevailing helplessness when it comes to advice on specific action. Reactionaries like Fess, Conservatives like Bulkley and Capper, Progressives like Borah and Norris, all can only say: “You have done as well as could be expected; you suffer many present disadvantages; there is nothing that we can do to help you, and your salvation lies in patience and further effort on your own part.” The Socialist, as represented by Norman Thomas in the February \sphinxstylestrong{Crisis}, invites the Negro as a worker to vote for the Socialist Party as the party of workers. He offers the Negro no panacea for prejudice and caste but assumes that the uplift of the white worker will automatically emancipate the yellow, brown and black.


\subsubsection{The Scottsboro Cases}
\label{\detokenize{Volumes/38/09/negro_and_communism:the-scottsboro-cases}}
\sphinxAtStartPar
Finally, the Scottsboro cases come and put new emphasis on the appeal of the Communists. Advocating the defense of the eight Alabama black boys, who without a shadow of doubt have been wrongly accused of crime, the Communists not only asked to take charge of the defense of these victims, but they proceeded to build on this case an appeal to the American Negro to join the Communist movement as the only solution of their problem.

\sphinxAtStartPar
Immediately, these two objects bring two important problems; first, can the Negroes with their present philosophy and leadership defend the Scottsboro cases successfully? Secondly, even if they can, will such defense help them to solve their problem of poverty and caste?

\sphinxAtStartPar
If the Communistic leadership in the United States had been broadminded and far\sphinxhyphen{}sighted, it would have acknowledged frankly that the honesty, earnestness and intelligence of the N.A.A.C.P. during twenty years of desperate struggle proved this organization under present circumstances to be the only one, and its methods the only methods available, to defend these boys and it would have joined capitalists and laborers north and south, black and white in every endeavor to win freedom for victims threatened with judicial murder. Then beyond that and with Scottsboro as a crimson and terrible text, Communists could have proceeded to point out that legal defense alone, even if successful, will never solve the larger Negro problem but that further and more radical steps are needed.


\subsubsection{Communist Strategy}
\label{\detokenize{Volumes/38/09/negro_and_communism:communist-strategy}}
\sphinxAtStartPar
Unfortunately, American Communists are neither wise nor intelligent. They sought to accomplish too much at one stroke. They tried to prove at once that the N.A.A.C.P. did not wish to defend the victims at Scottsboro and that the reason for this was that Negro leadership in the N.A.A.C.P. was allied with the capitalists. The first of these two efforts was silly and the Communists tried to accomplish it by deliberate lying and deception. They accused the N.A.A.C.P. of stealing, misuse of funds, lack of interest in the Scottsboro cases, cowardly surrender to malign forces, inefficiency and a policy of do\sphinxhyphen{}nothing.

\sphinxAtStartPar
Now whatever the N.A.A.C.P. has lacked, it is neither dishonest nor cowardly, and already events are proving clearly that the only effective defense of the Scottsboro boys must follow that which has been carefully organized, engineered and paid for by the N.A.A.C.P., and that the success of this defense is helped so far as the Communists cooperate by hiring bourgeois lawyers and appealing to bourgeois judges; but is hindered and made doubtful by ill\sphinxhyphen{}considered and foolish tactics against the powers in whose hands the fate of the Scottsboro victims lies.

\sphinxAtStartPar
If the Communists want these lads murdered, then their tactics of threatening judges and yelling for mass action on the part of white southern workers is calculated to insure this.

\sphinxAtStartPar
And, on the other hand, lying and deliberate misrepresentation of friends who are fighting for the same ideals as the Communists, are old capitalistic, bourgeois weapons of which the Communists ought to be ashamed. The final exploit at Camp Hill is worthy of the Russian Black Hundreds, whoever promoted it: black sharecroppers, half\sphinxhyphen{}starved and desperate were organized into a “Society for the Advancement of Colored People” and then induced to meet and protest against Scottsboro. Sheriff and white mob killed one and imprisoned 34. If this was instigated by Communists, it is too despicable for words; not because the plight of the black peons does not shriek for remedy but because this is no time to bedevil a delicate situation by drawing a red herring across the trail of eight innocent children.

\sphinxAtStartPar
Nevertheless, the N.A.A.C.P. will defend these 34 victims of Southern fear and communist irresponsibility.

\sphinxAtStartPar
The ultimate object of the Communists, was naturally not merely nor chiefly to save the boys accused at Scottsboro; it was to make this case a center of agitation to expose the helpless condition of Negroes, and to prove that anything less than the radical Communist program could not emancipate them.


\subsubsection{The Negro Bourgeoisie}
\label{\detokenize{Volumes/38/09/negro_and_communism:the-negro-bourgeoisie}}
\sphinxAtStartPar
The question of the honesty and efficiency of the N.A.A.C.P. in the defense of the Scottsboro boys, just as in a dozen other cases over the length and breadth of the United States, is entirely separate from the question as to whether or not Negro leadership is tending toward socialism and communism or toward capitalism.
The charge of the Communists that the present set\sphinxhyphen{}up of Negro America is that of the petit bourgeois minority dominating a helpless black proletariat, and surrendering to white profiteers is simply a fantastic falsehood. The attempt to dominate Negro Americans by purely capitalistic ideas died with Booker T. Washington. The battle against it was begun by the Niagara Movement and out of the Niagara Movement arose the N.A.A.C.P. Since that time there has never been a moment when the dominating leadership of the American Negro has been mainly or even largely dominated by wealth or capital or by capitalistic ideals.

\sphinxAtStartPar
There are naturally some Negro capitalists: some large landowners, some landlords, some industrial leaders and some investors; but the great mass of Negro capital is not owned or controlled by this group. Negro capital consists mainly of small individual savings invested in homes, and in insurance, in lands for direct cultivation and individually used tools and machines. Even the automobiles owned by Negroes represent to a considerable extent personal investments, designed to counteract the insult of the “Jim Crow” car. The Insurance business, which represents a large amount of Negro capital is for mutual co\sphinxhyphen{}operation rather than exploitation. Its profit is limited and its methods directed by the State. Much of the retail business is done in small stores with small stocks of goods, where the owner works side by side with one or two helpers, and makes a personal profit less than a normal American wage. Negro professional men—lawyers, physicians, nurses and teachers—represent capital invested in their education and in their office equipment, and not in commercial exploitation. There are few colored manufacturers of material who speculate on the products of hired labor. Nine\sphinxhyphen{}tenths of the hired Negro labor is under the control of white capitalists. There is probably no group of 12 million persons in the modern world which exhibits smaller contrasts in personal income than the American Negro group. Their emancipation will not come, as among the Jews, from an internal readjustment and ousting of exploiters; rather it will come from a wholesale emancipation from the grip of the white exploiters without.

\sphinxAtStartPar
It is, of course, always possible, with the ideals of America, that a full fledged capitalistic system may develop in the Negro group; but the dominant leadership of the Negro today, and particularly the leadership represented by the N.A.A.C.P. represents NO such tendency. For two generations the social leaders of the American Negro with very few exceptions have been poor men, depending for support on their salaries, owning little or no real property; few have been business men none have been exploiters, and while there have been wide differences of ultimate ideal these leaders on the whole have worked unselfishly for the uplift of the masses of Negro folk.

\sphinxAtStartPar
There is no group of leaders on earth who have so largely made common cause with the lowest of their race as educated American Negroes, and it is their foresight and sacrifice and theirs alone that has saved the American freedman from annihilation and degradation.

\sphinxAtStartPar
This is the class of leaders who have directed and organized and defended black folk in America and whatever their shortcomings and mistakes—and they are legion—their one great proof of success is the survival of the American Negro as the most intelligent and effective group of colored people fighting white civilization face to face and on its own ground, on the face of the earth.

\sphinxAtStartPar
The quintessence and final expression of this leadership is the N.A.A.C.P. For twenty years it has fought a battle more desperate than any other race conflict of modern times and it has fought with honesty and courage. It deserves from Russia something better than a kick in the back from the young jackasses who are leading Communism in America today.


\subsubsection{What is the N.A.A.CP.?}
\label{\detokenize{Volumes/38/09/negro_and_communism:what-is-the-n-a-a-cp}}
\sphinxAtStartPar
The N.A.A.C.P. years ago laid down a clear and distinct program. Its object was to make 12 million Americans:

\sphinxAtStartPar
\sphinxstyleemphasis{Physically free from peonage,Mentally free from ignorance,Politically free from disfranchisement,Socially free from insult.}

\sphinxAtStartPar
Limited as this platform may seem to perfectionists, it is so far in advance of anything ever attempted before in America, that it has gained an extraordinary following. On this platform we have succeeded in uniting white and black, employers and laborers, capitalists and communists, socialists and reformers, rich and poor. The funds which support this work come mainly from poor colored people, but on the other hand, we have in 20 years of struggle, enlisted the sympathy and cooperation of the rich, the white and the powerful; and so long as this cooperation is given upon the basis of the platform we have laid down, we seek and welcome it. On the other hand, we know perfectly well that the platform of the N.A.A.C.P. is no complete program of social reform. It is a pragmatic union of certain definite problems, while far beyond its program lies the whole question of the future of the darker races and the economic emancipation of the working classes.


\subsubsection{White Labor}
\label{\detokenize{Volumes/38/09/negro_and_communism:white-labor}}
\sphinxAtStartPar
Beyond the Scottsboro cases and the slurs on Negro leadership, there still remains for Negroes and Communists, the pressing major question: How shall American Negroes be emancipated from economic slavery? In answer to this both Socialists and Communists attempt to show the Negro that his interest lies with that of white labor. That kind of talk to the American Negro is like a red rag to a bull. Throughout the history of the Negro in America, white labor has been the black man’s enemy, his oppressor, his red murderer. Mobs, riots and the discrimination of trade unions have been used to kill, harass and starve black men, White labor disfranchised Negro labor in the South, is keeping them out of jobs and decent living quarters in the North, and is curtailing their education and civil and social privileges throughout the nation. White laborers have formed the backbone of the Ku Klux Klan and have furnished hands and ropes to lynch 3,560 Negroes since 1882.

\sphinxAtStartPar
Since the death of Terence Powderly not a single great white labor leader in the United States has wholeheartedly and honestly espoused the cause of justice to black workers.

\sphinxAtStartPar
Socialists and Communists explain this easily: white labor in its ignorance and poverty has been misled by the propaganda of white capital, whose policy is to divide labor, into classes, races and unions and pit one against the other. There is an immense amount of truth in this explanation: Newspapers, social standards, race pride, competition for jobs, all work to set white against black. But white American laborers are not fools. And with few exceptions the more intelligent they are, the higher they rise, the more efficient they become, the more determined they are to keep Negroes under their heels. It is no mere coincidence that Labor’s present representative in the President’s cabinet belongs to a union that will not admit a Negro, and himself was for years active in West Virginia in driving Negroes out of decent jobs. It is intelligent white labor that today keeps Negroes out of the trades, refuses them decent homes to live in and helps nullify their vote. Whatever ideals white labor today stives for in America, it would surrender nearly every one before it would recognize a Negro as a man.


\subsubsection{Communists and the Color Line}
\label{\detokenize{Volumes/38/09/negro_and_communism:communists-and-the-color-line}}
\sphinxAtStartPar
American Communists have made a courageous fight against the color line among the workers. They have solicited and admitted Negro members. They have insisted in their strikes and agitation to let Negroes fight with them and that the object of their fighting is for black workers as well as white workers. But in this they have gone dead against the thought and desire of the overwhelming mass of white workers, and face today a dead blank wall even in their own school in Arkansas. Thereupon instead of acknowledging defeat in their effort to make white labor abolish the color line, they turn and accuse Negroes of not sympathizing with the ideals of Labor!

\sphinxAtStartPar
Socialists have been franker. They learned that American labor would not carry the Negro and they very calmly unloaded him. They allude to him vaguely and as an afterthought in their books and platforms. The American Socialist party is out to emancipate the white worker and if this does not automatically free the colored man, he can continue in slavery. The only time that so fine a man and so logical a reasoner as Norman Thomas becomes vague and incoherent is when he touches the black man, and consequently he touches him as seldom as possible.

\sphinxAtStartPar
When, therefore, Negro leaders refuse to lay down arms and surrender their brains and action to “Nigger’hating white workers, liberals and socialists understand exactly the reasons for this and spend what energy they can spare in pointing out to white workers the necessity of recognizing Negroes. Sut the Communists, younger and newer, largely of foreign extraction, and thus discounting the hell of American prejudice, easily are led to blame the Negroes and to try to explain the intolerable American situation on the basis of an imported Marxist pattern, which does not at all fit the situation.

\sphinxAtStartPar
For instance, from Moscow comes this statement to explain Scottsboro and Camp Hill:
\begin{quote}

\sphinxAtStartPar
“Again, as in the case of Sacco and Vanzetti, the American Bourgeoisie is attempting to go against proletarian social opinion. It is attempting to carry through its criminal provocation to the very end.”
\end{quote}

\sphinxAtStartPar
This is a ludicrous misapprehension of local conditions and illustrates the error into which long distance interpretation, unsupported by real knowledge, may fall. The Sacco\sphinxhyphen{}Vanzetti cases in Massachusetts represented the fight of prejudiced, entrenched capital against radical propaganda; but in Jackson County, northeastern Alabama, where Scottsboro is situated, there are over 33,000 Native whites and less than 3,000 Negroes. The vast majority of these whites belong to the laboring class and they formed the white proletarian mob which is determined to kill the eight Negro boys. Such mobs of white workers demand the right to kill “n{[}******{]}” whenever their passions, especially in sexual matters, are inflamed by propaganda. The capitalists are willing to curb this blood lust when it interferes with their profits. They know that the murder of 8 innocent black boys will hurt organized industry and government in Alabama; but as long as 10,000 armed white workers demand these victims they do not dare move. Into this delicate and contradictory situation, the Communists hurl themselves and pretend to speak for the workers. They not only do not speak for the white workers but they even intensify the blind prejudices of these lynchers and leave the Negro workers helpless on the one hand and the white capitalists scared to death on the other.

\sphinxAtStartPar
The persons who are killing blacks in Northern Alabama and demanding blood sacrifice are the white workers— sharecroppers, trade unionists and artisans. The capitalists are against mob\sphinxhyphen{}law and violence and would listen to reason and justice in the long run because industrial peace increases their profits. On the other hand, the white workers want to kill the competition of “N{[}******{]}.” Thereupon, the Communists, seizing leadership of the poorest and most ignorant blacks head them toward inevitable slaughter and jail\sphinxhyphen{}slavery, while they hide safely in Chattanooga and Harlem.

\sphinxAtStartPar
American Negroes do not propose to be the shock troops of the Communist Revolution, driven out in front to death, cruelty and humiliation in order to win victories for white workers. They are picking no chestnuts from the fire, neither for capital nor white labor.

\sphinxAtStartPar
Negroes know perfectly well that whenever they try to lead revolution in America, the nation will unite as one fist to crush them and them alone. There is no conceivable idea that seems to the present overwhelming majority of Americans higher than keeping Negroes “in their place.”

\sphinxAtStartPar
Negroes perceive clearly that the real interests of the white worker are identical with the interests of the black worker; but until the white worker recognizes this, the black worker is compelled in sheer self\sphinxhyphen{}defense to refuse to be made the sacrificial goat.


\subsubsection{The Negro and the Rich}
\label{\detokenize{Volumes/38/09/negro_and_communism:the-negro-and-the-rich}}
\sphinxAtStartPar
The remaining grain of truth in the Communist attack on Negro leadership is the well\sphinxhyphen{}known fact that American wealth has helped the American Negro and that without this help the Negro could not have attained his present advancement. American courts from the Supreme Court down are dominated by wealth and Big Business, yet they are today the Negro’s only protection against complete disfranchisement, segregation and the abolition of his public schools. Higher education for Negroes is the gift of the Standard Oil, the Power Trust, the Steel Trust and the Mail Order Chain Stores, together with the aristocratic Christian Church; but these have given Negroes 40,000 black leaders to fight white folk on their own level and in their own language. Big industry in the last 10 years has opened occupations for a million Negro workers, without which we would have starved in jails and gutters.

\sphinxAtStartPar
Socialists and Communists may sneer and say that the capitalists sought in all this profit, cheap labor, strike\sphinxhyphen{}breakers and the training of conservative, reactionary leaders. They did. 3ut Negroes sought food, clothes, shelter and knowledge to stave off death and slavery and only damned fools would have refused the gift.

\sphinxAtStartPar
Moreover, we who receive education as the dole of the rich have not all become slaves of wealth.

\sphinxAtStartPar
Meanwhile, what have white workers and radical reformers done for Negroes? By strikes and agitation, by self\sphinxhyphen{}denial and sacrifice, they have raised wages and bettered working conditions; but they did this for themselves and only shared their gains with Negroes when they had to. They have preached freedom, political power, manhood rights and social uplift for everybody, when nobody objected; but for “white people only” when anybody demanded it. White labor segregated Dr. Sweet in Detroit; white laborers chased the Arkansas peons; white laborers steal the black children’s school funds in South Carolina, white laborers lynch Negroes in Alabama. Negroes owe much to white labor but it is not all, or mostly, on the credit side of the ledger.


\subsubsection{The Next Step}
\label{\detokenize{Volumes/38/09/negro_and_communism:the-next-step}}
\sphinxAtStartPar
Where does this leave the Negro?

\sphinxAtStartPar
As a practical program, it leaves him just where he was before the Russian Revolution; sympathetic with Russia and hopeful for its ultimate success in establishing a Socialistic state; sympathetic with the efforts of the American workingman to establish democratic control of industry in this land; absolutely certain that as a laborer his interests are the interests of all labor; but nevertheless fighting doggedly on the old battleground, led by the N.A.A.C.P. to make the Negro laborer a laborer on equal social footing with the white laborer: to maintain the Negro’s right to a political vote, notwithstanding the fact that this vote means increasingly less and less to all voters; to vindicate in the courts the Negro’s civil rights and American citizenship, even though he knows how the courts are prostituted to the power of wealth; and above all, determined by plain talk and agitation to show the intolerable injustice with which America and the world treats the colored peoples and to continue to insist that in this injustice, the white workers of Europe and America are just as culpable as the white owners of capital; and that these workers can gain black men as allies only and insofar as they frankly, fairly and completely abolish the Color Line.

\sphinxAtStartPar
Present organization of industry for private profit and control of government by concentrated wealth is doomed to disaster. It must change and fall if civilization survives. The foundation of its present world\sphinxhyphen{}wide power is the slavery and semislavery of the colored world including the American Negroes. Until the colored man, yellow, red, brown and black, becomes free, articulate, intelligent and the receiver of a decent income, white capital will use the profit derived from his degradation to keep white labor in chains.

\sphinxAtStartPar
There is no doubt, then, as to the future, or as to where the true interests of American Negroes lie. There is no doubt, too, but that the first step toward the emancipation of colored labor must come from white labor.


\bigskip\hrule\bigskip


\sphinxAtStartPar
\sphinxstyleemphasis{Citation:} Du Bois, W.E.B. 1931. “The Negro and Communism.”  38(9):313\sphinxhyphen{}315, 318, 320.


\subsection{Our Class Struggle}
\label{\detokenize{Volumes/40/07/our_class_struggle:our-class-struggle}}\label{\detokenize{Volumes/40/07/our_class_struggle::doc}}
\sphinxAtStartPar
In the Marxist patois, the “class struggle” means the natural antagonism and war between the exploiter and the exploited; that is, between those persons who own capital in the form of machines, raw material and money, and who can command credit, and that other large mass of people who have practically nothing to sell but their labor. Between these two classes, there can be no peace because the profit of the capitalist depends on the amount of surplus value he can extract from the work of the laborer.

\sphinxAtStartPar
One no sooner states this than the expert would say immediately that there is no trace of such class struggle among American Negroes. On second thought, however, he might modify this and say that the occupational differences of American Negroes show at least the beginnings of differentiation into capitalists and laborers.

\sphinxAtStartPar
Of Negroes, 10 years of age and over in gainful occupations, there are:


\begin{savenotes}\sphinxattablestart
\centering
\begin{tabulary}{\linewidth}[t]{|T|T|}
\hline
\sphinxstyletheadfamily 
\sphinxAtStartPar

&\sphinxstyletheadfamily 
\sphinxAtStartPar

\\
\hline
\sphinxAtStartPar
Skilled laborers
&
\sphinxAtStartPar
331,839
\\
\hline
\sphinxAtStartPar
Semi\sphinxhyphen{}Skilled laborers
&
\sphinxAtStartPar
734,951
\\
\hline
\sphinxAtStartPar
Farmers
&
\sphinxAtStartPar
873,653
\\
\hline
\sphinxAtStartPar
Common laborers
&
\sphinxAtStartPar
3,374,545
\\
\hline
\sphinxAtStartPar
Trade and Business
&
\sphinxAtStartPar
52,957
\\
\hline
\sphinxAtStartPar
Professional
&
\sphinxAtStartPar
119,827
\\
\hline
\sphinxAtStartPar
Civil Service
&
\sphinxAtStartPar
15,763
\\
\hline
\sphinxAtStartPar
Total
&
\sphinxAtStartPar
5,503,535
\\
\hline
\end{tabulary}
\par
\sphinxattableend\end{savenotes}

\sphinxAtStartPar
Of the farmers, 181,016 were owners. The others were tenants. We may, therefore, say that the capitalistic class among Negroes would be among the following :


\begin{savenotes}\sphinxattablestart
\centering
\begin{tabulary}{\linewidth}[t]{|T|T|}
\hline
\sphinxstyletheadfamily 
\sphinxAtStartPar

&\sphinxstyletheadfamily 
\sphinxAtStartPar

\\
\hline
\sphinxAtStartPar
Trade and Business
&
\sphinxAtStartPar
52,957
\\
\hline
\sphinxAtStartPar
Professional
&
\sphinxAtStartPar
119,827
\\
\hline
\sphinxAtStartPar
Farm Owners
&
\sphinxAtStartPar
181,016
\\
\hline
\sphinxAtStartPar
Civil Service
&
\sphinxAtStartPar
15,763
\\
\hline
\sphinxAtStartPar
Total
&
\sphinxAtStartPar
369,563
\\
\hline
\end{tabulary}
\par
\sphinxattableend\end{savenotes}

\sphinxAtStartPar
Most of these however depend for their income on labor rather than capital. Those in trade and business, include clerks, as well as about 30,000 investors of capital. And the professional men are not capitalists, except as some of them have saved money. The same thing can be said of the civil servants. The farm owners are by vast majority peasant proprietors, most of whom hire a little or no labor outside the family.

\sphinxAtStartPar
The most that can be said is that many of the people in this group have the American ambition to become rich and “independent;” to live on income rather than labor, and thus their ideology ranks them on the side of the white capitalists. On the other hand, the laborers, skilled and semi\sphinxhyphen{}skilled, and the tenants, are all a proletariat, exploited by white capital. One has, therefore, a rather curious arrangement, with the real class struggle not between colored classes, but rather between colored and white folk.

\sphinxAtStartPar
There is, however, an inner division that calls for attention because it emphasizes and foreshadows class distinctions within the race. And that is the existence of delinquency and dependency, of criminals and paupers. What is the extent of this class among American Negroes and what is the relation of the class to the workers and the more prosperous elements who have begun to accumulate property?

\sphinxAtStartPar
During the earlier history of the colored race, there was a natural social and class difference that came through the existence of mulattoes. In the West Indies, by French law, these mulattoes were free and often inherited wealth from their fathers. In many cases, they were carefully educated and formed a distinct social class, whose rank depended upon wealth, education and personal freedom. In the later history of the French colonies, and even more in Spanish and American colonies, persistent effort reduced this class to a semi\sphinxhyphen{}servile position, and it was the resentment against this that led the mulattoes to unite with the blacks in the Haitian Revolution and overthrow the whites.

\sphinxAtStartPar
In the United States, this color caste was dealt a death blow by the law that made children follow the condition of the mother, so that white fathers sold their colored children into slavery, and the mulatto ceased to be, in most cases, a free man. He inherited no property from his father and lost his right to education; although so far as he was free, he promoted schools, in centers like Washington, Charleston and New Orleans.

\sphinxAtStartPar
The color caste idea persisted after Emancipation, but was gradually driven out by the new economic organization. In this new economy there arose the criminals and paupers;—the direct result of the poverty of a suddenly emancipated class who had little or no capital.

\sphinxAtStartPar
The apparent criminality, however, of the Negro race is greatly exaggerated for two reasons: First, accusation of crime was used systematically in the South to keep Negroes in serfdom after the Civil War; and secondly, Negroes receive but scant justice in the courts. Most writers, today, have assumed on the basis of statistics, that because the Negro population in jails and penitentiary is proportionately much larger than the white population, that, therefore, the Negro is unusually criminal. But as Thorsten Sellin has pointed out in his note on the Negro criminal, “The American Negro lacks education and earthly goods. He has had very little political experience and industrial training. His contact with city life has been unfortunate, for it has forced him into the most dilapidated and vicious areas of our great cities. Like a shadow over his whole existence lies the oppressive race prejudice of his white neighbor, restricting his activities and thwarting his ambitions. It would be extraordinary, indeed, if this group were to prove more law\sphinxhyphen{}abiding than the white, which enjoys more fully the advantages of a civilization the Negro has helped to create.”*

\sphinxAtStartPar
On the other hand, the peculiar result of the assumed fact that Negroes are criminal is that within the race, a Negro accused or convicted of a crime immediately suffers a penalty, not only of ostracism, but lack of sympathy. Negroes make comparatively little effort to defend the accused; they do not systematically look after them; the churches take little interest in delinquents, and the general attitude of the race is one of irritation toward these members of their groups who have brought the whole race into disrepute. This makes a peculiarly bitter feeling among the unfortunate of the race and the more successful.

\sphinxAtStartPar
So far as dependents are concerned, again the material which we have to measure the amount of dependents is inconclusive and unsatisfactory. There are such differences in policy in various states, such difference in treatment that it is hard to say what the condition is. It would seem, by a study of states where there is a substantial uniform  policy toward the feeble\sphinxhyphen{}minded and paupers, regardless of race, that there is a higher rate of institutionization among Negroes than among whites. And this would be natural unless corrected by taking into account the unequal economic and social condition.

\sphinxAtStartPar
Here again within the race, there is a certain resentment against a colored person who fails to progress as rapidly as the Negro thinks a black man must. When, therefore, such a person becomes a subject of charity and must be put into an institution, he is regarded not so much as unfortunate as in some vague way blame worthy. He hinders the general advance and even if he is not at fault, his existence is a misfortune. The real question, then, in the Negro race, is how far the group can and should assume responsibility for its delinquents and dependents, and cultivate sympathy and help for these unfortunates, and how far in this way differentiation into class can keep economic exploitation from becoming a settled method of social advance.


\bigskip\hrule\bigskip


\sphinxAtStartPar
\sphinxstyleemphasis{Citation:} Du Bois, W.E.B. 1933. “Our Class Struggle.” \sphinxstyleemphasis{The Crisis} 40(7):164\sphinxhyphen{}165.


\chapter{Woman Suffrage}
\label{\detokenize{Sections/womansuffrage:woman-suffrage}}\label{\detokenize{Sections/womansuffrage::doc}}
\sphinxAtStartPar
Editorials on woman suffrage.
\begin{itemize}
\item {} 
\sphinxAtStartPar
{\hyperref[\detokenize{Volumes/04/04/ohio::doc}]{\sphinxcrossref{Ohio (1912)}}}

\item {} 
\sphinxAtStartPar
{\hyperref[\detokenize{Volumes/09/06/womansuffrage::doc}]{\sphinxcrossref{Woman Suffrage (1915)}}}

\item {} 
\sphinxAtStartPar
{\hyperref[\detokenize{Volumes/09/03/agility::doc}]{\sphinxcrossref{Agility (1915)}}}

\item {} 
\sphinxAtStartPar
{\hyperref[\detokenize{Volumes/15/01/votes_for_women::doc}]{\sphinxcrossref{Votes for Women (1917)}}}

\end{itemize}


\section{Ohio (1912)}
\label{\detokenize{Volumes/04/04/ohio:ohio-1912}}\label{\detokenize{Volumes/04/04/ohio::doc}}
\begin{sphinxShadowBox}
\sphinxstylesidebartitle{}

\sphinxAtStartPar
The Ohio suffrage amendment \sphinxhref{https://ballotpedia.org/Ohio\_Amendment\_23,\_Women\%27s\_Suffrage\_Measure\_(September\_1912)}{failed in 1912 by 15\%}. Male Wisconsin voters also turned down a suffrage amendment that year, although Arizona and Oregon both voted in favor.
\end{sphinxShadowBox}

\sphinxAtStartPar
This fall the colored voters of Ohio have a wonderful opportunity; the 40,000 or 50,000 votes which they cast will undoubtedly decide whether women shall vote in that State and whether the last of the infamous black laws shall be swept from the statute book.

\sphinxAtStartPar
The enfranchisement of women means the doubling of the black vote at the point where that vote is needed. If woman suffrage wins in Ohio, it will sweep the Middle West and East in less than a generation. As Negroes have a larger proportion of women than the whites our relative voting importance in the North will be increased.

\sphinxAtStartPar
Moreover, we need above all classes the women’s influence in politics—the influence of the mother, the wife, the teacher and the washerwoman. In the African fatherland the women stood high in counsel. We need them here again. It would be very bad, indeed, if the colored vote should be adverse to enfranchising women, even though it were not the deciding factor, for the day has gone by forever when colored men could get a respectful hearing for their protest against their own disfranchisement if, when offered the opportunity for voting for enfranchising their mothers, wives and sisters, they should fail to do so. For still another reason it will be unfortunate if the Ohio Negroes vote against votes for women; the vote will be analyzed with keen and eager intelligence, and the results studied for future use. The colored voters will turn many possible friends into critics, to put it mildly, if they inflict upon women that disfranchisement which all thinking people deplore when applied to the Negroes themselves. The general proposition that women ought to have the right to vote surely needs no argument among disfranchised colored folk:

\sphinxAtStartPar
Women are workers; workers should vote.

\sphinxAtStartPar
Women are taxpayers; taxpayers should vote.

\sphinxAtStartPar
Women have brains; voting needs brains.

\sphinxAtStartPar
Women organize, direct and largely support the family; families should vote.

\sphinxAtStartPar
Women are mothers of men; if men vote, why not women?

\sphinxAtStartPar
If politics are too nasty and rough for women voters, is it not time we asked the vote of women to cleanse them?

\sphinxAtStartPar
Is there a single argument for the right of men to vote, or for the right of black men to vote, that does not apply to the votes for women, and particularly for black women?


\bigskip\hrule\bigskip


\sphinxAtStartPar
\sphinxstyleemphasis{Citation:} Du Bois, W.E.B. 1912. “Ohio.”  \sphinxstyleemphasis{The Crisis}. 4(4):181\sphinxhyphen{}182.


\section{Woman Suffrage (1915)}
\label{\detokenize{Volumes/09/06/womansuffrage:woman-suffrage-1915}}\label{\detokenize{Volumes/09/06/womansuffrage::doc}}
\begin{sphinxShadowBox}
\sphinxstylesidebartitle{}

\sphinxAtStartPar
In the November 1915 elections after this editorial was published, male voters in each of the four states turned down woman suffrage.
\end{sphinxShadowBox}

\sphinxAtStartPar
In the fall of 1915 the colored voters of Massachusetts, New York, New Jersey and Pennsylvania are going to be asked for their opinion on the enfranchisement of women. There were in 1910, 151,341 Negro voters in these states. The number is probably near 200,000 today. It is safe to say that in an electorate of over 5,000,000 these 200,000 votes may easily hold a balance of power and certainly would be a valuable asset. Undoubtedly among Negro voters there is a good deal of indifference and lack of knowledge concerning woman suffrage. We tend to oppose the principle because we do not like the reactionary attitude of most white women toward our problems. We must remember, however, that we are facing a great question of right in which personal hatreds have no place. Every argument for Negro suffrage is an argument for woman’s suffrage; every argument for woman suffrage is an argument for Negro suffrage; both are great movements in democracy. There should be on the part of Negroes absolutely no hesitation whenever and wherever responsible human beings are without voice in their government. The man of Negro blood who hesitates to do them justice is false to his race, his ideals and his country.

\sphinxAtStartPar
\sphinxstyleemphasis{Citation:} “Woman Suffrage.” Editorial. 1915. \sphinxstyleemphasis{The Crisis}. 9(6): 57.


\section{Agility (1915)}
\label{\detokenize{Volumes/09/03/agility:agility-1915}}\label{\detokenize{Volumes/09/03/agility::doc}}
\begin{sphinxShadowBox}
\sphinxstylesidebartitle{}

\sphinxAtStartPar
\sphinxhref{https://en.wikipedia.org/wiki/Alva\_Belmont}{Alva Belmont}, was a prominent white New York suffragist and socialite.
\end{sphinxShadowBox}

\sphinxAtStartPar
Mrs. O.H.P. Belmont is coming in for considerable praise on account of her facile answer to a disturbing question while she was campaigning for suffrage in the South. At Chattanooga she was asked if her movement meant the giving of votes to colored women. Mrs. Belmont was most adroit. The expectant hush fell on the audience and instead of standing up like a frank woman and saying “Yes,” Mrs. Belmont quibbled and twisted after the most approved southern fashion. “We want,” she said, “the same voting privileges for colored women as are given colored men.” And there the adroitness stands naked and unashamed.

\sphinxAtStartPar
It will undoubtedly attract the support of those southerners who want aristocratic white women to vote and to vote their narrow\sphinxhyphen{}headed prejudices into a new southern oligarchy. But there are people whom such dishonesty will not attract. It will not, for instance, attract the tens of thousands of black voters who are going to cast their ballots in certain states this fall where the suffrage question will come up. They will not be satisfied in having their black sisters of the South disfranchised like their black brothers and they will hold in frank and logical suspicion a party that is working for that kind of democracy.

\sphinxAtStartPar
Moreover, there are thousands of white people in this country whom this kind of quibbling disgusts. Everybody knows what desperate effort has been made by certain elements among the suffragists to dodge the Negro problem, to try and work for democracy for white people while being dumb before slavery for blacks. This element has been squelched several times in the counsels of the party but it continually bobs up. Let the suffrage movement beware! In the turnings of time Mrs. Belmont may not be as adroit as she at present con­ceives herself.


\bigskip\hrule\bigskip


\sphinxAtStartPar
\sphinxstyleemphasis{Citation:} “Agility. 1915. \sphinxstyleemphasis{The Crisis}. 9(3): 133.


\section{Votes for Women (1917)}
\label{\detokenize{Volumes/15/01/votes_for_women:votes-for-women-1917}}\label{\detokenize{Volumes/15/01/votes_for_women::doc}}
\sphinxAtStartPar
Some 75,000 Negro voters in the State of New York will be asked to decide this month as to whether or not they are willing that women should have the vote in this State. It is an unpleas­ant but well\sphinxhyphen{}known fact that hither­ to American Negro voters have, in the majority of cases, not been favor able to woman suffrage. This attitude has been taken for two main reasons: \sphinxstyleemphasis{First}, the Negro, still imbued by the ideals of a past genera­tion, does not realize the new status of women in industrial and social life. Despite the fact that within his own­ group women are achieving economic independence even faster than whites, he thinks of these as excep­tional and abnormal and looks for­ ward to the time when his wages will be large enough to support his wife and daughters in comparative idleness at home.

\sphinxAtStartPar
\sphinxstyleemphasis{Secondly}, the American Negro is particularly bitter at the attitude of many white women: at the naive assumption that the height of his am­bition is to marry them, at their arti­ficially\sphinxhyphen{}inspired fear of every dark face, which leads to frightful accusa­tions and suspicions, and at their sometimes insulting behavior toward him in public places.

\sphinxAtStartPar
Notwithstanding the undoubted weight of these two reasons, the American Negro must remember, \sphinxstyleemphasis{First}, that the day when women can be considered as the mere appendages of men, dependent upon their bounty and educated chiefly for their pleasure, has gone by; that as an in­telligent, self\sphinxhyphen{}supporting human be­ing a woman has just as good a right to a voice in her own government as has any man; and that the denial of this right is as unjust as is the denial of the right to vote to American Negroes.

\sphinxAtStartPar
Secondly, two wrongs never made a right. We cannot punish the insolence of certain classes of American white women or correct their ridicu­lous fears by denying them their un­ doubted rights.

\sphinxAtStartPar
It goes without saying that the women’s vote, particularly in the South, will be cast almost unanimously, at first, for every reactionary Negro\sphinxhyphen{}hating piece of legislation that is proposed; that the Bourbons and the demagogs, who are today sitting in the Natural Legislature by the reason of stolen votes, will have additional backing for some years from the votes of white women.

\sphinxAtStartPar
But against this consideration it must be remembered that these same women are going to learn political justice a great deal more quickly than did their men and that despite their prejudices their very emergence into the real, hard facts of life and out of the silly fairy\sphinxhyphen{}land to which their Southern male masters beguile them is going to teach them sense in time.

\sphinxAtStartPar
Moreover, it is going to be more difficult to disfranchise colored women in the South than it was to disfranchise colored men. Even southern “gentlemen,” as used as they are to the mistreatment of colored wom­en, cannot in the blaze of present publicity physically beat them away from the polls. Their economic power over them will be smaller than their power over the men and while you can still bribe some pauperized Negro laborers with a few dollars at election time, you cannot bribe Negro women.

\begin{sphinxShadowBox}
\sphinxstylesidebartitle{}

\sphinxAtStartPar
In 1917, New Yorkers \sphinxhref{https://ballotpedia.org/New\_York\_Amendment\_1,\_Women\%27s\_Suffrage\_Measure\_(1917)}{voted} 54\%\sphinxhyphen{}46\% in favor of amending their constitution to provide women with the right to vote. They had turned it down by \sphinxhref{https://ballotpedia.org/New\_York\_Amendment\_2,\_Women\%27s\_Suffrage\_Measure\_(1915)}{15\%} two years earlier.
\end{sphinxShadowBox}

\sphinxAtStartPar
It is, therefore, of the utmost importance that every single black voter in the State of New York should this month cast his ballot in favor of woman suffrage and that every black voter in the United States should do the same thing whenever and as often as he has opportunity.

\sphinxAtStartPar
It is only in such broad\sphinxhyphen{}minded willingness to do justice to all, that the black man can prove his right not only to share, but to help direct modern culture.


\bigskip\hrule\bigskip


\sphinxAtStartPar
\sphinxstyleemphasis{Citation:} “Votes for Women. 1917. \sphinxstyleemphasis{The Crisis}. 15(1): 8.


\chapter{Race Riots}
\label{\detokenize{Sections/race_riots:race-riots}}\label{\detokenize{Sections/race_riots::doc}}
\sphinxAtStartPar
Editorials on race riots.
\begin{itemize}
\item {} 
\sphinxAtStartPar
{\hyperref[\detokenize{Volumes/18/06/shillady_and_texas::doc}]{\sphinxcrossref{Shillady and Texas (1919)}}}

\item {} 
\sphinxAtStartPar
{\hyperref[\detokenize{Volumes/34/06/mob_tactics::doc}]{\sphinxcrossref{Mob Tactics (1927)}}}

\item {} 
\sphinxAtStartPar
{\hyperref[\detokenize{Volumes/19/03/brothers_come_north::doc}]{\sphinxcrossref{Brothers, Come North (1920)}}}

\end{itemize}


\section{Shillady and Texas (1919)}
\label{\detokenize{Volumes/18/06/shillady_and_texas:shillady-and-texas-1919}}\label{\detokenize{Volumes/18/06/shillady_and_texas::doc}}
\sphinxAtStartPar
There was once a man who said that if he owned Hell and Texas, he would prefer to rent out Texas and live in Hell. He may have exaggerated, but he had some supporting facts: Texas was settled by white Southerners in order to extend slave territory; it was forcibly stolen from Mexico in 1837, largely because Mexico tried to abolish slavery in 1829. Thereupon Texas became a center of the African slave trade and the “most shameful violations” of United States slave trade laws were perpetrated through Texas. During the years of Texan independence slaves were rushed in at the rate of 15,000 or more a year and the annexation of Texas and the Mexican War were movements to extend Negro slavery.

\sphinxAtStartPar
Out of this past has risen the present Texas. In that state the first public burning alive of a Negro took place, at Paris. Since 1889 Texas has lynched 338 human beings—standing second only to Georgia and Mississippi in this horrible eminence.

\sphinxAtStartPar
Notwithstanding this, the Texas Negro has forged forward. Encouraged by his first great leader, Norris Wright Cuney, he has bought 21,182 farms with nearly two million acres of land, worth \$25,000,000. Starting with nothing fifty years ago nearly one\sphinxhyphen{}third of these black folk are now land owners.

\sphinxAtStartPar
To reward the Negroes for their thrift and struggles Texas gives them no voice in their own government, taxes them without representation and enforces “Jim\sphinxhyphen{}Crow’” travel, more irksome than in any other state because of the immense Texas distances. The Negro schools of Texas are better than in many Southern States and there are forty\sphinxhyphen{}four high schools for the 690,049 Negroes of the state, but 25 per cent of the Negro population is still absolutely illiterate and according to the white Houston \sphinxstyleemphasis{Post}: “The rural schools for Negro children where they exist at all are a joke.”

\sphinxAtStartPar
Is it not natural for the Negroes of such a state to endeavor to escape slavery?

\sphinxAtStartPar
They turned to the N.A.A.C.P. quite without solicitation. We made no special effort to organize branches in Texas, but 7,000 black Texans joined us to help make twelve million Americans “physically free from peonage, mentally free from ignorance, politically free from disfranchisement and socially free from insult.”

\sphinxAtStartPar
Is this revolution? Is this “stirring up trouble?” It is—in Texas, and to stop it a number of Texas gentlemen leaped on one unsuspecting, unarmed man and beat him nearly into unconsciousness, because he had come to their state to confer with colored and white people in the interests of this organization.

\sphinxAtStartPar
Mr. Shillady is a gentleman of training and experience, known to social workers all over the land. He was set upon by a judge, a constable and other officials, who have openly boasted their lawlessness and have been upheld by the Governor of the State.

\sphinxAtStartPar
This is Texas. This is the dominant white South. This is the answer of the Coward and the Brute to Reason and Prayer. This is the thing that America must conquer before it is civilized, and as long as Texas is this kind of Hell, civilization in America is impossible.


\bigskip\hrule\bigskip


\sphinxAtStartPar
\sphinxstyleemphasis{Citation:} Du Bois, W.E.B. 1919. “Shillady and Texas.”  18(6):283\sphinxhyphen{}284.


\section{Mob Tactics (1927)}
\label{\detokenize{Volumes/34/06/mob_tactics:mob-tactics-1927}}\label{\detokenize{Volumes/34/06/mob_tactics::doc}}
\sphinxAtStartPar
There has been developed in the United States a regular technique in matters of mob violence. Matters move somewhat as follows:

\sphinxAtStartPar
A crime is committed. The police hasten to accuse a Negro. This, of course, is popular because the white public readily believes in Negro crime. A Negro is arrested. If he is promptly lynched the police are vindicated and the guilty white persons saved from fear of detection. If lynching is delayed but threatened a mob usually attacks the Negro district. This gives a chance for looting and stealing. If any Negroes defend themselves, immediately the police, often assisted by the militia, promptly disarm all Negroes and charge a number with rioting. If any white people are arrested for rioting nearly all of them are discharged; but the Negroes are held and prosecuted. This serves to intimidate the Negro population and keeps it from attempting any self\sphinxhyphen{}defense, however innocent the defenders may be, and in no matter how grave danger to life, limb and property.

\sphinxAtStartPar
The result of all this is to mystify and mislead the public. By the time that the rioting is over, they are under the impression that the Negroes were partially responsible for starting the trouble and that they were armed and conspiring to kill innocent white people. Thus aggression against Negro Americans becomes an exciting form of sport for the lower order of white folk, in which they have practically nothing to lose and little to fear.

\sphinxAtStartPar
The technique of this procedure is, of course, taken from the acts of England and other countries in dealing with their colonies. Whenever the natives are subdued or punished and compelled to bow to the will of white folk, the explanation is that the natives were the aggressors; that the Colonial Power was acting in self\sphinxhyphen{}defense and that civilization was in danger.

\sphinxAtStartPar
The only solution to this kind of problem is not simply to permit but to encourage Negroes to keep and use arms in defense against lynchers and mobs.


\bigskip\hrule\bigskip


\sphinxAtStartPar
\sphinxstyleemphasis{Citation:} “Mob Tactics.” 1927. Editorial. \sphinxstyleemphasis{The Crisis} 34(6):204.


\section{Brothers, Come North (1920)}
\label{\detokenize{Volumes/19/03/brothers_come_north:brothers-come-north-1920}}\label{\detokenize{Volumes/19/03/brothers_come_north::doc}}
\sphinxAtStartPar
The migration of Negroes from South to North continues and ought to continue. The North is no paradise—as East St. Louis, Washington, Chicago, and Omaha prove; but the South is at best a system of caste and insult and at worst a Hell. With ghastly and persistent regularity, the lynching of Negroes in the South continues— every year, every month, every day; wholesale murders and riots have taken place at Norfolk, Longview, Arkansas, Knoxville, and 24 other places in a single year. The outbreaks in the North have been fiercer, but they have quickly been curbed; no attempt has been made to saddle the whole blame on Negroes; and the cities where riots have taken place are today safer and better for Negroes than ever before.

\sphinxAtStartPar
In the South, on the other hand, the outbreaks occurring daily but reveal the seething cauldron beneath—the unbending determination of the whites to subject and rule the blacks, to yield no single inch of their determination to keep Negroes as near slavery as possible.

\sphinxAtStartPar
There are, to be sure, Voices in the South—wise Voices and\_ troubled Consciences; souls that see the utter futility and impossibility of the southern program of race relations in work and travel and human intercourse. But these voices are impotent. Behold, Brough of Arkansas. He was an original leader of the most nromising recent group which sought Sense and Justice in the race problem—“The University Commission on Southern Race Questions.” He said, as chairman:
\begin{quote}

\sphinxAtStartPar
“As an American citizen the Negro is entitled to life, liberty, and the pursuit of happiness, and the equal protection of our laws for the safeguarding of these inalienable rights. … None but the most prejudiced Negro\sphinxhyphen{}hater, who oftentimes goes to the extreme of denying that any black man can have a white soul, would controvert the proposition that in the administration of quasi\sphinxhyphen{}public utilities and courts of justice the Negro is entitled to the fair and equal protection of the law. … The meanest Negro on a southern plantation is entitled to the same consideration in the administration of justice as the proudest scion of a cultured cavalier.”
\end{quote}

\begin{sphinxShadowBox}
\sphinxstylesidebartitle{}

\sphinxAtStartPar
The \sphinxhref{https://en.wikipedia.org/wiki/Elaine\_massacre}{Elaine massacre} was one of the worst mass killings of African\sphinxhyphen{}Americans in the United States.
\end{sphinxShadowBox}

\sphinxAtStartPar
Yet when he ran for Governor of Arkansas, he vehemently denied and explained away his liberal Negro sentiments,—and when the “uprising” occurred in Phillips. County, he let the slave barons make their own investigation, murder the innocent, and railroad ignorant, honest laborers to imprisonment and death in droves; contrast this with the actions of Governor Lowden of Illinois and Mayor Smith of Omaha!

\sphinxAtStartPar
On the other hand, we win through the ballot. We can vote in the North. We can hold office in the North. As workers in northern establishments, we are getting good wages, decent treatment, healthful homes and schools for our children. Can we hesitate? \sphinxstylestrong{Come North}! Not in a rush—not as aimless wanderers, but after quiet investigation and careful location. The demand for Negro labor is endless. Immigration is still cut off and a despicable and indefensible drive against all foreigners is shutting the gates of opportunity to the outcasts and victims of Europe. Very good. We will make America pay for her Injustice to us and to the poor foreigner by pouring into the open doors of mine and factory in increasing numbers.

\sphinxAtStartPar
Troubles will ensue with white unions and householders, but remember that the chief source of these troubles is rooted in the South; a million Southerners live in the North. These are the ones who by open and secret propaganda fomented trouble in these northern centers and are still at it. They have tried desperately to make trouble in Indianapolis, Cleveland, Pittsburgh, Philadelphia, Baltimore, and New York City.

\sphinxAtStartPar
This is a danger, but we have learned how to meet it by unwavering self\sphinxhyphen{}defense and by the ballot.

\sphinxAtStartPar
Meantime, if the South really wants the Negro and wants him at his best, it can have him permanently, on these terms and no others:
\begin{enumerate}
\sphinxsetlistlabels{\arabic}{enumi}{enumii}{}{.}%
\item {} 
\sphinxAtStartPar
The right to vote.

\item {} 
\sphinxAtStartPar
The abolition of lynching.

\item {} 
\sphinxAtStartPar
Justice in the courts.

\item {} 
\sphinxAtStartPar
The abolition of “Jim\sphinxhyphen{}Crow” cars.

\item {} 
\sphinxAtStartPar
A complete system of education, free and compulsory.

\end{enumerate}


\bigskip\hrule\bigskip


\sphinxAtStartPar
\sphinxstyleemphasis{Citation:} Du Bois, W.E.B. 1920. “Brothers, Come North.” \sphinxstyleemphasis{The Crisis}. 19(3): 105\sphinxhyphen{}106.


\chapter{Miscellaneous}
\label{\detokenize{Sections/other:miscellaneous}}\label{\detokenize{Sections/other::doc}}
\sphinxAtStartPar
Editorials on other topics.
\begin{itemize}
\item {} 
\sphinxAtStartPar
{\hyperref[\detokenize{Sections/racescience::doc}]{\sphinxcrossref{Science of race}}}
\begin{itemize}
\item {} 
\sphinxAtStartPar
{\hyperref[\detokenize{Volumes/02/04/races::doc}]{\sphinxcrossref{Races (1911)}}}

\end{itemize}

\item {} 
\sphinxAtStartPar
{\hyperref[\detokenize{Sections/criminal_justice::doc}]{\sphinxcrossref{Criminal Justice}}}
\begin{itemize}
\item {} 
\sphinxAtStartPar
{\hyperref[\detokenize{Volumes/15/03/thirteen::doc}]{\sphinxcrossref{Thirteen (1919)}}}

\item {} 
\sphinxAtStartPar
{\hyperref[\detokenize{Volumes/19/04/crime::doc}]{\sphinxcrossref{Crime (1920)}}}

\item {} 
\sphinxAtStartPar
{\hyperref[\detokenize{Volumes/32/01/crime::doc}]{\sphinxcrossref{Crime (1926)}}}

\item {} 
\sphinxAtStartPar
{\hyperref[\detokenize{Volumes/39/04/courts_and_jails::doc}]{\sphinxcrossref{Courts and Jails (1932)}}}

\end{itemize}

\item {} 
\sphinxAtStartPar
{\hyperref[\detokenize{Sections/intermarriage::doc}]{\sphinxcrossref{Intermarriage}}}
\begin{itemize}
\item {} 
\sphinxAtStartPar
{\hyperref[\detokenize{Volumes/05/04/intermarriage::doc}]{\sphinxcrossref{Intermarriage (1913)}}}

\item {} 
\sphinxAtStartPar
{\hyperref[\detokenize{Volumes/19/03/sex_equality::doc}]{\sphinxcrossref{Sex Equality (1920)}}}

\item {} 
\sphinxAtStartPar
{\hyperref[\detokenize{Volumes/31/05/correspondence::doc}]{\sphinxcrossref{Correspondence (1926)}}}

\end{itemize}

\item {} 
\sphinxAtStartPar
{\hyperref[\detokenize{Sections/reparations::doc}]{\sphinxcrossref{Reparations}}}
\begin{itemize}
\item {} 
\sphinxAtStartPar
{\hyperref[\detokenize{Volumes/24/02/whitecharity::doc}]{\sphinxcrossref{White Charity (1922)}}}

\end{itemize}

\item {} 
\sphinxAtStartPar
{\hyperref[\detokenize{Sections/colonialization::doc}]{\sphinxcrossref{Colonialization}}}
\begin{itemize}
\item {} 
\sphinxAtStartPar
{\hyperref[\detokenize{Volumes/09/01/worldwar::doc}]{\sphinxcrossref{World War and the Color Line (1914)}}}

\item {} 
\sphinxAtStartPar
{\hyperref[\detokenize{Volumes/17/04/reconstruction_and_africa::doc}]{\sphinxcrossref{Reconstruction and Africa (1919)}}}

\item {} 
\sphinxAtStartPar
{\hyperref[\detokenize{Volumes/19/03/race_pride::doc}]{\sphinxcrossref{Race Pride (1920)}}}

\item {} 
\sphinxAtStartPar
{\hyperref[\detokenize{Volumes/31/02/firing_line::doc}]{\sphinxcrossref{The Firing Line (1925)}}}

\item {} 
\sphinxAtStartPar
{\hyperref[\detokenize{Volumes/40/01/listen_japan_and_china::doc}]{\sphinxcrossref{Listen, Japan and China (1933)}}}

\end{itemize}

\item {} 
\sphinxAtStartPar
{\hyperref[\detokenize{Sections/misc::doc}]{\sphinxcrossref{Miscellaneous}}}
\begin{itemize}
\item {} 
\sphinxAtStartPar
{\hyperref[\detokenize{Volumes/03/05/lee::doc}]{\sphinxcrossref{Lee (1912)}}}

\item {} 
\sphinxAtStartPar
{\hyperref[\detokenize{Volumes/09/02/negro::doc}]{\sphinxcrossref{Negro (1914)}}}

\item {} 
\sphinxAtStartPar
{\hyperref[\detokenize{Volumes/06/07/national_emancipation_exposition::doc}]{\sphinxcrossref{The National Emancipation Exposition (1913)}}}

\item {} 
\sphinxAtStartPar
{\hyperref[\detokenize{Volumes/11/02/star_of_ethiopia::doc}]{\sphinxcrossref{The Star of Ethiopia (1915)}}}

\item {} 
\sphinxAtStartPar
{\hyperref[\detokenize{Volumes/18/03/reconstruction::doc}]{\sphinxcrossref{Reconstruction (1919)}}}

\item {} 
\sphinxAtStartPar
{\hyperref[\detokenize{Volumes/19/01/social_equity::doc}]{\sphinxcrossref{Social Equity (1919)}}}

\item {} 
\sphinxAtStartPar
{\hyperref[\detokenize{Volumes/34/02/farmers::doc}]{\sphinxcrossref{Farmers (1927)}}}

\item {} 
\sphinxAtStartPar
{\hyperref[\detokenize{Volumes/39/01/john_brown::doc}]{\sphinxcrossref{John Brown (1932)}}}

\end{itemize}

\end{itemize}


\section{Science of race}
\label{\detokenize{Sections/racescience:science-of-race}}\label{\detokenize{Sections/racescience::doc}}
\sphinxAtStartPar
Editorials on scientific understanding of race.
\begin{itemize}
\item {} 
\sphinxAtStartPar
{\hyperref[\detokenize{Volumes/02/04/races::doc}]{\sphinxcrossref{Races (1911)}}}

\end{itemize}


\subsection{Races (1911)}
\label{\detokenize{Volumes/02/04/races:races-1911}}\label{\detokenize{Volumes/02/04/races::doc}}
\begin{sphinxShadowBox}
\sphinxstylesidebartitle{}

\sphinxAtStartPar
The \sphinxhref{https://en.wikipedia.org/wiki/First\_Universal\_Races\_Congress}{First Universal Races Congress} was held at the University of London in 1911. They gathered:
\begin{quote}

\sphinxAtStartPar
To discuss, in the light of science and modern conscience, the general relations subsisting between the peoples of the West and those of the East, between the so\sphinxhyphen{}called “white” and the so\sphinxhyphen{}called “colored” peoples, with a view to encouraging between them a fuller understanding, the most friendly feelings, and the heartier co\sphinxhyphen{}operation.
\end{quote}

\sphinxAtStartPar
Du Bois delivered his paper, “The Negro Race in the United States of America.”
\end{sphinxShadowBox}

\sphinxAtStartPar
If Americans who have long since said the last word con­cerning the races of men and their proper relations will read the papers laid before the First Universal Races Congress, they will realize that America is fifty years behind the scientific world in its racial philosophy.

\sphinxAtStartPar
Before the publication of this epoch\sphinxhyphen{}making volume, \sphinxstylestrong{The Crisis} would not dare to express the statements which are contained therein. The leading scientists of the world have come forward in this book and laid down in categorical terms a series of proposi­tions which may be summarized as follows:
\begin{enumerate}
\sphinxsetlistlabels{\arabic}{enumi}{enumii}{}{.}%
\item {} 
\sphinxAtStartPar
(a) It is not legitimate to argue from differences in physical characteristics to differences in mental characteristics, (b) The mental characteristics differentiating a particular people or race are not (1) unchangeable, or (2) modifiable only through long ages of environmental pressure; but (3) marked improvements in mass education, in public sentiment, and in environment generally, may, apart from intermarriage—as the history of many a country proves—materially transform mental characteristics in a generation or two.

\item {} 
\sphinxAtStartPar
The civilization of a people or race at any particular moment of time offers no index to its innate or inherited capacities. In this respect it is of great importance to recognize that in the light of universal history civilizations are meteoric in nature, bursting out of relative obscurity only to plunge back into it.

\item {} 
\sphinxAtStartPar
(a) One ought to combat the irreconcilable contention prevalent among the various groups of mankind that their customs, their civilization, and their physique are superior to those of other groups, (b) In explanation of existing differences it would be pertinent to refer to the special needs arising from peculiar geographical and economic conditions and to related divergencies in national history; and, in explanation of the attitude of superiority assumed, it should be shown that intimacy leads to a love of our own customs, and unfamiliarity, among precipitate reasoners, to dislike and contempt for others’ customs.

\item {} 
\sphinxAtStartPar
(a) Divergencies in economic, hygienic, moral and educational standards are potent causes in keeping peoples apart who commercially or otherwise come in contact with each other, just as they keep classes apart, (b) These differences, like social differences generally, are in substance almost certainly due to passing social conditions, and not to inborn characteristics ; and the aim should therefore be, as in social differences generally, to remove these differences rather than to accentuate them by regarding them as fixed.

\item {} 
\sphinxAtStartPar
(a) The deepest cause of misunderstandings between peoples is perhaps the tacit assumption that the present characteristics of a people are the expression of permanent qualities. (b) If this is so, anthropologists, sociologists and scientific thinkers as a class could powerfully assist the movement for a juster appreciation of peoples by persistently pointing out in their lectures and in their works the fundamental fallacy involved in taking a static instead of a dynamic, a momentary instead of a historic, a fixed instead of a comparative, point of view of peoples; (c) and such dynamic teaching could be conveniently introduced into schools, more especially in the geography and history lessons, also into colleges for the training of teachers, diplomats, Colonial administrators, preachers and missionaries.

\item {} 
\sphinxAtStartPar
(a) The belief in racial or national superiority is largely due, as is suggested above, to unenlightened psychological repulsion and underestimation of the dynamic or environmental factors, (b) Since, therefore, there is no fair proof of some races being substantially superior to others in inborn capacity, our moral standard, or the manner of treating others— seeing how under favorable circumstances one people after another rises to fame, and how members of all human groups pass through universities with equal success— should remain the same whatever people we are dealing with.

\item {} 
\sphinxAtStartPar
(a) So far at least as intellectual and moral aptitudes are concerned, we ought to speak of civilizations where we now speak of races, (b) Indeed, even the physical characteristics, excluding the skin color of a people, are to no small extent the direct result of the physical and social environment under which it is living at any moment, and hence these characteristics differ measurably both in the history and in the different social strata of one and the same people; and (c) these physical characteristics are furthermore too indefinite and elusive to serve as a basis for any rigid classification or division of human groups, more especially as there has been much interbreeding among all peoples and because race characteris­tics are even said to distinguish every country and almost every province.

\item {} 
\sphinxAtStartPar
(a) The most fruitful cause of race crossing is ill\sphinxhyphen{}will—as illustrated by war, conquest, slavery, exploitation and persecution—for where there exists mutual respect the differences in differing traditions, etc., make it almost an invariable rule that intermarriage is avoided—as is shown by any two nations friendly to each other; (b) but intermarriage, we find — contrary to popular tenets— improves the vitality and capacity of a people, and cannot, therefore, be objectionable in itself, (c) The chief drawback to intermarriage between peoples is the same as the drawback to intermarriage between different social classes— i.e., the different traditions of the partners in marriage, (d) Those who dread intermarriage should, therefore, reflect both that there is no such thing as purity of race, and that the rate of crossing decreases with the increase of interracial and international amity.

\item {} 
\sphinxAtStartPar
(a) Each people  might study with advantage the customs and civilizations of other races or peoples, including those it thinks the lowest ones, for the definite purpose of improving its own customs and civilization, since the lowliest civilizations even have much to teach, (b) Dignified and unostentatious conduct and deferential respect for the customs of others, provided these are not morally objectionable to an unprejudiced mind, should be recommended to all who come in passing or permanent contact with members of human groups that are unfamiliar to them.

\end{enumerate}

\sphinxAtStartPar
These are a fair summary of the conclusions of writers who are among the best\sphinxhyphen{}known names in modern science. In the next number of \sphinxstylestrong{The Crisis} we shall give some of their views at length.


\bigskip\hrule\bigskip


\sphinxAtStartPar
\sphinxstyleemphasis{Citation:} Du Bois, W.E.B. 1911. “Races”  \sphinxstyleemphasis{The Crisis}. 2(4): 157\sphinxhyphen{}158.


\section{Criminal Justice}
\label{\detokenize{Sections/criminal_justice:criminal-justice}}\label{\detokenize{Sections/criminal_justice::doc}}
\sphinxAtStartPar
Editorials on  the criminal justice system
\begin{itemize}
\item {} 
\sphinxAtStartPar
{\hyperref[\detokenize{Volumes/15/03/thirteen::doc}]{\sphinxcrossref{Thirteen (1919)}}}

\item {} 
\sphinxAtStartPar
{\hyperref[\detokenize{Volumes/19/04/crime::doc}]{\sphinxcrossref{Crime (1920)}}}

\item {} 
\sphinxAtStartPar
{\hyperref[\detokenize{Volumes/32/01/crime::doc}]{\sphinxcrossref{Crime (1926)}}}

\item {} 
\sphinxAtStartPar
{\hyperref[\detokenize{Volumes/39/04/courts_and_jails::doc}]{\sphinxcrossref{Courts and Jails (1932)}}}

\end{itemize}


\subsection{Thirteen (1919)}
\label{\detokenize{Volumes/15/03/thirteen:thirteen-1919}}\label{\detokenize{Volumes/15/03/thirteen::doc}}
\begin{sphinxShadowBox}
\sphinxstylesidebartitle{}

\sphinxAtStartPar
This editorial is on the \sphinxhref{https://en.wikipedia.org/wiki/East\_St.\_Louis\_riots}{East St. Louis riots}.
\end{sphinxShadowBox}

\sphinxAtStartPar
The have gone to their death. Thirteen young, strong men; soldiers who have fought for a country which never was wholly theirs; men born to suffer ridicule, injustice, and, at last, death itself. They broke the law. Against their punishment, if it was legal, we cannot protest. But we can protest and we do protest against the shame­ful treatment which these men and which we, their brothers, receive all our lives, and which our fathers received, and our children await; and above all we raise our clenched hands against the hundreds of thousands of white murderers, rapists, and scoundrels who have oppressed, killed, ruined, robbed, and debased their black fellow men and fellow women, and yet, today, walk scot\sphinxhyphen{}free, un\sphinxhyphen{} whipped of justice, uncondemned by millions of their white fellow citizens, and unrebuked by the President of the United States.


\bigskip\hrule\bigskip


\sphinxAtStartPar
\sphinxstyleemphasis{Citation:} “Thirteen” Editorial. 1919. \sphinxstyleemphasis{The Crisis}. 15(3): 114.


\subsection{Crime (1920)}
\label{\detokenize{Volumes/19/04/crime:crime-1920}}\label{\detokenize{Volumes/19/04/crime::doc}}
\sphinxAtStartPar
We are not for a moment denying the existence of a criminal class among Negroes, who are guilty of deeds of violence. Every race in the world has such groups. No human efforts have yet been able wholly to rid society of crime. But if of all groups, the American Negro is to be singled out and punished \sphinxstylestrong{as a group} for the detestable deeds of its criminals, then this country is staging a race war of the bitterest kind, when the wronged and the innocent fight in desperate defense against the mob and murderer.

\sphinxAtStartPar
There is a curious assumption in some quarters that intelligent and law\sphinxhyphen{}abiding Negroes like, encourage, and sympathize with Negro crime and defend Negro criminals. They do not. They suffer more from the crime of their fellows than white ‘folk suffer, not only vicariously, but directly; the black criminal knows that he can prey on his own people with the least danger of punishment, because they control no police or courts.

\sphinxAtStartPar
But what can Negroes do to decrease crime? Some white Southerners have but one suggestion, which is that when a Negro is accused of crime, other Negroes turn to run him down and hand him over to the authorities.

\begin{sphinxShadowBox}
\sphinxstylesidebartitle{}

\sphinxAtStartPar
B.J. Keiley was the Catholic bishop of Savannah
\end{sphinxShadowBox}

\sphinxAtStartPar
But hold! Is there no difference between a person accused of crime and a criminal? Are black folks accused of crime in the South assured of a fair trial and just punishment? We will let a white southern ex\sphinxhyphen{}Confederate, Bishop B. J. Keiley, of Georgia, answer in the \sphinxstyleemphasis{Savannah Press}:
\begin{quote}

\sphinxAtStartPar
“Is it not the fact that fair and impartial justice is not meted out to white and colored men alike? The courts of this state either set the example, or follow the example set them, and they make a great distinction between the white and the black criminal brought before them. The latter, as a rule, gets the full limit of the law. Do you ever hear of a street difficulty in which a Negro and a white man were involved which was brought before a judge, in which, no matter what were the real facts of the case, the Negro did not get the worst of it?”
\end{quote}

\sphinxAtStartPar
This is bad enough, but this is not all. We have criminals who deserve punishment. Now the modern treatment of crime and criminals, is built on carefully considered principles: one, old as the English Common Law, and older, declares that it is better for the community that ten guilty men should escape, rather. than that one innocent man should be punished; moreover, it is beginning to be widely recognized that in crime, the criminal is not the only one guilty; you and I share in the guilt if we have not given him as a child an education, furnished him with a place to play, and seen that his body was nourished; we are guilty if as a man he was not allowed to do honest work, did not receive a living wage, and did not have proper social environment.

\sphinxAtStartPar
This social responsibility for crime is so widely recognized that when the criminal is arrested, the first desire of decent modern society is to reform him, and not to avenge itself on him. Penal servitude is being recognized only as it protects society and improves the criminal, and not because it makes him suffer as his victim suffered.

\sphinxAtStartPar
What, now, is the attitude of the white South toward Negro crime? First and foremost, it would rather that ten innocent Negroes suffer than that one guilty one escape; secondly, it furnishes Negro children, for the most part, wretched schools and no playgrounds; it usually pays the adult low wages, houses him in slums, and gives him neither care nor thought, until he steals or murders. It has few juvenile reformatories, and herds all kinds of criminals together, selling them into slavery to the highest bidder, under the “Lease” system. Its idea of punishment is vengeance— vengeance of the cruelest and most blood\sphinxhyphen{}curdling sort.

\sphinxAtStartPar
Under such circumstances, what can an honest Negro do to stop Negro crime?


\bigskip\hrule\bigskip


\sphinxAtStartPar
\sphinxstyleemphasis{Citation:} Du Bois, W.E.B. 1920. “Crime.” \sphinxstyleemphasis{The Crisis}. 19(4): 172\sphinxhyphen{}173.


\subsection{Crime (1926)}
\label{\detokenize{Volumes/32/01/crime:crime-1926}}\label{\detokenize{Volumes/32/01/crime::doc}}
\sphinxAtStartPar
The junior organizations of the Ku Klux Klan are receiving thousands of copies of “Educational Studies Number 10” which concerns “the Negro and his relation to America”. This pamphlet is arranged without the slightest regard for Truth. It asserts, for instance: “The Negro is constantly increasing in criminality”. As a matter of fact, between 1910 and 1923 the number of Negroes committed for crime decreased over 26 per cent.

\sphinxAtStartPar
It discovers that 70 per cent of Negro criminals are under 30 years of age, but apparently does not know that 80 per cent of the white criminals in New York City are under 22.

\sphinxAtStartPar
It repeats the canard that educated Negroes are more criminal than the illiterates, a conclusion contradicted by every known fact and resting simply upon a partial census report in 1910. In this case they asked Negro criminals if they could read and write and naturally most of them said they could.

\sphinxAtStartPar
With these falsehoods go the usual half\sphinxhyphen{}truths, namely, that the Negro is the most criminal element in our population and that he is more criminal in the North than in the South.

\sphinxAtStartPar
Crime is social disease; it is a complex result of poverty, ignorance and other sorts of social degradation. As the peculiar victim of these things the Negro in the United States suffers more from arrest and punishment at the hands of police and courts than any other element. This goes without saying.

\sphinxAtStartPar
The reason of it is clear. Practically the whole South traffics in Negro crime and makes money out of it. The convict lease system is in full blast in South Carolina, Florida, Georgia, Alabama, Mississippi, Louisiana and elsewhere. It is called by other names and technically “reformed” from time to time but one has only to read the terrible exposures in the New York World of the money made out of selling criminals in Alabama and Florida to realize what is happening.

\sphinxAtStartPar
The natural delinquency arising from his position in the United States is increased by the Negro’s treatment in the courts and in jails. A legislative committee of investigation in South Carolina reported in 1923: “Both the superintendent and directors testify that there are no written or printed rules to govern the conduct of and control over prisoners. The situation might well be summed up as follows: the discipline\sphinxhyphen{}of the penitentiary is largely based on the personal likes and dislikes of the captain of the guard.”

\sphinxAtStartPar
The United States is deliberately manufacturing Negro crime and has been doing so for 150 years. And despite this, Negro crime has decreased 26 per cent in the last 13 years.


\bigskip\hrule\bigskip


\sphinxAtStartPar
\sphinxstyleemphasis{Citation:} Du Bois, W.E.B. 1926. “Crime.”  32(1):9\sphinxhyphen{}10.


\subsection{Courts and Jails (1932)}
\label{\detokenize{Volumes/39/04/courts_and_jails:courts-and-jails-1932}}\label{\detokenize{Volumes/39/04/courts_and_jails::doc}}
\sphinxAtStartPar
It is to the disgrace of the American Negro, and particularly of his religious and philanthropic organizations, that they continually and systematically neglect Negroes who have been arrested, or who are accused of crime, or who have been convicted and incarcerated.

\sphinxAtStartPar
One can easily realize the reason for this: ever since Emancipation and even before, accused and taunted with being criminals, the emancipated and rising Negro has tried desperately to disassociate himself from his own criminal class. He has been all too eager to class criminals as outcasts, and to condemn every Negro who has the misfortune to be arrested or accused. He has joined with the bloodhounds in anathematizing every Negro in jail, and has called High Heaven to witness that he has absolutely no sympathy and no known connection with any black man who has committed crime.

\sphinxAtStartPar
All this, of course, is arrant nonsense; is a combination of ignorance and pharisaism which ought to put twelve million people to shame. There is absolutely no scientific proof, statistical, social or physical, to show that the American Negro is any more criminal than other elements in the American nation, if indeed as criminal. Moreover, even if he were, what is crime but disease, social or physical? In addition to this, every Negro knows that a frightful proportion of Negroes accused of crime are absolutely innocent. Nothing in the world is easier in the United States than to accuse a black man of crime. In the South, if any crime is committed, the first cry of the mob is, “Find the Negro!” And while they are finding him, the white criminal comfortably escapes. Nothing is easier, South and North, than for a white man to black his face, saddle a felony upon the Negro, and then go wash his body and his soul. Today, if a Negro is accused, whether he is innocent or guilty, he not only is almost certain of conviction, but of getting the limit of the law. What else is the meaning of the extraordinary fact that throughout the United States the number of Negroes hanged, sentenced for life, or for ten, twenty or forty years, is an amazingly large proportion of the total number?

\sphinxAtStartPar
Meantime, what are we doing about it? Here and there, in a few spectacular cases, we are defending persons, where race discrimination is apparent, and where the poor devil of a victim manages to get into the newspaper. But in most cases, the whole black world is dumb and acquiescent; they will not even visit the detention houses where the accused, innocent and guilty, are herded like cattle. They make few systematic attempts to reform the juvenile delinquent who may be guilty of nothing more than energy and mischief. Only in sporadic cases do we visit the jails and hear the tales of the damned.

\sphinxAtStartPar
For a race which boasts its Christianity, and for a Church which squanders its money upon carpets, organs, stained glass, bricks and stone, this attitude toward Negro crime is the most damning accusation yet made.


\bigskip\hrule\bigskip


\sphinxAtStartPar
\sphinxstyleemphasis{Citation:} Du Bois, W.E.B. 1932. “Courts and Jails” 39(4):132.


\section{Intermarriage}
\label{\detokenize{Sections/intermarriage:intermarriage}}\label{\detokenize{Sections/intermarriage::doc}}
\sphinxAtStartPar
Editorials on intermarriage
\begin{itemize}
\item {} 
\sphinxAtStartPar
{\hyperref[\detokenize{Volumes/05/04/intermarriage::doc}]{\sphinxcrossref{Intermarriage (1913)}}}

\item {} 
\sphinxAtStartPar
{\hyperref[\detokenize{Volumes/19/03/sex_equality::doc}]{\sphinxcrossref{Sex Equality (1920)}}}

\item {} 
\sphinxAtStartPar
{\hyperref[\detokenize{Volumes/31/05/correspondence::doc}]{\sphinxcrossref{Correspondence (1926)}}}

\end{itemize}


\subsection{Intermarriage (1913)}
\label{\detokenize{Volumes/05/04/intermarriage:intermarriage-1913}}\label{\detokenize{Volumes/05/04/intermarriage::doc}}
\sphinxAtStartPar
Few  groups of people are forced by their situation into such cruel dilem­mas as American Ne­groes. Nevertheless they must not allow anger or personal resentment to dim their clear vision.

\sphinxAtStartPar
Take, for instance, the question of the intermarrying of white and black folk; it is a question that colored people sel­dom discuss. It is about the last of the social problems over which they are dis­turbed, because they so seldom face it in fact or in theory. Their problems are problems of work and wages, of the right to vote, of the right to travel decently, of the right to frequent places of pub­ lic amusement, of the right to public security.

\sphinxAtStartPar
White people, on the other hand, for the most part profess to see but one prob­lem: “Do you want your sister to marry a N{[}*****{]}?”” Sometimes we are led to wonder if they are lying about their solicitude on this point; and if they are not, we are led to ask why under present laws anybody should be compelled to marry any person whom she does not wish to marry?

\sphinxAtStartPar
This brings us to the crucial question: so far as the present advisability of intermarrying between white and colored people in the United States is concerned, both races are practically in complete agreement. Colored folk marry colored folk and white marry white, and the exceptions are very few.

\sphinxAtStartPar
Why not then stop the exceptions? For three reasons: physical, social and moral.
\begin{enumerate}
\sphinxsetlistlabels{\arabic}{enumi}{enumii}{}{.}%
\item {} 
\sphinxAtStartPar
For the physical reason that to prohibit such intermarriage would be publicly to acknowledge that black blood is a physical taint—a thing that no decent, self\sphinxhyphen{}respecting black man can be asked to admit.

\item {} 
\sphinxAtStartPar
For the social reason that if two full\sphinxhyphen{}grown responsible human beings of any race and color propose to live to­gether as man and wife, it is only social decency not simply to allow, but to com­pel them to marry. Let those people who have yelled themselves purple in the face over Jack Johnson just sit down and ask themselves this q u e s t i o n : Granted that Johnson and Miss Cameron proposed to live together, was it better for them to be legally married or not? We know what the answer of the Bourbon South is. W e know that they would rather uproot the foundations of decent society than to call the consorts of their brothers, sons and fathers their legal wives. We infinitely prefer the methods of Jack Johnson to those of the brother of Governor Mann of Virginia.

\item {} 
\sphinxAtStartPar
The moral reason for opposing laws against intermarriage is the greatest of all: such laws leave the colored girl absolutely helpless before the lust of white men. It reduces colored women in the eyes of the law to the position of  dogs. Low as the white girl falls, she can compel her seducer to marry her. If it were proposed to take this last defense from poor white working girls, can you not hear the screams of the “white slave” defenders? What have these people to say to laws that propose to create in the United States 5,000,000 women, the ownership of whose bodies no white man is bound to respect?

\end{enumerate}

\sphinxAtStartPar
Note these arguments, my brothers and sisters, and watch your State legislatures. This winter will see a determined attempt to insult and degrade us by such non\sphinxhyphen{}intermarriage laws. We must kill them, not because we are anxious to marry white men’s sisters, but because we are determined that white men shall let our sisters alone.


\bigskip\hrule\bigskip


\sphinxAtStartPar
\sphinxstyleemphasis{Citation:} Du Bois, W.E.B. 1913. “Intermarriage.”  \sphinxstyleemphasis{The Crisis}. 5(4): 57.


\subsection{Sex Equality (1920)}
\label{\detokenize{Volumes/19/03/sex_equality:sex-equality-1920}}\label{\detokenize{Volumes/19/03/sex_equality::doc}}
\begin{sphinxShadowBox}
\sphinxstylesidebartitle{}

\sphinxAtStartPar
Attorney General \sphinxhref{https://en.wikipedia.org/wiki/A.\_Mitchell\_Palmer}{Mitchell Palmer}
\end{sphinxShadowBox}

\sphinxAtStartPar
The Department of Justice has discovered a new crime,— “Sex Equality.” This is not, as one might presume, equality of men and women, but it is the impudence of a man of Negro descent asserting his right to marry any human being who wants to marry him. With bated breath, Mr. Palmer (who has no power to prevent or punish lynching and who permits peonage to flourish untouched in Arkansas) tells an astonished Senate of this new sign of “Red” propaganda among blacks. Nonsense! Mr. Palmer is mistaken in assuming that it took a world war to make the Negro conscious of such an elementary right. No Negro with any sense has ever denied his right to marry another human being, for the simple reason that such denial would be frank admission of his own inferiority. For a man to stand up and say: I am not physically or morally or mentally fit to marry this woman, who wishes to marry me, would be a horrible admission. No healthy, decent man,—white, black, red, or blue—could for a moment admit so monstrous a fact.

\sphinxAtStartPar
He could, naturally, say: I do not WANT to marry this woman of another race, and this is what 999 black men out of every thousand DO say. Or a woman may say: I do not want to marry this black man, or this red man, or this white man,—this she has the absolute and unquestionable right to say. But the impudent and vicious demand that all colored folk shall write themselves down as brutes by a general assertion of their unfitness to marry other decent folk is a nightmare born only in the haunted brain of the bourbon South and transmitted by some astonishing power to the lips of the Attorney General of the United States.


\bigskip\hrule\bigskip


\sphinxAtStartPar
\sphinxstyleemphasis{Citation:} Du Bois, W.E.B. 1920. “Sex Equality.” \sphinxstyleemphasis{The Crisis}. 19(3): 106.


\subsection{Correspondence (1926)}
\label{\detokenize{Volumes/31/05/correspondence:correspondence-1926}}\label{\detokenize{Volumes/31/05/correspondence::doc}}
\sphinxAtStartPar
\sphinxstyleemphasis{Broken Bow, Nebraska.}

\sphinxAtStartPar
As president of a woman’s club I am writing you for information in regard to your views on race assimilation, intermarriage of Negroes and whites. We have just completed a two weeks’ study of your book “Souls of Black Folk” and are unable to arrive at a definite conclusion as to your attitude toward this question. I am asking you to please let me know your honest convictions along this line and greatly oblige,

\sphinxAtStartPar
Dr. Elizabeth Leonard



\sphinxAtStartPar
69 Fifth Ave., New York.
\begin{enumerate}
\sphinxsetlistlabels{\arabic}{enumi}{enumii}{}{.}%
\item {} 
\sphinxAtStartPar
I believe that from time to time the groups of human beings, which we call races, assimilate and again differentiate. No race is permanent in its physical or mental characteristics.

\item {} 
\sphinxAtStartPar
I believe that individuals usually will find the greatest happiness and the greatest chance to do their best work if they marry within their own racial group. There are, of course, exceptions to this and many marriages between persons of different races have turned out happily. But usually, for obvious reasons, marriages within the group are most likely to be happy.

\item {} 
\sphinxAtStartPar
Despite the above I maintain the perfect right of any individual of any race, who is sane and normal, to marry the person who wishes to marry him. Any denial of this fundamental right of human intercourse always results in more evil than the denial seeks to prevent.

\item {} 
\sphinxAtStartPar
Specifically and in regard to the intermarriage of Negroes and whites in the United States, I believe that when any group is disliked and ostracized, for historical and other reasons, its self\sphinxhyphen{}respect demands that it seek to minimize as far as possible any intermarriage with the group that assumes superiority.

\item {} 
\sphinxAtStartPar
Finally, I have no doubt that in large numbers of cases groups of persons working together and intermarrying have been enabled to make peculiar contributions to civilization and to preserve and hand down these gifts; and any group that has done this or wishes to do it has a right to confine its marriages to its own members so far as it does not seek also to insult or degrade other groups or deny them the same rights.

\end{enumerate}

\sphinxAtStartPar
\sphinxstyleemphasis{W. E. B. Du Bois}


\bigskip\hrule\bigskip


\sphinxAtStartPar
\sphinxstyleemphasis{Citation:} Du Bois, W.E.B. 1926. “Correspondence.”  31(5):218.


\section{Reparations}
\label{\detokenize{Sections/reparations:reparations}}\label{\detokenize{Sections/reparations::doc}}
\sphinxAtStartPar
Editorials on reperations
\begin{itemize}
\item {} 
\sphinxAtStartPar
{\hyperref[\detokenize{Volumes/24/02/whitecharity::doc}]{\sphinxcrossref{White Charity (1922)}}}

\end{itemize}


\subsection{White Charity (1922)}
\label{\detokenize{Volumes/24/02/whitecharity:white-charity-1922}}\label{\detokenize{Volumes/24/02/whitecharity::doc}}
\sphinxAtStartPar
Throughout the United States are numberless charities—schools, homes, hospitals and orphanages,—supported wholly or in part by white donors for the benefit of Negroes. As the Negroes have accumulated more means and become more self\sphinxhyphen{}assertive, the tendency has been for white givers to reduce their gifts or discontinue their interest entirely, putting many worthy and useful, indeed indispensable, institutions in grave distress.

\sphinxAtStartPar
The motives for this withdrawal of help are various: many charitable folk have been left straitened by the war and the new rich have not learned charity. Other folks think that Negroes are now rich enough to help themselves; while not a few others resent the Negroes’ new tone and demands so deeply that they say: Very well, help yourselves and make no more appeals to us!

\sphinxAtStartPar
These last two classes are ill\sphinxhyphen{}advised. The Negro is still a poor, a very poor group and cannot support the social reform, and eleemosynary work which he needs for social uplift. Moreover his great bond to the rich and powerful has been their charity—if they break this bond they break the last tie that holds him in leash. This may be best in time, but for them is it wise now? Is it wise for white folk to forget that no amount of alms­ giving on their part will half repay the 300 years of unpaid toil and the fifty years of serfdom by which the black man has piled up wealth and comfort for white America?

\sphinxAtStartPar
It would be a wiser and more far\sphinxhyphen{} sighted attitude today for white America to insist on paying back this debt which they owe to black America as a privilege—as a great peace\sphinxhyphen{}offering for wrong—rather than petulantly to vent their spleen on the sick, the degraded and the young for the growing self\sphinxhyphen{}assertion of the well, the risen and the old.

\sphinxAtStartPar
On our part the way is clear: the sooner we rise above charity, the sooner we shall be free.


\bigskip\hrule\bigskip


\sphinxAtStartPar
\sphinxstyleemphasis{Citation:} “White  charity.” Editorial. 1922. \sphinxstyleemphasis{The Crisis}. 24(2): 57.


\section{Colonialization}
\label{\detokenize{Sections/colonialization:colonialization}}\label{\detokenize{Sections/colonialization::doc}}
\sphinxAtStartPar
Editorials on colonialization
\begin{itemize}
\item {} 
\sphinxAtStartPar
{\hyperref[\detokenize{Volumes/09/01/worldwar::doc}]{\sphinxcrossref{World War and the Color Line (1914)}}}

\item {} 
\sphinxAtStartPar
{\hyperref[\detokenize{Volumes/17/04/reconstruction_and_africa::doc}]{\sphinxcrossref{Reconstruction and Africa (1919)}}}

\item {} 
\sphinxAtStartPar
{\hyperref[\detokenize{Volumes/19/03/race_pride::doc}]{\sphinxcrossref{Race Pride (1920)}}}

\item {} 
\sphinxAtStartPar
{\hyperref[\detokenize{Volumes/31/02/firing_line::doc}]{\sphinxcrossref{The Firing Line (1925)}}}

\item {} 
\sphinxAtStartPar
{\hyperref[\detokenize{Volumes/40/01/listen_japan_and_china::doc}]{\sphinxcrossref{Listen, Japan and China (1933)}}}

\end{itemize}


\subsection{World War and the Color Line (1914)}
\label{\detokenize{Volumes/09/01/worldwar:world-war-and-the-color-line-1914}}\label{\detokenize{Volumes/09/01/worldwar::doc}}
\index{colonialization@\spxentry{colonialization}}\ignorespaces 
\index{Africa@\spxentry{Africa}}\ignorespaces 
\sphinxAtStartPar
Many colored persons, and persons interested in them, may easily make the mistake of supposing that the present war is far removed from the color problem of America and that in the face of this great catastrophe we may forget for a moment such local problems and give all attention and contributions to the seemingly more pressing cause.

\sphinxAtStartPar
This attitude is a mistake. The present war in Europe is one of the great disasters due to race and color prejudice and it but foreshadows greater disasters in the future.

\sphinxAtStartPar
It is not merely national jealousy, or the so\sphinxhyphen{}called “race” rivalry of Slav, Teuton and Latin, that is the larger cause of this war. It is rather the wild quest for Imperial expansion among colored races between Germany, England and France primarily, and Belgium. Italy, Russia and Austria\sphinxhyphen{}Hungary in lesser degree. Germany long since found herself shut out from acquiring colonies. She looked toward South America, but the “Monroe Doctrine” stood in her way. She started for Africa and by bulldozing methods secured one good colony, one desert and two swamps. Her last efforts looked toward North Africa and Asia\sphinxhyphen{}Minor. Finally, she evidently decided at the first opportunity to seize English or French colonies and to this end feverishly expanded her navy, kept her army at the highest point of efficiency and has been for twenty years the bully of Europe with a chip on her shoulder and defiance in her mouth.

\sphinxAtStartPar
The colonies which England and France own and Germany covets are largely in tropical and semi\sphinxhyphen{}tropical lands and inhabited by black, brown and yellow peoples. In such colonies there is a chance to confiscate land, work the natives at low wages, make large profits and open wide markets for cheap European manufactures. Asia, Africa, the South Sea Islands, the West Indies, Mexico and Central America and much of South America have long been designated by the white world as fit field for this kind of commercial exploitation, for the benefit of Europe and with little regard for the welfare of the natives. One has only to remember the forced labor in South Africa, the outrages in Congo, the cocoa\sphinxhyphen{}slavery in Portuguese Africa, the land monopoly and peonage of Mexico, the exploitation of Chinese coolies and the rubber horror of the Amazon to realize what white imperialism is doing to\sphinxhyphen{}day in well\sphinxhyphen{}known cases, not to mention thousands of less\sphinxhyphen{}known instances.

\sphinxAtStartPar
In this way a theory of the inferiority of the darker peoples and a contempt for their rights and aspirations has become all but universal in the greatest centers of modern culture. Here it was that American color prejudice and race hatred received in recent years unexpected aid and sympathy. To\sphinxhyphen{}day civilized nations arc fighting like mad dogs over the right to own and exploit these darker peoples.

\sphinxAtStartPar
In such case where should our sympathy lie? Undoubtedly, with the Allies—with England and France in particular. Not that these nations are innocent. England was in the past blood\sphinxhyphen{} guilty above all lands in her wicked and conscienceless rape of darker races. England was primarily responsible for American slavery, for the starvation of India, and the Chinese opium traffic. But the salvation of England is that she has the ability to learn from her mistakes. To\sphinxhyphen{}day no white nation is fairer in its treatment of darker peoples than England. Not that England is yet fair. She is not yet just, and she still nourishes much disdain for colored races, erects contemptible and humiliating political and social barriers and steals their land and labor; but as compared with Germany England is an angel of light. The record of Germany as a colonizer toward weaker and darker people is the most barbarous of any civilized people and grows worse instead of better. France is less efficient than England as an administrator of colonies and has consequently been guilty of much neglect and injustice; but she is nevertheless the most kindly of all European nations in her personal relations with colored folk. She draws no dead line of color and colored Frenchmen always love France.

\sphinxAtStartPar
Belgium has been as pitiless and grasping as Germany and in strict justice deserves every pang she is suffering after her unspeakable atrocities in Congo. Russia has never drawn a color line but has rather courted the yellow races, although with ulterior motives. Japan, however, instilled wholesome respect in this line.

\sphinxAtStartPar
Undoubtedly, then the triumph of the allies would at least leave the plight of the colored races no worse than now. Indeed, considering the. fact than black Africans and brown Indians and yellow Japanese are fighting for France and England it may be that they will come out of this frightful welter of blood with new ideas of the essential equality of all men.

\sphinxAtStartPar
On the other hand, the triumph of Germany means the triumph of every force calculated to subordinate darker peoples. It would mean triumphant militarism, autocratic and centralized government and a studied theory of contempt for everything except Germany— “Germany above everything in the world.” The dispair and humiliation of Germany in the eighteenth century has brought this extraordinary rebound of self\sphinxhyphen{}exaltation and disdain for mankind. The triumph of this idea would mean a crucifixion of darker peoples unparalleled in history.

\sphinxAtStartPar
The writer speaks without anti\sphinxhyphen{}German bias; personally he has deep cause to love the German people. They made him believe in the essential humanity of white folk twenty years ago when he was near to denying it. But even then the spell of militarism was in the air, and the Prussian strut had caught the nation’s imagination. They were starting on the same road with the southern American whites toward a contempt toward human beings and a faith in their own utter superiority to all other breeds. This feeling had not then applied itself particularly to colored folk and has only begun to to\sphinxhyphen{}day; but it is going by leaps and bounds.. Germany needs but the role of world conquest to make her one of the most contemptible of “N{[}*****{]}” hating nations. Just as we go to press, the Berliner Tageblatt publishes a proclamation by “German representatives of Science and Art to the World of Culture” in which men like Harnack, Bode, Hauptmann, Suderman, Roentgen, Humperdink, Wundt and others, insult hundreds of millions of human beings by openly sneering at “Mongrels and N{[}******{]}.””

\sphinxAtStartPar
As colored Americans then, and as Americans who fear race prejudice as the greatest of War\sphinxhyphen{}makers, our sympathies in the awful conflict should be with France and England; not that they have conquered race prejudice, but they have at least begun to realize its cost and evil, while Germany exalts it.

\sphinxAtStartPar
If so great, a catastrophe has followed jealousies and greed built on a desire to steal from and oppress people whom the dominant culture despises, how much wilder and wider will be the conflict when black and brown and yellow people stand up together shoulder to shoulder and demand recognition as men!

\sphinxAtStartPar
Let us give then our sympathies to those nations whose triumph will most tend to postpone if not to make unnecessary a world war of races.


\bigskip\hrule\bigskip


\sphinxAtStartPar
“World War and the Color Line.” Editorial. 1913. \sphinxstyleemphasis{The Crisis} 9(1): 28\sphinxhyphen{}30.


\subsection{Reconstruction and Africa (1919)}
\label{\detokenize{Volumes/17/04/reconstruction_and_africa:reconstruction-and-africa-1919}}\label{\detokenize{Volumes/17/04/reconstruction_and_africa::doc}}
\index{colonialization@\spxentry{colonialization}}\ignorespaces 
\index{Africa@\spxentry{Africa}}\ignorespaces 
\sphinxAtStartPar
The suggestion has been made that these colonies which Germany has lost should not be handed over to any other nation of Europe but should, under the guidance of organized civilization, be brought to a point of development which shall finally result in an autonomous state. This plan has met with much criticism and ridicule. Let the natives develop along their own lines and they will “go back,” has been the cry. Back to what, in Heaven’s name?

\sphinxAtStartPar
Is a civilization naturally backward because it is different? Outside of cannibalism, which can be matched in this country, at least, by lynching, there is no vice and no degradation in native African customs which can begin to touch the horrors thrust upon them by white masters. Drunkenness, terrible diseases, immorality, all these things have been the gifts of European civilization. There is no need to dwell on German and Belgian atrocities, the world knows them too well. Nor have France and England been blameless. But even supposing that these masters had been models of kindness and rectitude, who shall say that any civilization is in itself so superior that it must be super­ imposed upon another nation with­ out the expressed and intelligent con­ sent of the people most concerned. The culture indigenous to a country, its folk\sphinxhyphen{}customs, its art, all this must have free scope or there is no such thing as freedom for the world.

\sphinxAtStartPar
The truth is, white men are merely juggling with words—or worse— when they declare that the withdrawal of Europeans from Africa will plunge that continent into chaos. What Europe, and indeed only a small group in Europe, wants in, Africa is not a field for the spread of European civilization, but a field for exploita­tion. They covet the raw materials,—ivory, diamonds, copper and rubber in which the land abounds, and even more do they covet cheap native labor to mine and produce these things. Greed,—naked, pitiless lust for wealth and power, lie back of all of Europe’s interest in Africa and the white world knows it and is not ashamed.

\sphinxAtStartPar
Any readjustment of Africa is not fair and cannot be lasting which does not consider the interests of native Africans and peoples of African descent. Prejudice, in European colonies in Africa, against the ambitious Negro is greater than in America, and that is saying much. But with the establishment of a form of gov­ernment which shall be based on the concept that Africa is for Africans, there would be a chance for the colored American to emigrate and to go as a pioneer to a country which must, sentimentally at least, possess for him the same fascination as England does for Indian\sphinxhyphen{}born English­men.


\bigskip\hrule\bigskip


\sphinxAtStartPar
\sphinxstyleemphasis{Citation:} “Reconstruction and Africa” Editorial. 1919. \sphinxstyleemphasis{The Crisis}. 17(4): 165\sphinxhyphen{}166.


\subsection{Race Pride (1920)}
\label{\detokenize{Volumes/19/03/race_pride:race-pride-1920}}\label{\detokenize{Volumes/19/03/race_pride::doc}}
\begin{sphinxShadowBox}
\sphinxstylesidebartitle{}

\sphinxAtStartPar
The Amazing Major is a reference to \sphinxhref{https://en.wikipedia.org/wiki/Robert\_Russa\_Moton}{Robert Moton}, a close ally of Booker T. Washington, and his successor as head of the  Tuskegee Institute.
\end{sphinxShadowBox}

\sphinxAtStartPar
Our friends are hard—very hard—to please. Only yesterday they were preaching fm! “Race Pride.” “Go to!” they said, and be PROUD of your race. If we hesitated or sought to explain— “Away,” they yelled; “Ashamed\sphinxhyphen{}of Yourself and Want\sphinxhyphen{}to\sphinxhyphen{}be\sphinxhyphen{}White!” Of course, the Amazing Major is still at it, but do you notice that others say less,—because they see that bullheaded worship of any “race,” as such, may lead and does lead to curious complications?

\sphinxAtStartPar
For instance: Today Negroes, Indians, Chinese, and other groups, are gaining new faith in themselves; they are beginning to “like” themselves; they are discovering that the current theories and stories of “backward” peoples are largely lies and assumptions; that human genius and possibility are not limited by color, race, or blood. What is this new self\sphinxhyphen{}consciousness leading to? Inevitably and directly to distrust and hatred of whites; to demands for self\sphinxhyphen{}government, separation, driving out of foreigners,—“Asia for the Asiatics,” “Africa for the Africans,” and “Negro officers for Negro troops!”

\sphinxAtStartPar
No sooner do whites see this unawaited development than they point out in dismay the inevitable consequences: “You lose our tutelage,” “You spurn our knowledge,” “You need our wealth and technique.” They point out how fine is the world role of Elder Brother.

\sphinxAtStartPar
Very well. Some of the darker brethren are convinced. They draw near in friendship; they seek to enter schools and churches; they would mingle in industry,—when lo! “Get out,” yells the White World—“You’re not our brothers and never will be’”’— “Go away, herd by yourselves’— “Eternal Segregation in the Lord!”

\sphinxAtStartPar
Can you wonder, Sirs, that we are a bit puzzled by all this and that we are asking gently, but more and more insistently: Choose one or the other horn of the dilemma:
\begin{enumerate}
\sphinxsetlistlabels{\arabic}{enumi}{enumii}{}{.}%
\item {} 
\sphinxAtStartPar
Leave the black and yellow world alone. Get out of Asia, Africa, and the Isles. Give us our states and towns and sections and let us rule them undisturbed. Absolutely segregate the races and sections of the world.

\item {} 
\sphinxAtStartPar
Let the world meet as men with men. Give utter Justice to all. Extend Democracy to all and treat all men according to their individual desert. Let it be possible for whites to rise to the highest positions in China and Uganda and blacks to the highest honors in England and Texas.

\end{enumerate}

\sphinxAtStartPar
Here is the choice. Which will you have, my masters?


\bigskip\hrule\bigskip


\sphinxAtStartPar
\sphinxstyleemphasis{Citation:} Du Bois, W.E.B. 1920. “Race Pride.” \sphinxstyleemphasis{The Crisis}. 19(3): 107.


\subsection{The Firing Line (1925)}
\label{\detokenize{Volumes/31/02/firing_line:the-firing-line-1925}}\label{\detokenize{Volumes/31/02/firing_line::doc}}
\sphinxAtStartPar
In the fight for human rights across the color bar where is the firing line? In the United States, in the West Indies, or in Africa?

\sphinxAtStartPar
Many sincere persons seem to think that this line is in Africa. This is not so. It may be so one hundred years from now but today Europe has so manipulated matters that the fight for rights in Africa is exceedingly difficult. The European countries with the army, the navy and the police in their hands; with their domination of the courts; with their ownership of capital and commerce and all lines of communication make any effort at uplift and reform a matter of disloyalty and rebellion which can bring summary punishment of the most violent sort. Free speech is impossible. There is no right to vote in the larger part of Africa. Consequently the fighting line\sphinxhyphen{}in Africa reduces itself to appeals to far\sphinxhyphen{}off powers who may perhaps never hear the appealing voices. Or it is a matter of revolt with every chance in favor of the oppressor. No, the firing line today is not in Africa.

\sphinxAtStartPar
Is it in the West Indies? Again, no. In the West Indies we have an overwhelming majority of Negroes and a large number of Negro leaders of wealth and education, but the ownership of land and capital is predominantly in the hands of Europeans while the political power of the Negro is so small that the fight is unequal; and again, as in Africa, the West Indies are a long way from the centers of power in the world.

\sphinxAtStartPar
The real fighting line for the Negro, then, is in the United States. Here he faces his foe. Here he can use unmolested the modern weapons of writing, talking and voting. He can make the world listen because he is right in the world. With his millions of votes no presidential candidate, few congressmen and fewer senators dare altogether to ignore his demands. They may play upon his ignorance and prejudices but as he becomes intelligent, sincere and clear\sphinxhyphen{}headed his phalanx marches forward. America, then, is today the firing line of the Negro.


\bigskip\hrule\bigskip


\sphinxAtStartPar
\sphinxstyleemphasis{Citation:} Du Bois, W.E.B. 1925. “The Firing Line.”  31(2):62.


\subsection{Listen, Japan and China (1933)}
\label{\detokenize{Volumes/40/01/listen_japan_and_china:listen-japan-and-china-1933}}\label{\detokenize{Volumes/40/01/listen_japan_and_china::doc}}
\sphinxAtStartPar
Colossi of Asia and leaders of all colored mankind: for God’s sake stop fighting and get together. Compose your quarrels on any reasonable basis. Unite in self\sphinxhyphen{}defense and assume that leadership of distracted mankind to which your four hundred millions of people entitle you.

\sphinxAtStartPar
Listen to a word from twelve little black millions who live in the midst of western culture and know it: the intervention of the League of Nations bodes ill for you and all colored folk. There are philanthropists and reformers in Europe and America genuinely interested in all mankind. But they do not rule, neither in England nor France, not in Germany nor in America. The real rulers of the world today, who stand back of Stimson, Macdonald and Herriot, are blood\sphinxhyphen{}sucking, imperial tyrants who see only one thing in the quarrel of China and Japan and that is a chance to crush and exploit both. Nothing has given them more ghoulish glee than the blood and smoke of Shanghai and Manchuria or led them to rub hands with more solemn unction and practised hypocrisy.

\sphinxAtStartPar
Unmask them, Asia; tear apart their double faces and double tongues and unite in peace. Remember Japan, that white America despises and fears you. Remember China, that England covets your land and labor. Unite! Beckon to the three hundred million Indians; drive Europe out of Asia and let her get her own raped and distracted house in order. Let the yellow and brown race, nine hundred million strong take their rightful leadership of mankind. Let the young Chinese and Japanese students and merchants of America and Europe cease debate and recrimination while gleeful whites egg them on. Get together and wire word to Asia. Get together China and Japan, cease quarrelling and fighting! Arise and lead! The world needs Asia.


\bigskip\hrule\bigskip


\sphinxAtStartPar
\sphinxstyleemphasis{Citation:} Du Bois, W.E.B. 1933. “Listen, Japan and China.”  40(1):20.


\section{Miscellaneous}
\label{\detokenize{Sections/misc:miscellaneous}}\label{\detokenize{Sections/misc::doc}}
\sphinxAtStartPar
Editorials on other topics.
\begin{itemize}
\item {} 
\sphinxAtStartPar
{\hyperref[\detokenize{Volumes/03/05/lee::doc}]{\sphinxcrossref{Lee (1912)}}}

\item {} 
\sphinxAtStartPar
{\hyperref[\detokenize{Volumes/09/02/negro::doc}]{\sphinxcrossref{Negro (1914)}}}

\item {} 
\sphinxAtStartPar
{\hyperref[\detokenize{Volumes/06/07/national_emancipation_exposition::doc}]{\sphinxcrossref{The National Emancipation Exposition (1913)}}}

\item {} 
\sphinxAtStartPar
{\hyperref[\detokenize{Volumes/11/02/star_of_ethiopia::doc}]{\sphinxcrossref{The Star of Ethiopia (1915)}}}

\item {} 
\sphinxAtStartPar
{\hyperref[\detokenize{Volumes/18/03/reconstruction::doc}]{\sphinxcrossref{Reconstruction (1919)}}}

\item {} 
\sphinxAtStartPar
{\hyperref[\detokenize{Volumes/19/01/social_equity::doc}]{\sphinxcrossref{Social Equity (1919)}}}

\item {} 
\sphinxAtStartPar
{\hyperref[\detokenize{Volumes/34/02/farmers::doc}]{\sphinxcrossref{Farmers (1927)}}}

\item {} 
\sphinxAtStartPar
{\hyperref[\detokenize{Volumes/39/01/john_brown::doc}]{\sphinxcrossref{John Brown (1932)}}}

\end{itemize}


\subsection{Lee (1912)}
\label{\detokenize{Volumes/03/05/lee:lee-1912}}\label{\detokenize{Volumes/03/05/lee::doc}}
\sphinxAtStartPar
In a recent review of Mr. Thomas Nelson Page’s life of Robert E. Lee in the New York Times we find the following sentence: “Of all the figures in history, it is he (Lee) who most nearly approaches Washington; in fact, there is little or nothing to choose between them except the fact that Lee failed.”

\sphinxAtStartPar
This statement is worth noting because it expresses a sentiment not uncommon to\sphinxhyphen{}day. Here are two gen­erals, both well born, scrupulously honorable, brave and efficient. The only difference between them is that one was victorious, while the other was obliged in the end to surrender. One won, the other lost; that is all.

\sphinxAtStartPar
In Memorial Hall, at Harvard University, are the names of the college graduates who fell in battle for the cause of the Union. Again and again has come the demand that with these names there be placed the names of the graduates who fell defending the Confederacy. Both were brave youths, the argument goes, both fought unselfishly. Why not give honor to both, since they only differed in that one lost and the other won in battle?

\sphinxAtStartPar
Now, what is the significance of this doctrine which many Americans believe should be preached in literature and history and on the walls of a noble building erected in ‘memory of the heroic dead? ‘This, that if the youths who go forth into the world, fight honorably; if they bear defeat bravely, it makes no difference what side they take in the battle. They may fight for the right of the individual to control the natural resources of the earth, to destroy the forests, to impoverish the land, or they may fight for the conservation of such resources; no matter, so that they fight well. Like Washington, their choice may be to lead the army of republicanism, or like Lee, they may choose to lead the aristocracy to battle for the right of one man to hold another as his chattel; the wisdom of their choice is of no importance; “there is little or nothing to choose between the two;” both are singularly alike, both are worthy of equal praise.

\sphinxAtStartPar
At this time of year, when we celebrate the birthdays of our two most famous Americans, let us denounce this philosophy in no uncertain terms. The choice that a man makes is his life. The present crisis faces every youthful spirit, and life for him is a failure or a success as he chooses “the good or evil side;” the side of spiritual, human progress, or the side of material, brutal enslavement. No sentiment can keep alive for long the names of those, however honorable, who chose to fight with the forces of darkness. If they live, they live in opprobrium. Washington lives and Lincoln lives because each, at the crisis of his life, chose the side of progress and civilization. Lincoln saw the “irrepressible conflict” and stood for freedom; otherwise he would be as great a nonentity as his rival, Stephen A. Douglas. Washington lives because he believed that taxation without representation was tyranny; otherwise he would have been forgotten like—but who remembers the name of one of the gentlemen who drew their swords for King George?


\bigskip\hrule\bigskip


\sphinxAtStartPar
\sphinxstyleemphasis{Citation:} Du Bois, W.E.B. 1912. “Lee” \sphinxstyleemphasis{The Crisis}. 3(5): 200\sphinxhyphen{}201.


\subsection{Negro (1914)}
\label{\detokenize{Volumes/09/02/negro:negro-1914}}\label{\detokenize{Volumes/09/02/negro::doc}}
\sphinxAtStartPar
There are indications that the custom of extending courtesy to 150,000,000 of human beings by capitalizing the racial name which is most in use, is slowly increasing. The manager of the  Associated Press writes us:
\begin{quote}

\sphinxAtStartPar
“We have a broad rule to the effect that the word ‘Negro’ should be capitalized in our service, but we do not control the typographical appearance of the word as it appears in the newspapers.
\end{quote}
\begin{quote}

\sphinxAtStartPar
“A little more than a year ago we sent out 900 copies of a letter from Mr. Lester A. Walton, of the Age, to the newspapers in the Associated Press, and I think the practice of capitalizing the word Negro is very general so far as I have been able to observe.”
\end{quote}


\bigskip\hrule\bigskip


\sphinxAtStartPar
“Negro.” Editorial. 1914. \sphinxstyleemphasis{The Crisis} 9(2): 28\sphinxhyphen{}30.


\subsection{The National Emancipation Exposition (1913)}
\label{\detokenize{Volumes/06/07/national_emancipation_exposition:the-national-emancipation-exposition-1913}}\label{\detokenize{Volumes/06/07/national_emancipation_exposition::doc}}
\sphinxAtStartPar
\sphinxstylestrong{In New York City, October 22\sphinxhyphen{}31, 1913}

\sphinxAtStartPar
\sphinxstyleemphasis{The pageant of Negro history as written by W. E. B. Du Bois and produced by Charles Burroughs, Master; Daisy Tapley, Dora Cole Norman, Marie Stuart Jackson, Augustus G. Dill and 350 others, during the exhibition, and entitled “The People of Peoples and Their Gifts to Men.”}


\subsubsection{Prelude}
\label{\detokenize{Volumes/06/07/national_emancipation_exposition:prelude}}
\sphinxAtStartPar
The lights of the Court of Freedom blaze. A trumpet blast is heard and four heralds, black and of gigantic stature, appear with silver trumpets and standing at the four corners of the temple of beauty cry:
\begin{quote}

\sphinxAtStartPar
“Hear ye, hear ye! Men of all the Americas, and listen to the tale of the eldest and strongest of the races of mankind, whose faces be black. Hear ye, hear ye, of the gifts of black men to this world, the Iron Gift and Gift of Faith, the Pain of Humility and the Sorrow Song of Pain, the Gift of Freedom and of Laughter, and the undying Gift of Hope. Men of the world, keep silence and hear ye this!”
\end{quote}

\sphinxAtStartPar
Four banner bearers come forward and stand along the four walls of the temple. On their banners is written:
\begin{quote}

\sphinxAtStartPar
“The First Gift of the Negro to the world, being the Gift of Iron. This picture shall tell how, in the deep and beast\sphinxhyphen{}bred forests of Africa, mankind first learned the welding of iron, and thus defense against the living and the dead.”
\end{quote}

\sphinxAtStartPar
What the banners tell the heralds solemnly proclaim.

\sphinxAtStartPar
Whereat comes the:


\subsubsection{First Episode. The Gift of Iron:}
\label{\detokenize{Volumes/06/07/national_emancipation_exposition:first-episode-the-gift-of-iron}}
\sphinxAtStartPar
The lights grow dim. The roar of beasts is heard and the crash of the storm. Lightnings flash. The dark figure of an African savage hurries across the foreground, frightened and cowering and \_ dancing. Another follows defying the lightning and is struck down; others come until the space is filled with 100 huddling, crowding savages. Some brave the storm, some pray their Gods with incantation and imploring dance. Mothers shield their children, and husbands their wives. At last, dimly enhaloed in mysterious light, the Veiled Woman appears, commanding in stature and splendid in garment, her dark face faintly visible, and in her right hand Fire, and Iron in her left. As she passes slowly round the Court the rhythmic roll of tomtoms begins. Then music is heard; anvils ring at the four corners. The arts flourish, huts arise, beasts are brought in and there is joy, feasting and dancing.

\sphinxAtStartPar
A trumpet blast calls silence and the heralds proclaim


\subsubsection{The Second Episode, saying:}
\label{\detokenize{Volumes/06/07/national_emancipation_exposition:the-second-episode-saying}}\begin{quote}

\sphinxAtStartPar
“Hear ye, hear ye! All them that come to know the Truth, and listen to the tale of the wisest and gentlest of the races of men whose faces be black. Hear ye, hear ye, of the Second Gift of black men to this world, the Gift of Civilization in the dark and splendid valley of the Nile. Men of the world, keep silence and hear ye this.”
\end{quote}

\sphinxAtStartPar
The banners of the banner bearers change and read:
\begin{quote}

\sphinxAtStartPar
“The Second Gift of the Negro to the world, being the Gift of the Nile. This picture tells how the meeting of Negro and Semite in ancient days made the civilization of Egypt the first in the world.”
\end{quote}

\sphinxAtStartPar
There comes a strain of mighty music, dim in the distance and drawing nearer. The 100 savages thronged round the whole Court rise and stand listening. Slowly there come fifty veiled figures and with them come the Sphinx, Pyramid, the Obelisk and the empty Throne of the Pharaoh drawn by oxen. As the cavalcade passes, the savages, wondering, threatening, inquiring, file by it. Suddenly a black chieftain appears in the entrance, with the Uraeus in one hand and the winged Beetle in the other. The Egyptians unveil and display Negroes and mulattoes clothed in the splendor of the Egyptian Court. The savages salaam; all greet him as Ra, the Negro. He mounts the throne and the cavalcade, led by posturing dancers and Ra, and followed by Egyptians and savages, pass in procession around to the right to the thunder of music and tomtoms. As they pass, Ra is crowned as Priest and King. While the Queen of Sheba and Candace of Ethiopia join the procession at intervals.

\sphinxAtStartPar
Slowly all pass out save fifty savages, who linger examining their gifts. The lights grow dim as Egyptian culture dies and the fifty savages compose themselves to sleep. As they sleep the light returns and the heralds proclaim


\subsubsection{The Third Episode, saying:}
\label{\detokenize{Volumes/06/07/national_emancipation_exposition:the-third-episode-saying}}\begin{quote}

\sphinxAtStartPar
“Hear ye, hear ye! All them that come to see the light and listen to the tale of the bravest and truest of the races of men, whose faces be black. Hear ye, hear ye, of the Third Gift of black men to this world—a Gift of Faith in Righteousness hoped for but unknown; men of the world, keep silence and hear ye this!”
\end{quote}

\sphinxAtStartPar
The banners change and read:

\sphinxAtStartPar
“The Third Gift of the Negro to the world, being a Gift of Faith. This episode tells how the Negro race spread the faith of Mohammed over half the world and built a new culture thereon.”

\sphinxAtStartPar
There is a sound of battle. The savages leap to their feet. Mohammed and fifty followers whirl in and rushing to the right beat the savages back. Fifty Songhay enter and attack the Mohammedans. Fifty other Mohammedans enter and attack the Songhay. Turning, the Songhay bear the last group of Mohammedans back to the left where they clash with the savages. Mohammedan priests strive and exhort among the warriors. At each of the four corners of the temple a priest falls on his face and cries: “God is God! God is God! There is no God but God, and Mohammed is his prophet!” Four more join, others join until gradually all is changed from battle to the one universal cry: “God is God! God is God! There is no God but God, and Mohammed is his prophet!” In each corner, however, some Mohammedans hold slaves in shackles, secretly.

\sphinxAtStartPar
Mansa Musa appears at the entrance with entourage on horseback, followed by black Mohammedan priests and scholars. The procession passes around to the right with musie and dancing, and passes out with Mohammedans and Songhay, leaving some Mohammedans and their slaves on the stage.

\sphinxAtStartPar
The herald proclaims


\subsubsection{The Fourth Episode, saying:}
\label{\detokenize{Volumes/06/07/national_emancipation_exposition:the-fourth-episode-saying}}\begin{quote}

\sphinxAtStartPar
“Hear ye, hear ye! All them that know the sorrow of the world. Hear ye, hear ye. and listen to the tale of the humblest and the mightiest of the races of men whose faces he black. Wear ye, hear ye, and learn how this race did suffer of Pain, of Death and Slavery and yet of this Humiliation did not die. Men of the world, keep silence and hear ye this!” The banners change again and say:
\end{quote}
\begin{quote}

\sphinxAtStartPar
“The Fourth Gift of the Negro to the world, being a Gift of Humiliation. This gift shows how men can bear even the Hell of Christian slavery and live.”
\end{quote}

\sphinxAtStartPar
The Mohammedans force their slaves forward as European traders enter. Other Negroes, with captives, enter. The Mohammedans take gold in barter. The Negroes refuse gold, but are seduced by beads and drink. Chains rattle. Christian missionaries enter, but the slave trade increases. The wail of the missionary grows fainter and fainter until all is a scene of carnage and captivity with whip and chain and only a frantic priest, staggering beneath a cross and crowned with bloody thorns, wanders to and fro in dumb despair.

\sphinxAtStartPar
There is silence. Then a confused moaning. Out of the moaning comes the slave song, “Nobody Knows the Trouble I’ve Seen,” and with it and through the chained and bowed forms of the slaves as they pass out is done the Dance of Death and Pain.

\sphinxAtStartPar
The stage is cleared of all its folk. There is a pause, in which comes the Dance of the Ocean, showing the transplantation of the Negro race over seas.

\sphinxAtStartPar
Then the heralds proclaim


\subsubsection{The Fifth Episode, saying:}
\label{\detokenize{Volumes/06/07/national_emancipation_exposition:the-fifth-episode-saying}}\begin{quote}

\sphinxAtStartPar
“Hear ye, hear ye! All them that strive and struggle. Hear ve, hear ye, and listen to the tale of the stoutest and the sturdiest of the races of men whose faces be black. Hear ye, hear ye, and learn how this race did rise out of slavery and the valley of the shadow of death. Men of the world, keep silence and hear ye this!”
\end{quote}

\sphinxAtStartPar
The banners change again and read:

\sphinxAtStartPar
“The Fifth Gift of the Negro to the world, being a Gift of Struggle Toward Freedom. This picture tells of Alonzo, the Negro pilot of Columbus, of Stephen Dorantes who discovered New Mexico, of the brave Maroons and valiant Haytians, of Crispus Attucks, George Lisle and Nat Turner.”

\sphinxAtStartPar
Twenty\sphinxhyphen{}five Indians enter, circling the Court right and left, stealthily and watchfully. As they sense the coming of the whites. they gather one side of the temple, watching.

\sphinxAtStartPar
Alonzo, the Negro, enters and after him Columbus and Spaniards, in mail, and one monk. They halt the other side of the temple and look about searchingly, pointing at the Indians. Slaves follow. One of the slaves, Stephen Dorantes, and the monk seek the Indians. The monk is killed and Stephen returns, circling the Court, tells his tale and dies. The Spaniards march on the Indians. Their slaves—the Maroons—revolt and march to the left and meet the Indians on the opposite side. The French, some of the mulattoes and Negroes, enter with more slaves. They march after the Spanish. Their slaves, helped by mulattoes and Toussaint, revolt and start back. The French follow the Spaniards, but the returning Haytians meet oncoming British. The Haytians fight their way through and take their place next to the Maroons. Still more slaves and white Americans follow the British. The British and Americans dispute. Attucks leads the Americans and the British are put to flight. Spanish, French and British, separated by dancing Indians, file around the Court and out, while Maroons, Haytians and slaves file around in the opposite direction and meet the Americans. As they pass the French, by guile induce Toussaint to go with them. There is a period of hesitation. Some slaves are freed, some Haytians resist aggression. George Lisle, a freed Negro, preaches the true religion as the masters listen. Peace ensues and the slaves sing at their tasks. Suddenly King Cotton arrives, followed by Greed, Vice, Luxury and Cruelty. The slaveholders are seduced. The old whips and chains appear. Nat Turner rebels and is killed. The slaves drop into despair and work silently and sullenly. The faint roll of tomtoms is heard.

\sphinxAtStartPar
The heralds proclaim


\subsubsection{The Sixth Episode, saying:}
\label{\detokenize{Volumes/06/07/national_emancipation_exposition:the-sixth-episode-saying}}\begin{quote}

\sphinxAtStartPar
“Hear ye, hear ye! Citizens of New York, and learn of the deeds of eldest and strongest of the races of men whose faces be black. Hear ye, hear ye, of the Sixth and Greatest Gift of black men to the world, the Gift of Freedom for the workers. Men of New York, keep silence and hear ye this.” The banners change and say:
\end{quote}
\begin{quote}

\sphinxAtStartPar
“The sixth and last episode, showing how the freedom of black slaves meant freedom for the world. In this episode shall be seen the work of Garrison and John Brown; of Abraham Lincoln and Frederick Douglass,  the marching of black soldiers to war and the hope that lies in little children.”
\end{quote}

\sphinxAtStartPar
The slaves work more and more dejectedly and drivers force them. Slave music comes. The tomtoms grow louder. The Veiled Woman appears with fire and iron. The slaves arise and begin to escape, passing through each other to and fro, confusedly. Benezet, Walker and Garrison enter, scattering their writings, and pass slowly to the right, threatened by slave drivers. Brown enters, gesticulating, A knot of Negroes follow him. The planters seize him and erect a gallows, but the slaves seize his body and begin singing Body.”

\sphinxAtStartPar
Frederick Douglass enters and passes to the right. Sojourner Truth enters and passes to the left. Sojourner Truth cries: “Frederick, is God dead?” Voices take up the ery, repeating: “Frederick, is God dead?” Douglass answers: “No, and therefore slavery must end in blood.” The heralds repeat: “Slavery must end in blood.”

\sphinxAtStartPar
The roll of drums is heard and the soldiers enter. First, a company in blue with Colonel Shaw on horseback.

\sphinxAtStartPar
A single voice sings “O Freedom.” A soprano chorus takes it up.

\sphinxAtStartPar
The Boy Scouts march in.

\sphinxAtStartPar
Full brasses take up “O Freedom.”

\sphinxAtStartPar
Little children enter, and among them symbolic figures of the Laborer, the Artisan, the Servant of Men, the Merchant, the Inventor, the Musician, the Actor, the Teacher, Law, Medicine and Ministry, the All\sphinxhyphen{}Mother, formerly the Veiled Woman, now unveiled in her chariot with her dancing brood, and the bust of Lincoln at her side.

\sphinxAtStartPar
With burst of music and blast of trumpets, the pageant ends and the heralds sing:
\begin{quote}

\sphinxAtStartPar
“Hear ye, hear ye, men of all the Americas, ye who have listened to the tale of the eldest and strongest of the races of mankind, whose faces be black. Hear ye, hear ye, and forget not the gift of black men to this world— the Iron Gift and Gift of Faith, the Pain of Humility and Sorrow Song of Pain, the Gift of Freedom and Laughter and the undying Gift of Hope. Men of America, break silence, for the play is done.”
\end{quote}

\sphinxAtStartPar
Then shall the banners announce:

\sphinxAtStartPar
“The play is done!”

\sphinxAtStartPar
\sphinxstyleemphasis{Citation:} Du Bois, W.E.B. 1913. “The National Emancipation Exposition.”  6(7):339\sphinxhyphen{}341.


\subsection{The Star of Ethiopia (1915)}
\label{\detokenize{Volumes/11/02/star_of_ethiopia:the-star-of-ethiopia-1915}}\label{\detokenize{Volumes/11/02/star_of_ethiopia::doc}}
\sphinxAtStartPar
I wrote it four years ago. I called it, or rather it called itself by various names. Finally it decided to be called “The Star of Ethiopia.” I acquiesced. My friends looked upon it with lack\sphinxhyphen{}luster eye. I did not blame them. The more I read it myself the more it seemed wanting: And yet as | turned over my rejected manuscripts this always bobbed up with a certain insistance, a kind of “Why not try me” tone. Its first resurrection came in 1913 when we were celebrating, in New York, Emancipation.

\sphinxAtStartPar
What a task that was! I have been through a good many laborious jobs and had to bear on many occasions accusations difficult to rest under, but without doubt the New York Emancipation Exposition was the worst of all my experiences. Such an avalanche of altogether unmerited and absurd attacks it had never been my fortune to experience. Yet through it all one thing became clearer—the Pageant must be tried. We must attempt, at least, this one new thing in the dead level of uninteresting exhibitions. We had our ups and downs with it. It was difficult to get hold of the people; it was more difficult to keep them. There were curious little wranglings and bickerings and none of us will forget that dress rehearsal. If there had been any reasonably convenient method of escape the Pageant would never have seen the light of day.

\sphinxAtStartPar
And then it came—four exhibitions, singular in their striking beauty, and above all in the grip they took upon men. Literally, thousands besieged our doors and the sight of the thing continually made the tears arise. After these audiences aggregating 14,000, I said: the Pageant is the thing. This is what the people want and long for. This is the gown and paraphernalia in which the message of education and reasonable race pride can deck itself. And yet a year went by and another year. I had upon me somehow the fear of doing. When one has done a thing and done it fairly well it seems like tempting fate to try it again. And then, after all, there was no money. There was war and there was trouble. So another year passed, and then with a start I felt that I was letting something go that was really worth while—letting it die because of certain moral cowardice and so I girded myself and looked around the world and hit upon Washington at once the most promising and the most difficult of places; the largest Negro city in the northern hemisphere and yet for that very reason exceedingly hard to reach and interest with a new and untried thing.

\sphinxAtStartPar
The few that came out to the preliminary meetings listened with interest. They even showed enthusiasm at times and they promised cooperation and they kept their word. I had the whole matter carefully planned but I early found that the best laid plans needed a curious personal joining together on the part of the one who has the vision of what he wants and cannot adequately tell it.

\sphinxAtStartPar
I remember that Friday when I opened the office. The beautiful sign which Richard Brown had painted was going up and I was helping boost it. A good friend came by and looked at me. There was pity way down in his soul. “Are you still encouraged?” he asked. His question was a revelation. The whole city doubted. A thousand actors, gowned and trained in three weeks? “Impossible!” said the city. Nevertheless they worked loyally.

\sphinxAtStartPar
My helpers came to town on the dot. Mr. Burroughs, with his deliberate genius and unwavering faith; Mrs. Norman with her great mastery of men—or rather of women—and Mrs. Curtis who came almost concealed under thousands of yards of cloth. Already in the field was the tireless Mrs. Glenn and eventually came Mr. Davidson. Richard Brown had long been working. So we started and our troubles started with us.

\sphinxAtStartPar
The School Board was slow but when it moved it moved generously. The rehearsals hitched and for interminable days we had a hundred people where we needed a thousand. The music went awry and it seemed an absolutely impossible task properly to advertise the thing we were doing and get the mass of the people to know and to think about it. And then the weather—oh! the weather! The pains of hell gat hold upon me and the terrors of the damned during that week of cold, drizzly weather that preceded the week of our entertainment. I was at the very nadir of my faith, and wondered rather helplessly why it was I could never let well enough alone, but was always pushing out into some frontier of wilderness endeavoring to do the impossible. Then (by what miracle who knows), came three nights, each more perfect than the other and sandwiched in between rains. Some said God did it and I am not disposed to dispute.

\sphinxAtStartPar
But the money. The way the funds rained through my fingers was quite unbelievable. I saw ahead of me the most shocking bankruptcy, the most unbelievable extravagance. Everything seemed to be costing just twice as much as it should. Final hitches came on the very end of our endeavor. The music looked dubious; the regulations of the ball field seemed about to interfere with even our walking firmly on tha grass; the electric lighting got into inextricable tangles. The tickets were late. The costumes were later and the properties latest of all. New expenses kept cropping up, and finally, when the ciyy inspector dropped around very late one evening and casually remarked that he might have to hold up the whole thing because a few red lights were missing, it seemed as though the cup of misfortune was full.

\sphinxAtStartPar
Then it was as it always is in things of this sort. Suddenly a great new spirit seemed born. The Thing that you have exorcized becomes a living, mighty, moving spirit. It sweeps on and you hang trembling to its skirts. Nothing can stop it. It is. It will. Wonderfully, irresistably the dream comes true. You feel no exaltation, you feel no personal merit. It is not yours. It is its own. You have simply called it, and it comes.

\sphinxAtStartPar
I shall never forget that last night. Six thousand human faces looked down from the shifting blaze of lights and on the field the shimmering streams of colors came and went, silently, miraculously save for the great cloud of music. that hovered over them and enveloped them. It was no mere picture: it was reality.

\sphinxAtStartPar
The Herald cried, “People of Washington arise, and go. The Play is done.” And yet the play was not done. Some things are quite too beautiful ever to be finished. So I walked home alone and
the joys of God.

\sphinxAtStartPar
\sphinxstyleemphasis{Citation:} Du Bois, W.E.B. 1927. “The Star of Ethopia.” 11(2):91\sphinxhyphen{}93.


\subsection{Reconstruction (1919)}
\label{\detokenize{Volumes/18/03/reconstruction:reconstruction-1919}}\label{\detokenize{Volumes/18/03/reconstruction::doc}}
\sphinxAtStartPar
This is a program of reconstruction within the Negro race in America, after the the revolution of world war. In \sphinxstyleemphasis{Education} we must take up the problem of the colored child in the white school. At present the tendency is to accept and even demand separate schools because our children so often are neglected, mistreated and humiliated in the public schools. This is a dangerous and inadvisable alternative and a wicked surrender of principle for which our descendants will pay dearly. Our policy should be to form in connection with each school and district effective Parents’ Associations, composed of the fathers, mothers and friends of colored pupils; these associations should establish friendly relations with teachers and school authorities, urge parents to wash and dress their children properly, help look after truancy and poverty, arrange for home work and tuition for the backward, curb delinquency and be, in fine, a vigilance committee to keep the public school open to all and fit the Negro child for it.

\sphinxAtStartPar
In \sphinxstyleemphasis{Religion} we must, in the larger cities, stop building and purchasing new church edifices and begin to invest the money of the church in homes, land and business, and philanthropic enterprises for the benefit of the people. Individual home ownership in most large cities is today difficult; but a group of people who can buy and pay for a hundred thousand dollar church can purchase a hundred thousand dollar apartment house and run it. It is a simple business proposition and requires only elementary honesty and ordinary executive ability. Churches can easily begin co\sphinxhyphen{}operative buying of coal, bread and meat, using their own premises for distribution; churches in the country and small towns can buy farms and rent or run them; the church can purchase automobile trucks and help the Negro farmer market his produce independent of the railroads and thieving commission merchants; even simple manufacturing, sewing and building are not beyond the reasonable activities of church bodies. Indeed, unless the church extends its economic functions beyond the simple program of building bigger and finer edifices—unless it organizes the Negro laborer so that his entire wage will not go in rent and supporting storekeepers who despise and cheat him—unless it thus helps the laborer, it will lose the laborer. The hope of the Negro church is character\sphinxhyphen{}building through economic co\sphinxhyphen{}operation.

\sphinxAtStartPar
In \sphinxstyleemphasis{Business} the Negro must branch out into certain new lines where he has long and foolishly hesitated: We must open drygoods and haberdashery shops, meat markets and clothing stores, shoe stores and hat stores. We must gradually but persistently get into manufacturing. The deft fingers of our young people are as easily adapted to machinery as the fingers of whites. We are denied opportunity by white trade unions and by lack of pioneering courage among colored capitalists and business men. Let us wake up. The era of manufactures in the United States is just begun. The expansion of domestic and foreign trade is going to be enormous. We raise the cotton—why not spin and weave it? We dig the iron —why not weld it? We mine the coal—why not turn it to steam and power? Do we lack brains and capital? No, we lack experience and courage. Get them.

\sphinxAtStartPar
In \sphinxstyleemphasis{Politics} the colored woman is going to vote. This is our chance. Away with the old regime, the pothouse politician and white bribery. Let us form clubs and study government in city, county, state and nation. Let us know the law and the officials and their duties. Let us keep continual and rigid tabs on every candidate. Away with parties—what we want is men. Away with promises—what we want is deeds. Study, learn, register and vote. Vote at every election and see that every friend of yours votes. Pay your poll taxes and register. Do not vote for a party. Never vote a straight ticket. Vote for men and measures—not for parties. But above all, vote! Let every Negro man and woman, always and everywhere, vote.


\bigskip\hrule\bigskip


\sphinxAtStartPar
\sphinxstyleemphasis{Citation:} Du Bois, W.E.B. 1919. “Reconstruction.” \sphinxstyleemphasis{The Crisis}. 18(3): 130\sphinxhyphen{}131.


\subsection{Social Equity (1919)}
\label{\detokenize{Volumes/19/01/social_equity:social-equity-1919}}\label{\detokenize{Volumes/19/01/social_equity::doc}}
\sphinxAtStartPar
Every time the American Negro seeks reasonably and earnestly to bring his case before the white South and the nation the bourbons proceed to throw dust in the eyes of the public by screaming frantically, “Social Equality.”

\sphinxAtStartPar
That bogey can be easily met: If “Social Equality” means the right to vote, the abolition of “Jim\sphinxhyphen{}Crow” cars, the stoppage of lynching, universal education and civil rights, then social equality is exactly what we want and what eventually we will and must have.

\sphinxAtStartPar
If, on the other hand, “Social Equality” involves the denial of the social right of any individual of any race or color to choose his own marital mate, his own friends and his own dinner companions—in fine, to be master of his own home, then no sane person ever dreamed of demanding the slightest interference with such an obvious right, and any one who accuses Negroes of such a demand writes himself down as an ass or a deliberate liar.


\bigskip\hrule\bigskip


\sphinxAtStartPar
\sphinxstyleemphasis{Citation:} Du Bois, W.E.B. 1919. “Social Equity.” \sphinxstyleemphasis{The Crisis}. 19(1): 337.


\subsection{Farmers (1927)}
\label{\detokenize{Volumes/34/02/farmers:farmers-1927}}\label{\detokenize{Volumes/34/02/farmers::doc}}
\sphinxAtStartPar
Let us understand this agitation for farm relief. We Negroes who take our thinking from provincial New York and who in darkest Mississippi and environs are dumb and blind perforce to the great movements of the world are being led to think that the Western demand for the protection and financial relief of farmers is nonsense— the Bloc devil raised against the Great God Party. But wait: remember that the tillers of the soil have always been exploited; always the dumb, driven cattle who filled the fat bellies of the rich merchants, kept the artisans from starvation and starved themselves. Well we the black peasants of sugar, rice, tobacco, corn and cotton know that. Why are we poor? Because the things we raise are filched from us for a song or a kick and sold at prices which enrich everybody but the farmer. When as in Germany and Austria great Agrarian parties have arisen to wresk a part of the loot from Commerce and Industry, it has been the land monopolists and not the ragged peasants who have gained. But because of the land policy of the United States in the West during the 19th Century a New Farmer has arisen: a farmer who owns his one hundred sixty acres, can read and write and think and who knows that in the past every human endeavor has received government aid except farming; toward farming we have the primitive psychology which is determined to exploit to the limit the man who raises food and clothes for the sake of the man who eats and dresses. “No”, says the Western American farmer. ‘This profit which makes millionaires of manufacturers and bankers is hereafter going to be divided in some part with us who feed and shelter them. Stand by us, Labor, and don’t let them subtract our just share from your all too low wages.”

\sphinxAtStartPar
The only rift in the lute is that the Southern Farmer cannot stand shoulder to shoulder with the West, for he is a land monopolist and slave driver whose history makes him the tool of Commerce and Industry.

\sphinxAtStartPar
But let the black peasants and laborers lift up their ears to the meaning of the Farm Bloc and the McNary\sphinxhyphen{}Haugen Bill.


\bigskip\hrule\bigskip


\sphinxAtStartPar
\sphinxstyleemphasis{Citation:} Du Bois, W.E.B. 1927. “Famers.”  34(2):69.


\subsection{John Brown (1932)}
\label{\detokenize{Volumes/39/01/john_brown:john-brown-1932}}\label{\detokenize{Volumes/39/01/john_brown::doc}}
\sphinxAtStartPar
A singular contretemps has arisen at Harpers Ferry from the project of United Daughters of the Confederacy to erect a monument in memory of the slaves who remained faithful during the John Brown raid. According to their statement: “John Brown, a Connecticut abolitionist, with a criminal record behind him,” planned a raid to free the slaves, but “not a single slave joined the conspirators.” A boulder was accordingly put in place and dedicated and at its dedication, to the surprise of the colored world, Henry T. McDonald, the white President of colored Storer College, and the Reverend George F. Bragg, Jr., colored rector of the St. James Episcopal Church, Baltimore, took part.

\sphinxAtStartPar
During the exercises, white speakers condemned the Haitian Revolution, lauded the “black mammy,” and called John Brown crazy. It was a pro\sphinxhyphen{}slavery celebration and most, people will agree with the Editor of the \sphinxstyleemphasis{Afro\sphinxhyphen{}American} that it was disgraceful for the President of Storer College and Dr. Bragg to have any part in this travesty.

\sphinxAtStartPar
The boulder is ostensibly erected to the memory of a free Negro who was killed in the fight, and the inscription says:
\begin{quote}

\sphinxAtStartPar
“This boulder is erected by the Sons of Confederate Veterans and the United Daughters of the Confederacy as a memorial to Heyward Shepherd, exemplifying the character and faithfulness of thousands of Negroes who, under many temptations throughout subsequent years of war, so conducted themselves that no stain was left upon a record which is the peculiar heritage of the American people, and an everlasting tribute to the best in both races.”
\end{quote}

\sphinxAtStartPar
The statement that no slave helped and fought with John Brown is historically incorrect and one is glad to remember that nearly two hundred thousand Negroes yielded to the “temptation” to fight against slavery in the Civil War and that most of them were former slaves.


\bigskip\hrule\bigskip


\sphinxAtStartPar
\sphinxstyleemphasis{Citation:} Du Bois, W.E.B. 1932. “John Brown” 39(1):467.


\chapter{Random Encounters}
\label{\detokenize{random:random-encounters}}\label{\detokenize{random::doc}}
\sphinxAtStartPar
A chapter of the United Daughters of the Confederacy gave a medal to Colin W. Covington, a University of South Carolina student, for his essay, “The Freedman’s Bureau” in 1911. As detailed in The Crisis, Covington plagiarized his submission from an essay of the same name written by Du Bois for the Atlantic Monthly. Du Bois published portions of each side\sphinxhyphen{}by\sphinxhyphen{}side in the April 1912 issue, concluding that a less modest person would want “that medal clipped from the proud young Southern breast that bears it and pinned on his own.”

\sphinxAtStartPar
Checking links:
{\hyperref[\detokenize{Volumes/24/02/whitecharity::doc}]{\sphinxcrossref{\DUrole{doc,std,std-doc}{White Charity (1922)}}}}







\renewcommand{\indexname}{Index}
\printindex
\end{document}